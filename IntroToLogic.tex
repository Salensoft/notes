% Created 2018-05-29 周二 05:34
\documentclass[11pt]{article}
\usepackage[utf8]{inputenc}
\usepackage[T1]{fontenc}
\usepackage{fixltx2e}
\usepackage{graphicx}
\usepackage{longtable}
\usepackage{float}
\usepackage{wrapfig}
\usepackage{rotating}
\usepackage[normalem]{ulem}
\usepackage{amsmath}
\usepackage{textcomp}
\usepackage{marvosym}
\usepackage{wasysym}
\usepackage{amssymb}
\usepackage{hyperref}
\tolerance=1000
\date{\today}
\title{范伟打天下}
\hypersetup{
  pdfkeywords={},
  pdfsubject={},
  pdfcreator={Emacs 25.2.1 (Org mode 8.2.10)}}
\begin{document}

\maketitle
\tableofcontents

\section{回顾}
\label{sec-1}
\begin{itemize}
\item 推理是从已知到未知的过程
\item 有效推理:两方面的正确
①认同结论 ②从前提到结论的过程也是同意的。 即内容与*形式*
\item[{逻辑的形式}] \begin{itemize}
\item 常项
\item 变项:概念、命题
\end{itemize}
\item 广义逻辑:如果只研究形式太狭隘,因而从必然性的要求下降到可能、得体,这涉及到个体的特性
于是研究有效交际,是否得体
\item 逻辑是识别谬误、驳斥诡辩的工具
\item 诡辩的三个特点:语言潜在的含义
\item 语言作为一个公共的平台,不能随意地误用
\end{itemize}
\section{题目}
\label{sec-2}
黑格尔:哲学的前提:追求真理的勇气、相信信仰的力量
      人的崇高感
\section{第二章 概念}
\label{sec-3}
\subsection{概念的概述}
\label{sec-3-1}
\subsubsection{什么是概念}
\label{sec-3-1-1}
\begin{itemize}
\item 概念是反映对象的特有属性的思维形式
\item 是主观性和客观性的统一
\item 是对象本质属性在人脑中的反映形式,属于意识的范畴,主观。
同时概念是主观对客观事物的反映,因而不能脱离可观,不是纯主观的东西
\item[{所指}] 概念
\item[{能指}] 音响形象(语言说出)
\item 符号由符形(文字,声音)、对象、符释。语言也是符号。
\item[{概念的作用}] \begin{itemize}
\item 概念具有区别的作用(概念通过更细的概念来帮助阐释)
能力:体能、技能、智能
实力、活力、魅力(递进关系)
老牌、名牌、大牌
产品、作品
\item 概念具有建构思想的作用
全新全意为人民服务-->建设有中国特色的社会主义-->先进性教育-->
和谐社会-->八荣八耻-->科学发展观
引进概念:逻辑
\item 概念具有凝结认识成果的作用
从两个文明到三个文明到四个文明
\end{itemize}
\item 概念与语词的关系
\begin{itemize}
\item 概念是思维的基本形式,而语词是语言的基本形式
\item 概念是语词的思想内容,而语词是概念的物质形式
\item 概念与语词的关系
1.任何概念都通过语词来表达,但并非任何语词都表达概念(了、吗、呢)
2.不同的语词有时可以表达同一个概念(语言的丰富性)
3.同一个语词有时可以表达不同的概念(意)思
\end{itemize}
\end{itemize}
\subsubsection{概念的内涵与外延}
\label{sec-3-1-2}
\begin{description}
\item[{内涵}] 概念所反映的对象的特有属性,也就是概念的含义
\item[{外延}] 概念所反映的对象,也就是概念所反映的范围
\item 概念有内涵与外延两个方面,因此要明确概念,既要明确概念的内涵,也要明确概念的外延
\item 概念内涵和外延的确定性和灵活性
概念的内涵和外延是互相依存、互相制约的,概念的内涵是指概念的质的方面,外延是量的
方面。确定某一概念的内涵,也就相应确定了这个概念的外延。从这方面,概念的内涵和外
延具有相对确定性,即在一定时间、地点、条件下,概念的内涵与外延总是确定不变的
另一方面,概念的内涵与外延又具有灵活性,因为概念是人们对客观事物的一种认识,而认
识具有发展性和不完整性,所以随着客观事物的发展和人们在实践中对客观事物认识的不断
深入,某些概念的内涵与外延会发生变化
\end{description}
\subsubsection{概念的种类}
\label{sec-3-1-3}
\begin{description}
\item[{单独概念和普遍概念}] 单独概念是反映某一特定对象的概念,它的外延是独一无二的对象
\begin{itemize}
\item 表达单独概念的话语形式有两种情况:
\begin{enumerate}
\item 专有名词:长城、名字(因为就个人而言)
\item 摹状词: 北京2002年的 \textbf{第一} 场雪
\end{enumerate}
\end{itemize}
摹状词就是通过对对象特征的描述来指称某一特定的事物的词组
普遍概念是反映某一类对象的概念,它的外延有两个或两个以上的分子所组成的类
\item[{集合概念和非集合概念}] \begin{itemize}
\item 集合概念就是反映集合体的概念。集合体的特点是,集合体所具有的属性,组成集合体的个体不一定具有
多半是不具有
中国共产党,人民,人类
×我是人民
\item 非集合概念就是反映非集合体的概念。非集合体,通常是指事物类,事物类的特点是,事物类所
具有的属性,组成事物类的分子一定个体都具有
党员,人,老虎
我是人
\item 分析一个概念是集合概念还是非集合体概念,要把这个概念放在具体的语言环境中进行
人定胜天:集合概念
人是会思维的:非集合概念
我们班的同学来自全国各地:集合概念
我们班的同学都是中国人:非集合概念
\end{itemize}
\item[{正概念、负概念}] \begin{itemize}
\item 正概念也叫肯定概念,就是以具有某种属性的事物为反映对象的概念
\item 负概念也叫否定概念,就是以不具有某种属性的事物为反映对象的概念
\item 表达负概念的语词都带有否定词
\item 表达负概念的语词中的否定词,如“非”、“不”、“未”都是可分离的
\end{itemize}
\item[{实体概念、属性概念}] \begin{itemize}
\item 分类标准:概念所反映的是对象的本身还是属性
\item 实体概念又称具体概念,是指以事物或现象的本身为反映的概念。如导体、三好学生
\item 属性概念又称抽象概念,是指以事物或现象的属性为反映的概念。导电性、优秀
\end{itemize}
\end{description}
\subsubsection{概念间的关系}
\label{sec-3-1-4}
两个概念间的外延关系有如下五种:
\begin{description}
\item[{全同关系}] \begin{itemize}
\item 有两个概念a和b,如果所有a都是b,同时所有b都是a,那么,a与b之间的关系就是全同关系
西红柿,番茄:内涵不一样,
\item 具有全同关系的两个概念,在表述时, \textbf{有时} 可以互相替代,涉及到内涵不可以替换,涉及到
外延是可以替换
\end{itemize}
\item[{真包含和真包含于关系}] \begin{itemize}
\item 如果所有b都是a,但有的a不是b,那么a与b之间的关系都是真包含关系,也叫属种关系,
这里的a为属概念,b为种概念
\item b对于a是真包含于
\item 属种关系同整体与部分关系,应加以区别,不可混淆。例如浙江大学与竺院的关系
(整体与部分)、树木与森林(整体与部分)、文化名城与杭州(属种关系)、概念与思维形式(属种关系)。
\end{itemize}
\item[{交叉关系}]
\item[{全异关系}] \begin{itemize}
\item 如果所有a都不是b,那么ab就是全异关系
\item 在属概念c中,如果a、b有全异关系,而且a与b的外延之和等于c的外延,那么a、b就是矛盾关系
机动车$\backslash$非机动车,党员/非党员
\item 小于c的外延,就是反对关系
漂亮/丑陋,同意$\backslash$反对
\end{itemize}
\end{description}
\subsubsection{概念的概括和限制}
\label{sec-3-1-5}
\begin{description}
\item[{概念的内涵与外延之间的反变关系}] 在具有属种关系的概念之间,外延越大,则内涵越少,反之,外延越小,内涵越多,这种关系就叫做
概念的内涵与外延之间的反变关系
\begin{itemize}
\item “反变关系”一例
刘晓庆在《我的路》中说:
“做人难。做女人难。做名女人更难。做单身的名女人,难乎其难”
人——女人——名女人——单身名女人
这四个概念构成了一个属种关系的系列,相互之间的内涵与外延之间就存在一个反变关系
\end{itemize}
\item[{概括和限制}] \begin{itemize}
\item 概念的概括,亦称概念外延的扩大法,它是通过减少某一概念的内涵,从而使该概念变为外延较大
的一个概念的逻辑方法。概念的概括是一个由种概念过渡到属概念的思维过程
\begin{itemize}
\item 风声雨声读书声 \textbf{声声} 入耳
\item 家事国事天下事 \textbf{事事} 关心
\item 常表现为减去附加语或限制词。如:中国工人阶级 -> 工人阶级 基本职称 -> 职称
\item 概念的概括可以连续进行:概括只能在属种关系的概念之间进行,概念概括的极限是哲学范畴
\end{itemize}
\item 概念的限制,亦称概念外延的缩小法,它是通过增加某一概念的内涵,从而使改概念变为外延较小
的一个概念的逻辑方法。概念的限制是由属概念
\item 概念与限制的作用
正确地应用概念的概括与限制,有助于准确地使用概念,表达思想
\end{itemize}
\item[{注意点}] \begin{itemize}
\item 单独概念不能限制
\item 外延不能过宽
\end{itemize}
\end{description}
\subsection{定义}
\label{sec-3-2}
\subsubsection{定义及其组成}
\label{sec-3-2-1}
\begin{description}
\item[{真实定义的组成}] \begin{itemize}
\item 被定义项,就是其内涵要被揭示的概念。$D_s$
\item 定义项,揭示被定义项内涵的概念,$D_P$
\item 定义联项,把定义项和定义项联结起来组成定义
\end{itemize}
\end{description}
\subsubsection{下定义的方法}
\label{sec-3-2-2}
\begin{itemize}
\item 最常用的真实定义是邻近的属加种差的定义,步骤
\begin{itemize}
\item 找被定义项的邻近的属概念
\item 找出种差,也就是它的特有属性
\item 按照$D_S$ 就是\$D$_{\text{P}}$\$这一形式来定义
\end{itemize}
\end{itemize}
\subsubsection{语词定义}
\label{sec-3-2-3}
\begin{itemize}
\item 用来说明或规定语词意义的定义,叫做语词定义。
\end{itemize}
\subsubsection{定义的规则}
\label{sec-3-2-4}
\begin{description}
\item[{定义项的外延和被定义项的外延必须是全同的}] \begin{itemize}
\item “定义过宽” : 刑法是国家制定的法律
\item “定义过窄” : 刑法是惩治贪污犯的法律
\end{itemize}
\item[{定义项不能直接或间接地包含被定义项}] \begin{itemize}
\item “同语反复” : 罪犯是犯了罪的人
\item “循环定义” : 辩证法是同形而上学根本对立的宇宙观,形而上学就是同辩证法根本对立的宇宙观
\end{itemize}
\item[{定义必须使用含义确定的语词,不能隐喻}]
\item[{不能用否定形式}]
\end{description}
\subsection{划分}
\label{sec-3-3}
\subsubsection{什么是划分}
\label{sec-3-3-1}
\begin{itemize}
\item 划分就是通过把一个属概念分为若干种概念,从而明确概念外延的逻辑方法
\item 划分有三个要素:母项、子项与划分的标准。母项就是被划分的属概念,子项是划分所得的种概念,每次
划分必须以对象的一定属性作根据,作根据的一定属性就是划分的标准
\item 知识: 意会知识、言传知识
\end{itemize}
\subsubsection{划分的方法}
\label{sec-3-3-2}
\begin{itemize}
\item 一次划分与连续划分
只有母项与子项两层的划分,就是一次划分。母项与子项有三层或三层以上,就是连续划分
\item 二分法
以对象有无某种属性为根据的划分,叫做二分法
\item 科学分类
根据对象的本质属性或显著特征将对象分为若干类。它是划分的特殊形式
当代科技革命最有影响的三大领域
\end{itemize}
\subsection{练习题}
\label{sec-3-4}
\begin{itemize}
\item
\end{itemize}
\section{简单命题及其推理}
\label{sec-4}
\subsection{命题和推理的概述}
\label{sec-4-1}
\begin{itemize}
\item 什么是命题
命题是有真假意义的语句所表达的思想
\item 什么是判断
判断是对思维对象断定的思维形式
\item 命题与判断的关系
共同点:都有真假
不通点:判断有所断定而命题未加断定(如果A比B大,A强)
\item 命题与语句的关系(命题是形式化的,公式)
\begin{itemize}
\item 命题是语句的思想内容,语句是命题的语言形式
\item 命题与语句的对应关系:
\begin{itemize}
\item 任何一个命题都要通过语句表达,但语句不一定表达命题
\begin{itemize}
\item 索引语句就是含有称谓代词、指示代词、时间名词、时间副词、时间助词等索引
语词的语句
\item 祈使句、疑问句、感叹句不是命题
\end{itemize}
\item 同一个命题可用不同的语句来表达(感情不在意义之内)
\begin{itemize}
\item 任何人都会犯错误
\item 没有人不会犯错误
\end{itemize}
\item 同一个语句可以表达不同的命题
\end{itemize}
\end{itemize}
\item 简单命题与复合命题
\begin{itemize}
\item 简单命题就是不包含其它命题作为其组成部分的命题
\item 复合命题就是包含了其它命题作为其组成部分的命题
\item 并非所有的鸟都会飞 复合命题
\end{itemize}
\item 推理
\begin{itemize}
\item 推理是依据已知的命题得到新命题的思维形式
\item 组成:有前提和结论两个部分组成,蕴涵关系
\item 推理的分类:
\begin{itemize}
\item 根据前提到结论的思维进程的不同,分为:演绎推理、归纳推理和类比推理
\begin{itemize}
\item 演绎: 一般->一般 或 一般->特殊 外延未扩大
\item 归纳: 特殊->一般
\item 类比: 个体->个体
\end{itemize}
\item 根据前提和结论之间是否有蕴涵关系,推理可分为:必然性推理(演绎)、或然性推理
\item 根据前提的数量,分为:直接推理、间接推理
\end{itemize}
\end{itemize}
\item 推理形式的有效性
\begin{itemize}
\item 一个推理形式是有效的,当且仅当具有此推理形式的任一推理都不出现真前提和假结论
\end{itemize}
\end{itemize}
\subsection{性质命题}
\label{sec-4-2}
\begin{description}
\item[{性质命题}] \begin{itemize}
\item 性质命题就是反映对象具有或不具有某性质的命题。在性质命题中所作的断定是直接的,因此
也叫做直言命题
\end{itemize}
\item 性质命题的组成
\begin{itemize}
\item 主项 S
\item 谓项 P
\item 联项:表示主项与谓项之间的联系性质的词项。一是肯定联项(是),而是否定联项(不是)
\item 量项:表示对象的数量的词项。全称量词(所有)、特称量项(有)
\end{itemize}
\item 性质命题的种类
\begin{itemize}
\item 按 \textbf{性质命题的质} 的不同来分:
肯定命题:A
否定命题
\item 按性质命题的 \textbf{量} 的不同来分:
全称命题
单称命题
特称命题:有的
\item 按命题 \textbf{质和量} 的结合来分:
质$\times$ 量
A E I O
全称肯定 A 全程否定 E 特称肯定 I 特称否定 O 单称肯定 U 单称否定 V
\end{itemize}
\begin{center}
\begin{tabular}{llllll}
外延关系 & 全同 & 真包含于 & 真包含 & 交叉 & 全异\\
SAP & T & T & F & F & F\\
SEP & F & F & F & F & T\\
SIP & T & T & T & T & F\\
SOP & F & F & T & T & T\\
命题 & 主项 & 谓项\\
\end{tabular}
\end{center}
\begin{tikzpicture}
\node (A) at (0, 1) {A};
\node (T) at (0, 0) {T};
\node (E) at (1, 1) {E};
\node (O) at (1, 0) {O};
\draw (A) -- (E) {反对};
\draw (A) -- (T) {差等};
\draw (A) -- (O) {矛盾};
\draw (T) -- (E) {矛盾};
\draw (T) -- (O) {下反对};
\draw (E) -- (O) {差等};
\end{tikzpicture}
\item 对当关系的直接推理
\begin{itemize}
\item 以矛盾关系为依据的对当推理
由真推假:
A->!O E->!I I->!E O->!A
由假推真
!A->O !E->I !I->E !O->A
\item 以反对关系为依据的对当推理
由真推假:
A->!E E->!A
\item 以下反对为依据的对当推理
由假推真:
!I->O !O->I
\item 以差等关系为依据的对当推理
全称真推特称真
特称假推全称假
\end{itemize}
\item 性质命题主、谓项的周延性
\begin{itemize}
\item 项的周延性:指在性质命题中对主项谓项外延数量的反映情况。如果在一个命题中,对它的主项的
全部外延作了反映,那这个命题的主项(或谓项)就是周延的;如果卫队主项(或谓项)的全部外延
作反映,那这个命题主项(或谓项)就不周延
有的、全部
\begin{center}
\begin{tabular}{lll}
命题 & 主项 & 谓项\\
所有S都是P & 周延 & 不周延\\
所有S都不是P & 周延 & 周延\\
有的S是P & 不周延 & 不周延\\
有的S不是P & 不周延 & 周延\\
\end{tabular}
\end{center}
看谓项前的联项
是:未涵盖所有P 不是:涵盖了所有P
\end{itemize}
\item[{命题变形的直接推理}] \begin{itemize}
\item 命题变形推理
通过改变性质命题的联项(肯定->否定,否定->肯定),或者改变性质命题的主项与谓项的位置,
或既改变联项又改变主项与谓项的位置,从而得出结论的推理
\begin{itemize}
\item 换质法
e.g. 双重否定句
\begin{enumerate}
\item 结论与前提不同质(肯定->否定)
\item 前提的主项保持不变,结论的谓项是前提谓项的矛盾概念
\item A->E E->A I->O O->I
\end{enumerate}
\item 换位法
\begin{enumerate}
\item 结论与前提的质相同
\item A->I E->E I->I O不能换位
\item 规则:
只更换主项与谓项的位置命题,质不变
原命题中不周延的项换位后仍不周延
\item
\end{enumerate}
\end{itemize}
\end{itemize}
\end{description}
\subsection{三段论}
\label{sec-4-3}
\begin{itemize}
\item 三段论及其组成
\begin{itemize}
\item 三段论是借助于一个共同概念把两个直言命题联结起来而得到一个新的直言命题的演绎推理
\item 有 \textbf{一般性前提推出个别性结论}
\item we
\begin{center}
\begin{tabular}{l}
PAM\\
SAM\\
\hline
SAP\\
\end{tabular}
\end{center}
\item[{三段论的组成}] \begin{itemize}
\item 概念的角度分析
\begin{enumerate}
\item 小项,结论的主项,S
\item 大项,谓项,P
\item 中项,前提中的共同项,M
\end{enumerate}
\item 从命题的角度分析
\begin{enumerate}
\item 大前提,包含大项的前提
\item 小前提,包含小项的前提
\end{enumerate}
\end{itemize}
\end{itemize}
\item 一般规则
\begin{enumerate}
\item 一个正确的三段论有且只有三个
\item 中项至少要周延一次
\item 前提中不周延的项,到结论中不得周延
熊猫是应当受到保护的,白天鹅不是熊猫,白天鹅不应当受到国家保护
人非草木,孰能无情
人不是草木
有一部分无情的是草木
人不能无情
\item 两个否定前提不能推出结论
两个前提都否定,大项与小项都排斥,中项就不能起到联结大项和小项的媒介作用
\item 如果前提有一否定,则结论否定;如果结论否定,则前提有一否定
\begin{itemize}
\item 错:
草木是无情的,我不是草木,所以我不是无情地。(无情被扩大了)
凡鸟皆会飞,我不是鸟,所以我不会飞。(飞是不周延的)
\end{itemize}
\item 两个特称前提不能推出结论
两个都I:I命题没有周延的项
一I一O:I不周延,O有周延,周延项是中项,结论就是否定的,大项原来是周延的
两否定:4.错
\item 如果前提有一特称,则结论特称
一A一I:结论肯定,I不周延,如果结论不特称,因为只有一个周延项,结论是A,但结论又得是否定,矛盾
一E一I:E有两个周延的项,结论是否定的,周延的给了否定,小项就不能否定
一A一O:结论谓项周延,小项不周延
一E一O:两否定
\end{enumerate}
\item 三段论的格
\begin{itemize}
\item 三段论的格就是有中项在前提中的不同位置所构成的不同形式
\item[{三段论的四个格}] \begin{itemize}
\item 第一格(审判格):中项为大前提的主项和小前提的谓项
\begin{itemize}
\item 规则
大前提是全称
小前提是肯定
\begin{itemize}
\item 因为:
M周延,若S-M中周延,S-M否定,S-P否定,M-P中P周延,两个否定;若M-P周延,
\end{itemize}
\item 特点
从一般推出特殊,称为审判格
M - P
  $\backslash$
S - M

\rule{\linewidth}{0.5pt}
S - P
\end{itemize}
\item 第二个(区别格):中项为大小前提的谓项
\begin{itemize}
\item 规则:
大前提全称
前提中有一个否定(中项必须周延一次,结论为否定,P周延,P-M中P就周延了)
\item 特点
前提中必有一个是否定的,被称为“区别格“
P - M
\begin{center}
\begin{tabular}{}
\\
\end{tabular}
\end{center}
S - M

\rule{\linewidth}{0.5pt}
S - P
\end{itemize}
\item 例证格、反驳格:中项为大、小前提的主项
\begin{itemize}
\item 规则:
小前提必须肯定
结论必须特称
\item 特点
结论特称
M - P
\begin{center}
\begin{tabular}{}
\\
\end{tabular}
\end{center}
M - S

\rule{\linewidth}{0.5pt}
S - P
\end{itemize}
\item 第四格
P - M
  /
M - S

\rule{\linewidth}{0.5pt}
S - P
\end{itemize}
\end{itemize}
\item 三段论的式
\begin{itemize}
\item 三段论的式就是前提和结论的质(肯定或否定)量(全称或特称)的组合形式。
\item 第一格有六个式
大     A      E
      /$\backslash$    / $\backslash$
小   A  I   A  I
    /$\backslash$  |  /$\backslash$  |
结论A I  I E O  O
\item 第二格
大     A      E
      /$\backslash$     /$\backslash$
小    E O    A I
     /$\backslash$ \   /$\backslash$ \
结论 E O O  E O O
\item 第三格
结论   A  I A  E O E
       $\backslash$/ /    $\backslash$/ /
小      A I     A I
        $\backslash$/      $\backslash$/
大       I       O
\item 三段论的有效式()表示弱式,因为已经在前面了
第一格:AAA、EAE、AII、EIO、(AAI)、(EAO)
第二格:AEE、EAE、AOO、EIO、(AEO)、(EAO)
第三格:AAI、EAO、AII、EIO、IAI、OAO
第四格:AAI、EAO、AEE、EIO、IAI、(AEO
\end{itemize}
\item 三段论的省略式
\begin{itemize}
\item 在日常语言的表达中,省略某个部分的三段论,叫做三段论的省略式
\item 省略三段论的恢复
\begin{enumerate}
\item 根据联词,确定结论是否被省略
\item 根据结论确定大、小项,进而确定大前提或小前提是否被省略
\item 补充被省略的部分,从而构成三段论的完整形式
\end{enumerate}
\end{itemize}
\item 练习:
\begin{enumerate}
\item ()()()
S  O  M

\rule{\linewidth}{0.5pt}
S  () P
\item M  O  P
() () ()

\rule{\linewidth}{0.5pt}
S () P
\item () E ()
M  I S

\rule{\linewidth}{0.5pt}
\end{enumerate}
\end{itemize}
\subsection{关系命题}
\label{sec-4-4}
\section{复合命题及其推理}
\label{sec-5}
\subsection{复合命题及其特点}
\label{sec-5-1}
\begin{itemize}
\item 含有其他命题的命题
\end{itemize}
\subsection{联言命题及其推理}
\label{sec-5-2}
\begin{itemize}
\item 反映几种事物情况同时存在的命题
\item p 并且 q
\item 组成
\begin{enumerate}
\item 联言支:p,q
\item 合取符号:$\wedge$
\end{enumerate}
\item 一个联言命题是真的,当且仅当,全部联言支都是真的
真值表
\end{itemize}
\subsection{选言命题及其推理}
\label{sec-5-3}
\begin{itemize}
\item 反映几种可能事物情况至少有一种存在,也可以同时存在的命题
\item $\vee$
\end{itemize}
\subsection{不相容选言命题}
\label{sec-5-4}
\begin{itemize}
\item 反映几种食物情况有且只有一种存在的宣言命题
\item 要么p要么q
\item xor
\end{itemize}
\subsection{假言命题}
\label{sec-5-5}
\begin{itemize}
\item 反映某一事物情况是另一事物情况的条件的命题,叫做假言命题,又叫条件命题
\item 条件关系的种类
\begin{enumerate}
\item 充分条件
A->B
\item 必要条件
没有A就没有B,有A不一定有B
\item 充分必要条件
有A就有B,没有A就没有B
\end{enumerate}
\end{itemize}
\subsubsection{充分条件假言命题}
\label{sec-5-5-1}
\begin{itemize}
\item 真假有前件和后件的真假来确定
\item 为真当且仅当不会出现前件真而后件假
\end{itemize}
\subsubsection{必要条件假言命题}
\label{sec-5-5-2}
\begin{itemize}
\item 只有p才有q
\end{itemize}
\subsubsection{充分必要条件假言命题}
\label{sec-5-5-3}
\subsection{负命题}
\label{sec-5-6}
\begin{itemize}
\item 否定某个命题的命题
\item 负命题与性质命题的否定命题是不同的,前者是复合命题,后者是简单命题
\item 并非p
\item 负命题的真假由支命题的真假来确定
\item 性质命题的负命题及等值命题
\begin{enumerate}
\item 并非所有S是P
等值命题:有S不是P
\item 并非所有S不是P
有S是P
\item 并非有S是P
所有S不是P
\item 并非有S不是P
所有S是P
\end{enumerate}
\end{itemize}
\subsubsection{复合命题的负命题及其推理}
\label{sec-5-6-1}
\begin{itemize}
\item $\neg(p\wedge q)\equiv \neg p\lor \neg q$
\item $\neg(p\lor q)\equiv \neg p\wedge \neg q$
\end{itemize}
% Emacs 25.2.1 (Org mode 8.2.10)
\end{document}
