% Created 2019-04-01 一 09:51
% Intended LaTeX compiler: pdflatex
\documentclass[11pt]{article}
\usepackage[utf8]{inputenc}
\usepackage[T1]{fontenc}
\usepackage{graphicx}
\usepackage{grffile}
\usepackage{longtable}
\usepackage{wrapfig}
\usepackage{rotating}
\usepackage[normalem]{ulem}
\usepackage{amsmath}
\usepackage{textcomp}
\usepackage{amssymb}
\usepackage{capt-of}
\usepackage{hyperref}
\usepackage{tikz}
\author{wugouzi}
\date{\today}
\title{Compiler}
\hypersetup{
 pdfauthor={wugouzi},
 pdftitle={Compiler},
 pdfkeywords={},
 pdfsubject={},
 pdfcreator={Emacs 26.1 (Org mode 9.1.14)}, 
 pdflang={English}}
\begin{document}

\maketitle
\tableofcontents

\section{Chap5 Bottom-up parsing}
\label{sec:org6337b5e}
\subsection{Overview of bottom-up parsing}
\label{sec:org72eeec2}
\begin{itemize}
\item A bottom-up parser uses an \textbf{explicit stack} to perform a parse
\item The parsing stack will contain both tokens and nonterminals
\begin{center}
\begin{tabular}{ll}
\$ & inputstring \$\\
\ldots{} & \ldots{}\\
\$StartSymbol & \$accept\\
\end{tabular}
\end{center}
\item \textbf{right-most} derivation -- backward
start with the tokens; end with the start symbol
(1+2+(3+4))+5 ⇐
(E+2+(3+4))+5 ⇐
(S+2+(3+4))+5 ⇐
(S+E+(3+4))+5 ⇐
(S+(3+4))+5 ⇐
(S+(E+4))+5 ⇐
(S+(S+4))+5 ⇐
(S+(S+E))+5 ⇐
(S+(S))+5 ⇐
(S+E)+5 ⇐
(S)+5 ⇐
E+5 ⇐
S+5 ⇐
S+E ⇐
S
\item \textbf{parsing actions}: a sequence of \textbf{shift} and \textbf{reduce} operations
\textbf{parser state}: a stack of terminals and non-terminals
\textbf{current derivation step} = always stack + input
\begin{center}
\begin{tabular}{llr}
derivation & step stack & unconsumed input\\
(1+2+(3+4))+5 ⇐ &  & (1+2+(3+4))+5\\
 & ( & 1+2+(3+4))+5\\
(E+2+(3+4))+5 ⇐ & (E & +2+(3+4))+5 ⇐\\
(S+2+(3+4))+5 ⇐ & (S & +2+(3+4))+5\\
 & (S+ & 2+(3+4))+5\\
 & (S+2 & +(3+4))+5\\
(S+E+(3+4))+5 ⇐ & (S+E & +(3+4))+5\\
\end{tabular}
\end{center}
\item 1. \textbf{shift}: shift a terminal from the front of the input to the top of the
stack 
\begin{enumerate}
\item \textbf{reduce}: reduce a string α at the top of the stack to a nonterminal A,
given the BNF choice A ⟶ α
\end{enumerate}

A bottom-up parser: \textbf{shift-reduce parser}
\item One further feature of bottom-up parsers: grammars are always augmented
with a \textbf{new start symbol}. if S is the start symbol, a new start symbol S' is
added to the grammar :  S' →S

\item example
S'->S
S ->(S)S|ε

S'=>S=>(S)S=>(S)=>()
\begin{center}
\begin{tabular}{rlrl}
 & Parsing stack & Input & Action\\
1 & \$ & ( ) \$ & Shift\\
2 & \$ ( & ) \$ & Reduce  S -> ε\\
3 & \$ (S & ) \$ & Shift\\
4 & \$ (S ) & \$ & Reduce  S -> ε\\
5 & \$ (S ) S & \$ & Reduce S --> (S) S\\
6 & \$S & \$ & Reduce S'--> S\\
7 & \$S' & \$ & Accept\\
\end{tabular}
\end{center}

\item example
E'->E
E->E+n|n

E'=>E=>E+n=>n+n
\begin{center}
\begin{tabular}{rlrl}
 & Parsing stack & Input & Action\\
1 & \$ & n+n\$ & Shift\\
2 & \$n & +n\$ & Reduce  E->n\\
3 & \$E & +n\$ & Shift\\
4 & \$E+ & n\$ & Shift\\
5 & \$E+n & \$ & Reduce E->E+n\\
6 & \$E & \$ & Reduce E'->E\\
7 & \$E' & \$ & Accept\\
\end{tabular}
\end{center}
\item[{Right sentential form}] \begin{itemize}
\item A \textbf{sentential} form is any string derivable from the start symbol. Note
that this includes the forms with non-terminals at intermediate steps as
well.
\item A \textbf{right-sentential form} is a sentential form that occurs in a step of
rightmost derivation (RMD).
\item A \textbf{sentence} is a sentential form consisting only of terminals
\end{itemize}

E,E+,E+n are \textbf{viable prefixes} of the right sentential form E+n.
The sequence of symbols on the parsing stack is called \textbf{viable prefix} of the
right sentential form
\end{itemize}

\subsection{Finite automata of LR(0) items and LR(0) parsing}
\label{sec:orgeb36e07}
\begin{itemize}
\item An \textbf{LR(0) item} of a context-free grammar: a production choice with a
distinguished position in its right-hand side
\item If \textbf{A → α}, \textbf{βγ = α}, then \textbf{A → β · γ} is an LR(0) item
\item Example
S' → S
S → (S)S | ε

S' → ·S
S' → S·
S → ·(S)S
S → (·S)S
S → (S·)S
S → (S)·S
S → (S)S·
S → ·
\end{itemize}
\subsubsection{Finite automata of items}
\label{sec:org66b2f99}
\begin{itemize}
\item The LR(0) items: as the state of a finite automata
\item construct the DFA of sets of LR(0) using the subset construction from NFA
\item If X is a token or a nonterminal
\begin{tikzpicture}
[place/.style={circle,minimum size=5mm}]
\node (x1) at (0,0) [place] {$A\to\alpha\cdot X\eta$};
\node (x2) at (5,0) [place] {$A\to\alpha X\cdot\eta$};
\draw [->] (x1) to node [above] {X} (x2);
\end{tikzpicture}
\item If X is a token, then this transition corresponds to a shift of X from the
input to the top of the stack during a parse
\item if X is a nonterminal
X will never appear as an input symbol
\begin{tikzpicture}
\node (x1) at (0,0) [circle] {$A\to\alpha\cdot X\eta$};
\node (x2) at (5,0) [circle] {$X\to\cdot\beta$};
\draw [->] (x1) to node [above] {$\epsilon$} (x2);
\end{tikzpicture}
\item The \textbf{start state} of the NFA ↔ the \textbf{initial state} of the parser: the stack is
empty
\item the solution is to augment the grammar by a single production S' -> S
\item \textbf{S'->·S} the \textbf{start state} of the NFA
\end{itemize}
\subsubsection{The LR(0) parsing algorithm}
\label{sec:orgae19f64}
\begin{itemize}
\item the parsing stack to store: \textbf{symbols} and \textbf{state numbers}
\item pushing the new \textbf{state number} onto the parsing stack after each push of \textbf{a
symbol}
\item Let s be the current state. Then actions are
\begin{enumerate}
\item if state s contains any item of the form \textbf{A -> α·Xβ} (X is a terminal).
Then the action is to shift the current input token onto the stack
\item If state s contains any \textbf{complete item} (an item of the form \textbf{A->γ·}),
then the action is to reduce by the rule \textbf{A->γ·}
\begin{itemize}
\item A \textbf{reduction} by the rule \textbf{S'->S} where S' is the start state
\item \textbf{acceptance} if the input is empty
\item \textbf{Error} if the input is not empty
\end{itemize}
\end{enumerate}
\item A grammar is \textbf{LR(0)} grammar if the above rules are unambiguous
\item A grammar is \textbf{LR(0)} iff
\begin{itemize}
\item Each state is a shift state
\item A reduce state containing a single complete item
\end{itemize}
\end{itemize}
\end{document}