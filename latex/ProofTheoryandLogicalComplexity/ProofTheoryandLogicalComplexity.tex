% Created 2020-01-30 四 15:39
% Intended LaTeX compiler: pdflatex
\documentclass[11pt]{article}
\usepackage[utf8]{inputenc}
\usepackage[T1]{fontenc}
\usepackage{graphicx}
\usepackage{grffile}
\usepackage{longtable}
\usepackage{wrapfig}
\usepackage{rotating}
\usepackage[normalem]{ulem}
\usepackage{amsmath}
\usepackage{textcomp}
\usepackage{amssymb}
\usepackage{capt-of}
\usepackage{hyperref}
\usepackage{minted}
% TIPS
% \substack{a\\b} for multiple lines text





% pdfplots will load xolor automatically without option
\usepackage[dvipsnames]{xcolor}

\usepackage{forest}
% two-line text in node by [two \\ lines]
% \begin{forest} qtree, [..] \end{forest}
\forestset{
  qtree/.style={
    baseline,
    for tree={
      parent anchor=south,
      child anchor=north,
      align=center,
      inner sep=1pt,
    }}}
%\usepackage{flexisym}
% load order of mathtools and mathabx, otherwise conflict overbrace

\usepackage{mathtools}
%\usepackage{fourier}
\usepackage{pgfplots}
\usepackage{amsthm}
\usepackage{amsmath}
%\usepackage{unicode-math}
%
\usepackage{commath}
%\usepackage{,  , }
\usepackage{amsfonts}
\usepackage{amssymb}
% importing symbols https://tex.stackexchange.com/questions/14386/importing-a-single-symbol-from-a-different-font
%mathabx change every symbol
% use instead stmaryrd
%\usepackage{mathabx}
\usepackage{stmaryrd}
\usepackage{empheq}
\usepackage{tikz}
\usepackage{tikz-cd}
%\usepackage[notextcomp]{stix}
\usetikzlibrary{arrows.meta}
\usepackage[most]{tcolorbox}
%\utilde
%\usepackage{../../latexpackage/undertilde/undertilde}
% left and right superscript and subscript
\usepackage{actuarialsymbol}
\usepackage{threeparttable}
\usepackage{scalerel,stackengine}
\usepackage{stackrel}
% \stackrel[a]{b}{c}
\usepackage{dsfont}
% text font
\usepackage{newpxtext}
%\usepackage{newpxmath}

%\newcounter{dummy} \numberwithin{dummy}{section}
\newtheorem{dummy}{dummy}[section]
\theoremstyle{definition}
\newtheorem{definition}[dummy]{Definition}
\newtheorem{corollary}[dummy]{Corollary}
\newtheorem{lemma}[dummy]{Lemma}
\newtheorem{proposition}[dummy]{Proposition}
\newtheorem{theorem}[dummy]{Theorem}
\theoremstyle{definition}
\newtheorem{example}[dummy]{Example}
\theoremstyle{remark}
\newtheorem*{remark}{Remark}


\newcommand\what[1]{\ThisStyle{%
    \setbox0=\hbox{$\SavedStyle#1$}%
    \stackengine{-1.0\ht0+.5pt}{$\SavedStyle#1$}{%
      \stretchto{\scaleto{\SavedStyle\mkern.15mu\char'136}{2.6\wd0}}{1.4\ht0}%
    }{O}{c}{F}{T}{S}%
  }
}

\newcommand\wtilde[1]{\ThisStyle{%
    \setbox0=\hbox{$\SavedStyle#1$}%
    \stackengine{-.1\LMpt}{$\SavedStyle#1$}{%
      \stretchto{\scaleto{\SavedStyle\mkern.2mu\AC}{.5150\wd0}}{.6\ht0}%
    }{O}{c}{F}{T}{S}%
  }
}

\newcommand\wbar[1]{\ThisStyle{%
    \setbox0=\hbox{$\SavedStyle#1$}%
    \stackengine{.5pt+\LMpt}{$\SavedStyle#1$}{%
      \rule{\wd0}{\dimexpr.3\LMpt+.3pt}%
    }{O}{c}{F}{T}{S}%
  }
}

\newcommand{\bl}[1] {\boldsymbol{#1}}
\newcommand{\Wt}[1] {\stackrel{\sim}{\smash{#1}\rule{0pt}{1.1ex}}}
\newcommand{\wt}[1] {\widetilde{#1}}
\newcommand{\tf}[1] {\textbf{#1}}


%For boxed texts in align, use Aboxed{}
%otherwise use boxed{}

\DeclareMathSymbol{\widehatsym}{\mathord}{largesymbols}{"62}
\newcommand\lowerwidehatsym{%
  \text{\smash{\raisebox{-1.3ex}{%
    $\widehatsym$}}}}
\newcommand\fixwidehat[1]{%
  \mathchoice
    {\accentset{\displaystyle\lowerwidehatsym}{#1}}
    {\accentset{\textstyle\lowerwidehatsym}{#1}}
    {\accentset{\scriptstyle\lowerwidehatsym}{#1}}
    {\accentset{\scriptscriptstyle\lowerwidehatsym}{#1}}
}

\usepackage{graphicx}
    
% text on arrow for xRightarrow
\makeatletter
%\newcommand{\xRightarrow}[2][]{\ext@arrow 0359\Rightarrowfill@{#1}{#2}}
\makeatother


\newcommand{\dom}[1]{%
\mathrm{dom}{(#1)}
}

% Roman numerals
\makeatletter
\newcommand*{\rom}[1]{\expandafter\@slowromancap\romannumeral #1@}
\makeatother

\def \fR {\mathfrak{R}}
\def \bx {\boldsymbol{x}}
\def \bz {\boldsymbol{z}}
\def \ba {\boldsymbol{a}}
\def \bh {\boldsymbol{h}}
\def \bo {\boldsymbol{o}}
\def \bU {\boldsymbol{U}}
\def \bc {\boldsymbol{c}}
\def \bV {\boldsymbol{V}}
\def \bI {\boldsymbol{I}}
\def \bK {\boldsymbol{K}}
\def \bt {\boldsymbol{t}}
\def \bb {\boldsymbol{b}}
\def \bA {\boldsymbol{A}}
\def \bX {\boldsymbol{X}}
\def \bu {\boldsymbol{u}}
\def \bS {\boldsymbol{S}}
\def \bZ {\boldsymbol{Z}}
\def \bz {\boldsymbol{z}}
\def \by {\boldsymbol{y}}
\def \bw {\boldsymbol{w}}
\def \bT {\boldsymbol{T}}
\def \bF {\boldsymbol{F}}
\def \bS {\boldsymbol{S}}
\def \bm {\boldsymbol{m}}
\def \bW {\boldsymbol{W}}
\def \bR {\boldsymbol{R}}
\def \bQ {\boldsymbol{Q}}
\def \bS {\boldsymbol{S}}
\def \bP {\boldsymbol{P}}
\def \bT {\boldsymbol{T}}
\def \bY {\boldsymbol{Y}}
\def \bH {\boldsymbol{H}}
\def \bB {\boldsymbol{B}}
\def \blambda {\boldsymbol{\lambda}}
\def \bPhi {\boldsymbol{\Phi}}
\def \btheta {\boldsymbol{\theta}}
\def \bTheta {\boldsymbol{\Theta}}
\def \bmu {\boldsymbol{\mu}}
\def \bphi {\boldsymbol{\phi}}
\def \bSigma {\boldsymbol{\Sigma}}
\def \lb {\left\{}
\def \rb {\right\}}
\def \la {\langle}
\def \ra {\rangle}
\def \caln {\mathcal{N}}
\def \dissum {\displaystyle\Sigma}
\def \dispro {\displaystyle\prod}
\def \E {\mathbb{E}}
\def \Q {\mathbb{Q}}
\def \N {\mathbb{N}}
\def \V {\mathbb{V}}
\def \R {\mathbb{R}}
\def \P {\mathbb{P}}
\def \A {\mathbb{A}}
\def \Z {\mathbb{Z}}
\def \I {\mathbb{I}}
\def \C {\mathbb{C}}
\def \cala {\mathcal{A}}
\def \calb {\mathcal{B}}
\def \calq {\mathcal{Q}}
\def \calp {\mathcal{P}}
\def \cals {\mathcal{S}}
\def \calg {\mathcal{G}}
\def \caln {\mathcal{N}}
\def \calr {\mathcal{R}}
\def \calm {\mathcal{M}}
\def \calc {\mathcal{C}}
\def \calf {\mathcal{F}}
\def \calk {\mathcal{K}}
\def \call {\mathcal{L}}
\def \calu {\mathcal{U}}
\def \bcup {\bigcup}


\def \uin {\underline{\in}}
\def \oin {\overline{\in}}
\def \uR {\underline{R}}
\def \oR {\overline{R}}
\def \uP {\underline{P}}
\def \oP {\overline{P}}

\def \Ra {\Rightarrow}

\def \e {\enspace}

\def \sgn {\operatorname{sgn}}
\def \gen {\operatorname{gen}}
\def \ker {\operatorname{ker}}
\def \im {\operatorname{im}}

\def \tril {\triangleleft}

% \varprod
\DeclareSymbolFont{largesymbolsA}{U}{txexa}{m}{n}
\DeclareMathSymbol{\varprod}{\mathop}{largesymbolsA}{16}

% \bigtimes
\DeclareFontFamily{U}{mathx}{\hyphenchar\font45}
\DeclareFontShape{U}{mathx}{m}{n}{
      <5> <6> <7> <8> <9> <10>
      <10.95> <12> <14.4> <17.28> <20.74> <24.88>
      mathx10
      }{}
\DeclareSymbolFont{mathx}{U}{mathx}{m}{n}
\DeclareMathSymbol{\bigtimes}{1}{mathx}{"91}
% \odiv
\DeclareFontFamily{U}{matha}{\hyphenchar\font45}
\DeclareFontShape{U}{matha}{m}{n}{
      <5> <6> <7> <8> <9> <10> gen * matha
      <10.95> matha10 <12> <14.4> <17.28> <20.74> <24.88> matha12
      }{}
\DeclareSymbolFont{matha}{U}{matha}{m}{n}
\DeclareMathSymbol{\odiv}         {2}{matha}{"63}


\newcommand\subsetsim{\mathrel{%
  \ooalign{\raise0.2ex\hbox{\scalebox{0.9}{$\subset$}}\cr\hidewidth\raise-0.85ex\hbox{\scalebox{0.9}{$\sim$}}\hidewidth\cr}}}
\newcommand\simsubset{\mathrel{%
  \ooalign{\raise-0.2ex\hbox{\scalebox{0.9}{$\subset$}}\cr\hidewidth\raise0.75ex\hbox{\scalebox{0.9}{$\sim$}}\hidewidth\cr}}}

\newcommand\simsubsetsim{\mathrel{%
  \ooalign{\raise0ex\hbox{\scalebox{0.8}{$\subset$}}\cr\hidewidth\raise1ex\hbox{\scalebox{0.75}{$\sim$}}\hidewidth\cr\raise-0.95ex\hbox{\scalebox{0.8}{$\sim$}}\cr\hidewidth}}}
\newcommand{\stcomp}[1]{{#1}^{\mathsf{c}}}


\usepackage{bussproofs}
\def \EBA {\EnableBpAbbreviations}
\def \RL[#1]{\RightLabel{#1}}
\setcounter{tocdepth}{2}
\author{Jean-Yves Girard}
\date{\today}
\title{Proof Theory and Logical Complexity}
\hypersetup{
 pdfauthor={Jean-Yves Girard},
 pdftitle={Proof Theory and Logical Complexity},
 pdfkeywords={},
 pdfsubject={},
 pdfcreator={Emacs 26.3 (Org mode 9.3.1)}, 
 pdflang={English}}
\begin{document}

\maketitle
\tableofcontents \clearpage
\section{Preliminaries}
\label{sec:orga0d278e}
\subsection{Languages}
\label{sec:org25bb01f}
A (first-order) language is defined as follows: \(\bL\) is built up from the
following atmoic symbols:
\begin{enumerate}
\item for all integers \(n\), \textbf{predicate letters} \(p_j^n\) (\(p_j^n\) is \emph{n-ary})
\item for all integers \(n\), \textbf{function letters} \(f_j^n\) (\(f_j^n\) is \emph{n-ary}). A \emph{0-ary}
function letter is a \textbf{constant}
\item the \textbf{connectives} \(\wedge,\vee,\neg,\to\)
\item variables \(x_j\) (\(j\in\N\))
\item the quantifiers \(\forall\) and \(\exists\)
\end{enumerate}


We shall always assume that our languages are \textbf{denumerable}; this means that the
set of function and predicate letters is denumerable.

The \textbf{terms} of \(\bL\) are inductively defined as follows:
\begin{enumerate}
\item a variable \(x_j\) is a term
\item if \(t_1,\dots,t_n\) are terms and \(f_j^n\) is an n-ary function letter, then
\(f_j^nt_1\dots t_n\) is a term
\item the only terms of \(\bL\) are given by (1) and (2)
\end{enumerate}


The \textbf{formulas} of \(\bL\) are inductively defined as follows:
\begin{enumerate}
\item if \(t_1,\dots,t_n\) are terms and \(p_j^n\) is an n-ary function letter, then
\(p_j^nt_1\dots t_n\) is a formula (\textbf{atomic formula})
\item if \(A\) and \(B\) are formulas, so are \(\wedge AB,\vee AB,\to AB,\neg A\)
\item if \(A\) is a formula and \(x_j\) is a variable, \(\forall x_jA\) and \(\exists
   x_jA\) are formulas
\item the only formulas of \(\bL\) are given by (1)-(3)
\end{enumerate}
\subsection{occurrences}
\label{sec:org430d477}
\begin{enumerate}
\item in \(\forall x(px\to pfx)\) there are
\begin{itemize}
\item three occurrences of \(x\)
\item two occurrences of \(p\)
\item one occurrence of \(\forall\)
\item one occurrence of \(\to\)
\item one occurrence of \(f\)
\end{itemize}
\item in \(A,A\to A,A\vdash A\) there are
\begin{itemize}
\item three occurrences of \(A\)
\item one occurrence of \(A\to A\)
\item one occurrence of \(\vdash\)
\end{itemize}
\item in the proof
\begin{gather*}
\EBA\AXC{$A\vdash A$}
\AXC{$A\vdash A$}\RL{$\wedge$I}
\BIC{$A,A\vdash A\wedge A$}
\AXC{$A\vdash A$}\RL{$\wedge$I}
\BIC{$A,A,A\vdash(A\wedge A)\wedge A$}\DP
\end{gather*}
\begin{itemize}
\item the sequent \(A\vdash A\) occurs three times
\item \(A,A\vdash A\wedge A\) occurs once
\item \(A,A,A\vdash(A\wedge A)\wedge A\) occurs once
\item \(\wedge I\) occurs twice
\end{itemize}
\end{enumerate}


If one wants to distinguish between various occurrences of sequents and rules,
one can add indices, say:
\begin{gather*}
\EBA\AXC{$A\vdash^1 A$}
\AXC{$A\vdash^2 A$}\RL{$\wedge^1$I}
\BIC{$A,A\vdash^1 A\wedge A$}
\AXC{$A\vdash^3 A$}\RL{$\wedge^2$I}
\BIC{$A,A,A\vdash^1(A\wedge A)\wedge A$}\DP
\end{gather*}
\subsection{free and bound variables}
\label{sec:orgc584d54}
We shall use square brackets to denote all free occurrences of variables of one
or several variables in an expression; if \(A\) is \(A[x_1,\dots,x_n]\), then
\(A[x_1,\dots,x_n]\) denotes \(A[t_1,\dots,t_n/x_1,\dots,x_n]\)

\emph{A bound variable has no individuality}
\section{The Fall of Hilbert Program}
\label{sec:orgd1daeed}
\subsection{Hilbert's Program}
\label{sec:org14bfe09}
\subsubsection{the formalist philosophy}
\label{sec:org529689e}
For the \textbf{formalist}, mathematical activity is \emph{mechanical}: a machine could as well
form sequences of strings of symbols, according to fixed laws.
\subsubsection{Hilbert's ontology}
\label{sec:orgf578d23}
The idea of Hilbert was to use this formal aspect of mathematics (which has the
consequence that a mathematical proof can be viewed as a mathematical object
itself) in order to \emph{prove} some general facts concerning mathematical activity.
Hilbert's ontology of mathematics distinguished between:
\begin{enumerate}
\item \emph{Real} (or \emph{elementray}, \emph{finitist}) objects which \emph{do} exists
\item \emph{Abstract} objects which do not actually exist
\end{enumerate}
\subsubsection{Hilbert's program}
\label{sec:org8a1a6a7}
\emph{purity of methods}
\subsubsection{consistency proofs}
\label{sec:org272de54}
\subsubsection{the fall}
\label{sec:org184ef5f}
\subsection{Recursive Functions}
\label{sec:orgb23199c}
\begin{definition}[]
A function is \textbf{recursive} iff it maps \(\N^k\) into \(\N\) (\(k\ge 0\)), and is obtained
by means of the following schemes:

(\(R1\)) \(I_i^n(a_1,\dots,a_n)=a_i\), \(a_1+a_2,a_1\cdot a_2,\chi_<(a_1,a_2)\)

(\(R2\)) composition

(\(R3\)) \textbf{\(\mu\)-operator}
\end{definition}


\textbf{Church's Thesis}: \emph{every computable function is recursive}.

\begin{theorem}[]
\begin{enumerate}
\item The set of recursive functions is denumerable
\item the set of recursive functions cannot be enumerated by a recursive function
\end{enumerate}
\end{theorem}
\begin{proof}
(2) means that if \((f_n)_{n\in\N}\) is an enumeration of all recursive functions,
then the function \(F(n,m)=f_n(m)\) is not recursive: one easily sees that the
function \(g(n)=F(n,m)+1\) would otherwise be recursive, but if \(g=f_k\), one would
have \(g(k)=g(k)-1\)
\end{proof}

\begin{theorem}[]
(\(R4\)) Constant functions are recursive

(\(R5\)) Let \(F\) and \(G\) be recursive functions with respectively \(n\) and \(n+2\)
arguments; then one can define a recursive function \(H\), with \(n+1\) arguments
and such that
\begin{align*}
&H(a_1,\dots,a_n,0)=F(a_1,\dots,a_n)\\
&H(a_1,\dots,a_n,k+1)=G(a_1,\dots,a_n,k,H(a_1,\dots,a_n,k))
\end{align*}
\end{theorem}

\begin{definition}[]
\(F\) is \textbf{primitive recursive} if \(F\) can be obtained by means of
\((R1),(R2),(R4),(R5)\) 
\end{definition}


\begin{theorem}
\begin{enumerate}
\item the set of primitive recursive functions is denumerable
\item the set of unary primitive recursive functions can be enumerated by means of
a recursive binary function, called the \textbf{Ackermann function}
\item The Ackermann function is not primitive recursive
\end{enumerate}
\end{theorem}

\begin{definition}[]
\begin{enumerate}
\item A predicate \(P\) is \textbf{recursive} iff its characteristic function \(\chi_p\) is
recursive
\item A problem is \textbf{decidable} iff the predicate which represents the problem is
recursive
\end{enumerate}
\end{definition}

\begin{examplle}[]
\begin{enumerate}
\item Predicate calculus is undecidable: if one encodes formulas by integers, then
the set of integers which are codes of theorems of predicate calculus is not
recursive
\item The \textbf{word problem} is undecidable: take the free group \(G\) generated by a
finite number of points, and let \(g_1,\dots,g_k\) be elements of this group;
let \(H\) be the normal subgroup generated by \(g_1,\dots,g_k\); then the
equivalence relation \(st^{-1}\in H\) is undecidable for a suitable choice of
\(G\) and \(g_1,\dots,g_k\)
\end{enumerate}
\end{examplle}

\begin{definition}[]
\(\bL_o\) is the language of arithmetic: constant: \(\overline{o}\), one unary
function letter \(S\), two binary function letters \(+\) and \(\cdot\), and two binary
predicate letters \(=\) and \(<\)
\begin{enumerate}
\item \(\sum\) is the smallest class formulas of \(\bL_0\) s.t.
\begin{enumerate}
\item atomic formulas and their negation belong to \(\sum\)
\item if \(A,B\in\sum\), then \(A\wedge B,A\vee B\in\sum\)
\item if \(A\in\Sigma\), \(x\) is not free in term \(t\), then \(\forall x<t\; A\in\sum\)
\item if \(A\in\sum\) and \(x\) is a variable, then \(\exists xA\in\sum\)
\end{enumerate}
\item \(\Delta\) with the following differences:
\begin{enumerate}
\setcounter{enumii}{3}
\item if \(A\in\Delta\) and \(x\) is not free in term \(t\), then \(\exists
      x<t\; A\in\Delta\)
\item if \(A\in\Delta\), then \(\neg A\in\Delta\)
\end{enumerate}
\item A formula is \(\sum^0_n\) (resp. \(\prod_n^0\)) iff it can be written 
\(Q_0x_0\dots Q_{n-1}x_{n-1}A\) where \(A\) is \(\prod\) and the quantifiers \(Q_i\)
are alternating, and \(Q_0=\exists\) (resp. \(Q_0=\forall\)). For instance,
Fermat's last theorem for a given \(n\) is \(\prod_n^0\):
\begin{equation*}
\forall z\forall a<z\forall b<z\forall c<z(abc\neq\overline{o}\to
a^n+b^n\neq c^n)
\end{equation*}
\end{enumerate}
\end{definition}

\begin{proposition}[]
Any \(\sum\) formula is equivalent to a \(\sum_1^0\)-formula
\end{proposition}

\begin{proof}
If \(A\in\sum\), form \(A^x\) by replacing all existential quantifiers \(\exists z\)
of \(A\) by bounded quantifiers \((\exists z<x:A^x)\in\Delta\). And \(A\) is equivalent
to the \(\sum_1^0\)-formula \(\exists xA^x\)
\end{proof}

\begin{theorem}[]
The properties \(F(x_1,\dots,x_n)=y\) and \(P(x_1,\dots,x_n)\) when \(F\) is a partial
recursive function and \(P\) and r.e. predicate, can be expressed by means of \(\sum\)
formulas 
\end{theorem}

\begin{definition}[]
\begin{enumerate}
\item Given integers \(a_0,a_{n-1}\), one defines \(\la
   a_0,\dots,a_{n-1}\ra=2^{a_0+1}3^{a_1+1}\dots p_{n-1}^{a_{n-1}+1}\), \(Seq(x)\)
is the predicate: for some
\(x_0,\dots,x_{n-1}\) , \(x=\la x_0,\dots,x_{n-1}\ra\)
\item \textbf{length}
\begin{equation*}
lh(x)=
\begin{cases}
0&x\not\in Seq\\
n&x=\la x_0,\dots,x_{n-1}\ra
\end{cases}
\end{equation*}

\item \textbf{projection}
\begin{equation*}
(x)_i=
\begin{cases}
a_i&i<lh(x)\\
0&i\ge lh(x)
\end{cases}
\end{equation*}
\item \textbf{concatenation}

\(\la a_0,\dots,a_{n-1}\ra*\la b_0,\dots,b_{m-1}\ra=\la a_0,\dots,a_{n-1},b_0,\dots,b_{m-1}\ra\)
\item \(\la a_0,\dots,a_{n-1}\ra\restriction i=\la a_0,\dots,a_{i-1}\ra\) if \(i<n\),
otherwise \(x\restriction i=0\)
\end{enumerate}
\end{definition}


\begin{definition}[]
\begin{enumerate}
\item A \textbf{numeral} is a term of \(\bL_o\) which canonically represents an integer;
\(\overline{n}\) is the \(n\)th numeral; hence \(\overline{0}\) is the constant
of \(\bL_0\), and \(\overline{n+1}=S\overline{n}\)
\item One defines the following prim. rec. predicates:
\begin{itemize}
\item \(Term(a)\): \(a\) is a term
\item \(Form(a)\): \(a\) is a formula
\item \(Fr(a,b)\): \(b\) is the Gödel number of a variable occurring freely in the
expression encoded by \(a\)
\item \(Cl(a)\): \(a\) is the Gödel number of a closed expression \(\bL_0\)
\item \(Subst(a,b,c)\): \(c\) is the Gödel number of a term substitutable for the
variable with Gödel \(b\) in the formula with Gödel number \(a\)
\end{itemize}
\item prim. rec. functions
\begin{itemize}
\item \(Num(a)=\ulcorner\overline{a}\urcorner\), the Gödel number of the \(a\)th
numeral
\item \(Sub(a,b,c)=\) the Gödel number of \(A[t/x_n]\) if \(a=\ulcorner
     A\urcorner,b=\ulcorner x_n\urcorner,c=\ulcorner t\urcorner\)
\end{itemize}
\end{enumerate}
\end{definition}


\begin{theorem}[]
\begin{enumerate}
\item There exists a prim. rec. function \(Val\) s.t. if \(a\) is the Gödel number of a
closed term of \(\bL_0\), \(Val(a)\) is the integer represented by this term; in
particular, \(Val(Num(a))=a\)
\item There exists a prim. rec. predicate \(Tr\) s.t. if \(a\) is the Gödel number of a
closed \(\Delta\)-formula of \(\bL_0\), then \(Tr(a)\) iff the formula is true
\end{enumerate}
\end{theorem}

\begin{theorem}[Kleene Normal Form Theorem]
\begin{enumerate}
\item For each integer \(n\ge 0\), one can define a prim. rec. predicate \(T_n\) with
\(n+2\) arguments and a prim. rec. function \(U\) with the following property: if
\(F\) is a partial recursive function of \(n\) arguments, then there is an
integer \(e\) (an \textbf{index} of \(F\)) s.t. for all \(x_1,\dots,x_n\)
\begin{equation*}
F(x_1,\dots,x_n)\simeq U(\mu yT_n(e,x_1,\dots,x_n,y))
\end{equation*}
\item If \(P\) is an r.e. predicate of \(n\) arguments, then there is an integer \(c\)
(an \textbf{index} of \(P\)) s.t. for all \(x_1,\dots,x_n\)
\begin{equation*}
P(x_1,\dots,x_n)\leftrightarrow\exists yT_n(e,x_1,\dots,x_n,y)
\end{equation*}
\end{enumerate}
\end{theorem}

\begin{proof}
\begin{enumerate}
\item We represent \(F(x_1)\simeq y\) by a formula \(A[x_1,y]\) which is \(\sum\); hence
there is a \(\Delta\)-formula \(B[z,x_1,y]\) s.t. \(F(x_1)\simeq y\) iff \(\exists
   zB[z,x_1,y])\). If one defines \(T_1(e,a_1,b)\) by 
\begin{equation*}
Tr(Sub(Sub(Sub(e,\ulcorner x_0\urcorner, Num((b)_0)),
\ulcorner x_1\urcorner,Num(a_1)),\ulcorner x_2\urcorner,Num((b)_1))))
\end{equation*}
and \(U(b)=(b)_1\), one sees that the result holds with \(e=\ulcorner
   B\urcorner\)
\item \(P(x)\) can be written "\(F(x)\) is defined" for an appropriate \(F\) (if
\(P(x)\leftrightarrow G(x)\simeq 0\), let \(F(x)\simeq\mu y(G(x)\simeq 0)\))
\end{enumerate}
\end{proof}

\begin{corollary}[]
\begin{enumerate}
\item A non-void subset of \(\N\) is r.e. iff it is the range of a prim. rec.
function
\item A set is recursive iff it and its complement are r.e.
\item A partial recursive function which is total is a recursive function
\end{enumerate}
\end{corollary}

\begin{proof}
\begin{enumerate}
\item The range of a partial recursive function is always an r.e. set. Conversely,
if \(A\subset\N\) is defined by the index \(e\), and \(a_0\in A\) define \(F\) by
\(F(\la x,b\ra))=x\) if \(T_1(e,x,b)\), \(F(x)=a_0\) otherwise
\end{enumerate}
\end{proof}
\end{document}