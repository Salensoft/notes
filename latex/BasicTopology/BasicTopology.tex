% Created 2020-06-11 四 22:49
% Intended LaTeX compiler: pdflatex
\documentclass[11pt]{article}
\usepackage[utf8]{inputenc}
\usepackage[T1]{fontenc}
\usepackage{graphicx}
\usepackage{grffile}
\usepackage{longtable}
\usepackage{wrapfig}
\usepackage{rotating}
\usepackage[normalem]{ulem}
\usepackage{amsmath}
\usepackage{textcomp}
\usepackage{amssymb}
\usepackage{capt-of}
\usepackage{hyperref}
\usepackage{minted}
% TIPS
% \substack{a\\b} for multiple lines text





% pdfplots will load xolor automatically without option
\usepackage[dvipsnames]{xcolor}

\usepackage{forest}
% two-line text in node by [two \\ lines]
% \begin{forest} qtree, [..] \end{forest}
\forestset{
  qtree/.style={
    baseline,
    for tree={
      parent anchor=south,
      child anchor=north,
      align=center,
      inner sep=1pt,
    }}}
%\usepackage{flexisym}
% load order of mathtools and mathabx, otherwise conflict overbrace

\usepackage{mathtools}
%\usepackage{fourier}
\usepackage{pgfplots}
\usepackage{amsthm}
\usepackage{amsmath}
%\usepackage{unicode-math}
%
\usepackage{commath}
%\usepackage{,  , }
\usepackage{amsfonts}
\usepackage{amssymb}
% importing symbols https://tex.stackexchange.com/questions/14386/importing-a-single-symbol-from-a-different-font
%mathabx change every symbol
% use instead stmaryrd
%\usepackage{mathabx}
\usepackage{stmaryrd}
\usepackage{empheq}
\usepackage{tikz}
\usepackage{tikz-cd}
%\usepackage[notextcomp]{stix}
\usetikzlibrary{arrows.meta}
\usepackage[most]{tcolorbox}
%\utilde
%\usepackage{../../latexpackage/undertilde/undertilde}
% left and right superscript and subscript
\usepackage{actuarialsymbol}
\usepackage{threeparttable}
\usepackage{scalerel,stackengine}
\usepackage{stackrel}
% \stackrel[a]{b}{c}
\usepackage{dsfont}
% text font
\usepackage{newpxtext}
%\usepackage{newpxmath}

%\newcounter{dummy} \numberwithin{dummy}{section}
\newtheorem{dummy}{dummy}[section]
\theoremstyle{definition}
\newtheorem{definition}[dummy]{Definition}
\newtheorem{corollary}[dummy]{Corollary}
\newtheorem{lemma}[dummy]{Lemma}
\newtheorem{proposition}[dummy]{Proposition}
\newtheorem{theorem}[dummy]{Theorem}
\theoremstyle{definition}
\newtheorem{example}[dummy]{Example}
\theoremstyle{remark}
\newtheorem*{remark}{Remark}


\newcommand\what[1]{\ThisStyle{%
    \setbox0=\hbox{$\SavedStyle#1$}%
    \stackengine{-1.0\ht0+.5pt}{$\SavedStyle#1$}{%
      \stretchto{\scaleto{\SavedStyle\mkern.15mu\char'136}{2.6\wd0}}{1.4\ht0}%
    }{O}{c}{F}{T}{S}%
  }
}

\newcommand\wtilde[1]{\ThisStyle{%
    \setbox0=\hbox{$\SavedStyle#1$}%
    \stackengine{-.1\LMpt}{$\SavedStyle#1$}{%
      \stretchto{\scaleto{\SavedStyle\mkern.2mu\AC}{.5150\wd0}}{.6\ht0}%
    }{O}{c}{F}{T}{S}%
  }
}

\newcommand\wbar[1]{\ThisStyle{%
    \setbox0=\hbox{$\SavedStyle#1$}%
    \stackengine{.5pt+\LMpt}{$\SavedStyle#1$}{%
      \rule{\wd0}{\dimexpr.3\LMpt+.3pt}%
    }{O}{c}{F}{T}{S}%
  }
}

\newcommand{\bl}[1] {\boldsymbol{#1}}
\newcommand{\Wt}[1] {\stackrel{\sim}{\smash{#1}\rule{0pt}{1.1ex}}}
\newcommand{\wt}[1] {\widetilde{#1}}
\newcommand{\tf}[1] {\textbf{#1}}


%For boxed texts in align, use Aboxed{}
%otherwise use boxed{}

\DeclareMathSymbol{\widehatsym}{\mathord}{largesymbols}{"62}
\newcommand\lowerwidehatsym{%
  \text{\smash{\raisebox{-1.3ex}{%
    $\widehatsym$}}}}
\newcommand\fixwidehat[1]{%
  \mathchoice
    {\accentset{\displaystyle\lowerwidehatsym}{#1}}
    {\accentset{\textstyle\lowerwidehatsym}{#1}}
    {\accentset{\scriptstyle\lowerwidehatsym}{#1}}
    {\accentset{\scriptscriptstyle\lowerwidehatsym}{#1}}
}

\usepackage{graphicx}
    
% text on arrow for xRightarrow
\makeatletter
%\newcommand{\xRightarrow}[2][]{\ext@arrow 0359\Rightarrowfill@{#1}{#2}}
\makeatother


\newcommand{\dom}[1]{%
\mathrm{dom}{(#1)}
}

% Roman numerals
\makeatletter
\newcommand*{\rom}[1]{\expandafter\@slowromancap\romannumeral #1@}
\makeatother

\def \fR {\mathfrak{R}}
\def \bx {\boldsymbol{x}}
\def \bz {\boldsymbol{z}}
\def \ba {\boldsymbol{a}}
\def \bh {\boldsymbol{h}}
\def \bo {\boldsymbol{o}}
\def \bU {\boldsymbol{U}}
\def \bc {\boldsymbol{c}}
\def \bV {\boldsymbol{V}}
\def \bI {\boldsymbol{I}}
\def \bK {\boldsymbol{K}}
\def \bt {\boldsymbol{t}}
\def \bb {\boldsymbol{b}}
\def \bA {\boldsymbol{A}}
\def \bX {\boldsymbol{X}}
\def \bu {\boldsymbol{u}}
\def \bS {\boldsymbol{S}}
\def \bZ {\boldsymbol{Z}}
\def \bz {\boldsymbol{z}}
\def \by {\boldsymbol{y}}
\def \bw {\boldsymbol{w}}
\def \bT {\boldsymbol{T}}
\def \bF {\boldsymbol{F}}
\def \bS {\boldsymbol{S}}
\def \bm {\boldsymbol{m}}
\def \bW {\boldsymbol{W}}
\def \bR {\boldsymbol{R}}
\def \bQ {\boldsymbol{Q}}
\def \bS {\boldsymbol{S}}
\def \bP {\boldsymbol{P}}
\def \bT {\boldsymbol{T}}
\def \bY {\boldsymbol{Y}}
\def \bH {\boldsymbol{H}}
\def \bB {\boldsymbol{B}}
\def \blambda {\boldsymbol{\lambda}}
\def \bPhi {\boldsymbol{\Phi}}
\def \btheta {\boldsymbol{\theta}}
\def \bTheta {\boldsymbol{\Theta}}
\def \bmu {\boldsymbol{\mu}}
\def \bphi {\boldsymbol{\phi}}
\def \bSigma {\boldsymbol{\Sigma}}
\def \lb {\left\{}
\def \rb {\right\}}
\def \la {\langle}
\def \ra {\rangle}
\def \caln {\mathcal{N}}
\def \dissum {\displaystyle\Sigma}
\def \dispro {\displaystyle\prod}
\def \E {\mathbb{E}}
\def \Q {\mathbb{Q}}
\def \N {\mathbb{N}}
\def \V {\mathbb{V}}
\def \R {\mathbb{R}}
\def \P {\mathbb{P}}
\def \A {\mathbb{A}}
\def \Z {\mathbb{Z}}
\def \I {\mathbb{I}}
\def \C {\mathbb{C}}
\def \cala {\mathcal{A}}
\def \calb {\mathcal{B}}
\def \calq {\mathcal{Q}}
\def \calp {\mathcal{P}}
\def \cals {\mathcal{S}}
\def \calg {\mathcal{G}}
\def \caln {\mathcal{N}}
\def \calr {\mathcal{R}}
\def \calm {\mathcal{M}}
\def \calc {\mathcal{C}}
\def \calf {\mathcal{F}}
\def \calk {\mathcal{K}}
\def \call {\mathcal{L}}
\def \calu {\mathcal{U}}
\def \bcup {\bigcup}


\def \uin {\underline{\in}}
\def \oin {\overline{\in}}
\def \uR {\underline{R}}
\def \oR {\overline{R}}
\def \uP {\underline{P}}
\def \oP {\overline{P}}

\def \Ra {\Rightarrow}

\def \e {\enspace}

\def \sgn {\operatorname{sgn}}
\def \gen {\operatorname{gen}}
\def \ker {\operatorname{ker}}
\def \im {\operatorname{im}}

\def \tril {\triangleleft}

% \varprod
\DeclareSymbolFont{largesymbolsA}{U}{txexa}{m}{n}
\DeclareMathSymbol{\varprod}{\mathop}{largesymbolsA}{16}

% \bigtimes
\DeclareFontFamily{U}{mathx}{\hyphenchar\font45}
\DeclareFontShape{U}{mathx}{m}{n}{
      <5> <6> <7> <8> <9> <10>
      <10.95> <12> <14.4> <17.28> <20.74> <24.88>
      mathx10
      }{}
\DeclareSymbolFont{mathx}{U}{mathx}{m}{n}
\DeclareMathSymbol{\bigtimes}{1}{mathx}{"91}
% \odiv
\DeclareFontFamily{U}{matha}{\hyphenchar\font45}
\DeclareFontShape{U}{matha}{m}{n}{
      <5> <6> <7> <8> <9> <10> gen * matha
      <10.95> matha10 <12> <14.4> <17.28> <20.74> <24.88> matha12
      }{}
\DeclareSymbolFont{matha}{U}{matha}{m}{n}
\DeclareMathSymbol{\odiv}         {2}{matha}{"63}


\newcommand\subsetsim{\mathrel{%
  \ooalign{\raise0.2ex\hbox{\scalebox{0.9}{$\subset$}}\cr\hidewidth\raise-0.85ex\hbox{\scalebox{0.9}{$\sim$}}\hidewidth\cr}}}
\newcommand\simsubset{\mathrel{%
  \ooalign{\raise-0.2ex\hbox{\scalebox{0.9}{$\subset$}}\cr\hidewidth\raise0.75ex\hbox{\scalebox{0.9}{$\sim$}}\hidewidth\cr}}}

\newcommand\simsubsetsim{\mathrel{%
  \ooalign{\raise0ex\hbox{\scalebox{0.8}{$\subset$}}\cr\hidewidth\raise1ex\hbox{\scalebox{0.75}{$\sim$}}\hidewidth\cr\raise-0.95ex\hbox{\scalebox{0.8}{$\sim$}}\cr\hidewidth}}}
\newcommand{\stcomp}[1]{{#1}^{\mathsf{c}}}


\author{M. A. Armstrong}
\date{\today}
\title{Basic Topology}
\hypersetup{
 pdfauthor={M. A. Armstrong},
 pdftitle={Basic Topology},
 pdfkeywords={},
 pdfsubject={},
 pdfcreator={Emacs 26.3 (Org mode 9.4)}, 
 pdflang={English}}
\begin{document}

\maketitle
\tableofcontents \clearpage\section{Introduction}
\label{sec:org9e09d41}
\subsection{Abstract spaces}
\label{sec:orgec75618}
We ask for a set \(X\) and for each point \(x\) of \(X\) a nonempty
collection of subsets of \(X\), called neighbourhoods of \(x\). These
neighbourhoods are required to satisfy four axioms
\begin{enumerate}
\item \(x\) lies in each of its neighbourhoods
\item The intersection of two neighbourhoods of \(x\) is itself a neighbourhood
of \(x\)
\item If \(N\) is a neighbourhood of \(x\) and if \(U\) is a subset of \(X\)
which contains \(N\), then \(U\) is a neighbourhood of \(x\)
\item If \(N\) is a neighbourhood of \(x\) and if \(\mathring{N}\) denotes the set
\(\{z\in N\mid N\text{ is a neighbourhood of }z\}\), then \(\mathring{N}\) is a
neighbourhood of \(x\). (The set \(\mathring{N}\) is called the \textbf{interior} of \(N\))
\end{enumerate}


This whole structure is called a \textbf{topological space}. The assignment of a
collection of neighbourhoods satisfying axioms \((1)\to(4)\) to each point
\(x\in X\) is called a \textbf{topology} on the set \(X\).

Let \(X\) and \(Y\) be topological spaces. A function \(f:X\to Y\) is
\textbf{continuous} if for each point \(x\in X\) and each neighbourhood \(N\) of
\(f(x)\) in \(Y\) the set\(f^{-1}(N)\) is a neighbourhood of \(x\) in \(X\).
A function \(h:X\to Y\) is called a \textbf{homeomorphism} if it is one-one, onto,
continuous and has a continuous inverse. When such a function exists, \(X\)
and \(Y\) are called \textbf{homeomorphic} spaces

\begin{examplle}[]
\begin{enumerate}
\item Let \(X\) be a topological space and let \(Y\) be a subset of \(X\). We
can define a topology on \(Y\) as follows. Given a point \(y\in Y\) take
the collection of its neighbourhoods in the topological space \(X\) and
intersect each of these neighbourhood with \(Y\). The resulting sets are
the neighbourhoods of \(y\) in \(Y\). We say that \(Y\) has the \textbf{subspace topology}.
\end{enumerate}
\end{examplle}

\begin{definition}[]
A \textbf{surface} is a topological space in which each point has a neighbourhood
homeomorphic to the plane, and for which any two distinct points possess
disjoint neighbourhoods
\end{definition}

\section{Continuity}
\label{sec:org0d3a083}

\subsection{Open and closed sets}
\label{sec:orga08c079}
Let \(X\) be a topological space and call a subset \(O\) of \(X\) \textbf{open} if
it is a neighbourhood of each of its points. The union of any collection of
open sets is open by axiom (3) and the intersection of \emph{finite} number of
open sets is open by axiom (2).


Suppose we have a set \(X\) together with a nonempty collection of subsets of
\(X\), which we call open sets, such that any union of open sets is itself
open, any finite intersection of open sets is open, and both the whole set
and the empty set are open. Given a point \(x\) of \(X\), we shall call a
subset \(N\) of \(x\) a \textbf{neighbourhood of} \(x\) if we can find an open set
\(O\) s.t. \(x\in O\subseteq N\)


We claim that this definition of neighbourhood makes \(X\) into a topological
space.

Verification for axiom (4). Suppose \(N\) is a neighbourhood of \(x\) and let
\(\mathring{N}\) denote the set of points \(z\) s.t. \(N\) is a neighbourhood of
\(z\). Choose an open set \(O\) s.t. \(x\in O\subseteq N\). Now \(O\), being
open, is a neighbourhood of each of its points, so \(O\) is contained in
\(\mathring{N}\).

\begin{definition}[]
A \textbf{topology} on a set \(X\) is a nonempty collection of subsets of \(X\),
called open sets, such that any union of open sets is open, any finite
intersection of open sets is open, and both \(X\) and the empty set are open.
A set together with a topology on it is called a \textbf{topological space}
\end{definition}

The open sets of the ``usual'' topology on \(\R^n\) are characterized as
follows. A set \(U\) is open if given \(x\in U\) we can always find a
positive real number \(\epsilon\) s.t. the ball with centre \(x\) and radius
\(\epsilon\) lies entirely in \(U\).

For \textbf{discrete topology} on \(X\), every subset of \(X\) is an open set and we
call \(X\) discrete space.

A subset of a topological space is \textbf{closed} if its complement is open.

Consider the set \(A\) on \(\R^2\) whose points \(x,y\) satisfy \(x\ge0\) and
\(y>0\). This set is neither closed nor open. Take the space \(X\) whose
points are those points \((x,y)\in\R^2\) s.t. \(x\ge1\) or \(x\le-1\) and
whose topology is induced from \(\R^2\). The subsets of \(X\) consisting of
those points with positive first coordinate is both open and closed.

Let \(A\) be a subset of a topological space \(X\) and call a point \(p\) of
\(X\) a \textbf{limit point} (or accumulation point) of \(A\) if every neighbourhood
of \(p\) contains at least one point of \(A-\{p\}\)

\begin{examplle}[]
\begin{enumerate}
\item Take \(X\) to be the real line \(\R\), and let \(A\) consist of the points
\(1/n\), \(n=1,2,\dots\). Then \(A\) has exactly one limit point, namely
the origin
\item X the real line, take \(A=[0,1)\). Then each point of \(A\) is a limit
point of \(A\), and in addition \(1\) is a limit point
\item Let \(X\subseteq\R^3\) and let \(A\) consist of those points all of whose
coordinates are rational. Then every point of \(\R^3\) is a limit point of \(A\)
\item Let \(A\subseteq\R^3\) be the set of points which have integer
coordinates. Then \(A\) does not have any limit points
\item Take \(X\) to be the set of all real numebers with the so called
\textbf{finite-complement topology}. Here a set is open if its complement is
finite or all of \(X\). If we now take \(A\) to be an infinite subset of
\(X\) (say the set of all integers), then every point of \(X\) is a limit
point of \(A\). On the other hand a finite subset of \(X\) has no limit
points in this topology
\end{enumerate}
\end{examplle}

\begin{theorem}[]
A set is closed if and only if it contains all its limit points
\end{theorem}

\begin{proof}
If \(A\) is closed, its complement \(X-A\) is open. Since an open set is a
neighbourhood of each of its points, no point of \(X-A\) can be a limit point
of \(A\).

Suppose \(A\) contains all its limit point and let \(x\in X-A\). Since \(x\)
is not a limit point of \(A\), we can find a neighbourhood \(N\) of \(x\)
which does not meet \(A\). So \(N\) is inside \(X-A\), showing \(X-A\) to be
a neighbourhood of each of its points and consequently open. Therefore \(A\)
is clsoed.
\end{proof}

The union of \(A\) and all its limit points is called the \textbf{closure} of \(A\)
and is written \(\overbar{A}\)

\begin{theorem}[]
The closure of \(A\) is the smallest closed set containing \(A\), in other
words the intersection of all closed sets which contain \(A\)
\end{theorem}

\begin{proof}
For if \(x\in X-\overbar{A}\), we can find an open neighbourhood \(U\) of \(x\)
which does not contain any points of \(A\). Since an open set is a
neighbourhood of each of its points, \(U\) cannot contain any of the limit
points of \(A\). Therefore we have an open set \(U\) s.t.
\(x\in U\subseteq X-\overbar{A}\). Consequently \(X-\overbar{A}\) is a
neighbourhood of each of its points and must be open.

Now let \(B\) be a closed set which contains \(A\). Then every limit point of
\(A\) is a limit point of \(B\) and therefore must lie in \(B\) since \(B\)
is closed. This gives \(\overbar{A}\subseteq B\)
\end{proof}

\begin{corollary}[]
A set is closed if and only if it is equal to its closure
\end{corollary}

A set whose closure is the whole space is said to be \textbf{dense} in the space

The \textbf{interior} of a set, usually written \(\mathring{A}\), is the union of
all open sets contained in \(A\). A point lies in \(\mathring{A}\) if and
only if it's a neighbourhood of \(A\).

We define the \textbf{frontier} of \(A\) to be the \(\overbar{A}\cap\overbar{X-A}\).

Suppose we have a topology on a set \(X\), and a collection \(\beta\) of open set
s.t. every open set is a union of members of \(\beta\). Then \(\beta\) is called a
\textbf{base} for the topology and elements of \(\beta\) are called \textbf{basic open sets}.
An equivalent formulation is to ask that given a point \(x\in X\), and a
neighbourhood \(N\) of \(x\), there is always an element \(B\) of \(\beta\) s.t.
\(x\in B\subseteq N\).

\begin{theorem}[]
Let \(\beta\) be a nonempty collection of subsets of a set \(X\). If the
intersection of any finite number of members of \(\beta\) is always in \(\beta\), and
if \(\bigcup\beta=X\), then \(\beta\) is a base for a topology on \(X\)
\end{theorem}


\subsection{Continuous functions}
\label{sec:org15cf029}
\begin{theorem}[]
A function from \(X\) to \(Y\) is continuous if and only if the inverse image
of each open set of \(Y\) is open in \(X\)
\end{theorem}

A continuous function is often called a \textbf{map}

\begin{theorem}[]
The composition of two maps is a map
\end{theorem}

\begin{theorem}[]
Suppose \(f:X\to Y\) is continuous, and let \(A\subseteq X\) have the
subspace topology. Then the restriction \(f|A:A\to Y\) is continuous
\end{theorem}

\begin{theorem}[]
The following are equivalent
\begin{enumerate}
\item \(f:X\to Y\) is a map
\item If \(\beta\) is a base for the topology of \(Y\), the inverse image of every
member of \(\beta\) is open in \(X\)
\item \(f(\overbar{A})\subseteq\overbar{f(A)}\) for any subset \(A\) of \(X\)
\item \(\overbar{f^{-1}(B)}\subseteq f^{-1}(\overbar{B})\) for any subset \(B\)
of \(Y\)
\item The inverse image of each closed set in \(Y\) is closed in \(X\)
\end{enumerate}
\end{theorem}
\end{document}
