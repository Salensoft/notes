% Created 2019-07-14 日 15:48
% Intended LaTeX compiler: pdflatex
\documentclass[11pt]{article}
\usepackage[utf8]{inputenc}
\usepackage[T1]{fontenc}
\usepackage{graphicx}
\usepackage{grffile}
\usepackage{longtable}
\usepackage{wrapfig}
\usepackage{rotating}
\usepackage[normalem]{ulem}
\usepackage{amsmath}
\usepackage{textcomp}
\usepackage{amssymb}
\usepackage{capt-of}
\usepackage{hyperref}
\usepackage{minted}
% TIPS
% \substack{a\\b} for multiple lines text





% pdfplots will load xolor automatically without option
\usepackage[dvipsnames]{xcolor}

\usepackage{forest}
% two-line text in node by [two \\ lines]
% \begin{forest} qtree, [..] \end{forest}
\forestset{
  qtree/.style={
    baseline,
    for tree={
      parent anchor=south,
      child anchor=north,
      align=center,
      inner sep=1pt,
    }}}
%\usepackage{flexisym}
% load order of mathtools and mathabx, otherwise conflict overbrace

\usepackage{mathtools}
%\usepackage{fourier}
\usepackage{pgfplots}
\usepackage{amsthm}
\usepackage{amsmath}
%\usepackage{unicode-math}
%
\usepackage{commath}
%\usepackage{,  , }
\usepackage{amsfonts}
\usepackage{amssymb}
% importing symbols https://tex.stackexchange.com/questions/14386/importing-a-single-symbol-from-a-different-font
%mathabx change every symbol
% use instead stmaryrd
%\usepackage{mathabx}
\usepackage{stmaryrd}
\usepackage{empheq}
\usepackage{tikz}
\usepackage{tikz-cd}
%\usepackage[notextcomp]{stix}
\usetikzlibrary{arrows.meta}
\usepackage[most]{tcolorbox}
%\utilde
%\usepackage{../../latexpackage/undertilde/undertilde}
% left and right superscript and subscript
\usepackage{actuarialsymbol}
\usepackage{threeparttable}
\usepackage{scalerel,stackengine}
\usepackage{stackrel}
% \stackrel[a]{b}{c}
\usepackage{dsfont}
% text font
\usepackage{newpxtext}
%\usepackage{newpxmath}

%\newcounter{dummy} \numberwithin{dummy}{section}
\newtheorem{dummy}{dummy}[section]
\theoremstyle{definition}
\newtheorem{definition}[dummy]{Definition}
\newtheorem{corollary}[dummy]{Corollary}
\newtheorem{lemma}[dummy]{Lemma}
\newtheorem{proposition}[dummy]{Proposition}
\newtheorem{theorem}[dummy]{Theorem}
\theoremstyle{definition}
\newtheorem{example}[dummy]{Example}
\theoremstyle{remark}
\newtheorem*{remark}{Remark}


\newcommand\what[1]{\ThisStyle{%
    \setbox0=\hbox{$\SavedStyle#1$}%
    \stackengine{-1.0\ht0+.5pt}{$\SavedStyle#1$}{%
      \stretchto{\scaleto{\SavedStyle\mkern.15mu\char'136}{2.6\wd0}}{1.4\ht0}%
    }{O}{c}{F}{T}{S}%
  }
}

\newcommand\wtilde[1]{\ThisStyle{%
    \setbox0=\hbox{$\SavedStyle#1$}%
    \stackengine{-.1\LMpt}{$\SavedStyle#1$}{%
      \stretchto{\scaleto{\SavedStyle\mkern.2mu\AC}{.5150\wd0}}{.6\ht0}%
    }{O}{c}{F}{T}{S}%
  }
}

\newcommand\wbar[1]{\ThisStyle{%
    \setbox0=\hbox{$\SavedStyle#1$}%
    \stackengine{.5pt+\LMpt}{$\SavedStyle#1$}{%
      \rule{\wd0}{\dimexpr.3\LMpt+.3pt}%
    }{O}{c}{F}{T}{S}%
  }
}

\newcommand{\bl}[1] {\boldsymbol{#1}}
\newcommand{\Wt}[1] {\stackrel{\sim}{\smash{#1}\rule{0pt}{1.1ex}}}
\newcommand{\wt}[1] {\widetilde{#1}}
\newcommand{\tf}[1] {\textbf{#1}}


%For boxed texts in align, use Aboxed{}
%otherwise use boxed{}

\DeclareMathSymbol{\widehatsym}{\mathord}{largesymbols}{"62}
\newcommand\lowerwidehatsym{%
  \text{\smash{\raisebox{-1.3ex}{%
    $\widehatsym$}}}}
\newcommand\fixwidehat[1]{%
  \mathchoice
    {\accentset{\displaystyle\lowerwidehatsym}{#1}}
    {\accentset{\textstyle\lowerwidehatsym}{#1}}
    {\accentset{\scriptstyle\lowerwidehatsym}{#1}}
    {\accentset{\scriptscriptstyle\lowerwidehatsym}{#1}}
}

\usepackage{graphicx}
    
% text on arrow for xRightarrow
\makeatletter
%\newcommand{\xRightarrow}[2][]{\ext@arrow 0359\Rightarrowfill@{#1}{#2}}
\makeatother


\newcommand{\dom}[1]{%
\mathrm{dom}{(#1)}
}

% Roman numerals
\makeatletter
\newcommand*{\rom}[1]{\expandafter\@slowromancap\romannumeral #1@}
\makeatother

\def \fR {\mathfrak{R}}
\def \bx {\boldsymbol{x}}
\def \bz {\boldsymbol{z}}
\def \ba {\boldsymbol{a}}
\def \bh {\boldsymbol{h}}
\def \bo {\boldsymbol{o}}
\def \bU {\boldsymbol{U}}
\def \bc {\boldsymbol{c}}
\def \bV {\boldsymbol{V}}
\def \bI {\boldsymbol{I}}
\def \bK {\boldsymbol{K}}
\def \bt {\boldsymbol{t}}
\def \bb {\boldsymbol{b}}
\def \bA {\boldsymbol{A}}
\def \bX {\boldsymbol{X}}
\def \bu {\boldsymbol{u}}
\def \bS {\boldsymbol{S}}
\def \bZ {\boldsymbol{Z}}
\def \bz {\boldsymbol{z}}
\def \by {\boldsymbol{y}}
\def \bw {\boldsymbol{w}}
\def \bT {\boldsymbol{T}}
\def \bF {\boldsymbol{F}}
\def \bS {\boldsymbol{S}}
\def \bm {\boldsymbol{m}}
\def \bW {\boldsymbol{W}}
\def \bR {\boldsymbol{R}}
\def \bQ {\boldsymbol{Q}}
\def \bS {\boldsymbol{S}}
\def \bP {\boldsymbol{P}}
\def \bT {\boldsymbol{T}}
\def \bY {\boldsymbol{Y}}
\def \bH {\boldsymbol{H}}
\def \bB {\boldsymbol{B}}
\def \blambda {\boldsymbol{\lambda}}
\def \bPhi {\boldsymbol{\Phi}}
\def \btheta {\boldsymbol{\theta}}
\def \bTheta {\boldsymbol{\Theta}}
\def \bmu {\boldsymbol{\mu}}
\def \bphi {\boldsymbol{\phi}}
\def \bSigma {\boldsymbol{\Sigma}}
\def \lb {\left\{}
\def \rb {\right\}}
\def \la {\langle}
\def \ra {\rangle}
\def \caln {\mathcal{N}}
\def \dissum {\displaystyle\Sigma}
\def \dispro {\displaystyle\prod}
\def \E {\mathbb{E}}
\def \Q {\mathbb{Q}}
\def \N {\mathbb{N}}
\def \V {\mathbb{V}}
\def \R {\mathbb{R}}
\def \P {\mathbb{P}}
\def \A {\mathbb{A}}
\def \Z {\mathbb{Z}}
\def \I {\mathbb{I}}
\def \C {\mathbb{C}}
\def \cala {\mathcal{A}}
\def \calb {\mathcal{B}}
\def \calq {\mathcal{Q}}
\def \calp {\mathcal{P}}
\def \cals {\mathcal{S}}
\def \calg {\mathcal{G}}
\def \caln {\mathcal{N}}
\def \calr {\mathcal{R}}
\def \calm {\mathcal{M}}
\def \calc {\mathcal{C}}
\def \calf {\mathcal{F}}
\def \calk {\mathcal{K}}
\def \call {\mathcal{L}}
\def \calu {\mathcal{U}}
\def \bcup {\bigcup}


\def \uin {\underline{\in}}
\def \oin {\overline{\in}}
\def \uR {\underline{R}}
\def \oR {\overline{R}}
\def \uP {\underline{P}}
\def \oP {\overline{P}}

\def \Ra {\Rightarrow}

\def \e {\enspace}

\def \sgn {\operatorname{sgn}}
\def \gen {\operatorname{gen}}
\def \ker {\operatorname{ker}}
\def \im {\operatorname{im}}

\def \tril {\triangleleft}

% \varprod
\DeclareSymbolFont{largesymbolsA}{U}{txexa}{m}{n}
\DeclareMathSymbol{\varprod}{\mathop}{largesymbolsA}{16}

% \bigtimes
\DeclareFontFamily{U}{mathx}{\hyphenchar\font45}
\DeclareFontShape{U}{mathx}{m}{n}{
      <5> <6> <7> <8> <9> <10>
      <10.95> <12> <14.4> <17.28> <20.74> <24.88>
      mathx10
      }{}
\DeclareSymbolFont{mathx}{U}{mathx}{m}{n}
\DeclareMathSymbol{\bigtimes}{1}{mathx}{"91}
% \odiv
\DeclareFontFamily{U}{matha}{\hyphenchar\font45}
\DeclareFontShape{U}{matha}{m}{n}{
      <5> <6> <7> <8> <9> <10> gen * matha
      <10.95> matha10 <12> <14.4> <17.28> <20.74> <24.88> matha12
      }{}
\DeclareSymbolFont{matha}{U}{matha}{m}{n}
\DeclareMathSymbol{\odiv}         {2}{matha}{"63}


\newcommand\subsetsim{\mathrel{%
  \ooalign{\raise0.2ex\hbox{\scalebox{0.9}{$\subset$}}\cr\hidewidth\raise-0.85ex\hbox{\scalebox{0.9}{$\sim$}}\hidewidth\cr}}}
\newcommand\simsubset{\mathrel{%
  \ooalign{\raise-0.2ex\hbox{\scalebox{0.9}{$\subset$}}\cr\hidewidth\raise0.75ex\hbox{\scalebox{0.9}{$\sim$}}\hidewidth\cr}}}

\newcommand\simsubsetsim{\mathrel{%
  \ooalign{\raise0ex\hbox{\scalebox{0.8}{$\subset$}}\cr\hidewidth\raise1ex\hbox{\scalebox{0.75}{$\sim$}}\hidewidth\cr\raise-0.95ex\hbox{\scalebox{0.8}{$\sim$}}\cr\hidewidth}}}
\newcommand{\stcomp}[1]{{#1}^{\mathsf{c}}}


\author{wu}
\date{\today}
\title{Rough Sets: Theoretical aspects of reasoning about data}
\hypersetup{
 pdfauthor={wu},
 pdftitle={Rough Sets: Theoretical aspects of reasoning about data},
 pdfkeywords={},
 pdfsubject={},
 pdfcreator={Emacs 26.2 (Org mode 9.2.4)}, 
 pdflang={English}}
\begin{document}

\maketitle
\tableofcontents

\section{Knowledge}
\label{sec:org7a43b82}
\subsection{Knowledge base}
\label{sec:org91c75de}
Given a finite set \(U\neq \emptyset\) (the universe). Any subset \(X\subset U\)
of the universe is called a \textbf{concept} or a \textbf{category} in \(U\). And any family of
concepts in \(U\) will be referred to as \textbf{abstract knowledge} about \(U\).

\textbf{partition} or \textbf{classification} of a certain universe \(U\) is a family 
\(C=\lb X_1,X_2,\dots,X_n\rb\) s.t. \(X_i\subset U,X_i\neq\emptyset,X_i\cap
   X_j=\emptyset\) and \(\bigcup X_i=U\)

A family of classifications is called a \textbf{knowledge base} over \(U\)


\(R\) an equivalence relation over \(U\), \(U/R\) family of all equivalence classes
of \(R\), referred to be \textbf{categories} or \textbf{concepts} of \(R\), and \([x]_R\) denotes a
category in \(R\) containing an element \(x\in U\)

By a \textbf{knowledge base} we can understand a relational system \(K=(U,\bR)\), \(\bR\)
is a family of equivalence relations over \(U\)

If \(\bP\subset \bR\) and \(\bP\neq\emptyset\), then \(\bigcap\bP\) is also an
equivalence relation, and will be denoted by \(IND(\bP)\), called an
\textbf{indiscernibility relation} over \(\bP\)
\begin{equation*}
[x]_{IND(\bP)}=\bigcap_{R\in\bP}[x]_R
\end{equation*}

\(U/IND(\bP)\) called \(\bP\textbf{-basic}\) \textbf{knowledge about} \(U\) in \(K\). For
simplicity, \(U/\bP=U/IND(\bP)\) and \(\bP\) will be also called
\(\bP\textbf{-basic}\) \textbf{knowledge}
. Equivalence classes of \(IND(\bP)\) are called
\textbf{basic categories} of knowledge \(\bP\). If \(Q\in\bR\), then \(Q\) is a
\(Q\textbf{-elementary}\) \textbf{knowledge} and equivalence classes of \(Q\) are referred
to as \(Q\textbf{-elementary}\) \textbf{concepts} of knowledge \(\bR\)

The family of all \(\bP\text{-basic}\) categories for all
\(\empty\neq\bP\subset\bR\) will be called the \textbf{family of basic categories} in
knowledge base \(K=(U,\bR)\)

Let \(K=(U,\bR)\) be a knowledge base. By \(IND(K)\) we denote the family of all
equivalence relations defined in \(K\) as \(IND(K)=\lb
   IND(\bP):\emptyset\neq\bP\subseteq\bR\rb\).

Thus \(IND(K)\) is the minimal set of equivalence relations.

Every union of \(\bP\text{-basic}\) categories will be \(\bP\textbf{-category}\)

The family of all categories in the knowledge base \(K=(U,\bR)\) will be
referred to as \(K\textbf{-categories}\)
\subsection{Equivalence, generalization and specialization of knowledge}
\label{sec:orgce4347c}
Let \(K=(U,\bP),K'=(U,\bQ)\). \(K\) and \(K'\) are \textbf{equivalent} \(K\simeq
   K',(\bP\simeq\bQ)\) if \(IND(\bP)=IND(\bQ)\). Hence \(K\simeq K'\) if both \(K\) and
\(K'\) have the same set of elememtary categories. \emph{This means that knowledge in
knowledge bases \(K\) and \(K'\) enables us to express exactly the same facts about the universe.}

If \(IND(\bP)\subset IND(\bQ)\) then knowledge \(\bP\) is \textbf{finer} than knowledge
\(\bQ\) (\textbf{coarser}). \(\bP\) is \textbf{specialization} of \(\bQ\) and \(\bQ\) is \textbf{generalization}
of \(\bP\)
\section{Imprecise categories, approximations and rough sets}
\label{sec:org1f69ccc}
\subsection{Rough sets}
\label{sec:orgc4792a8}
Let \(X\subseteq U\). \(X\) is \(R\textbf{-definable}\) or \(R\textbf{-exact}\) if \(X\) is the union of some
\(R\text{-basic}\) categories. otherwise
\(R\textbf{-undefinable},R\textbf{-rough},R\textbf{-inexact}\)  .
\subsection{Approximations of set}
\label{sec:org7eab3e5}
Given \(K=(U,\bR), R\in IND(K)\)
\begin{align*}
&\underline{R}X=\bigcup\lb Y\in U/R:Y\subseteq X\rb\\
&\overline{R}X=\bigcup\lb Y\in U/R:Y\cap X\neq\emptyset\rb\\
\end{align*}
called the \(R\textbf{-lower}\) and \(R\textbf{-upper}\) \textbf{approximation} of \(X\)

\(BN_R(X)=\overline{R}X-\underline{R}X\) is \(R\textbf{-boundary}\) of \(X\).
\(BN_R(X)\) is the set of elements which cannot be classified either to \(X\) or
to \(-X\) having knowledge \(R\)

\begin{align*}
&POS_R(X)=\underline{R}X,R\text{-positive region of } X\\
&NEG_R(X)=U-\overline{R}X,R\text{-negative region of } X\\
&BN_R(X) - R\text{-borderline region of } X\\
\end{align*}

If \(x\in POS(X)\), then \(x\) will be called an \(R\textbf{-positive}\) \textbf{example of} \(X\)

\begin{proposition}
\begin{enumerate}
\item $X$ is $R$-definable if and only if $\underline{R}X=\overline{R}X$
\item $X$ is rought w.r.t. $R$ if and only if $\underline{R}X\neq\overline{R}X$
\end{enumerate}
\end{proposition}
\subsection{Properties of approximations}
\label{sec:org20d597d}
\begin{proposition}[2.2]
\begin{enumerate}
\item \(\uR X\subseteq X\subseteq \oR X\)
\item \(\uR\emptyset=\uR\emptyset=\emptyset;\quad \uR U=\oR U=U\)
\item \(\oR(X\cup Y)=\oR X\cup \oR Y\)
\item \(\uR(X\cap Y)=\uR X\cap \uR Y\)
\item \(X\subseteq Y\) implies \(\uR X\subseteq \uR Y\)
\item \(X\subseteq Y\) implies \(\oR X\subseteq\oR Y\)
\item \(\uR(X\cup Y)\subseteq \uR X\cup \uR Y\)
\item \(\uR(-X)=-\oR X\)
\item \(\oR(-X)=-\uR X\)
\item \(\oR(-X)=-\uR X\)
\item \(\uR\uR X=\oR\uR X=\uR X\)
\item \(\oR\oR X=\uR\oR X=\oR X\)
\end{enumerate}
\end{proposition}

The equivalence relation \(R\) over \(U\) uniquely defines a topological space
\(T=(U,DIS(R))\) where \(DIS(R)\) is the familty of all open and closed set in
\(T\) and \(U/R\) is a base for \(T\). The \(R\text{-lower}\) and \(R\text{-upper}\)
approximation of \(X\) in \(A\) are \textbf{interior} and \textbf{closure} operations in the
topological space \(T\)
\subsection{Approximations and membership relation}
\label{sec:org1852cb7}
\begin{align*}
&x\underline{\in}_RX \text{if and only if } x\in\underline{R}X\\
&x\overline{\in}_RX \text{if and only if } x\in\overline{R}X\\
\end{align*}
where \(\underline{\in}_R\) read "\(x\) \textbf{surely belongs} to \(X\) w.r.t. \(R\)" and
\(\overline{\in}_R\) - "\(x\) \textbf{possibly belongs} to \(X\) w.r.t. \(R\)". The \textbf{lower} and
\textbf{upper} membership.
\begin{proposition}
\begin{enumerate}
\item $x\uin X$ implies $x\in X$ implies $x\oin X$
\item $X\subset Y$ implies ($x\uin X$ implies $x\uin Y$ and $x\oin X$ implies $x\oin Y$)
\item $x\oin(X\cup Y)$ if and only if $x\oin X$ or $x\oin Y$
\item $x\uin(X\cap Y)$ if and only if $x\uin X$ and $x\uin Y$
\item $x\uin X$ or $x\uin Y$ implies $x\uin (X\cup Y)$
\item $x\oin X\cap Y$ implies $x\oin X$ and $x\oin Y$
\item $x\uin (-X)$ if and only if non $x\oin X$
\item $x\oin (-X)$ if and only if non $x\uin X$
\end{enumerate}
\end{proposition}
\subsection{Numerical characterization of imprecision}
\label{sec:orgfda11eb}
\textbf{accuracy measure}
\begin{equation*}
\alpha_R(X)=\frac{card\;\uR}{card\;\oR}
\end{equation*}
\subsection{Topological characterization of imprecision}
\label{sec:org05c2b68}
\begin{definition}[]
\begin{enumerate}
\item If \(\uR X\neq\emptyset\) and \(\oR X\neq U\), then we say that \(X\) is
\textbf{roughly R-definable}. We can decide whether some elements belong to \(X\)
or \(-X\)
\item If \(\uR X=\emptyset\) and \(\oR X\neq U\), then we say that \(X\) is
\textbf{internally R-undefinable}. We can decide whether some elemnts belong
to \(-X\)
\item If \(\uR X\neq\emptyset\) and \(\oR X=U\), then we say that \(X\) is
\textbf{externally R-undefinable}. We can decide whether some elements belong
to \(X\)
\item If \(\uR X=\emptyset\) and \(\oR X=U\), then we say that \(X\) is
\textbf{totally R-undefinable}. unable to decide
\end{enumerate}
\end{definition}

\begin{proposition}[2.4]
\begin{enumerate}
\item Set \(X\) is R-definable(roughly R-definable, totally R-undefinable) if and
only if so is \(-X\)
\item Set \(X\) is externally R-undefinable if and only if \(-X\) is internally
R-undefinable
\end{enumerate}
\end{proposition}

\begin{proof}
\begin{enumerate}
\item \begin{align*}
R\text{-definable}&\Leftrightarrow \uR X=\oR X, \uR\neq\emptyset,\oR\neq U\\
&\Leftrightarrow -\uR X=-\oR X\\
&\Leftrightarrow \oR(-X)=\uR(-X)\\
\end{align*}

\begin{align*}
X \text{ is roughly } R\text{-definable}
&\Leftrightarrow \uR X\neq \emptyset\wedge\oR X\neq U\\
&\Leftrightarrow -\uR X\neq U\wedge -\oR X\neq \emptyset\\
&\Leftrightarrow \oR(-X)\neq U\wedge \uR(-X)\neq \emptyset\\
\end{align*}
\end{enumerate}
\end{proof}
\subsection{Approximation of classifications}
\label{sec:org910351b}
If \(F=\lb X_1,\dots,X_n\rb\) is a family of non empty sets, then
\(\uR F=\lb \uR X_1,\dots,\uR X_n\rb\) and \(\oR F=\lb\oR X_1,\dots,\oR X_n\rb\),
called the \(R\textbf{-lower}\) \textbf{approximation} and the \(R\textbf{-upper}\)
\textbf{approximation} of the family \(F\)

The \textbf{accuracy of approximation} of \(F\) by \(R\) is
\begin{equation*}
\alpha_R(F)=\frac{\displaystyle\sum card\;\uR X_i}
{\displaystyle\sum card\;\oR X_i}
\end{equation*}

\textbf{quality of approximation} of \(F\) by \(R\)
\begin{equation*}
\gamma_R(F)=\frac{\displaystyle\sum card\;\uR X_i}{card\; U}
\end{equation*}

\begin{proposition}[2.5]
Let \(F=\lb X_1,\dots,X_n\rb\) where \(n>1\) be a classification of \(U\) and let
\(R\) be an equivalence relation. If there exists \(i\in\lb 1,2,\dots,n\rb\) s.t.
\(\uR X_i\neq\emptyset\), then for each \(j\neq i\) and \(j\in\lb 1,\dots,n\rb\),
\(\oR X_j\neq U\)
\end{proposition}

\begin{proof}
If \(\uR X_i\neq\emptyset\) then there exists \(x\in X\) s.t. \([x]_R\subseteq X\),
which implies \([x]_R\cap X_j=\emptyset\) for each \(j\neq i\). This yields \(\oR
   X_j\cap[x]_R=\emptyset\).
\end{proof}

\begin{proposition}[2.6]
Let \(F=\lb X_1,\dots,X_n\rb,n>1\) be a classification of \(U\) and let \(R\) be an
equivalence relation. If there exists \(i\in\lb 1,\dots,n\rb\) s.t. \(\oR
   X_i=U\), then for each \(j\neq i\) and \(j\in\lb 1,\dots,n\rb\) \(\uR X_j=\emptyset\)
\end{proposition}

\begin{proposition}[2.7]
Let \(F=\lb X_1,\dots,X_n\rb,n>1\) be a classification of \(U\) and let \(R\) be an
equivalence relation. If for each \(i\in\lb 1,2,\dots,n\rb\) \(\uR
   X_i\neq\emptyset\) holds, then \(\oR X_i\neq U\) for each \(i\in\lb 1,\dots,n,\rb\)
\end{proposition}

\begin{proposition}[]
Let \(F=\lb X_1,\dots,X_n\rb,n>1\) be a classification of \(U\) and let \(R\) be an
equivalence relation. If for each \(i\in\lb 1,2,\dots,n\rb\) \(\oR X_i=U\) holds,
then \(\uR X_i=\emptyset\) for each \(i\in\lb 1,\dots,n\rb\)
\end{proposition}
\subsection{Rough equality of sets}
\label{sec:orgbc84b32}
\begin{definition}[]
Let \(K=(U,\bR)\) be a knowledge base, \(X,Y\subseteq U\) and \(R\in IND(K)\), then
\begin{enumerate}
\item Sets \(X\) and \(Y\) are \textbf{bottom} \(R\textbf{-equal}\) \((X\eqsim_R Y)\) if \(\uR X=\uR
      Y\)
\item Sets \(X\) and \(Y\) are \textbf{top} \(R-\textbf{equal}\) \((X\simeq_R Y)\) if \(\oR X=\oR
      Y\)
\item Sets \(X\) and \(Y\) are \(R\textbf{-equal}\) \((X\approx_R Y)\) if \(X\simeq_R Y\)
and \(X\eqsim_R Y\)
\end{enumerate}
\end{definition}

\begin{proposition}[2.9]
\begin{enumerate}
\item \(X\eqsim Y\) iff \(X\cap Y\eqsim X\) and \(X\cap Y\eqsim Y\)
\item \(X\simeq Y\) iff \(X\cup Y\simeq X\) and \(X\cup Y\simeq Y\)
\item If \(X\simeq X'\) and \(Y\simeq Y'\) then \(X\cup Y\simeq X'\cup Y'\)
\item If \(X\eqsim X'\) and \(Y\eqsim Y'\) then \(X\cap Y\eqsim X'\cap Y'\)
\item If \(X\simeq Y\), then \(X\cup -Y\simeq U\)
\item If \(X\eqsim Y\), then \(X\cap -Y\eqsim\emptyset\)
\item If \(X\subseteq Y\) and \(Y\simeq\emptyset\), then \(X\simeq\emptyset\)
\item If \(X\subseteq Y\) and \(X\subseteq U\) then \(Y\subseteq U\)
\item \(X\simeq Y\) iff \(-X\eqsim -Y\)
\item If \(X\eqsim \emptyset\) or \(Y\eqsim\emptyset\), then \(X\cap
       Y\eqsim\emptyset\)
\item If \(X\simeq U\) or \(Y\simeq U\), then \(X\cup Y\simeq U\)
\end{enumerate}
\end{proposition}

\begin{proposition}[2.10 ]
For any equivalence relation \(R\)
\begin{enumerate}
\item \(\uR X\) is the intersection of all \(Y\subseteq U\) s.t. \(X\eqsim_R Y\)
\item \(\oR\) is the union of all \(Y\subseteq U\) s.t. \(X\simeq_R Y\)
\end{enumerate}
\end{proposition}
\subsection{Rough inclusion of sets}
\label{sec:org6fb091a}
\begin{definition}[]
Let \(K=(U,\bR)\) be a knowledge base, \(X,Y\subseteq U\) and \(R\in IND(K)\).
\begin{enumerate}
\item Set \(X\) is \textbf{bottom} \(R\textbf{-included}\) in \(Y\) \((X\subsetsim_R  Y)\) iff \(\uR
      X\subseteq\uR Y\)
\item Set \(X\) is \textbf{top} \(R\textbf{-included}\) in \(Y\) \((X\simsubset_R Y)\) iff \(\oR
      X\subseteq \oR Y\)
\item Set \(X\) is \(R\textbf{-included}\) in \(Y\) \((X\simsubsetsim_R Y)\) iff
\(X\simsubset_R Y\) and \(X\subsetsim_R Y\)
\end{enumerate}
\end{definition}

\begin{proposition}[2.11]
\begin{enumerate}
\item If \(X\subseteq Y\), then \(X\subsetsim Y, X\simsubset Y\) and \(X\simsubsetsim
      Y\)
\item If \(X\subsetsim Y\) and \(Y\subsetsim X\), then \(X\eqsim Y\)
\item If \(X\simsubset Y\) and \(Y\simsubset X\), then \(X\simeq Y\)
\item If \(X\simsubsetsim Y\) and \(Y\simsubsetsim X\) then \(X\approx Y\)
\item \(X\simsubset Y\) iff \(X\cup Y\simeq Y\)
\item \(X\subsetsim Y\) iff \(X\cap Y\eqsim X\)
\item If \(X\subseteq Y, X\eqsim X',Y\eqsim Y'\), then \(X'\subsetsim Y'\)
\item If \(X\subseteq Y, X\simeq X',Y\simeq Y'\), then \(X'\simsubset Y'\)
\item If \(X\subseteq Y, X\approx X',Y\approx Y'\), then \(X'\simsubsetsim Y'\)
\item If \(X'\simsubset X\) and \(Y'\simsubset Y\), then \(X'\cup Y'\simsubset X\cup
       Y\)
\item If \(X'\subsetsim X,Y'\subsetsim\) then \(X'\cap Y'\subsetsim X\cap Y\)
\item \(X\cap Y\subsetsim X\simsubset X\cup Y\)
\item If \(X\subsetsim Y\) and \(X\eqsim Z\) then \(Z\subsetsim Y\)
\item If \(X\simsubset Y\) and \(X\simeq Z\) then \(Z\simsubset Y\)
\item If \(X\simsubsetsim Y\) and \(X\approx\) then \(Z\simsubsetsim Y\)
\end{enumerate}
\end{proposition}
\section{Reduction of knowledge}
\label{sec:org2f3b47b}
\subsection{Reduct and Core of Knowledge}
\label{sec:orgc277c3a}
Let \(\bR\) be a family of equivalence relations and let \(P\in\bR\). \(R\) is
\textbf{dispensable} in \(\bR\) if \(IND(\bR)=IND(\bR-\lb R\rb)\). Otherwise \(R\) is
\textbf{indispensable} in \(\bR\). The family of \(\bR\) is \textbf{independent} if each \(R\in\bR\)
is indispensable in \(\bR\). Otherwise \(\bR\) is \textbf{dependent}

\begin{proposition}[3.1]
If \(\bR\) is independent and \(\bP\subseteq \bR\), then \(\bP\) is also independent
\end{proposition}

\begin{proof}
\(IND(\bR)=IND(\bP\cup(\bR-\bP))=IND(\bP)\cap IND(\bR-\bP)\)
\end{proof}

\(\bQ\subseteq \bR\) is a \textbf{reduct} of \(\bP\) if \(\bQ\) is independent and
\(IND(\bQ)=IND(\bP)\)


The set of all indispensable relations in \(\bP\) is called the \textbf{core} of \(\bP\)
denoted by \(CORE(\bP)\)

\begin{proposition}[3.2]
\begin{equation*}
CORE(\bP)=\bigcap RED(\bP)
\end{equation*}
where \(RED(\bP)\) is the family of all reducts of \(\bP\)
\end{proposition}

\begin{proof}
If \(\bQ\) is a reduct of \(\bP\) and \(R\in\bP-\bQ\), then \(IND(\bP)=IND(\bQ)\). If
\(\bQ\subseteq\bR\subseteq\bP\) then \(IND(\bQ)=IND(\bR)\). Assuming \(\bR=\bP-\lb
   R\rb\) then \(R\notin CORE(\bP)\)

If \(R\notin CORE(\bP)\). This means \(IND(\bP)=IND(\bP-\lb R\rb)\) which implies
that there exists an independent subset \(\bS\subseteq \bP-\lb R\rb\) s.t.
\(IND(\bS)=IND(\bP)\). Hence \(R\notin\bigcap RED(\bP)\)
\end{proof}
\subsection{Relative reduct and relative core of knowledge}
\label{sec:org65224e1}
Let \(P\) and \(Q\) be equivalence relations over \(U\)

\(P\textbf{-positive}\)
\begin{equation*}
POS_P(Q)=\displaystyle\bigcup_{X\in U/Q}\uP X
\end{equation*}
The \(P\text{-positive}\) region of \(Q\) is the set of all objects of the
universe \(U\) which can be properly classified to classes of \(U/Q\) employing
knowledge expressed by the classification \(U/P\)


Let \(\bP\) and \(\bQ\) be families of equivalence relations over \(U\)

\(R\in\bP\) is \(\bQ\textbf{-dispensable}\) in \(\bP\) if
\begin{equation*}
POS_{IND(\bP)}(IND(\bQ))=POS_{IND(\bP-\lb R\rb)}(IND(\bQ))
\end{equation*}
otherwise \(R\) is \(\bQ\text{-indispensable}\) in \(\bP\)

If every \(R\) in \(\bP\) is \(\bQ\text{-indispensable}\), we will say that \(\bP\)
is \(\bQ\textbf{-independent}\) or \(\bP\) is \textbf{independent w.r.t.} \(\bQ\)

The family \(\bS\subseteq \bP\) will be called a \(\bQ\textbf{-reduct}\) of \(\bP\)
if and only if \(\bS\) is the \(\bQ\text{-independent}\) subfamily of \(\bP\) and
\(POS_{\bS}(\bQ)=POS_{\bP}(\bQ)\)

The set of all \(\bQ\text{-indispensable}\) elmentary relations in \(\bP\) will
be called the \(\bQ\textbf{-core}\) of \(\bP\) and will be denoted as
\(CORE_{\bQ}(\bP)\)


\begin{proposition}[3.3]
\begin{equation*}
CORE_{\bQ}(\bP)=\bigcap RED_{\bQ}(\bP)
\end{equation*}
where \(RED_{\bQ}(\bP)\) is the family of all \(\bQ\text{-reducts}\) of \(\bP\)
\end{proposition}
\subsection{Reduction of categories}
\label{sec:org110d6a5}
Basic categories are pieces of knowledge, which can be considered as
"building blocks" of concepts. Every concept in the knowledge base can be
only expressed (exactly or approximately) in terms of basic categories. On
the other hand, every  basic category is "built up" (is an intersection) of
some elementary categories. Thus the question arises whether all the
elementary categories are necessary to define the basic categories in
question. 

Let \(F=\lb X_1,\dots,X_n\rb\) be a family of sets s.t. \(X_i\subseteq U\).

\(X_i\) is \textbf{dispensable} in \(F\) if \(\bigcap(F-\lb X_i\rb)=\bigcap F\), otherwise
the set \(X_i\) is \textbf{indispensable} in \(F\)

The family \(F\) is \textbf{independent} if all of its components are indispensable in
\(F\). Otherwise \(F\) is \textbf{dependent}

The family \(H\subseteq F\) is a \textbf{reduct} of \(F\) if \(H\) is independent and
\(\bigcap H=\bigcap F\)

The family of all indispensable sets in \(F\) will be called the \textbf{core} of \(F\),
denoted \(CORE(F)\)

\begin{proposition}[3.4]
\begin{equation*}
CORE(F)=\bigcap RED(F)
\end{equation*}
\end{proposition}
\subsection{Relative reduct and core of categories}
\label{sec:orge2b7bff}
\(F=\lb X_1,\dots,X_n\rb,X_i\subseteq U\) and a subset \(Y\subseteq U\) s.t.
\(\bigcap F\subseteq Y\)

\(X_i\) is \(Y\textbf{-dispensable}\) in \(\bigcap F\) if \(\bigcap(F-\lb
   X_i\rb)\subseteq Y\). Otherwise \(X_i\) is \(Y\textbf{-indispensable}\)

The family \(F\) is \(Y\textbf{-independent}\) in \(\bigcap F\) if all of its
components are \(Y\textbf{-indispensable}\) in \(\bigcap F\)

The family \(H\subseteq F\) is a \(Y\textbf{-reduct}\) of \(\bigcap F\) if \(H\) is
\(Y\text{-independent}\) in \(\bigcap F\) and \(\bigcap H\subseteq Y\)

The family of all \(Y\text{-indispensable}\) sets in \(\bigcap F\) will be called
the \(Y\textbf{core}\) of \(F\) and will be denoted by \(CORE_Y(F)\)

\begin{proposition}[3.5]
\begin{equation*}
CORE_Y(F)=\bigcap RED_Y(F)
\end{equation*}
\end{proposition}

\section{Dependencies in knowledge base}
\label{sec:org55869c6}
\subsection{Dependency of knowledge}
\label{sec:orgc580510}
Knowledge \(\bQ\) is \textbf{derivable} from knowledge \(\bP\) if all elementary
categories of \(\bQ\) can be defined in terms of some elementary categories of
knowledge \(\bP\). If \(\bQ\) is derivable from \(\bP\) we will also say that \(\bQ\)
\textbf{depends} on \(\bP\) and can be written \(\bP\Rightarrow \bQ\)

Let \(K=(U,\bR)\) be a knowledge base and let \(\bP,\bQ\subseteq \bR\)
\begin{enumerate}
\item Knowledge \(\bQ\) \textbf{depends on knowledge} \(\bP\) iff \(IND(\bP)\subseteq
      IND(\bQ)\) note that \(IND(\bP)\) is a set of pair
\item Knowledge \(\bP\) and \(\bQ\) are \textbf{equivalent} denoted as \(\bP\equiv\bQ\) iff
\(\bP\Rightarrow\bQ\) and \(\bQ\Rightarrow\bP\)
\item Knowledge \(\bP\) and \(\bQ\) are \textbf{independent} denoted as \(\bP\not\equiv\bQ\)
iff neither \(\bP\Rightarrow\bQ\) nor \(\bQ\Rightarrow\bP\)
\end{enumerate}


Obiviously \(\bP\equiv\bQ\) if and only if \(IND(\bP)=IND(\bQ)\)


\begin{proposition}[4.1]
The following conditions are equivalent
\begin{enumerate}
\item \(\bP\Rightarrow\bQ\)
\item \(IND(\bP\cup\bQ)=IND(\bP)\)
\item \(POS_{\bP}(\bQ)=POS_{IND(\bP)}(\bQ)=U\)
\item \(\underline{\bP} X=X\) for all \(X\in U/Q\)
\end{enumerate}


where \(\underline{\bP} X\) denotes \(\underline{IND(\bP)} X\)
\end{proposition}
\begin{proposition}[4.2]
If \(\bP\) is a reduct of \(\bQ\) then \(\bP\Rightarrow \bQ-\bP\) and \(IND(\bP)=IND(\bQ)\)
\end{proposition}

\begin{proof}
\begin{enumerate}
\item \((1)\to (2)\)

\(IND(\bP)\subseteq IND(\bP\cup \bQ)\subseteq IND(\bP)\)
\item \((2)\to (3)\)
\begin{align*}
POS_{IND(\bP)}(\bQ)&=\displaystyle\bigcup_{X\in U/\bQ} 
\underline{IND(\bP)} X\\
&=\displaystyle\bigcup_{X\in U/\bQ} \underline{IND(\bP\cup \bQ)} X
\end{align*}
Since \(\bQ\subseteq \bP\cup\bQ\), \(IND(\bP\cup\bQ)\subseteq IND(\bQ)\) and
for each \(x\in U\), \([x]_{IND(\bP\cup\bQ)}\subseteq [x]_{IND(\bQ)}\), which
means for any \(Y\in U/\bP\cup\bQ\), there exists some \(X\in U/\bQ\) s.t.
\(Y\subseteq X\). Hence \(POS_{\bP}(\bQ)=U\)
\item \((3)\to(4)\)
\begin{align*}
POS_{\bP}(\bQ)&=\displaystyle\bigcup_{X\in U/\bQ}\underline{IND(\bP)} X\\
&=\displaystyle\bigcup_{X\in U/bQ}\underline{\bP} X=U\\
\end{align*}
And \(\underline{\bP} X\subseteq X\)
\item \((4)\to (1)\)
\begin{align*}
\bP\Rightarrow\bQ&\Leftrightarrow IND(\bP)\subseteq IND(\bQ)\\
&\Leftrightarrow \forall x\in U, [x]_{IND(\bP)}\subseteq [x]_{IND(\bQ)}\\
\end{align*}
\end{enumerate}
\end{proof}

\begin{proof}
\(\bP\Rightarrow\bQ-\bP\Leftrightarrow IND(\bP\cup\bQ-\bP)=IND(\bP)\)
\end{proof}

\begin{proposition}[4.3]
\begin{enumerate}
\item If \(\bP\) is dependent, then there exists a subset \(\bQ\subset \bP\) s.t.
\(\bQ\) is a reduct of \(\bP\)
\item If \(\bP\subseteq\bQ\) and \(\bP\) is dependent, then all basic relations in
\(\bP\) are pairwise independent
\item If \(\bP\subseteq\bQ\) and \(\bP\) is independent, then every subset \(\bR\) of
\(\bP\) is independent
\end{enumerate}
\end{proposition}

\begin{proposition}[4.4]
\begin{enumerate}
\item If \(\bP\Rightarrow\bQ\) and \(\bP'\supset\bP\), then \(\bP'\Rightarrow\bQ\)
\item If \(\bP\Ra\bQ\) and \(\bQ'\subset\bQ\) then \(\bP\Ra\bQ'\)
\item \(\bP\Ra\bQ\) and \(\bQ\Ra\bR\) imply \(\bP\Ra\bR\)
\item \(\bP\Ra\bR\) and \(\bQ\Ra\bR\) imply \(\bP\cup\bQ\Ra\bR\)
\item \(\bP\Ra\bR\cup\bQ\) implies \(\bP\Ra\bR\) and \(\bP\Ra\bQ\)
\item \(\bP\Ra\bQ\) and \(\bR\Ra\bT\) imply \(\bP\cup\bR\Ra\bQ\cup\bT\)
\item \(\bP\Ra\bQ\) and \(\bR\Ra\bT\) imply \(\bP\cup\bR\Ra\bQ\cup\bT\)
\end{enumerate}
\end{proposition}
\subsection{Partial dependency of knowledge}
\label{sec:orgd27d6fe}
Let \(K=(U,\bR)\) be the knowledge base and \(\bP,\bQ\subset \bR\). Knowledge
\(\bQ\) \textbf{depends in a degree} \(k(0\le k \le 1)\) from knowledge \(\bP\),
symbolically \(\bP\Ra_k\bQ\) if and only if
\begin{equation*}
k=\gamma_{\bP}(\bQ)=\frac{card\; POS_{\bP}(\bQ)}{card\; U}
\end{equation*}

If \(k=1\), \(\bQ\) \textbf{totally depends from} \(\bP\). If \(0<k<1\), \(\bQ\) \textbf{roughly}
\textbf{depends from} \(\bP\). If \(k=0\), \(\bQ\) is \textbf{totally independent from} \(\bP\)

Ability to classify objects.
\begin{proposition}[4.5]
\begin{enumerate}
\item If \(\bR\Ra_k\bP\) and \(\bQ\Ra_l\bP\), then \(\bR\cup\bQ\Ra\bP\) for some
\(m\ge\max(k,l)\)
\item If \(\bR\cup\bP\Ra_k\bQ\), then \(\bR\Ra_l\bQ\) and \(\bP\Ra_m\bQ\) for some
\(l,m\le k\)
\item If \(\bR\Ra_k\bQ\) and \(\bR\Ra_l\bP\) then \(\bR\Ra_m\bQ\cup\bP\) for some
\(m\le\min(k,l)\)
\item If \(\bR\Ra_k\bQ\cup\bP\) then \(\bR\Ra_l\bQ\) and \(\bR\Ra_m\bP\) for some 
\(l,m\ge k\)
\item If \(\bR\Ra_k\bP\) and \(\bP\Ra_l\bQ\) then \(\bR\Ra_m\bQ\) for some \(m\ge l+k-1\)
\end{enumerate}
\end{proposition}
\section{Knowledege prepresentation}
\label{sec:orgd025b6d}
\subsection{Formal definition}
\label{sec:orga4e977b}
\textbf{Knowledge representation system} is a pair \(S=(U,A)\) where \(U\) is a nonempty
finite set called the \textbf{universe}, and \(A\) is a nonempty finite set of
\textbf{primitive attributes}

Every primitive attribute \(a\in A\) is a total function \(a:U\to V_a\) where
\(V_a\) is the \textbf{domain} of \(a\)

With every subset of attributes \(B\subseteq A\) we associate a binary relation
\(IND(B)\) called and \textbf{indiscernibility relation}
\begin{equation*}
IND(B)=\lb(x,y)\in U^2:\text{for every } a\in B,a(x)=a(y)\rb
\end{equation*}
\(IND(B)\) is an euivalence relation and
\begin{equation*}
IND(B)=\displaystyle\bigcap_{a\in B} IND(a)
\end{equation*}

Every subset \(B\subseteq A\) will be called an \textbf{attribute}. If \(B\) is a single
element set, then \(B\) is called \textbf{primitive} otherwise \textbf{compound}

\(a(x)\) can be viewed as a name of \([x]_{IND(a)}\). The name of an elementary
category of attribute \(B\subseteq A\) containing object \(x\) is a set of pairs
\(\lb a,a(x):a\in B\rb\)

There is a one-to-one correspondence between knowledge bases and knowledge
representation system up to isomorphism of attributes and attribute names

Suppose

\begin{center}
\begin{tabular}{rrrrrr}
U & a & b & c & d & e\\
\hline
1 & 1 & 0 & 2 & 2 & 0\\
2 & 0 & 1 & 1 & 1 & 2\\
3 & 2 & 0 & 0 & 1 & 1\\
4 & 1 & 1 & 0 & 2 & 2\\
5 & 1 & 0 & 2 & 0 & 1\\
6 & 2 & 2 & 0 & 1 & 1\\
7 & 2 & 1 & 1 & 1 & 2\\
8 & 0 & 1 & 1 & 0 & 1\\
\end{tabular}
\end{center}

The universe \(U=\{1,2,3,4,5,6,7,8\}\). \(V=V_a=\dots=V_e=\{0,1,2\}\)
\begin{align*}
&U/IND\{a\}=\{\{2,8\},\{1,4,5\},\{3,6,7\}\}\\
&U/IND\{c,d\}=\{\{1\},\{3,6\},\{2,7\},\{4\},\{5\},\{8\}\}
\end{align*}
\subsection{Discernibility matrix}
\label{sec:org1b75e74}
Let \(S=(U,A)\) be a knowledge representation system with \(U=\lb
   x_1,x_2,\dots,x_n\rb\). By an \textbf{discernibility matrix of} \(S\) is
\begin{equation*}
M(S)=(c_{ij})=\lb a\in A:a(x_i)\neq a(x_j)\rb \quad\text{for } i,j=1,2,\dots,n
\end{equation*}

Now the core can be defined as the set of all single element entries of the
discernibility matrix

\(B\subseteq A\) is the reduct of \(A\) if \(B\) is the minimal subset of A s.t.
\begin{equation*}
B\cap c\neq\emptyset \text{ for any nonempty entry } c(c\neq\emptyset) \text{ in }
M(S)
\end{equation*}
\section{Decision tables}
\label{sec:orgcbd2394}
\subsection{Formal definition and some properties}
\label{sec:org779f289}
Let \(K=(U,A)\) be a knowledge representation system and let \(C,D\subset A\) be
two subsets of attributes called \textbf{condition} and \textbf{decision attributes}
repectively. KR-system with distinguished condition ad decision attributes
will be called a \textbf{decision table} and will be denoted by \(T=(U,A,C,D)\) or in
short \(CD\)

Equivalence classes of the relations \(IND(C)\) and \(IND(D)\) will be called
\textbf{condition} and \textbf{decision classes}

With every \(x\in U\) we associate a function \(d_x:A\to V\) s.t. \(d_x(a)=a(x)\)
for every \(a\in C\cup D\). The function \(d_x\) will be called a \textbf{decision rule}

If \(d_x\) is a decision rule, then the restriction of \(d_x\) to C, denoted
\(d_x|C\) and the restriction of \(d_x\) to \(D\), denoted \(d_x|D\) will be called
\textbf{conditions} and \textbf{decisions} of \(d_x\)

The decision rule \(d_x\) is \textbf{consistent} if for every \(y\neq x,d_x|C=d_y|C\)
implies \(d_x|D=d_y|D\). Otherwise \textbf{inconsistent}

A decision table is \textbf{consistent} if al its decision rules are consistent

\begin{proposition}[6.1]
A decision table \(T=(U,A,C,D)\) is consistent if and only if \(C\Ra D\)
\end{proposition}

\begin{proposition}[6.2]
Each decision table \(T=(U,A,C,D)\) can be uniquely decomposed into two
decision tables \(T_1=(U,A,C,D)\) and \(T_2=(U,A,C,D)\) s.t. \(C\Ra_1 D\) in \(T_1\)
and \(C\Ra_0 D\) in \(T_2\) where \(U_1=POS_{C}(D)\) and 
\(U_2=\displaystyle\bigcup_{X\in U/IND(D)}BN_C(X)\)
\end{proposition}

Example. Consider
\begin{table}[htbp]
\caption{\label{tab:dec-tab-1}
Knowledge representation system}
\centering
\begin{tabular}{rrrrrr}
U & a & b & c & d & e\\
\hline
1 & 1 & 0 & 2 & 2 & 0\\
2 & 0 & 1 & 1 & 1 & 2\\
3 & 2 & 0 & 0 & 1 & 1\\
4 & 1 & 1 & 0 & 2 & 2\\
5 & 1 & 0 & 2 & 0 & 1\\
6 & 2 & 2 & 0 & 1 & 1\\
7 & 2 & 1 & 1 & 1 & 2\\
8 & 0 & 1 & 1 & 0 & 1\\
\end{tabular}
\end{table}

Assume that a,b,c are condition attributes and d,e are decision attributes. 
\begin{align*}
  &U/\{a\}=\{\{2,8\},\{1,4,5\},\{3,6,7\}\}\\
  &U/\{b\}=\{\{1,3,5\},\{2,4,7,8\},\{6\}\}\\
  &U/\{c\}=\{\{3,4,6\},\{2,7,8\},\{1,5\}\}\\
  &U/\{d\}=\{\{5,8\},\{2,3,6,7\},\{1,4\}\}\\
  &U/\{e\}=\{\{1\},\{3,5,6,8\},\{2,4,7\}\}\\
  &U/\{a,b,c\}=\{\{1,5\},\{2,8\},\{3\},\{4\},\{6\},\{7\}\}\\
  &U/\{d,e\}=\{\{1\},\{2,7\},\{3,6\},\{4\},\{5,8\}\}\\
  &POS_C(D)=\{3,4,6,7\}\\
  &\displaystyle\bigcup_{X\in/IND(D)}BN_C(X)=\{1,2,5,8\}\\
\end{align*}

\begin{table}[htbp]
\caption{\label{tab:dec-tab-2}
}
\centering
\begin{tabular}{rrrrrr}
\(U_1\) & a & b & c & d & e\\
\hline
3 & 2 & 0 & 0 & 1 & 1\\
4 & 1 & 1 & 0 & 2 & 2\\
6 & 2 & 2 & 0 & 1 & 1\\
7 & 2 & 1 & 1 & 1 & 2\\
\end{tabular}
\end{table}

\begin{table}[htbp]
\caption{\label{tab:dec-tab-3}
}
\centering
\begin{tabular}{rrrrrr}
\(U_2\) & a & b & c & d & e\\
\hline
1 & 1 & 0 & 2 & 2 & 0\\
2 & 0 & 1 & 1 & 1 & 2\\
5 & 1 & 0 & 2 & 0 & 1\\
8 & 0 & 1 & 1 & 0 & 1\\
\end{tabular}
\end{table}

Table \ref{tab:dec-tab-2} is consistent whereas table \ref{tab:dec-tab-3} is totally inconsistent
\subsection{Simplification of decision tables}
\label{sec:org29e52f0}
Step
\begin{enumerate}
\item Computation of reducts of condition attributes which is equivalent to
elimination of some column from the decision table
\item elimination of duplicate rows
\item elimination of superfluous values of attributes
\end{enumerate}


Thus the proposed method consists in removing superfluous condition
attributes (columns), duplicate rows and, in addition to that, irrelevant
values of condition attributes.


Suppose \(B\subseteq A\) and an object \(x\). \(\forall C,
   [x]_C=\displaystyle_{a\in C}[x]_a\). Each \([x]_a\) is uniquely determined by
attribute value \(a(x)\). hence in order to remove superfluous values of
condition attributes, we have to eliminate all superfluous equivalence
classes \([x]_a\) from the equivalence class \([x]_C\) 

Given 
\begin{center}
\begin{tabular}{rrrrrr}
U & a & b & c & d & e\\
\hline
1 & 1 & 0 & 0 & 1 & 1\\
2 & 1 & 0 & 0 & 0 & 1\\
3 & 0 & 0 & 0 & 0 & 0\\
4 & 1 & 1 & 0 & 1 & 0\\
5 & 1 & 1 & 0 & 2 & 2\\
6 & 2 & 1 & 0 & 2 & 2\\
7 & 2 & 2 & 2 & 2 & 2\\
\end{tabular}
\end{center}
where a,b,c,d are condition attributes and e is a decision attribute.
\end{document}