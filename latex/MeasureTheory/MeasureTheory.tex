% Created 2020-01-30 四 22:25
% Intended LaTeX compiler: pdflatex
\documentclass[11pt]{article}
\usepackage[utf8]{inputenc}
\usepackage[T1]{fontenc}
\usepackage{graphicx}
\usepackage{grffile}
\usepackage{longtable}
\usepackage{wrapfig}
\usepackage{rotating}
\usepackage[normalem]{ulem}
\usepackage{amsmath}
\usepackage{textcomp}
\usepackage{amssymb}
\usepackage{capt-of}
\usepackage{hyperref}
\usepackage{minted}
% TIPS
% \substack{a\\b} for multiple lines text





% pdfplots will load xolor automatically without option
\usepackage[dvipsnames]{xcolor}

\usepackage{forest}
% two-line text in node by [two \\ lines]
% \begin{forest} qtree, [..] \end{forest}
\forestset{
  qtree/.style={
    baseline,
    for tree={
      parent anchor=south,
      child anchor=north,
      align=center,
      inner sep=1pt,
    }}}
%\usepackage{flexisym}
% load order of mathtools and mathabx, otherwise conflict overbrace

\usepackage{mathtools}
%\usepackage{fourier}
\usepackage{pgfplots}
\usepackage{amsthm}
\usepackage{amsmath}
%\usepackage{unicode-math}
%
\usepackage{commath}
%\usepackage{,  , }
\usepackage{amsfonts}
\usepackage{amssymb}
% importing symbols https://tex.stackexchange.com/questions/14386/importing-a-single-symbol-from-a-different-font
%mathabx change every symbol
% use instead stmaryrd
%\usepackage{mathabx}
\usepackage{stmaryrd}
\usepackage{empheq}
\usepackage{tikz}
\usepackage{tikz-cd}
%\usepackage[notextcomp]{stix}
\usetikzlibrary{arrows.meta}
\usepackage[most]{tcolorbox}
%\utilde
%\usepackage{../../latexpackage/undertilde/undertilde}
% left and right superscript and subscript
\usepackage{actuarialsymbol}
\usepackage{threeparttable}
\usepackage{scalerel,stackengine}
\usepackage{stackrel}
% \stackrel[a]{b}{c}
\usepackage{dsfont}
% text font
\usepackage{newpxtext}
%\usepackage{newpxmath}

%\newcounter{dummy} \numberwithin{dummy}{section}
\newtheorem{dummy}{dummy}[section]
\theoremstyle{definition}
\newtheorem{definition}[dummy]{Definition}
\newtheorem{corollary}[dummy]{Corollary}
\newtheorem{lemma}[dummy]{Lemma}
\newtheorem{proposition}[dummy]{Proposition}
\newtheorem{theorem}[dummy]{Theorem}
\theoremstyle{definition}
\newtheorem{example}[dummy]{Example}
\theoremstyle{remark}
\newtheorem*{remark}{Remark}


\newcommand\what[1]{\ThisStyle{%
    \setbox0=\hbox{$\SavedStyle#1$}%
    \stackengine{-1.0\ht0+.5pt}{$\SavedStyle#1$}{%
      \stretchto{\scaleto{\SavedStyle\mkern.15mu\char'136}{2.6\wd0}}{1.4\ht0}%
    }{O}{c}{F}{T}{S}%
  }
}

\newcommand\wtilde[1]{\ThisStyle{%
    \setbox0=\hbox{$\SavedStyle#1$}%
    \stackengine{-.1\LMpt}{$\SavedStyle#1$}{%
      \stretchto{\scaleto{\SavedStyle\mkern.2mu\AC}{.5150\wd0}}{.6\ht0}%
    }{O}{c}{F}{T}{S}%
  }
}

\newcommand\wbar[1]{\ThisStyle{%
    \setbox0=\hbox{$\SavedStyle#1$}%
    \stackengine{.5pt+\LMpt}{$\SavedStyle#1$}{%
      \rule{\wd0}{\dimexpr.3\LMpt+.3pt}%
    }{O}{c}{F}{T}{S}%
  }
}

\newcommand{\bl}[1] {\boldsymbol{#1}}
\newcommand{\Wt}[1] {\stackrel{\sim}{\smash{#1}\rule{0pt}{1.1ex}}}
\newcommand{\wt}[1] {\widetilde{#1}}
\newcommand{\tf}[1] {\textbf{#1}}


%For boxed texts in align, use Aboxed{}
%otherwise use boxed{}

\DeclareMathSymbol{\widehatsym}{\mathord}{largesymbols}{"62}
\newcommand\lowerwidehatsym{%
  \text{\smash{\raisebox{-1.3ex}{%
    $\widehatsym$}}}}
\newcommand\fixwidehat[1]{%
  \mathchoice
    {\accentset{\displaystyle\lowerwidehatsym}{#1}}
    {\accentset{\textstyle\lowerwidehatsym}{#1}}
    {\accentset{\scriptstyle\lowerwidehatsym}{#1}}
    {\accentset{\scriptscriptstyle\lowerwidehatsym}{#1}}
}

\usepackage{graphicx}
    
% text on arrow for xRightarrow
\makeatletter
%\newcommand{\xRightarrow}[2][]{\ext@arrow 0359\Rightarrowfill@{#1}{#2}}
\makeatother


\newcommand{\dom}[1]{%
\mathrm{dom}{(#1)}
}

% Roman numerals
\makeatletter
\newcommand*{\rom}[1]{\expandafter\@slowromancap\romannumeral #1@}
\makeatother

\def \fR {\mathfrak{R}}
\def \bx {\boldsymbol{x}}
\def \bz {\boldsymbol{z}}
\def \ba {\boldsymbol{a}}
\def \bh {\boldsymbol{h}}
\def \bo {\boldsymbol{o}}
\def \bU {\boldsymbol{U}}
\def \bc {\boldsymbol{c}}
\def \bV {\boldsymbol{V}}
\def \bI {\boldsymbol{I}}
\def \bK {\boldsymbol{K}}
\def \bt {\boldsymbol{t}}
\def \bb {\boldsymbol{b}}
\def \bA {\boldsymbol{A}}
\def \bX {\boldsymbol{X}}
\def \bu {\boldsymbol{u}}
\def \bS {\boldsymbol{S}}
\def \bZ {\boldsymbol{Z}}
\def \bz {\boldsymbol{z}}
\def \by {\boldsymbol{y}}
\def \bw {\boldsymbol{w}}
\def \bT {\boldsymbol{T}}
\def \bF {\boldsymbol{F}}
\def \bS {\boldsymbol{S}}
\def \bm {\boldsymbol{m}}
\def \bW {\boldsymbol{W}}
\def \bR {\boldsymbol{R}}
\def \bQ {\boldsymbol{Q}}
\def \bS {\boldsymbol{S}}
\def \bP {\boldsymbol{P}}
\def \bT {\boldsymbol{T}}
\def \bY {\boldsymbol{Y}}
\def \bH {\boldsymbol{H}}
\def \bB {\boldsymbol{B}}
\def \blambda {\boldsymbol{\lambda}}
\def \bPhi {\boldsymbol{\Phi}}
\def \btheta {\boldsymbol{\theta}}
\def \bTheta {\boldsymbol{\Theta}}
\def \bmu {\boldsymbol{\mu}}
\def \bphi {\boldsymbol{\phi}}
\def \bSigma {\boldsymbol{\Sigma}}
\def \lb {\left\{}
\def \rb {\right\}}
\def \la {\langle}
\def \ra {\rangle}
\def \caln {\mathcal{N}}
\def \dissum {\displaystyle\Sigma}
\def \dispro {\displaystyle\prod}
\def \E {\mathbb{E}}
\def \Q {\mathbb{Q}}
\def \N {\mathbb{N}}
\def \V {\mathbb{V}}
\def \R {\mathbb{R}}
\def \P {\mathbb{P}}
\def \A {\mathbb{A}}
\def \Z {\mathbb{Z}}
\def \I {\mathbb{I}}
\def \C {\mathbb{C}}
\def \cala {\mathcal{A}}
\def \calb {\mathcal{B}}
\def \calq {\mathcal{Q}}
\def \calp {\mathcal{P}}
\def \cals {\mathcal{S}}
\def \calg {\mathcal{G}}
\def \caln {\mathcal{N}}
\def \calr {\mathcal{R}}
\def \calm {\mathcal{M}}
\def \calc {\mathcal{C}}
\def \calf {\mathcal{F}}
\def \calk {\mathcal{K}}
\def \call {\mathcal{L}}
\def \calu {\mathcal{U}}
\def \bcup {\bigcup}


\def \uin {\underline{\in}}
\def \oin {\overline{\in}}
\def \uR {\underline{R}}
\def \oR {\overline{R}}
\def \uP {\underline{P}}
\def \oP {\overline{P}}

\def \Ra {\Rightarrow}

\def \e {\enspace}

\def \sgn {\operatorname{sgn}}
\def \gen {\operatorname{gen}}
\def \ker {\operatorname{ker}}
\def \im {\operatorname{im}}

\def \tril {\triangleleft}

% \varprod
\DeclareSymbolFont{largesymbolsA}{U}{txexa}{m}{n}
\DeclareMathSymbol{\varprod}{\mathop}{largesymbolsA}{16}

% \bigtimes
\DeclareFontFamily{U}{mathx}{\hyphenchar\font45}
\DeclareFontShape{U}{mathx}{m}{n}{
      <5> <6> <7> <8> <9> <10>
      <10.95> <12> <14.4> <17.28> <20.74> <24.88>
      mathx10
      }{}
\DeclareSymbolFont{mathx}{U}{mathx}{m}{n}
\DeclareMathSymbol{\bigtimes}{1}{mathx}{"91}
% \odiv
\DeclareFontFamily{U}{matha}{\hyphenchar\font45}
\DeclareFontShape{U}{matha}{m}{n}{
      <5> <6> <7> <8> <9> <10> gen * matha
      <10.95> matha10 <12> <14.4> <17.28> <20.74> <24.88> matha12
      }{}
\DeclareSymbolFont{matha}{U}{matha}{m}{n}
\DeclareMathSymbol{\odiv}         {2}{matha}{"63}


\newcommand\subsetsim{\mathrel{%
  \ooalign{\raise0.2ex\hbox{\scalebox{0.9}{$\subset$}}\cr\hidewidth\raise-0.85ex\hbox{\scalebox{0.9}{$\sim$}}\hidewidth\cr}}}
\newcommand\simsubset{\mathrel{%
  \ooalign{\raise-0.2ex\hbox{\scalebox{0.9}{$\subset$}}\cr\hidewidth\raise0.75ex\hbox{\scalebox{0.9}{$\sim$}}\hidewidth\cr}}}

\newcommand\simsubsetsim{\mathrel{%
  \ooalign{\raise0ex\hbox{\scalebox{0.8}{$\subset$}}\cr\hidewidth\raise1ex\hbox{\scalebox{0.75}{$\sim$}}\hidewidth\cr\raise-0.95ex\hbox{\scalebox{0.8}{$\sim$}}\cr\hidewidth}}}
\newcommand{\stcomp}[1]{{#1}^{\mathsf{c}}}


\author{Claudio Landim}
\date{\today}
\title{Measure Theory}
\hypersetup{
 pdfauthor={Claudio Landim},
 pdftitle={Measure Theory},
 pdfkeywords={},
 pdfsubject={},
 pdfcreator={Emacs 26.3 (Org mode 9.3.1)}, 
 pdflang={English}}
\begin{document}

\maketitle
\tableofcontents \clearpage
\section{Introduction: a non-measurable set}
\label{sec:org57c4ac1}
Suppose we want a measure that satisfies:
\begin{enumerate}
\setcounter{enumi}{-1}
\item \(\lambda:\calp(\R)\to\R_+\cup\{+\infty\}\)
\item \(\lambda((a,b])=b-a\)
\item \(A\subseteq\R,A+x=\{x+y:y\in A\}\)
\begin{equation*}
\forall A\subseteq\R\forall x\in\R,\lambda(A+x)=\lambda(A)
\end{equation*}
\item \(A=\bigcup_{j\ge 1}A_j,A_j\cap A_k=\emptyset\)
\begin{equation*}
\lambda(A)=\displaystyle\sum_{k\ge1}\lambda(A_k)
\end{equation*}
\end{enumerate}



Define \(x\sim y\) for \(x,y\in\R\) if \(y-x\in\Q\). \(\Lambda=\R/\sim\) and
\(\alpha,\beta\in\Lambda\). \(\Gamma\) is uncountable since each equivalent class
is countable.

By the \textbf{Axiom of Choice}, we have a \(\Omega\subseteq\R\) s.t. for each
\([x]\in\R/\sim\), there is a \(x\in[x]\) s.t. \(x\in\Omega\). Hence we can assume
\(\Omega\subseteq(0,1)\). 

\begin{claim}
For \(p,q\in\Q\), either \(\Omega+p=\Omega+q\) or
\(\Omega+p\cap\Omega+q=\emptyset\).
\end{claim}

\begin{proof}
Assume \((\Omega+p)\cap(\Omega+q)\neq\emptyset\), \(x=\alpha+p=\beta+q\). Hence
\(\alpha-\beta=q-p\in\Q\), which implies \(\alpha=\beta\).
\end{proof}

\begin{claim}
\(\Omega+q\subseteq(-1,2)\) since \(-1<q<1\).
\end{claim}

In particular,
\begin{equation*}
\displaystyle\bigcup_{\substack{q\in\Q\\-1<q<1}}(\Omega+q)\subseteq(-1,2)
\end{equation*}

\begin{claim}
If \(E\subseteq F\), then \(\lambda(E)\le\lambda(F)\)
\end{claim}

\begin{proof}
\(\lambda(F)=\lambda(E\cup(F-E))=\lambda(E)+\lambda(F-E)\)
\end{proof}

If \(q\neq p\),
\begin{equation*}
\lambda(\displaystyle\bigcup_{\substack{q\in\Q\\-1<q<1}}(\Omega+q))
=\displaystyle\sum_{\substack{q\in\Q\\-1<q<1}}\lambda(\Omega+q)
=\displaystyle\sum_{\substack{q\in\Q\\-1<q<1}}\lambda(\Omega)
\le\lambda((-1,2))
=3
\end{equation*}

Hence \(\lambda(\Omega)=0\)

\begin{claim}
\((0,1)\subseteq\sum_{q\in\Q,-1<q<1}(\Omega+q)\)
\end{claim}

\begin{proof}
Fix \(x\in[0,1]\), \(\exists\alpha\in[x]\cap\Omega\) and \(\alpha\in(0,1)\). Hence
\(\alpha-x=q\in\Q\). Then \(x\in\Omega+q\)
\end{proof}

Hence we have a contradiction and there is no such \(\lambda\) function.

\section{Classes of subsets}
\label{sec:org4b91bf0}
\begin{definition}[]
For \(\cals\subseteq\calp(\Omega)\), \(\cals\) is a \textbf{semi-algebra} if
\begin{enumerate}
\item \(\Omega\in\cals\)
\item If \(A,B\in\cals\), then \(A\cap B\in\cals\)
\item For all \(A\in\cals\), there are \(E_1,\dots,E_n\in\cals\) s.t. \(A^c=\sqcup E_j\)
\end{enumerate}
\end{definition}

\begin{examplle}[]
If \(\Omega=\R\) and 
\begin{align*}
\cals&=\R\cup\{(a,b]:a<b,a,b\in\R\}\\
&\cup\{(-\infty,b]:b\in\R\}\\
&\cup\{(a,\infty):a\in\R\}\\
&\cup\emptyset
\end{align*}
then \(\cals\) is a semi-algebra
\end{examplle}

\begin{definition}[]
Take \(\cala\subseteq\calp(\Omega)\), \(\cala\) is an \textbf{algebra} if
\begin{enumerate}
\item \(\Omega\in\cala\)
\item If \(A,B\in\cala\), then \(A\cap B\in\cala\)
\item If \(A\in\cala\), then \(A^c\in\cala\)
\end{enumerate}
\end{definition}

If \(\cala\) is an algebra, then it is also semi-algebra.

\begin{definition}[]
\(\calf\subseteq\calp(\Omega)\) is a \textbf{\(\sigma\)-algebra} if
\begin{enumerate}
\item \(\Omega\in\calf\)
\item If \(A_j\in\calf\) for \(j\ge 1\), then \(\bigcap_{j\ge1}A_j\in\calf\)
\item If \(A\in\calf\), then \(A^c\in\calf\)
\end{enumerate}
\end{definition}

\begin{proposition}[]
Suppose \(\cala_\alpha\subseteq\calp(\Omega)\), \(\cala_\alpha\) is an (\(\sigma\)-)algebra, 
\(\alpha\in I\). Then \(\cala=\bigcap_{\alpha\in I}\cala_\alpha\) is an (\(\sigma\)-)algebra
\end{proposition}

\begin{definition}[]
Suppose class \(\calc\subseteq\calp(\Omega)\). An (\(\sigma\)-)algebra \(\cala(\calc)\) \textbf{generated}
\textbf{by} \(\calc\) is the smallest (\(\sigma\)-)algebra s.t. for any (\(\sigma\)-)algebra \(\calb\supseteq\calc\)
and \(\calb\) is an (\(\sigma\)-)algebra, then \(\calb\supseteq \cala\). Hence
\(\cala(\calc)=\bigcap_\alpha \cala_\alpha\)
\end{definition}


\begin{lemma}[]
Let \(\cals\) be a semi-algebra and \(\cals\subseteq\calp(\Omega)\).
\(A\in\cala(\cals)\) 
iff there exists \(1\le j\le n,E_j\in\cals\) s.t.
\begin{equation*}
A=\displaystyle\sum_{j=1}^nE_j
\end{equation*}
\end{lemma}

\begin{proof}
Suppose \(A=\sum_{j=1}^nE_j,E_j\in\cals\subseteq \cala(\cals)\)
\begin{itemize}
\item If \(E,F\in\cala\) then \(E\cup F\in\cala\) since \(E\cup F=(E^c\cap F^c)^c\)
\end{itemize}


Suppose \(A\in \cala(\cals)\), let
\(\calb=\{\sum_{j=1}^nF_j:F_j\in\cals\}\subseteq\calp(\Omega)\) 
\begin{itemize}
\item \(\calb\) is an algebra
\item \(\calb\supseteq\cals\)
\end{itemize}
\end{proof}

\begin{definition}[]
\(\calc\subseteq\calp(\Omega),\emptyset\in\calc\) and
\(\mu:\calc\to\R_+\cup\{+\infty\}\), \(\mu\) is \textbf{additive} if
\begin{enumerate}
\item \(\mu(\emptyset)=0\)
\item \(E_1,\dots,E_n\in\calc,E=\sum_{j=1}^nE_j\in\calc\), then
\(\mu(E)=\sum_{j=1}^n\mu(E_j)\)
\end{enumerate}
\end{definition}

\begin{remark}
\begin{enumerate}
\item Assume there is a set \(A\in\calc,\mu(A)<\infty\), then \(A=A\cup\emptyset\) and
\(\mu(\emptyset)=0\)
\item If \(E\subseteq F\) and \(F-E\in\calc\), if \(\mu(E)=+\infty\), \(F=E\cup(F-E)\),
hence \(\mu(F)=\mu(E)+\mu(F-E)=+\infty\). If \(\mu(E)<\infty\), 
\(\mu(F-E)=\mu(F)-\mu(E)\). We can conclude \(\mu(E)\le\mu(F)\)
\end{enumerate}
\end{remark}

\begin{examplle}[]
Discrete measure. Suppose we have \(\{X_j:j\ge 1\},\{P_j:j\ge 1\}\)
and \(\mu(A)=\sum_{j}P_j1\{X_j\in A\}\) (indicator function), then \(\mu\) is
additive 
\end{examplle}

\begin{definition}[]
\(\emptyset\in\calc\subseteq\calp(\Omega),\mu:\calc\to\R_+\cup\{+\infty\}\). \(\mu\) is
\textbf{\(\sigma\)-additive} if
\begin{enumerate}
\item \(\mu(\emptyset)=0\)
\item \(E=\sum_{j\ge 1}E_j\in\calc,\mu(E)=\sum_{j\ge 1}\mu(E_j)\)
\end{enumerate}
\end{definition}

\begin{examplle}[]
\(\Omega=(0,1),\calc=\{(a,b]:0\le a<b<1\}\)
\begin{equation*}
\mu((a,b])=
\begin{cases}
+\infty&a=0\\
b-a&b>a
\end{cases}
\end{equation*}
\((a,b]=\sum_{j=1}^n(a_j,b_j]\). \(\mu\) is additive but not sigma-additive. If \(x_j\)
converges to 0, \(x_0=1/2\), then \(\mu((0,1/2])=+\infty,\mu(\sum_{j\ge1}(x_{j+1},x_j])=1/2\)
\end{examplle}

\section{Set functions}
\label{sec:org35e3fb4}
\begin{definition}[]
\begin{enumerate}
\item \(\mu\) is \textbf{continuous from below} at \(E\) if for all \((E_n)_{n\ge1}\),
\(E_n\in\calc,E_n\subseteq E_{n+1},\bigcup_{n\ge1}E_n=E\),
\(\lim_{n\to\infty}\mu(E_n)=\mu(E)\)
\item \(\mu\) is \textbf{continuous from above} at \(E\) if \(E_n\supseteq
   E_{n+1},\bigcap_{n\ge1}E_n=E,\exists n\;\mu(E_{n_0})<\infty\), then
\(\lim_{n\to\infty}\mu(E_n)=\mu(E)\)
\end{enumerate}
\end{definition}

\begin{lemma}[]
Algebra \(\cala\subseteq\calp(\Omega),\mu:\cala\to\R_+\cup\{+\infty\}\) additive,
then
\begin{enumerate}
\item If \(\mu\) is \(\sigma\)-additive, then \(\mu\) is continuous at \(E\) for all \(E\in\cala\)
\item If \(\mu\) is continuous from below, then \(\mu\) is \(\sigma\)-additive
\item If \(\mu\) is continuous from above at \(\emptyset\) and \(\mu\) is finite, then \(\mu\) is
\(\sigma\)-additive
\end{enumerate}
\end{lemma}

\begin{proof}
\begin{enumerate}
\item Suppose \(E_n\uparrow E\), define \(F_1=E_1,F_2=E_2-E_1,\dots,F_n=E_n-E_{n-1}\),
then \(\bigcup E_n=\sum
   F_n=E\),\(\mu(E)=\sum\mu(F_k)=\lim_{n\to\infty}\sum_{k=1}^n\mu(F_k)=\lim\mu(\sum
   F_k)=\lim\mu(E_n)\) 

Suppose \(E,E_n\in\cala,E_n\downarrow E,\mu(E_{n_0})<\infty\). Define
\(G_1=E_{n_0}-E_{n_0+1}, G_2=E_{n_0}}-E_{n_0+2},\dots,G_k=E_{n_0}-E_{n_0+k}\),
we know \(G_k\uparrow E_{n_0}-E\). By the first part,
\(\mu(G_k)\uparrow\mu(E_{n_0}-E)\).
\(\mu(E_{n_0}-E)=\mu(E_{n_0})-\mu(E)=\lim_k\mu(E_{n_0}-E_{n_0+k})=\lim_k(\mu(E_{n_0})-\mu(E_{n_0+k}))\)
\item \(\displaystyle\sum_{k=1}^nE_k\subseteq E\), \(\mu(\sum E_k)\le\mu(E)\),
\(\sum(\mu(E_k))\le\mu(E)\), hence \(\sum\mu(E_k)\le\mu(E)\). Let
\(F_n=\displaystyle\sum_{k=1}^nE_k\in\cala\), \(F_n\uparrow E\).
\(\sigma\)-additivity follows.
\item \(F_n=\sum_{k\ge n}E_k\). \(\mu(E)=\mu(\displaystyle\sum_{k=1}^nE_k\cup
   \displaystyle\sum_{k>n}E_k)=\displaystyle\sum_{k=1}^n\mu(E_k)+\mu(F_{n+1})\)
converges to \(\displaystyle\sum_{k\ge1}\mu(E_k)\)
\end{enumerate}
\end{proof}

\begin{examplle}[]
\((a,b],0\le a<b<1\),
\begin{equation*}
\mu((a,b])=\begin{cases}
          b-a&a>0\\
          +\infty&a=0
         \end{cases}
\end{equation*}
Take \(E_n\downarrow\emptyset\), \(\mu\) won't be finite in some cases
\end{examplle}

\begin{theorem}[]
Suppose we have a semi-algebra \(\cals\subseteq\calp(\Omega)\), \(\mu\) is additive,
there is a 
\(\nu\) s.t. \(\nu:\cala(\cals)\to\R_+\cup\{+\infty\}\) 
\begin{enumerate}
\item \(\nu\) is additive
\item \(\nu\)(A)=\(\mu\)(A) for all \(A\in\cals\)
\item If \(\mu_1,\mu_2:\cala(\cals)\to\R_+\cup\{+\infty\}\),\(\forall
   A\in\cals,\mu_1(A)=\mu_2(A)\), then \(\forall
   E\in\cala(\cals),\mu_1(E)=\mu_2(E)\)
\item \(\nu\) is \(\sigma\)-additive
\end{enumerate}
\end{theorem}

\begin{proof}
If \(A\in\cala(\cals)\), then \(A=\displaystyle\sum_{j=1}^nE_j,E_j\in\cals\).
\(\nu(A)=\displaystyle\sum_{j=1}^n\nu(E_j)=\displaystyle\sum_{j=1}^n\mu(E_j)\) 
\begin{enumerate}
\item \(\nu\) is well-defined

If \(A=\displaystyle\sum_{k=1}^mF_k,F_k\in\cals\), \(E_j=E_j\cap
   \displaystyle\sum_{k=1}^mF_k=\displaystyle\sum_{k=1}^m E_j\cap F_k\) and
\(\mu(E_j)=\displaystyle\sum_{k=1}^m \mu(E_j\cap F_k)\). Hence
\(\nu(A)=\displaystyle\sum_{j=1}^n \displaystyle\sum_{k=1}^m\mu(E_j\cap F_k)\)
\item \(\nu\) is additive
\item unique
\item \(A=\displaystyle\sum_{j\ge 1}A_j,A,A_j\in\cala(\cals)\),
\(A=\displaystyle\sum_{j=1}^nE_j,E_j\in\cals\), 
\(A_k=\displaystyle\sum_{l=1}^{m_k}E_{k,l}\in\cals\). 
\begin{align*}
E_j&=E_j\cap A
=E_j\cap(\displaystyle\sum_{k\ge 1}A_k)\\
&=E_j\cap
(\displaystyle\sum_{k\ge 1}\displaystyle\sum_{l=1}^{m_k}E_{k,l})
=\displaystyle\sum_{k\ge 1}\displaystyle\sum_{l=1}^{m_k}E_j\cap E_{k,l}
\end{align*}
\(\mu(E_j)=\displaystyle\sum_{k\ge1}\displaystyle\sum_{l=1}^{m_k}\mu(E_j\cap
   E_{k,l})\) and \(\nu(A)=\displaystyle\sum_{j=1}^n\mu(E_j)=
   \displaystyle\sum_{j=1}^n
   \displaystyle\sum_{k\ge 1}\displaystyle\sum_{l=1}^{m_k}\mu(E_j\cap E_{k,l})\).
\begin{align*}
E_{k,l}&=E_{k,l}\cap A=\displaystyle\sum_{j=1}^nE_{k,l}\cap E_j
\end{align*}
\(\mu(E_{k,l})=\displaystyle\sum_{j=1}^n\mu(E_{k,l}\cap E_j)\). Finish.
\end{enumerate}
\end{proof}
\end{document}