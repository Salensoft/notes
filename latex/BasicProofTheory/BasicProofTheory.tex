% Created 2019-12-27 五 22:22
% Intended LaTeX compiler: pdflatex
\documentclass[11pt]{article}
\usepackage[utf8]{inputenc}
\usepackage[T1]{fontenc}
\usepackage{graphicx}
\usepackage{grffile}
\usepackage{longtable}
\usepackage{wrapfig}
\usepackage{rotating}
\usepackage[normalem]{ulem}
\usepackage{amsmath}
\usepackage{textcomp}
\usepackage{amssymb}
\usepackage{capt-of}
\usepackage{hyperref}
\usepackage{minted}
% TIPS
% \substack{a\\b} for multiple lines text





% pdfplots will load xolor automatically without option
\usepackage[dvipsnames]{xcolor}

\usepackage{forest}
% two-line text in node by [two \\ lines]
% \begin{forest} qtree, [..] \end{forest}
\forestset{
  qtree/.style={
    baseline,
    for tree={
      parent anchor=south,
      child anchor=north,
      align=center,
      inner sep=1pt,
    }}}
%\usepackage{flexisym}
% load order of mathtools and mathabx, otherwise conflict overbrace

\usepackage{mathtools}
%\usepackage{fourier}
\usepackage{pgfplots}
\usepackage{amsthm}
\usepackage{amsmath}
%\usepackage{unicode-math}
%
\usepackage{commath}
%\usepackage{,  , }
\usepackage{amsfonts}
\usepackage{amssymb}
% importing symbols https://tex.stackexchange.com/questions/14386/importing-a-single-symbol-from-a-different-font
%mathabx change every symbol
% use instead stmaryrd
%\usepackage{mathabx}
\usepackage{stmaryrd}
\usepackage{empheq}
\usepackage{tikz}
\usepackage{tikz-cd}
%\usepackage[notextcomp]{stix}
\usetikzlibrary{arrows.meta}
\usepackage[most]{tcolorbox}
%\utilde
%\usepackage{../../latexpackage/undertilde/undertilde}
% left and right superscript and subscript
\usepackage{actuarialsymbol}
\usepackage{threeparttable}
\usepackage{scalerel,stackengine}
\usepackage{stackrel}
% \stackrel[a]{b}{c}
\usepackage{dsfont}
% text font
\usepackage{newpxtext}
%\usepackage{newpxmath}

%\newcounter{dummy} \numberwithin{dummy}{section}
\newtheorem{dummy}{dummy}[section]
\theoremstyle{definition}
\newtheorem{definition}[dummy]{Definition}
\newtheorem{corollary}[dummy]{Corollary}
\newtheorem{lemma}[dummy]{Lemma}
\newtheorem{proposition}[dummy]{Proposition}
\newtheorem{theorem}[dummy]{Theorem}
\theoremstyle{definition}
\newtheorem{example}[dummy]{Example}
\theoremstyle{remark}
\newtheorem*{remark}{Remark}


\newcommand\what[1]{\ThisStyle{%
    \setbox0=\hbox{$\SavedStyle#1$}%
    \stackengine{-1.0\ht0+.5pt}{$\SavedStyle#1$}{%
      \stretchto{\scaleto{\SavedStyle\mkern.15mu\char'136}{2.6\wd0}}{1.4\ht0}%
    }{O}{c}{F}{T}{S}%
  }
}

\newcommand\wtilde[1]{\ThisStyle{%
    \setbox0=\hbox{$\SavedStyle#1$}%
    \stackengine{-.1\LMpt}{$\SavedStyle#1$}{%
      \stretchto{\scaleto{\SavedStyle\mkern.2mu\AC}{.5150\wd0}}{.6\ht0}%
    }{O}{c}{F}{T}{S}%
  }
}

\newcommand\wbar[1]{\ThisStyle{%
    \setbox0=\hbox{$\SavedStyle#1$}%
    \stackengine{.5pt+\LMpt}{$\SavedStyle#1$}{%
      \rule{\wd0}{\dimexpr.3\LMpt+.3pt}%
    }{O}{c}{F}{T}{S}%
  }
}

\newcommand{\bl}[1] {\boldsymbol{#1}}
\newcommand{\Wt}[1] {\stackrel{\sim}{\smash{#1}\rule{0pt}{1.1ex}}}
\newcommand{\wt}[1] {\widetilde{#1}}
\newcommand{\tf}[1] {\textbf{#1}}


%For boxed texts in align, use Aboxed{}
%otherwise use boxed{}

\DeclareMathSymbol{\widehatsym}{\mathord}{largesymbols}{"62}
\newcommand\lowerwidehatsym{%
  \text{\smash{\raisebox{-1.3ex}{%
    $\widehatsym$}}}}
\newcommand\fixwidehat[1]{%
  \mathchoice
    {\accentset{\displaystyle\lowerwidehatsym}{#1}}
    {\accentset{\textstyle\lowerwidehatsym}{#1}}
    {\accentset{\scriptstyle\lowerwidehatsym}{#1}}
    {\accentset{\scriptscriptstyle\lowerwidehatsym}{#1}}
}

\usepackage{graphicx}
    
% text on arrow for xRightarrow
\makeatletter
%\newcommand{\xRightarrow}[2][]{\ext@arrow 0359\Rightarrowfill@{#1}{#2}}
\makeatother


\newcommand{\dom}[1]{%
\mathrm{dom}{(#1)}
}

% Roman numerals
\makeatletter
\newcommand*{\rom}[1]{\expandafter\@slowromancap\romannumeral #1@}
\makeatother

\def \fR {\mathfrak{R}}
\def \bx {\boldsymbol{x}}
\def \bz {\boldsymbol{z}}
\def \ba {\boldsymbol{a}}
\def \bh {\boldsymbol{h}}
\def \bo {\boldsymbol{o}}
\def \bU {\boldsymbol{U}}
\def \bc {\boldsymbol{c}}
\def \bV {\boldsymbol{V}}
\def \bI {\boldsymbol{I}}
\def \bK {\boldsymbol{K}}
\def \bt {\boldsymbol{t}}
\def \bb {\boldsymbol{b}}
\def \bA {\boldsymbol{A}}
\def \bX {\boldsymbol{X}}
\def \bu {\boldsymbol{u}}
\def \bS {\boldsymbol{S}}
\def \bZ {\boldsymbol{Z}}
\def \bz {\boldsymbol{z}}
\def \by {\boldsymbol{y}}
\def \bw {\boldsymbol{w}}
\def \bT {\boldsymbol{T}}
\def \bF {\boldsymbol{F}}
\def \bS {\boldsymbol{S}}
\def \bm {\boldsymbol{m}}
\def \bW {\boldsymbol{W}}
\def \bR {\boldsymbol{R}}
\def \bQ {\boldsymbol{Q}}
\def \bS {\boldsymbol{S}}
\def \bP {\boldsymbol{P}}
\def \bT {\boldsymbol{T}}
\def \bY {\boldsymbol{Y}}
\def \bH {\boldsymbol{H}}
\def \bB {\boldsymbol{B}}
\def \blambda {\boldsymbol{\lambda}}
\def \bPhi {\boldsymbol{\Phi}}
\def \btheta {\boldsymbol{\theta}}
\def \bTheta {\boldsymbol{\Theta}}
\def \bmu {\boldsymbol{\mu}}
\def \bphi {\boldsymbol{\phi}}
\def \bSigma {\boldsymbol{\Sigma}}
\def \lb {\left\{}
\def \rb {\right\}}
\def \la {\langle}
\def \ra {\rangle}
\def \caln {\mathcal{N}}
\def \dissum {\displaystyle\Sigma}
\def \dispro {\displaystyle\prod}
\def \E {\mathbb{E}}
\def \Q {\mathbb{Q}}
\def \N {\mathbb{N}}
\def \V {\mathbb{V}}
\def \R {\mathbb{R}}
\def \P {\mathbb{P}}
\def \A {\mathbb{A}}
\def \Z {\mathbb{Z}}
\def \I {\mathbb{I}}
\def \C {\mathbb{C}}
\def \cala {\mathcal{A}}
\def \calb {\mathcal{B}}
\def \calq {\mathcal{Q}}
\def \calp {\mathcal{P}}
\def \cals {\mathcal{S}}
\def \calg {\mathcal{G}}
\def \caln {\mathcal{N}}
\def \calr {\mathcal{R}}
\def \calm {\mathcal{M}}
\def \calc {\mathcal{C}}
\def \calf {\mathcal{F}}
\def \calk {\mathcal{K}}
\def \call {\mathcal{L}}
\def \calu {\mathcal{U}}
\def \bcup {\bigcup}


\def \uin {\underline{\in}}
\def \oin {\overline{\in}}
\def \uR {\underline{R}}
\def \oR {\overline{R}}
\def \uP {\underline{P}}
\def \oP {\overline{P}}

\def \Ra {\Rightarrow}

\def \e {\enspace}

\def \sgn {\operatorname{sgn}}
\def \gen {\operatorname{gen}}
\def \ker {\operatorname{ker}}
\def \im {\operatorname{im}}

\def \tril {\triangleleft}

% \varprod
\DeclareSymbolFont{largesymbolsA}{U}{txexa}{m}{n}
\DeclareMathSymbol{\varprod}{\mathop}{largesymbolsA}{16}

% \bigtimes
\DeclareFontFamily{U}{mathx}{\hyphenchar\font45}
\DeclareFontShape{U}{mathx}{m}{n}{
      <5> <6> <7> <8> <9> <10>
      <10.95> <12> <14.4> <17.28> <20.74> <24.88>
      mathx10
      }{}
\DeclareSymbolFont{mathx}{U}{mathx}{m}{n}
\DeclareMathSymbol{\bigtimes}{1}{mathx}{"91}
% \odiv
\DeclareFontFamily{U}{matha}{\hyphenchar\font45}
\DeclareFontShape{U}{matha}{m}{n}{
      <5> <6> <7> <8> <9> <10> gen * matha
      <10.95> matha10 <12> <14.4> <17.28> <20.74> <24.88> matha12
      }{}
\DeclareSymbolFont{matha}{U}{matha}{m}{n}
\DeclareMathSymbol{\odiv}         {2}{matha}{"63}


\newcommand\subsetsim{\mathrel{%
  \ooalign{\raise0.2ex\hbox{\scalebox{0.9}{$\subset$}}\cr\hidewidth\raise-0.85ex\hbox{\scalebox{0.9}{$\sim$}}\hidewidth\cr}}}
\newcommand\simsubset{\mathrel{%
  \ooalign{\raise-0.2ex\hbox{\scalebox{0.9}{$\subset$}}\cr\hidewidth\raise0.75ex\hbox{\scalebox{0.9}{$\sim$}}\hidewidth\cr}}}

\newcommand\simsubsetsim{\mathrel{%
  \ooalign{\raise0ex\hbox{\scalebox{0.8}{$\subset$}}\cr\hidewidth\raise1ex\hbox{\scalebox{0.75}{$\sim$}}\hidewidth\cr\raise-0.95ex\hbox{\scalebox{0.8}{$\sim$}}\cr\hidewidth}}}
\newcommand{\stcomp}[1]{{#1}^{\mathsf{c}}}


\author{A. S. Troelstra and H. Schwichtenberg}
\date{\today}
\title{Basic Proof Theory}
\hypersetup{
 pdfauthor={A. S. Troelstra and H. Schwichtenberg},
 pdftitle={Basic Proof Theory},
 pdfkeywords={},
 pdfsubject={},
 pdfcreator={Emacs 26.3 (Org mode 9.3)}, 
 pdflang={English}}
\begin{document}

\maketitle
\tableofcontents \clearpage
\section{Introduction}
\label{sec:org732dd42}
\subsection{Simple type theories}
\label{sec:org7cf9ff4}
\begin{definition}[the set of simple types]
the set of \tf{simple types} \(\calt_\to\) is constructed from a countable set
of \tf{type variables} \(P_0,P_1,\dots\) by means of a type-forming operation
(\tf{function-type constructor}) \(\to\)
\begin{enumerate}
\item type variables belong to \(\calt_\to\)
\item if \(A,B\in\calt_\to\), then \((A\to B)\in\calt_\rightarrow\)
\end{enumerate}


A type of the form \(A\to B\) is called a \tf{function type}
\end{definition}

\begin{definition}[Terms of the simply typed lambda calculus $\lambda_\to$]
All terms appear with a type; for terms of type \(A\) we use \(t^A,s^A,r^A\). The
terms are generated by the following three clauses
\begin{enumerate}
\item For each \(A\in T_\to\) there is a countably infinite supply of variables of
type \(A\); for arbitrary variables of type \(A\) we use
\(u^A,v^A,w^A,x^A,y^A,z^A\)
\item if \(t^{A\to B},s^A\) are terms, then \(\app(t^{A\to B},s^A)^B\) is a term of
type \(B\)
\item if \(t^B\) is a term of type \(B\) and \(x^A\) a variable of type \(A\), then
\((\lambda x^A.t^B)^{A\to B}\)
\end{enumerate}
\end{definition}
For \(\app(t^{A\to B},s^A)^B\) we usually write simply \((t^{A\to B}s^A)^B\)
\begin{definition}[]
The set \(\fv(t)\) of variables free in \(t\) is specified by
\begin{alignat*}{2}
&\fv(x^A)&&:=x^A\\
&\fv(ts)&&:=\fv(t)\cup\fv(s)\\
&\fv(\lambda x.t)&&:=\fv(t)\backslash\{x\}
\end{alignat*}
\end{definition}


\begin{definition}[Substitution]
The operation of substitution of a term \(s\) for a variable \(x\) in a term \(t\)
(notation \(t[x/s]\)) may be defined by recursion on the complexity of \(t\), as
follows
\begin{alignat*}{2}
&x[x/s]&&:=s\\
&y[x/s]&&:=y\text{ for } y\not\equiv x\\
&(t_1t_2)[x/s]&&:=t_1[x/s]t_2[x/s]\\
&(\lambda x.t)[x/s]&&:=\lambda x.t\\
&(\lambda y.t)[x/s]&&=\lambda y.t[x/s]\text{ for } y\not\equiv x; \text{
w.l.o.g. } y\not\in\fv(s)
\end{alignat*}
\end{definition}

\begin{lemma}[Substitution lemma]
If \(x\not\equiv y, x\not\in\fv(t_2)\), then
\begin{equation*}
t[x/t_1][y/t_2]\equiv t[y/t_2][x/t_1[y/t_2]]
\end{equation*}
\end{lemma}

\begin{definition}[Conversion, reduction, normal form]
Let \(\mathsf{T}\) be a set of terms, and let conv be a binary relation on
\(\mathsf{T}\), written in infix notation: \(t\) conv \(s\). If \(t\) conv \(s\), we
say that \(t\) \tf{converts to} \(s\); \(t\) is called a \tf{redex} or
\tf{convertible} term and \(s\) the \tf{conversum} of \(t\). The replacement of a
redex by its conversum is called a \tf{conversion}. We write \(t\succ_1 s\)
(\(t\) \tf{reduces in one step to } \(s\)) if \(s\) is obtained from \(t\) by
replacement of a redex \(t'\) of \(t\) by a conversum \(t''\) of \(t'\). The relation
\(\succ\) (\tf{properly reduces to}) is the transitive closure of \(\succ_1\) and
\(\succeq\) (\tf{reduces to}) is the reflexive and transitive closure of
\(\succ_1\). The relation \(\succeq\) is said to be the notion of reduction
\tf{generated} by cont.

With the notion of reduction generated by cony we associate a relation on
\(\mathsf{T}\) called \textbf{conversion equality}: \(t=_{\conv}s\) (\(t\) is equal by
conversion to \(s\)) if there 
is a sequence \(t_0,\dots,t_n\) with \(t_0\equiv t,t_n\equiv s\), and \(t_i\preceq
   t_{i+1}\) or \(t_i\succeq t_{i+1}\) for each
\(i, 0\le i < n\). The subscript "conv" is usually omitted when clear from the
context

A term \(t\) is in \textbf{normal form}, or \(t\) is \textbf{normal}, if \(t\) does not contain a redex. \(t\)
\textbf{has a normal form} if there is a normal \(s\) such that \(t\succeq s\).

A \textbf{reduction sequence} is a (finite or infinite) sequence of pairs
\((t_0,\delta_0),(t_1,\delta_1),\dots\) 
with \(\delta_i\) an (occurrence of a) redex in \(t_i\) and \(t_i\succ t_{i+1}\) by
conversion 
of \(\delta_i\), for all \(i\). This may be written as
\begin{equation*}
t_0\overset{\delta_0}{\succ}_1 t_1\overset{\delta_1}{\succ}_1 t_2
\overset{\delta_2}{\succ}_1\dots
\end{equation*}
We often omit the \(\delta_i\), simply writing \(t_0\succ_1 t_1\succ_1 t_2\)

Finite reduction sequences are partially ordered under the initial part
relation ("sequence \(\sigma\) is an initial part of sequence \(\tau\)"); the collection of
finite 
reduction sequences starting from a term \(g\) forms a tree, the \textbf{reduction tree}
of \(t\). The branches of this tree may be identified with the collection of all
infinite and all terminating finite reduction sequences.


A term is \textbf{strongly normalizing} (is SN) if its reduction tree is finite
\end{definition}

\(\beta\)-conversion:
\begin{equation*}
(\lambda x^A.t^B) s^A\e\cont_\beta\e t^B[x^A/s^A]
\end{equation*}
\(\eta\)-conversion:
\begin{equation*}
\lambda x^A.tx\e\cont_\eta\e  t\quad(x\not\in\fv(t))
\end{equation*}
\(\beta \eta\)-conversion \(\cont_{\beta\eta}\) is \(\cont_\beta\cup\cont_\eta\)

\begin{definition}[]
A relation \(R\) is said to be \textbf{confluent}, or to have the \textbf{Church-Rosser property}
(CR), if whenever \(t_0 Rt_1\) and \(t_0Rt_2\), then there is a \(t_3\) s.t.
\(t_1Rt_3\) and \(t_2Rt_3\). A relation \(R\) is said to be \textbf{weakly confluent} or to
have the \textbf{weak Church-Rosser property} if whenever \(t_0Rt_1,t_0Rt_2\) there is a
\(t_3\) s.t. \(t_1R^*t_3\) and \(t_2R^* t_3\) where \(R^*\) is the reflexive and
transitive closure of \(T\)
\end{definition}

\begin{theorem}[]
For a confluent reduction relation \(\succeq\) the normal forms of terms are
unique. Furthermore, if \(\succeq\) is a confluent reduction relation we have
\(t=t'\) iff there is a term \(t''\) s.t. \(t\succ t''\) and \(t'\succ t''\)
\end{theorem}

\begin{theorem}[Newman's lemma]
Let \(\succeq\) be the transitive and reflexive closure of \(\succ_1\), and let
\(\succ_1\) be weakly confluent. Then the normal form w.r.t. \(\succ_1\) of a
strongly normalizing \(t\) is unique. Moreover, if all terms are strongly
normalizing w.r.t. \(\succ_1\) then the relation \(\succeq\) is confluent.
\end{theorem}

\begin{proof}
Assume WCR, and let write \(s\in UN\) to indicate that \(s\) has a unique normal
form. Assume \(t\in SN, t\not\in UN\). Then there are two reduction sequences
\(t\succ_1 t_1'\dots\succ_1 t'\) and \(t\succ_1 T_1''\succ_1\dots\succ_1 t''\) with
\(t'\not\equiv t''\). Then either \(t'_1=t''_1\) or \(t'_1\neq t_1''\)

In the first case we can take \(t_1:=t_1'=t_1''\). In the second case, by WCR
we can find a \(t^*\) s.t. \(t^*\prec t_1',t_1''\); \(t\in SN\) hence \(t^*\succ
   t'''\) for some normal \(t'''\). Since \(t'\neq t'''\) or \(t''\neq t'''\), either
\(t_1'\not\in UN\) or \(t_1''\not\in UN\); so take \(t_1:=t_1'\) if \(t'\neq t'''\),
\(t_1:=t_1''\) otherwise.

Hence we can always find a \(t_1\prec t\) with \(t_1\not\in UN\) and get an
infinite sequence contradicting the SN of \(t\)
\end{proof}

\begin{definition}[]
The \textbf{simple typed lambda calculus} \(\lambda_\to\) is the calculus of
\(\beta\)-reduction and \(\beta\)-equality on the set of terms of \(\lambda_\to\).
\(\lambda_\to\) has the term system as described with the following axioms and
rules for \(\prec\) (\(\prec_\beta\)) and \(=\) (is \(=_\beta\))
\begin{align*}
&t\succeq t\quad(\lambda x^A.t^B)s^A\succeq t^B[x^A/s^A]\\
&\frac{t\succeq s}{rt\succeq rs}\quad
\frac{t\succ s}{tr\succ sr}\quad
\frac{t\succeq s}{\lambda x.t\succeq\lambda x.s}\quad
\frac{t\succeq s\quad s\succeq r}{t\succeq r}\\
&\frac{t\succeq s}{t=s}\quad\frac{t=s}{s=t}\quad
\frac{t=s\quad s=r}{t=r}
\end{align*}
The \textbf{extensional simple typed lambda calculus} \(\lambda\eta_\to\) is the
calculus of \(\beta \eta\)-reduction and \(\beta \eta\)-equality and the ser of terms
of \(\lambda_\to\); in addition there is the axiom
\begin{equation*}
\lambda x.tx\succeq t\quad(x\not\in\fv(t))
\end{equation*}
\end{definition}

\begin{lemma}[Substitutivity of $\succ_\beta$ and $\succ_{\beta\eta}$]
For \(\succeq\) either \(\succeq_\beta\) or \(\succ_{\beta\eta}\) we have
\begin{equation*}
\text{if } s\succeq s' \text{ then } s[y/s'']\succeq s'[y/s'']
\end{equation*}
\end{lemma}
\begin{proof}
By induction on the depth of a proof of \(s\succeq s'\). It suffices to check
the crucial basis step, where \(s\) is \((\lambda x.t)t'\) and \(s'\) is \(t[x/t']\).
\begin{equation*}
(\lambda x.t)t'[y/s'']=(\lambda x.(t[y/s''])t'[y/s''])=
t[y/s''][x/t'[y/s'']]=t[x/t'][y/s'']
\end{equation*}
\end{proof}

\begin{proposition}[]
\(\succ_{\beta,1}\) and \(\succ_{\beta\eta,1}\) are weakly confluent
\end{proposition}
\begin{proof}
If the conversions leading from \(t\) to \(t'\) and \(t\) to \(t''\) concern disjoint
redexes, then \(t'''\) is simply obtained by converting both redexes

If \(t\equiv\dots(\lambda x.s)s'\dots\), \(t'\equiv\dots s[x/s']\dots\) and
\(t''\equiv\dots(\lambda x.s)s''\dots\), \(s'\succ_1 s''\), then \(t'''\equiv\dots
   s[x/s'']\dots\)

If \(t\equiv\dots(\lambda x.s)s'\dots\), \(t'\equiv\dots s[x/s']\dots\) and
\(t''\equiv\dots(\lambda x.s'')s'\dots\), \(s\succ_1 s''\), then \(t'''\equiv\dots
   s''[x/s']\dots\)

If \(t\equiv\dots(\lambda x.sx)s'\), \(t'=\dots (sx)[x/s']\dots\),
\(t''\equal\dots ss'\dots\)
\end{proof}

\begin{theorem}[]
The terms of \(\lambda_\to,\lambda\beta_\to\) are SN for \(\succeq_\beta\) and
\(\succeq_{\beta\eta}\) respectively, then hence the \(\beta\)- and
\(\beta \eta\)-normal forms are unique
\end{theorem}


\begin{definition}[]
\(\succeq_p\) on \(\lambda_\to\) is generated by the axiom and rules
\begin{alignat*}{2}
&(\text{id})x\succeq_p x\\
&(\lambda\text{mon})\frac{t\succeq_p t'}{\lambda x.t\succeq_p \lambda x.t'}&&
(\text{appmon})\frac{t\succeq_p t'\quad s\succeq_p s'}{ts\succeq_p t's'}\\
&(\beta\text{par})\frac{t\succeq_p t'\quad s\succeq_ps'}{(\lambda x.t)s\succeq_pt'[x/s']}
&&(\eta\text{par})\frac{t\succeq_p t'}{\lambda x.tx\succeq_pt'}
(x\not\in\fv(t))
\end{alignat*}
\end{definition}

\begin{lemma}[Substitutivity of $\succ_p$]
If \(t\succ_p t',s\succ_p s'\), then \(t[x/s]\succ_p t'[x/s']\)
\end{lemma}

\begin{proof}
By induction on \(t\). 

\begin{enumerate}
\item \(t\equiv(\lambda y.t_1)t_2\), then
\begin{align*}
&t\succeq_p t_1'[y/t_2']\\
&t[x/s]\equiv(\lambda y.t_1[x/s])t_2[x/s]\succeq_p
t_1'[x/s'][y/t_2'[x/s']]\succeq_p
t_1'[y/t_2'][x/s']
\end{align*}
\end{enumerate}
\end{proof}

\begin{lemma}[]
\(\succeq_p\) is confluent
\end{lemma}

\begin{proof}
Induction on \(t\)
\end{proof}

\begin{theorem}[]
\(\beta\)- and \(\beta \eta\)-reduction are confluent
\end{theorem}
\begin{proof}
The reflexive closure of \(\succ_1\) for \(\beta \eta\)-reduction is contained in
\(\succeq_p\), and \(\succeq\) is therefore the transitive closure of
\(\succeq_p\). Write \(t\succeq_{p,n}t'\) if there is a chain
\(t\equiv t_0\succeq_p t_1\succeq_p\dots\succeq_pt_n\equiv t'\). Then we show
by induction on \(n+m\) using the preceding lemma, that if
\(t\succeq_{p,n}t',t\succeq_{p,m}t''\) then there is a \(t'''\) s.t.
\(t'\succeq_{p,m}t''',t''\succeq_{p,n}t'''\) 
\begin{center}
\begin{tikzcd}
t \arrow[r,"\alpha-1"] \arrow[rd,"n+m+1-\alpha"{left}]&
t_0' \arrow[r,"1"] \arrow[rd,"n+m+1-\alpha"]&
t' \arrow[rd]\\
&t'' \arrow[r,"\alpha-1"] &
t_0''' \arrow[r]&t'''
\end{tikzcd}
\end{center}
\end{proof}

\begin{definition}[Terms of typed combinatory logic $\cl_\to$]
The terms are inductive defined as in the case of \(\lambda_\to\), but now with
the clauses
\begin{enumerate}
\item For each \(A\in\calt_\to\) there is a countably infinite supply of variables
of type \(A\); for arbitrary variables of type \(A\) we use
\(u^A,v^A,w^A,x^A,y^A,z^A\)
\item for each \(A,B,C\in\calt\) there are constant terms
\begin{align*}
&\bk^{A,B}\in A\to(B\to A)\\
&\bs^{A,B,C}\in (A\to(B\to C))\to((A\to B)\to(A\to C))
\end{align*}
\item if \(t^{A,B},s^A\) are terms, then so is \(t^{A,B}s\)
\end{enumerate}


\(\fv(\bk)=\fv(\bs)=\emptyset\)
\end{definition}

\begin{definition}[]
The \textbf{weak reduction} relation \(\succeq_w\) on the terms of \(\cl_\to\) is
generated by a conversion relation \(\cont_w\) consisting of the following
pairs
\begin{equation*}
\bk^{A,B}x^Ay^B\e\cont_w\e x,\quad\bs^{A,B,C}x^{A\to(B\to C)}y^{A\to B}z^A
\e\cont_w\e xz(yz)
\end{equation*}

In otherwords, \(\cl_\to\) is the term system defined above with the following
axioms and rules for \(\succeq_w\) and \(=_w\)
\begin{alignat*}{3}
&t\succeq t&&\bk xy\succeq x\quad&&\bs xyz\succeq xz(yz)\\
&\frac{t\succeq s}{rt\succeq rs}\quad&&\frac{t\succeq s}{tr\succeq sr}&&
\frac{t\succeq s\quad s\succeq r}{t\succeq r}\\
&\frac{t\succeq s}{t=s}&&\frac{t=s}{s=t}&&\frac{t=s\quad s=r}{t=r}
\end{alignat*}
\end{definition}

\begin{theorem}[]
The weak reduction relation in \(\cl_\to\), is confluent and
strongly normalizing, so normal forms are unique.
\end{theorem}

\begin{theorem}[]
To each term \(t\) in \(\cl_\to\), there is another term \(\lambda^*x^A.t\) such
that
\begin{enumerate}
\item \(x^A\not\in\fv(\lambda^*x^A.t)\)
\item \((\lambda^*x^A.t)s^A\succ_wt[x^A/s^A]\)
\end{enumerate}
\end{theorem}
\begin{proof}
\begin{align*}
&\lambda^*x^A.x:=\bs^{A,A\to A,A}\bk^{A,A\to A}\bk^{A,A}\\
&\lambda^*x^A.y^B:=\bk^{B,A}y^B\text{ for }y\not\equiv x\\
&\lambda^*x^A.t_1^{B\to C}t_2^B:=\bs^{A,B,C}(\lambda^*x.t_1)(\lambda^*x.t_2)
\end{align*}
\end{proof}

\begin{corollary}[]
\(\cl_\to\) is \textbf{combinatorially complete}, i.e. for every applicative
combination \(t\) of \(\bk,\bs\) and variables \(x_1,x_2,\dots x_n\) there is a
closed term \(s\) s.t. in \(\cl_\to\vdash sx_1\dots x_n=_w t\), in fact even
\(\cl_\to\vdash sx_1\dots x_n\succeq_w t\)
\end{corollary}

\begin{remark}
Note that: it's not true that if \(t=t'\) then \(\lambda^*x.t=\lambda^*x.t'\). 
\(\bk x\bk=x\) but \(\lambda^*x.\bk x\bk=\bs(\bs(\bk\bk)(\bs\bk\bk))(\bk\bk)\),
\(\lambda^*x.x=\bs\bk\bk\)
\end{remark}

\begin{definition}[]
The \textbf{Church numerals} of type \(A\) are \(\beta\)-normal terms \(\bar{n}_A\) of type 
\((A\to A)\to(A\to A), n\in\N\), defined by
\begin{equation*}
\bar{n}_A:=\lambda f^{A\to A}\lambda x^A.f^n(x)
\end{equation*}
where \(f^0(x):=x,f^{n+1}(x):=f(f^n(x))\). \(N_A=\{\bar{n}_A\}\)
\end{definition}
N.B. If we want to use \(\beta \eta\)-normal terms, we must use \(\lambda f^{A\to
   A}.f\) instead of \(\lambda fx.fx\) for \(\bar{1}_A\)

\begin{definition}[]
A function ff\(f:\N^k\to\N\) is said to be \textbf{A-representable} if there is a term \(F\)
of \(\lambda_\to\) s.t. (abbreviating \(\bar{n}_A\) as \(\bar{n}\))
\begin{equation*}
F\bar{n}_1\dots\bar{n}_k=\bar{f(n_1,\dots,n_k)}
\end{equation*}
for all \(n_1,\dots,n_k\in\N,\bar{n}_i=(\bar{n}_i)_A\)
\end{definition}


\begin{definition}[]
\textbf{Polynomials}, \textbf{extended polynomials}
\begin{enumerate}
\item The \(n\)-argument \textbf{projections} \(\bp_i^n\) are given by
\(\bp_i^n(x_1,\dots,x_n)=x_i\), the unary constant functions \(\bc_m\) by
\(\bc_m(x)=m\), and \(\sg\), \(\bar{sg}\) are unary functions which satisfy
\(\sg(S_n)=1\), \(\sg(0)=0\), where \(S\) is the successor function.
\item The \(n\)-argument function \(f\) is the \textbf{composition} of \(m\)-argument \(g\),
\(n\)-argument \(h_1,\dots,h_m\) if \(f\) satisfies
\(f(\bar{x})=g(h_1(\bar{x}),\dots,h_m(\bar{x}))\)
\item The \textbf{polynomials} in \(n\) variables are generated from \(\bp_i^n,\bc_m\),
addition and multiplication by closure under composition. The \textbf{extended
polynomials} are generated from \(\bp_i^n,\bc_m,\sg,\bar{sg}\), addition and
multiplication by closure under proposition
\end{enumerate}
\end{definition}

\begin{exercise}
Show that all terms in \(\beta\)-normal form of type \((P\to P)\to(P\to P)\), \(P\) a
propositional variable, are either of the form \(\bar{n}_P\) or of the form
\(\lambda f^{P\to P}.f\)
\end{exercise}
\begin{proof}
\begin{enumerate}
\item \(\lambda f^{P\to P}.g^{P\to P}\), if \(g\neq f\), then \(g\) is of the form
\(\lambda x^P.y^P\) and hence \(\lambda f^{P\to P}\lambda x^P.y^P\)
\end{enumerate}
\end{proof}
\end{document}