% Created 2020-06-16 二 10:34
% Intended LaTeX compiler: pdflatex
\documentclass[11pt]{article}
\usepackage[utf8]{inputenc}
\usepackage[T1]{fontenc}
\usepackage{graphicx}
\usepackage{grffile}
\usepackage{longtable}
\usepackage{wrapfig}
\usepackage{rotating}
\usepackage[normalem]{ulem}
\usepackage{amsmath}
\usepackage{textcomp}
\usepackage{amssymb}
\usepackage{capt-of}
\usepackage{hyperref}
\usepackage{minted}
% TIPS
% \substack{a\\b} for multiple lines text





% pdfplots will load xolor automatically without option
\usepackage[dvipsnames]{xcolor}

\usepackage{forest}
% two-line text in node by [two \\ lines]
% \begin{forest} qtree, [..] \end{forest}
\forestset{
  qtree/.style={
    baseline,
    for tree={
      parent anchor=south,
      child anchor=north,
      align=center,
      inner sep=1pt,
    }}}
%\usepackage{flexisym}
% load order of mathtools and mathabx, otherwise conflict overbrace

\usepackage{mathtools}
%\usepackage{fourier}
\usepackage{pgfplots}
\usepackage{amsthm}
\usepackage{amsmath}
%\usepackage{unicode-math}
%
\usepackage{commath}
%\usepackage{,  , }
\usepackage{amsfonts}
\usepackage{amssymb}
% importing symbols https://tex.stackexchange.com/questions/14386/importing-a-single-symbol-from-a-different-font
%mathabx change every symbol
% use instead stmaryrd
%\usepackage{mathabx}
\usepackage{stmaryrd}
\usepackage{empheq}
\usepackage{tikz}
\usepackage{tikz-cd}
%\usepackage[notextcomp]{stix}
\usetikzlibrary{arrows.meta}
\usepackage[most]{tcolorbox}
%\utilde
%\usepackage{../../latexpackage/undertilde/undertilde}
% left and right superscript and subscript
\usepackage{actuarialsymbol}
\usepackage{threeparttable}
\usepackage{scalerel,stackengine}
\usepackage{stackrel}
% \stackrel[a]{b}{c}
\usepackage{dsfont}
% text font
\usepackage{newpxtext}
%\usepackage{newpxmath}

%\newcounter{dummy} \numberwithin{dummy}{section}
\newtheorem{dummy}{dummy}[section]
\theoremstyle{definition}
\newtheorem{definition}[dummy]{Definition}
\newtheorem{corollary}[dummy]{Corollary}
\newtheorem{lemma}[dummy]{Lemma}
\newtheorem{proposition}[dummy]{Proposition}
\newtheorem{theorem}[dummy]{Theorem}
\theoremstyle{definition}
\newtheorem{example}[dummy]{Example}
\theoremstyle{remark}
\newtheorem*{remark}{Remark}


\newcommand\what[1]{\ThisStyle{%
    \setbox0=\hbox{$\SavedStyle#1$}%
    \stackengine{-1.0\ht0+.5pt}{$\SavedStyle#1$}{%
      \stretchto{\scaleto{\SavedStyle\mkern.15mu\char'136}{2.6\wd0}}{1.4\ht0}%
    }{O}{c}{F}{T}{S}%
  }
}

\newcommand\wtilde[1]{\ThisStyle{%
    \setbox0=\hbox{$\SavedStyle#1$}%
    \stackengine{-.1\LMpt}{$\SavedStyle#1$}{%
      \stretchto{\scaleto{\SavedStyle\mkern.2mu\AC}{.5150\wd0}}{.6\ht0}%
    }{O}{c}{F}{T}{S}%
  }
}

\newcommand\wbar[1]{\ThisStyle{%
    \setbox0=\hbox{$\SavedStyle#1$}%
    \stackengine{.5pt+\LMpt}{$\SavedStyle#1$}{%
      \rule{\wd0}{\dimexpr.3\LMpt+.3pt}%
    }{O}{c}{F}{T}{S}%
  }
}

\newcommand{\bl}[1] {\boldsymbol{#1}}
\newcommand{\Wt}[1] {\stackrel{\sim}{\smash{#1}\rule{0pt}{1.1ex}}}
\newcommand{\wt}[1] {\widetilde{#1}}
\newcommand{\tf}[1] {\textbf{#1}}


%For boxed texts in align, use Aboxed{}
%otherwise use boxed{}

\DeclareMathSymbol{\widehatsym}{\mathord}{largesymbols}{"62}
\newcommand\lowerwidehatsym{%
  \text{\smash{\raisebox{-1.3ex}{%
    $\widehatsym$}}}}
\newcommand\fixwidehat[1]{%
  \mathchoice
    {\accentset{\displaystyle\lowerwidehatsym}{#1}}
    {\accentset{\textstyle\lowerwidehatsym}{#1}}
    {\accentset{\scriptstyle\lowerwidehatsym}{#1}}
    {\accentset{\scriptscriptstyle\lowerwidehatsym}{#1}}
}

\usepackage{graphicx}
    
% text on arrow for xRightarrow
\makeatletter
%\newcommand{\xRightarrow}[2][]{\ext@arrow 0359\Rightarrowfill@{#1}{#2}}
\makeatother


\newcommand{\dom}[1]{%
\mathrm{dom}{(#1)}
}

% Roman numerals
\makeatletter
\newcommand*{\rom}[1]{\expandafter\@slowromancap\romannumeral #1@}
\makeatother

\def \fR {\mathfrak{R}}
\def \bx {\boldsymbol{x}}
\def \bz {\boldsymbol{z}}
\def \ba {\boldsymbol{a}}
\def \bh {\boldsymbol{h}}
\def \bo {\boldsymbol{o}}
\def \bU {\boldsymbol{U}}
\def \bc {\boldsymbol{c}}
\def \bV {\boldsymbol{V}}
\def \bI {\boldsymbol{I}}
\def \bK {\boldsymbol{K}}
\def \bt {\boldsymbol{t}}
\def \bb {\boldsymbol{b}}
\def \bA {\boldsymbol{A}}
\def \bX {\boldsymbol{X}}
\def \bu {\boldsymbol{u}}
\def \bS {\boldsymbol{S}}
\def \bZ {\boldsymbol{Z}}
\def \bz {\boldsymbol{z}}
\def \by {\boldsymbol{y}}
\def \bw {\boldsymbol{w}}
\def \bT {\boldsymbol{T}}
\def \bF {\boldsymbol{F}}
\def \bS {\boldsymbol{S}}
\def \bm {\boldsymbol{m}}
\def \bW {\boldsymbol{W}}
\def \bR {\boldsymbol{R}}
\def \bQ {\boldsymbol{Q}}
\def \bS {\boldsymbol{S}}
\def \bP {\boldsymbol{P}}
\def \bT {\boldsymbol{T}}
\def \bY {\boldsymbol{Y}}
\def \bH {\boldsymbol{H}}
\def \bB {\boldsymbol{B}}
\def \blambda {\boldsymbol{\lambda}}
\def \bPhi {\boldsymbol{\Phi}}
\def \btheta {\boldsymbol{\theta}}
\def \bTheta {\boldsymbol{\Theta}}
\def \bmu {\boldsymbol{\mu}}
\def \bphi {\boldsymbol{\phi}}
\def \bSigma {\boldsymbol{\Sigma}}
\def \lb {\left\{}
\def \rb {\right\}}
\def \la {\langle}
\def \ra {\rangle}
\def \caln {\mathcal{N}}
\def \dissum {\displaystyle\Sigma}
\def \dispro {\displaystyle\prod}
\def \E {\mathbb{E}}
\def \Q {\mathbb{Q}}
\def \N {\mathbb{N}}
\def \V {\mathbb{V}}
\def \R {\mathbb{R}}
\def \P {\mathbb{P}}
\def \A {\mathbb{A}}
\def \Z {\mathbb{Z}}
\def \I {\mathbb{I}}
\def \C {\mathbb{C}}
\def \cala {\mathcal{A}}
\def \calb {\mathcal{B}}
\def \calq {\mathcal{Q}}
\def \calp {\mathcal{P}}
\def \cals {\mathcal{S}}
\def \calg {\mathcal{G}}
\def \caln {\mathcal{N}}
\def \calr {\mathcal{R}}
\def \calm {\mathcal{M}}
\def \calc {\mathcal{C}}
\def \calf {\mathcal{F}}
\def \calk {\mathcal{K}}
\def \call {\mathcal{L}}
\def \calu {\mathcal{U}}
\def \bcup {\bigcup}


\def \uin {\underline{\in}}
\def \oin {\overline{\in}}
\def \uR {\underline{R}}
\def \oR {\overline{R}}
\def \uP {\underline{P}}
\def \oP {\overline{P}}

\def \Ra {\Rightarrow}

\def \e {\enspace}

\def \sgn {\operatorname{sgn}}
\def \gen {\operatorname{gen}}
\def \ker {\operatorname{ker}}
\def \im {\operatorname{im}}

\def \tril {\triangleleft}

% \varprod
\DeclareSymbolFont{largesymbolsA}{U}{txexa}{m}{n}
\DeclareMathSymbol{\varprod}{\mathop}{largesymbolsA}{16}

% \bigtimes
\DeclareFontFamily{U}{mathx}{\hyphenchar\font45}
\DeclareFontShape{U}{mathx}{m}{n}{
      <5> <6> <7> <8> <9> <10>
      <10.95> <12> <14.4> <17.28> <20.74> <24.88>
      mathx10
      }{}
\DeclareSymbolFont{mathx}{U}{mathx}{m}{n}
\DeclareMathSymbol{\bigtimes}{1}{mathx}{"91}
% \odiv
\DeclareFontFamily{U}{matha}{\hyphenchar\font45}
\DeclareFontShape{U}{matha}{m}{n}{
      <5> <6> <7> <8> <9> <10> gen * matha
      <10.95> matha10 <12> <14.4> <17.28> <20.74> <24.88> matha12
      }{}
\DeclareSymbolFont{matha}{U}{matha}{m}{n}
\DeclareMathSymbol{\odiv}         {2}{matha}{"63}


\newcommand\subsetsim{\mathrel{%
  \ooalign{\raise0.2ex\hbox{\scalebox{0.9}{$\subset$}}\cr\hidewidth\raise-0.85ex\hbox{\scalebox{0.9}{$\sim$}}\hidewidth\cr}}}
\newcommand\simsubset{\mathrel{%
  \ooalign{\raise-0.2ex\hbox{\scalebox{0.9}{$\subset$}}\cr\hidewidth\raise0.75ex\hbox{\scalebox{0.9}{$\sim$}}\hidewidth\cr}}}

\newcommand\simsubsetsim{\mathrel{%
  \ooalign{\raise0ex\hbox{\scalebox{0.8}{$\subset$}}\cr\hidewidth\raise1ex\hbox{\scalebox{0.75}{$\sim$}}\hidewidth\cr\raise-0.95ex\hbox{\scalebox{0.8}{$\sim$}}\cr\hidewidth}}}
\newcommand{\stcomp}[1]{{#1}^{\mathsf{c}}}


\author{Heinz-Dieter Ebbinghaus\\Jörg Flum}
\date{\today}
\title{Finite Model Theory}
\hypersetup{
 pdfauthor={Heinz-Dieter Ebbinghaus\\Jörg Flum},
 pdftitle={Finite Model Theory},
 pdfkeywords={},
 pdfsubject={},
 pdfcreator={Emacs 26.3 (Org mode 9.4)}, 
 pdflang={English}}
\begin{document}

\maketitle
\tableofcontents \clearpage

\section{Preliminaries}
\label{sec:org5835688}

\subsection{Structures}
\label{sec:orgeb2f0cf}
\textbf{Vocabularies} are finite sets that consist of \textbf{relation symbols} and
\textbf{constant symbols}. We denote vocabularies by \(\tau\), \(\sigma\),\(\dots\). A
*vocabulary is *relational if it does not contain constants.


\subsection{Graph}
\label{sec:orge75c1a5}
Let \(\tau=\{E\}\) with a binary relation symbol \(E\). A \textbf{graph} (or
\textbf{undirected graph}) is a \(\tau\)-structure \(\calg=(G,E^G)\) satisfying
\begin{enumerate}
\item for all \(a\in G\): not \(E^Gaa\)
\item for all \(a,b\in G\): if \(E^Gab\) then \(E^Gba\)
\end{enumerate}


By GRAPH we denote the class of \textbf{finite} graphs. If only (1) is required, we
speak of a \textbf{digraph}

A subset \(X\) of the universe of a graph \(\calg\) is a \textbf{clique}, if
\(E^Gab\) for all \(a,b\in X\), \(a\neq b\)


Let \(\calg\) be a digraph. If \(n\ge1\) and
\begin{equation*}
E^Ga_0a_1,E^Ga_1a_2,\dots,E^Ga_{n-1}a_n
\end{equation*}
then \(a_0,\dots,a_n\) is a \textbf{path} from \(a_0\) to \(a_n\) of \textbf{length} \(n\).
If \(a_0=a_n\), then \(a_0,\dots,a_n\) is a \textbf{cycle}. A path \(a_0,\dots,a_n\)
is \textbf{Hamiltonian} if \(G=\{a_0,\dots,a_n\}\) and \(a_i\neq a_j\) for
\(i\neq j\). If, in addition, \(E^Ga_na_0\) we speak of a \textbf{Hamiltonian
circuit}

Let \(\calg\) be a graph. Write \(a\sim b\) if \(a=b\) or if there is a path
from \(a\) to \(b\). The equivalence class of \(a\) is called the
\textbf{(connected) component} of \(a\). Let CONN be the class of finite connected
graphs

Denote by \(d(a,b)\) the length of a shortest path from \(a\) to \(b\); more
precisely, define the \textbf{distance function} \(d:G\times G\to\N\cup\{\infty\}\)
by
\begin{equation*}
d(a,b)=\infty\text{ iff }a\not\sim b,\quad d(a,b)=0\text{ iff }a=b
\end{equation*}
and otherwise
\begin{equation*}
d(a,b)=\min\{n\ge1\mid\text{there is a path from $a$ to $b$ of length $n$}\}
\end{equation*}

We give the following definitions only for \textbf{finite} digraphs. A vertex \(b\)
is a successor of a vertex \(a\) if \(E^Gab\). The \textbf{in-degree} of a vertex is
the number of its predecessors, the \textbf{out-degree} the number of its
successors.

A \textbf{root} of a digraph is a vertex with in-degree 0 and a \textbf{leaf} a vertex with
out-degree 0.

A \textbf{forest} is an acyclic digraph where each vertex has in-degree at most 1. A
\textbf{tree} is a forest with connected underlying graph. Let TREE be the class of
finite trees.

\subsection{Syntax and Semantics of First-Order Logic}
\label{sec:org16bb348}
Denote \(\FO[\tau]\) the set of formulas of first-order logic of vocabulary
\(\tau\).

When only taking into consideration finite structures, we use the notation
\(\Phi\models_{\fin}\psi\)

The \textbf{quantifier rank} \(\qr(\varphi)\) of a formula \(\varphi\) is the maximum
number of nested quantifiers occurring in it

It can be shown that every first-order formula is logically equivlent to a
formula in prenex normal form, that is, to a formula of the form
\(Q_1x_1,\dots,Q_sx_s\psi\) where \(Q_1,\dots,Q_s\in\{\forall,\exists\}\),
and where \(\psi\) is quantifier-free. Such a formula is called \(\Sigma_n\) if
the string of \(n\) consecutive blocks, where in each block all quantifiers
all of the same type, adjacent blocks contain quantifiers of different type,
and the first block is existential. \(\Pi_n\) formulas are defined in the
same way but now we require that the first block consists of universal
quantifiers. A \(\Delta_n\)-formula is a formula logically equivalent to
both a \(\Sigma_n\)-formula and a \(\Pi_n\)-formula

Given a formula \(\varphi(x,\overbar{z})\) and \(n\ge1\),
\begin{equation*}
\exists^{\ge n}x\varphi(x,\overbar{z})
\end{equation*}
is an abbreviation for the formula
\begin{equation*}
\exists x_1,\dots\exists x_n(
\bigwedge_{1\le i\le n}\varphi(x_i,\overbar{z})\wedge
\bigwedge_{1\le i<j\le n}\neg x_i=x_j)
\end{equation*}

We set
\begin{equation*}
\varphi_{\ge n}:=\exists^{\ge n}x\;x=x
\end{equation*}
Clearly
\begin{equation*}
\cala\models\varphi_{\ge n}\quad\text{ iff }\quad\norm{A}\ge n
\end{equation*}

\subsection{Some Classical Results of First-Order Logic}
\label{sec:org14a6b13}
\begin{theorem}[]
\label{thm1.0.2}
The set of logically valid sentences of first-order logic is r.e.
\end{theorem}
\begin{theorem}[Compactness Theorem]
\label{thm1.0.3}
\(\Phi\) is satisfiable iff every finite subset of \(\Phi\) is satisfiable
\end{theorem}

Neither Theorem \ref{thm1.0.2} nor \ref{thm1.0.3} remain valid if one only
considers finite structures. A counterexample for the Compactness Theorem is
given by the set \(\Phi_\infty:=\{\varphi_{\ge n}\mid n\ge1\}\): Each finite
subset of \(\Phi_\infty\) has a finite model, but \(\Phi_\infty\) has no
finite model

The failure of Theorem \ref{thm1.0.2} is documented by
\begin{theorem}[Trahtenbrot's Theorem]
The set of sentences of first-order logic valid in all finite structures is
not r.e.
\end{theorem}

\begin{lemma}[]
Let \(\varphi\in\FO[\tau]\) and for \(i\in I\), let
\(\Phi^i\subseteq\FO[\tau]\). Assume that
\begin{equation*}
\models\varphi\leftrightarrow\bigvee_{i\in I}\bigwedge\Phi^i
\end{equation*}
Then there is a finite \(I_0\subseteq I\) and for every \(i\in I_0\), a
finite \(\Phi^i_0\subseteq\Phi^i\) s.t.
\begin{equation*}
\models\varphi\leftrightarrow\bigvee_{i\in I_0}\bigwedge\Phi^i_0
\end{equation*}
\end{lemma}

\begin{proof}
For simplicity we assume that \(\varphi\) is a sentence and that every
\(\Phi^i\) is a set of sentences. By hypothesis, for some \(i\in I\), we
have \(\Phi^i\models\varphi\); hence, by the Compactness Theorem,
\(\Phi^i_0\models\varphi\) for some finite \(\Phi^i_0\subseteq\Phi^i\).

If there is not such \(I_0\) with \(\models\varphi\to\bigvee_{i\in
    I_0}\bigwedge\Phi_0^i\), then each finite subset of
\(\{\varphi\}\cup\{\neg\bigwedge\Phi^i_0\mid i\in I\}\) has a model. Hence by
the Compactness Theorem, there is a contradiction
\end{proof}

\begin{corollary}[]
Let \(\Phi\) be a set of first-order sentences. Assume that any two structures
that satisfy the same sentences of \(\Phi\) are elementarily equivalent. Then
any first-order sentence is equivalent to a boolean combination of sentences
of \(\Phi\)
\end{corollary}

\begin{proof}
For any structure \(\cala\) set
\begin{equation*}
\Phi(\cala):=\{\psi\mid\psi\in\Phi,\cala\models\psi\}\cup
\{\neg\psi\mid\psi\in\Phi,\cala\models\neg\psi\}
\end{equation*}
Let \(\varphi\) be any first-order sentence. By the preceding lemma it
suffices to show that
\begin{equation*}
\models\varphi\leftrightarrow\bigvee_{\cala\models\phi}\bigwedge\Phi(\cala)
\end{equation*}
If \(\calb\models\varphi\) then
\(\calb\models\bigvee_{\cala\models\varphi}\bigwedge\Phi(\cala)\). Suppose
\(\cala\models\bigvee_{\cala\models\varphi}\bigwedge\Phi(\cala)\). Then for
some model \(\cala\) of \(\varphi\), \(\calb\models\Phi(\cala)\). By the
definition of \(\Phi(\cala)\), \(\cala\) and \(\calb\) satisfy the same
sentences of \(\Phi\)
\end{proof}

\subsection{Model Classes and Global Relations}
\label{sec:org8b57e7c}
Fix a vocabulary \(\tau\). For a sentence \(\varphi\) of \(\FO[\tau]\) we denote by
\(\Mod(\varphi)\) the class of \textbf{finite} models of \(\varphi\).

\(\Mod(\varphi)\) is closed under isomorphisms

For \(\varphi(x_1,\dots,x_n)\in\FO[\tau]\) and a structure \(\cala\) let
\begin{equation*}
\varphi^{\cala}(-):=\{(a_1,\dots,a_n)\mid\cala\models\varphi[a_1,\dots,a_n]\}
\end{equation*}
be the set of \(n\)-tuples \textbf{defined by \(\varphi\) in \(\cala\)}

\begin{definition}[]
Let \(K\) be a class of \(\tau\)-structures. An \(n\)-ary
\textbf{global relation
\(\Gamma\) on \(K\)} is a mapping assigning to each \(A\in K\) an \(n\)-ary
relation \(\Gamma(\cala)\) on \(\cala\) satisfying
\begin{equation*}
\Gamma(\cala)a_1\dots a_n\quad\text{ iff }\quad\Gamma(\calb)\pi(a_1)\dots\pi(a_n)
\end{equation*}
for every isomorphism \(\pi:\cala\cong\calb\) and every
\(a_1,\dots,a_n\in A\). If \(K\) is the class of all finite \(\tau\)-structures,
then we just speak of an \(n\)-ary \textbf{global relation}
\end{definition}

\begin{examplle}[]
\begin{enumerate}
\item Any formula \(\varphi(x_1,\dots,x_n)\in\FO[\tau]\) defines the global
relation
\(\cala\mapsto\varphi^{\cala}(-)\)
\item The ``transitive closure relation'' TC is the binary global relation on
GRAPH with
\begin{equation*}
\TC(\calg):=\{(a,b)\mid a,b\in G,\text{ there is a path from $a$ to $b$}\}
\end{equation*}
\end{enumerate}
\end{examplle}

\subsection{Relational Databases and Query Languages}
\label{sec:orga927361}

\section{Ehrenfeucht–Fraïssé Method}
\label{sec:orgf607dd6}

\subsection{Elementary Classes}
\label{sec:org2f5b5be}
\begin{proposition}[]
Every finite structure can be characterized in first-order logic up to
isomorphism, i.e., for every finite structure \(\cala\) there is a sentence
\(\varphi_\cala\) of first-order logic s.t. for all structures \(\calb\) we
have
\begin{equation*}
\calb\models\varphi_{\cala}\quad\text{ iff }\quad\cala\cong\calb
\end{equation*}
\end{proposition}

\begin{proof}
Suppose \(A=\{a_1,\dots,a_n\}\). Set \(\overbar{a}=a_1\dots a_n\). Let
\begin{align*}
\Theta_n:=\{\psi\mid\psi\text{ has the form }&Rx_1\dots x_k,x=y\text{ or }c=x,\\
&\text{and variables among }v_1,\dots,v_n\}
\end{align*}
and
\begin{align*}
\varphi_{\cala}&:=\exists v_1\dots\exists v_n(\bigwedge\{\psi\mid\psi\in\Theta_n,
\cala\models\psi[\overbar{a}]\}\wedge\\
&\bigwedge\{\neg\psi\mid\psi\in\Theta_n,\cala\models\neg\psi[\overbar{a}]\}\wedge
\forall v_{n+1}(v_{n+1}=v_n\vee\dots\vee v_{n+1}=v_n))
\end{align*}
\end{proof}

\begin{corollary}[]
Let \(K\) be a class of finite structures. Then there is a set \(\Phi\) of
first-order sentences s.t.
\begin{equation*}
K=\Mod(\Phi)
\end{equation*}
that is, \(K\) is the class of finite models of \(\Phi\)
\end{corollary}

\begin{proof}
For each \(n\) there is only a finite number of pairwise nonisomorphic
structures of cardinality \(n\). Let \(\cala_1,\dots,\cala_k\) be a maximal
subset of \(K\) of pairwise nonisomorphic structures of cardinality \(n\).
Set
\begin{equation*}
\psi_n:=(\varphi_{=n}\to(\varphi_{\cala_1}\vee\dots\vee\varphi_{\cala_k}))
\end{equation*}
Then \(K=\Mod(\{\psi_n\mid n\ge1\})\)
\end{proof}

\begin{definition}[]
Let \(K\) be a class of finite structures. \(K\) is called
\textbf{axiomatizable in first-order logic} or \textbf{elementary} if there is a setence
\(\varphi\) of first-order logic s.t. \(K=\Mod(\varphi)\)
\end{definition}

For structures \(\cala\) and \(\calb\) and \(m\in\N\) we write
\(\cala\equiv_m\calb\) and say that \(\cala\) and \(\calb\) are
\textbf{\(m\)-equivalent} if \(\cala\) and \(\calb\) satisfy the same first-order
sentences of quantifier rank \(\le m\)

\begin{theorem}[]
Let \(K\) be a class of finite structures. Suppose that for every \(m\) there
are fintie structures \(\cala\) and \(\calb\) s.t.
\begin{equation*}
\cala\in K,\calb\not\in K,\text{ and }\cala\equiv_m\calb
\end{equation*}
Then \(K\) is not axiomatizable in first-order logic
\end{theorem}

\begin{proof}
Let \(\varphi\) be any first-order sentence. Set \(m:=\qr(\varphi)\) and \(\cala\equiv_m\calb\)
\end{proof}

\subsection{Ehrenfeucht's Theorem}
\label{sec:org5b44346}
\begin{definition}[]
Assume \(\cala\) and \(\calb\) are structures. Let \(p\) be a map with
\(\dom(p)\subseteq A\) and \(\im(p)\subseteq B\). Then \(p\) is said to be a
\textbf{partial isomorphism} from \(\cala\) to \(\calb\) if
\begin{itemize}
\item \(p\) is injective
\item for every \(c\in\tau\): \(c^A\in\dom(p)\) and \(p(c^A)=c^B\)
\item for every \(n\)-ary \(R\in\tau\) and all \(a_1,\dots,a_n\in\dom(p)\)
\begin{equation*}
R^Aa_1\dots a_n \quad\text{ iff }\quad R^Bp(a_1)\dots p(a_n)
\end{equation*}
\end{itemize}


We write \(\Part(\cala,\calb)\) for the set of partial isomorphisms from
\(\cala\) to \(\calb\)
\end{definition}


Let \(\cala\) and \(\calb\) be \(\tau\)-structures, \(\bar{a}\in A^s\),
\(\overbar{b}\in B^s\), and \(m\in\N\). The \textbf{Ehrenfeucht game}
\(G_m(\cala,\overbar{a},\calb,\overbar{b})\) is played by two players called
the \textbf{spoiler} and the \textbf{duplicator}. Each player has to make \(m\) moves in
the course of a play. In his \(i\)-th move the spoiler first selects a
structure, \(\cala\) or \(\calb\), and an element in this structure. If the
spoiler chooses \(e_i\) in \(\cala\) the nthe duplicator in his \(i\)-th move
must choose an element \(f_i\) in \(\calb\). If the spoiler chooses \(f_i\)
in \(\calb\) then the duplicator must choose an element \(e_i\) in \(\cala\)

\begin{center}
\begin{tabular}{cc|c}
 & \(\cala,\overbar{a}\) & \(\calb,\overbar{b}\)\\
\hline
first move & \(e_1\) & \(f_1\)\\
second move & \(e_2\) & \(f_2\)\\
\(\vdots\) & \(\vdots\) & \(\vdots\)\\
\(m\)-th move & \(e_m\) & \(f_m\)\\
\end{tabular}
\end{center}
\end{document}
