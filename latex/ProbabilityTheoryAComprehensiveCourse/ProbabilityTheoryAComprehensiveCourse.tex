% Created 2020-04-24 五 14:02
% Intended LaTeX compiler: pdflatex
\documentclass[11pt]{article}
\usepackage[utf8]{inputenc}
\usepackage[T1]{fontenc}
\usepackage{graphicx}
\usepackage{grffile}
\usepackage{longtable}
\usepackage{wrapfig}
\usepackage{rotating}
\usepackage[normalem]{ulem}
\usepackage{amsmath}
\usepackage{textcomp}
\usepackage{amssymb}
\usepackage{capt-of}
\usepackage{imakeidx}
\usepackage{hyperref}
\usepackage{minted}
% TIPS
% \substack{a\\b} for multiple lines text





% pdfplots will load xolor automatically without option
\usepackage[dvipsnames]{xcolor}

\usepackage{forest}
% two-line text in node by [two \\ lines]
% \begin{forest} qtree, [..] \end{forest}
\forestset{
  qtree/.style={
    baseline,
    for tree={
      parent anchor=south,
      child anchor=north,
      align=center,
      inner sep=1pt,
    }}}
%\usepackage{flexisym}
% load order of mathtools and mathabx, otherwise conflict overbrace

\usepackage{mathtools}
%\usepackage{fourier}
\usepackage{pgfplots}
\usepackage{amsthm}
\usepackage{amsmath}
%\usepackage{unicode-math}
%
\usepackage{commath}
%\usepackage{,  , }
\usepackage{amsfonts}
\usepackage{amssymb}
% importing symbols https://tex.stackexchange.com/questions/14386/importing-a-single-symbol-from-a-different-font
%mathabx change every symbol
% use instead stmaryrd
%\usepackage{mathabx}
\usepackage{stmaryrd}
\usepackage{empheq}
\usepackage{tikz}
\usepackage{tikz-cd}
%\usepackage[notextcomp]{stix}
\usetikzlibrary{arrows.meta}
\usepackage[most]{tcolorbox}
%\utilde
%\usepackage{../../latexpackage/undertilde/undertilde}
% left and right superscript and subscript
\usepackage{actuarialsymbol}
\usepackage{threeparttable}
\usepackage{scalerel,stackengine}
\usepackage{stackrel}
% \stackrel[a]{b}{c}
\usepackage{dsfont}
% text font
\usepackage{newpxtext}
%\usepackage{newpxmath}

%\newcounter{dummy} \numberwithin{dummy}{section}
\newtheorem{dummy}{dummy}[section]
\theoremstyle{definition}
\newtheorem{definition}[dummy]{Definition}
\newtheorem{corollary}[dummy]{Corollary}
\newtheorem{lemma}[dummy]{Lemma}
\newtheorem{proposition}[dummy]{Proposition}
\newtheorem{theorem}[dummy]{Theorem}
\theoremstyle{definition}
\newtheorem{example}[dummy]{Example}
\theoremstyle{remark}
\newtheorem*{remark}{Remark}


\newcommand\what[1]{\ThisStyle{%
    \setbox0=\hbox{$\SavedStyle#1$}%
    \stackengine{-1.0\ht0+.5pt}{$\SavedStyle#1$}{%
      \stretchto{\scaleto{\SavedStyle\mkern.15mu\char'136}{2.6\wd0}}{1.4\ht0}%
    }{O}{c}{F}{T}{S}%
  }
}

\newcommand\wtilde[1]{\ThisStyle{%
    \setbox0=\hbox{$\SavedStyle#1$}%
    \stackengine{-.1\LMpt}{$\SavedStyle#1$}{%
      \stretchto{\scaleto{\SavedStyle\mkern.2mu\AC}{.5150\wd0}}{.6\ht0}%
    }{O}{c}{F}{T}{S}%
  }
}

\newcommand\wbar[1]{\ThisStyle{%
    \setbox0=\hbox{$\SavedStyle#1$}%
    \stackengine{.5pt+\LMpt}{$\SavedStyle#1$}{%
      \rule{\wd0}{\dimexpr.3\LMpt+.3pt}%
    }{O}{c}{F}{T}{S}%
  }
}

\newcommand{\bl}[1] {\boldsymbol{#1}}
\newcommand{\Wt}[1] {\stackrel{\sim}{\smash{#1}\rule{0pt}{1.1ex}}}
\newcommand{\wt}[1] {\widetilde{#1}}
\newcommand{\tf}[1] {\textbf{#1}}


%For boxed texts in align, use Aboxed{}
%otherwise use boxed{}

\DeclareMathSymbol{\widehatsym}{\mathord}{largesymbols}{"62}
\newcommand\lowerwidehatsym{%
  \text{\smash{\raisebox{-1.3ex}{%
    $\widehatsym$}}}}
\newcommand\fixwidehat[1]{%
  \mathchoice
    {\accentset{\displaystyle\lowerwidehatsym}{#1}}
    {\accentset{\textstyle\lowerwidehatsym}{#1}}
    {\accentset{\scriptstyle\lowerwidehatsym}{#1}}
    {\accentset{\scriptscriptstyle\lowerwidehatsym}{#1}}
}

\usepackage{graphicx}
    
% text on arrow for xRightarrow
\makeatletter
%\newcommand{\xRightarrow}[2][]{\ext@arrow 0359\Rightarrowfill@{#1}{#2}}
\makeatother


\newcommand{\dom}[1]{%
\mathrm{dom}{(#1)}
}

% Roman numerals
\makeatletter
\newcommand*{\rom}[1]{\expandafter\@slowromancap\romannumeral #1@}
\makeatother

\def \fR {\mathfrak{R}}
\def \bx {\boldsymbol{x}}
\def \bz {\boldsymbol{z}}
\def \ba {\boldsymbol{a}}
\def \bh {\boldsymbol{h}}
\def \bo {\boldsymbol{o}}
\def \bU {\boldsymbol{U}}
\def \bc {\boldsymbol{c}}
\def \bV {\boldsymbol{V}}
\def \bI {\boldsymbol{I}}
\def \bK {\boldsymbol{K}}
\def \bt {\boldsymbol{t}}
\def \bb {\boldsymbol{b}}
\def \bA {\boldsymbol{A}}
\def \bX {\boldsymbol{X}}
\def \bu {\boldsymbol{u}}
\def \bS {\boldsymbol{S}}
\def \bZ {\boldsymbol{Z}}
\def \bz {\boldsymbol{z}}
\def \by {\boldsymbol{y}}
\def \bw {\boldsymbol{w}}
\def \bT {\boldsymbol{T}}
\def \bF {\boldsymbol{F}}
\def \bS {\boldsymbol{S}}
\def \bm {\boldsymbol{m}}
\def \bW {\boldsymbol{W}}
\def \bR {\boldsymbol{R}}
\def \bQ {\boldsymbol{Q}}
\def \bS {\boldsymbol{S}}
\def \bP {\boldsymbol{P}}
\def \bT {\boldsymbol{T}}
\def \bY {\boldsymbol{Y}}
\def \bH {\boldsymbol{H}}
\def \bB {\boldsymbol{B}}
\def \blambda {\boldsymbol{\lambda}}
\def \bPhi {\boldsymbol{\Phi}}
\def \btheta {\boldsymbol{\theta}}
\def \bTheta {\boldsymbol{\Theta}}
\def \bmu {\boldsymbol{\mu}}
\def \bphi {\boldsymbol{\phi}}
\def \bSigma {\boldsymbol{\Sigma}}
\def \lb {\left\{}
\def \rb {\right\}}
\def \la {\langle}
\def \ra {\rangle}
\def \caln {\mathcal{N}}
\def \dissum {\displaystyle\Sigma}
\def \dispro {\displaystyle\prod}
\def \E {\mathbb{E}}
\def \Q {\mathbb{Q}}
\def \N {\mathbb{N}}
\def \V {\mathbb{V}}
\def \R {\mathbb{R}}
\def \P {\mathbb{P}}
\def \A {\mathbb{A}}
\def \Z {\mathbb{Z}}
\def \I {\mathbb{I}}
\def \C {\mathbb{C}}
\def \cala {\mathcal{A}}
\def \calb {\mathcal{B}}
\def \calq {\mathcal{Q}}
\def \calp {\mathcal{P}}
\def \cals {\mathcal{S}}
\def \calg {\mathcal{G}}
\def \caln {\mathcal{N}}
\def \calr {\mathcal{R}}
\def \calm {\mathcal{M}}
\def \calc {\mathcal{C}}
\def \calf {\mathcal{F}}
\def \calk {\mathcal{K}}
\def \call {\mathcal{L}}
\def \calu {\mathcal{U}}
\def \bcup {\bigcup}


\def \uin {\underline{\in}}
\def \oin {\overline{\in}}
\def \uR {\underline{R}}
\def \oR {\overline{R}}
\def \uP {\underline{P}}
\def \oP {\overline{P}}

\def \Ra {\Rightarrow}

\def \e {\enspace}

\def \sgn {\operatorname{sgn}}
\def \gen {\operatorname{gen}}
\def \ker {\operatorname{ker}}
\def \im {\operatorname{im}}

\def \tril {\triangleleft}

% \varprod
\DeclareSymbolFont{largesymbolsA}{U}{txexa}{m}{n}
\DeclareMathSymbol{\varprod}{\mathop}{largesymbolsA}{16}

% \bigtimes
\DeclareFontFamily{U}{mathx}{\hyphenchar\font45}
\DeclareFontShape{U}{mathx}{m}{n}{
      <5> <6> <7> <8> <9> <10>
      <10.95> <12> <14.4> <17.28> <20.74> <24.88>
      mathx10
      }{}
\DeclareSymbolFont{mathx}{U}{mathx}{m}{n}
\DeclareMathSymbol{\bigtimes}{1}{mathx}{"91}
% \odiv
\DeclareFontFamily{U}{matha}{\hyphenchar\font45}
\DeclareFontShape{U}{matha}{m}{n}{
      <5> <6> <7> <8> <9> <10> gen * matha
      <10.95> matha10 <12> <14.4> <17.28> <20.74> <24.88> matha12
      }{}
\DeclareSymbolFont{matha}{U}{matha}{m}{n}
\DeclareMathSymbol{\odiv}         {2}{matha}{"63}


\newcommand\subsetsim{\mathrel{%
  \ooalign{\raise0.2ex\hbox{\scalebox{0.9}{$\subset$}}\cr\hidewidth\raise-0.85ex\hbox{\scalebox{0.9}{$\sim$}}\hidewidth\cr}}}
\newcommand\simsubset{\mathrel{%
  \ooalign{\raise-0.2ex\hbox{\scalebox{0.9}{$\subset$}}\cr\hidewidth\raise0.75ex\hbox{\scalebox{0.9}{$\sim$}}\hidewidth\cr}}}

\newcommand\simsubsetsim{\mathrel{%
  \ooalign{\raise0ex\hbox{\scalebox{0.8}{$\subset$}}\cr\hidewidth\raise1ex\hbox{\scalebox{0.75}{$\sim$}}\hidewidth\cr\raise-0.95ex\hbox{\scalebox{0.8}{$\sim$}}\cr\hidewidth}}}
\newcommand{\stcomp}[1]{{#1}^{\mathsf{c}}}


\author{Achim Klenke}
\date{\today}
\title{\aunclfamily\Huge Probability Theory\\ A \\Comprehensive Course}
\hypersetup{
 pdfauthor={Achim Klenke},
 pdftitle={\aunclfamily\Huge Probability Theory\\ A \\Comprehensive Course},
 pdfkeywords={},
 pdfsubject={},
 pdfcreator={Emacs 26.3 (Org mode 9.3.6)}, 
 pdflang={English}}
\begin{document}

\maketitle \clearpage
\tableofcontents \clearpage\section{Basic Measure Theory}
\label{sec:orgea0c38c}
\subsection{Classes of Sets}
\label{sec:orgff5ba97}
\begin{definition}[]
A class of sets \(\cala\) is called
\begin{itemize}
\item \textbf{\(\cap\)-closed} or a \textbf{\(\pi\)-system} if \(A\cap B\in\cala\) whenever \(A,B\in\cala\)
\item \textbf{\(\sigma\text{-}\cap\)-closed} (closed under countable intersection)
\item \textbf{\(\cup\)-closed} (closed under unions)
\item \textbf{\(\sigma\text{-}\cup\)-closed}
\item \textbf{\textbackslash-closed}
\item closed under complements
\end{itemize}
\end{definition}

\begin{definition}[$\sigma$-algebra]
A class of sets \(\cala\subset 2^{\Omega}\) is called a \textbf{\(\sigma\)-algebra} if it
fullills the following three conditions
\begin{enumerate}
\item \(\Omega\in\cala\)
\item \(\cala\) is closed under complements
\item \(\cala\) is closed under countable unions
\end{enumerate}
\end{definition}


\begin{theorem}[]
If \(\cala\) is closed under complements, then we have the equivalence
\begin{align*}
\cala\text{ is }\cap\text{-closed}\quad&\Longleftrightarrow\quad\cala\text{ is }
\cup\text{-closed}\\
\cala\text{ is }\sigma\text{-}\cap\text{-closed}\quad&\Longleftrightarrow\quad\cala\text{ is }
\sigma\text{-}\cup\text{-closed}
\end{align*}
\end{theorem}

\begin{theorem}[]
\label{thm1.4}
Assume that \(\cala\) is \textbackslash-closed. Then the following statements hold:
\begin{enumerate}
\item \(\cala\) is \(\cup\)-closed
\item If in addition \(\cala\) is \(\sigma\text{-}\cup\)-closed, then \(\cala\) is 
\(\sigma\text{-}\cup\)-closed
\item Any countable (repectively finite) union of sets in \(\cala\) can be
expressed as a countable (respectively finite) disjoint union of sets in \(\cala\)
\end{enumerate}
\end{theorem}

\begin{proof}
\begin{enumerate}
\setcounter{enumi}{2}
\item Assume that \(A_1,A_2,\dots\in\cala\)
\begin{equation*}
\displaystyle\bigcup_{n=1}^\infty A_n=A_1\uplus(A_2\textbackslash
A_1)\uplus((A_3\textbackslash
A_2)\textbackslash A_1)\uplus\dots
\end{equation*}
\end{enumerate}
\end{proof}

\begin{definition}[]
A class of sets \(\cala\subset 2^\Omega\) is called an \textbf{algebra} if the
following three conditions are fulfilled
\begin{enumerate}
\item \(\Omega\in\cala\)
\item \(\cala\) is \textbackslash-closed
\item \(\cala\) is \(\cup\)-closed
\end{enumerate}
\end{definition}

\begin{theorem}[]
A class of sets \(\cala\subset2^\Omega\) is an algebra if and only if the
following  three properties hold
\begin{enumerate}
\item \(\Omega\in\cala\)
\item \(\cala\) is closed under complements
\item \(\cala\) is closed under intersections
\end{enumerate}
\end{theorem}

\begin{definition}[]
A class of sets \(\cala\subset2^\Omega\) is called a \textbf{ring} if the following
conditions hold
\begin{enumerate}
\item \(\emptyset\in\cala\)
\item \(\cala\) is \textbackslash-closed
\item \(\cala\) is \(\cup\)-closed
\end{enumerate}
\end{definition}


\begin{definition}[]
A class of sets \(\cala\subset2^\Omega\) is called a \textbf{semiring} if
\begin{enumerate}
\item \(\emptyset\in\cala\)
\item for any two sets \(A,B\in\cala\) the difference set \(B\textbackslash A\) is a
finite union of mutually disjoint sets in \(\cala\)
\item \(\cala\) is \(\cap\)-closed
\end{enumerate}
\end{definition}

\begin{definition}[]
A class of sets \(\cala\subset2^\Omega\) is called a \textbf{\(\lambda\)-system} if
\begin{enumerate}
\item \(\Omega\in\cala\)
\item for any two sets \(A,B\in\cala\) with \(A\subset B\), \(B\textbackslash A\in\cala\)
\item \(\biguplus_{n=1}^\infty A_n\in\cala\) for any choice of countably many
pairwise disjoint sets \(A_1,\dots\in\cala\)
\end{enumerate}
\end{definition}

\begin{theorem}[]
\begin{enumerate}
\item Every \(\sigma\)-algebra also is a \(\lambda\)-system, an algebra and a
\(\sigma\)-ring
\item Every \(\sigma\)-ring is a ring, and every ring is a semiring
\item Every algebra is a ring. An algebra on a finite set \(\Omega\) is a \(\sigma\)-algebra
\end{enumerate}
\end{theorem}

\begin{definition}[liminf and limsup]
Let \(A_1,A_2,\dots\) be a subset of \(\Omega\). The sets
\begin{equation*}
\liminf_{n\to\infty}A_n:=\displaystyle\bigcup_{n=1}^\infty
\bigcap_{m=n}^\infty A_m\hspace{1.5cm}
\limsup_{n\to\infty}A_n:=\bigcap_{n=1}^\infty\bigcup_{m=n}^\infty
A_m
\end{equation*}
are called \textbf{limes inferior} and \textbf{limes superior}, respectively, of the sequence
\((A_n)_{n\in\N}\)
\end{definition}

\begin{remark}
\begin{enumerate}
\item lim\hspace{0.03cm}inf and lim\hspace{0.03cm}sup can be rewritten as
\begin{align*}
\liminf_{n\to\infty}A_n&=\{\omega\in\Omega:\#\{n\in\N:\omega\not\in A_n\}<\infty\}\\
\limsup_{n\to\infty}A_n&=\{\omega\in\Omega:\#\{n\in\N:\omega\in A_n\}=\infty\}
\end{align*}
In other words, limes inferior is the event where \textbf{eventually all} of the
\(A_n\) occur. On the other hand, limes superior is the event where
\textbf{infinitely many} of the \(A_n\) occur. In particular,
\(A_*:=\liminf_{n\to\infty}A_n\subset A^*:=\limsup_{n\to\infty}A_n\)
\item We define the \textbf{indicator function} on the set \(A\) by
\begin{equation*}
\mathbbm{1}_A(x):=
\begin{cases}
1,&x\in A\\
0,&x\not\in A
\end{cases}
\end{equation*}
With this notation 
\begin{equation*}
\mathbbm{1}_{A_*}=\liminf_{n\to\infty}\mathbbm{1}_{A_n}\quad\text{and}\quad
\mathbbm{1}_{A^*}=\limsup_{n\to\infty}\mathbbm{1}_{A_n}
\end{equation*}
\item If \(\cala\subset2^\Omega\) is a \(\sigma\)-algebra and if \(A_n\in\cala\) for
every \(n\in\N\), then \(A_*\in\cala\) and \(A^*\in\cala\)
\end{enumerate}
\end{remark}

\begin{theorem}[Intersection of classes of sets]
Let \(I\) be an arbitrary index set, and assume that \(\cala_i\) is a
\(\sigma\)-algebra for every \(i\in I\). Hence the intersection
\begin{equation*}
\cala_I:=\{A\subset\Omega:A\in\cala_i\text{ for every }i\in I\}=
\displaystyle\bigcap_{i\in I}\cala_i
\end{equation*}
is a \(\sigma\)-algebra. The analogous statement holds for rings, \(\sigma\)-rings,
algebras and \(\lambda\)-systems. However, it fails for semirings
\end{theorem}

\begin{theorem}[Generated $\sigma$-algebra]
Let \(\cale\subset2^\Omega\). Then there exists a smallest \(\sigma\)-algebra 
\(\sigma(\cale)\) with \(\cale\subset\sigma(\cale)\)
\begin{equation*}
\sigma(\cale):=\displaystyle\bigcap_{\substack{\cala\subset2^\Omega
\text{ is a }\sigma\text{-algebra}\\\cala\supset\cale}}\cala
\end{equation*}
\(\sigma(\cale)\) is called the \(\sigma\)-algebra generated by \(\cale\). \(\cale\) is
called a generator of \(\sigma(\cale)\). Similarly, we define \(\delta(\cale)\)
as the \(\lambda\)-system generated by \(\cale\)
\end{theorem}

\begin{remark}
The following three statements hold 
\begin{enumerate}
\item \(\cale\subset\sigma(\cale)\)
\item If \(\cale_1\subset\cale_2\), then \(\sigma(\cale_1)\subset\sigma(\cale_2)\)
\item \(\cala\) is a \(\sigma\)-algebra if and only if \(\sigma(\cala)=\cala\)
\end{enumerate}
\end{remark}

\begin{theorem}[$\cap$-closed $\lambda$-system]
\label{thm1.18}
Let \(\cald\subset2^\Omega\) be a \(\lambda\)-system. Then 
\begin{equation*}
\cald\text{ is a }\pi\text{-system}\quad\Longleftrightarrow\quad
\cald\text{ is a }\sigma\text{-algebra}
\end{equation*}
\end{theorem}

\begin{proof}
"\(\Longrightarrow\)"
\begin{enumerate}
\setcounter{enumi}{2}
\item Let \(A,B\in\cald\). By assumption, \(A\cap B\in\cald\) and trivially
\(A\cap B\subset A\). Thus \(A\textbackslash B=A\textbackslash(A\cap
      B)\in\cald\). This implies that \(\cald\) is \textbackslash-closed. Thus by
Theorem \ref{thm1.4}, works.
\end{enumerate}
\end{proof}


\begin{theorem}[Dynkin's $\pi$-$\lambda$ theorem]
If \(\cale\subset2^\Omega\) is a \(\pi\)-system, then
\begin{equation*}
\sigma(\cale)=\delta(\cale)
\end{equation*}
\end{theorem}

\begin{proof}
\begin{enumerate}
\item \(\supseteq\). \(A^c=\Omega\textbackslash A\).
\item \(\subseteq\). By Theorem \ref{thm1.18}, it is enough to show that
\(\delta(\cale)\) is a \(\pi\)-system. For any \(B\in\delta(\cale)\) define
\begin{equation*}
\cald_B:=\{A\in\delta(\cale):A\cap B\in\delta(\cale)\}
\end{equation*}
In order to show that \(\delta(\cale)\) is a \(\pi\) system, it is enough to
show that 
\begin{equation*}
\delta(\cale)\subset\cald_B\quad\text{for any }B\in\delta(\cale)
\end{equation*}

\(\cald_E\) is a \(\lambda\)-system
\begin{enumerate}
\item \(\Omega\cap E=E\in\delta(\cale)\). Hence \(\Omega\in\cald_E\)
\item For any \(A,B\in\cald_E\) with \(A\subset B\), we have 
\((B\textbackslash A)\cap E=(B\cap E)\textbackslash(A\cap E)\in\delta(E)\)
\item Assume that \(A_1,\dots,\in\cald_E\) are mutually disjoint. Hence
\begin{equation*}
\left(\displaystyle\bigcup_{n=1}^\infty\right)\cap E=
\biguplus_{n=1}^\infty(A_n\cap E)\in\delta(\cale)
\end{equation*}
\end{enumerate}
\end{enumerate}


By assumption, \(A\cap E\in\cale\) if \(A,E\in\cale\); thus
\(\cale\subset\cale_E\) if \(E\in\cale\). Hence
\(\delta(\cale)\subset\delta(\cald_E)=\cald_E\) for any \(E\in\cale\). Hence we
get that \(B\cap E\in\delta(\cale)\) for any \(B\in\delta(\cale)\) and
\(E\in\cale\). This implies that \(E\in\cale_B\) for any
\(B\in\delta(\cale)\). Thus \(\cale\subset\cald_B\) for any
\(B\in\delta(\cale)\). 
\end{proof}
\end{document}