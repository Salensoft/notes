% Created 2019-12-09 一 19:29
% Intended LaTeX compiler: pdflatex
\documentclass[11pt]{article}
\usepackage[utf8]{inputenc}
\usepackage[T1]{fontenc}
\usepackage{graphicx}
\usepackage{grffile}
\usepackage{longtable}
\usepackage{wrapfig}
\usepackage{rotating}
\usepackage[normalem]{ulem}
\usepackage{amsmath}
\usepackage{textcomp}
\usepackage{amssymb}
\usepackage{capt-of}
\usepackage{hyperref}
\usepackage{minted}
\author{wu}
\date{\today}
\title{}
\hypersetup{
 pdfauthor={wu},
 pdftitle={},
 pdfkeywords={},
 pdfsubject={},
 pdfcreator={Emacs 26.3 (Org mode 9.3)}, 
 pdflang={English}}
\begin{document}

\tableofcontents \clearpage\% Created 2019-12-09 一 19:28
\% Intended \LaTeX{} compiler: pdflatex
\documentclass[11pt]{article}
\usepackage[utf8]{inputenc}
\usepackage[T1]{fontenc}
\usepackage{graphicx}
\usepackage{grffile}
\usepackage{longtable}
\usepackage{wrapfig}
\usepackage{rotating}
\usepackage[normalem]{ulem}
\usepackage{amsmath}
\usepackage{textcomp}
\usepackage{amssymb}
\usepackage{capt-of}
\usepackage{hyperref}
\usepackage{minted}
% TIPS
% \substack{a\\b} for multiple lines text





% pdfplots will load xolor automatically without option
\usepackage[dvipsnames]{xcolor}

\usepackage{forest}
% two-line text in node by [two \\ lines]
% \begin{forest} qtree, [..] \end{forest}
\forestset{
  qtree/.style={
    baseline,
    for tree={
      parent anchor=south,
      child anchor=north,
      align=center,
      inner sep=1pt,
    }}}
%\usepackage{flexisym}
% load order of mathtools and mathabx, otherwise conflict overbrace

\usepackage{mathtools}
%\usepackage{fourier}
\usepackage{pgfplots}
\usepackage{amsthm}
\usepackage{amsmath}
%\usepackage{unicode-math}
%
\usepackage{commath}
%\usepackage{,  , }
\usepackage{amsfonts}
\usepackage{amssymb}
% importing symbols https://tex.stackexchange.com/questions/14386/importing-a-single-symbol-from-a-different-font
%mathabx change every symbol
% use instead stmaryrd
%\usepackage{mathabx}
\usepackage{stmaryrd}
\usepackage{empheq}
\usepackage{tikz}
\usepackage{tikz-cd}
%\usepackage[notextcomp]{stix}
\usetikzlibrary{arrows.meta}
\usepackage[most]{tcolorbox}
%\utilde
%\usepackage{../../latexpackage/undertilde/undertilde}
% left and right superscript and subscript
\usepackage{actuarialsymbol}
\usepackage{threeparttable}
\usepackage{scalerel,stackengine}
\usepackage{stackrel}
% \stackrel[a]{b}{c}
\usepackage{dsfont}
% text font
\usepackage{newpxtext}
%\usepackage{newpxmath}

%\newcounter{dummy} \numberwithin{dummy}{section}
\newtheorem{dummy}{dummy}[section]
\theoremstyle{definition}
\newtheorem{definition}[dummy]{Definition}
\newtheorem{corollary}[dummy]{Corollary}
\newtheorem{lemma}[dummy]{Lemma}
\newtheorem{proposition}[dummy]{Proposition}
\newtheorem{theorem}[dummy]{Theorem}
\theoremstyle{definition}
\newtheorem{example}[dummy]{Example}
\theoremstyle{remark}
\newtheorem*{remark}{Remark}


\newcommand\what[1]{\ThisStyle{%
    \setbox0=\hbox{$\SavedStyle#1$}%
    \stackengine{-1.0\ht0+.5pt}{$\SavedStyle#1$}{%
      \stretchto{\scaleto{\SavedStyle\mkern.15mu\char'136}{2.6\wd0}}{1.4\ht0}%
    }{O}{c}{F}{T}{S}%
  }
}

\newcommand\wtilde[1]{\ThisStyle{%
    \setbox0=\hbox{$\SavedStyle#1$}%
    \stackengine{-.1\LMpt}{$\SavedStyle#1$}{%
      \stretchto{\scaleto{\SavedStyle\mkern.2mu\AC}{.5150\wd0}}{.6\ht0}%
    }{O}{c}{F}{T}{S}%
  }
}

\newcommand\wbar[1]{\ThisStyle{%
    \setbox0=\hbox{$\SavedStyle#1$}%
    \stackengine{.5pt+\LMpt}{$\SavedStyle#1$}{%
      \rule{\wd0}{\dimexpr.3\LMpt+.3pt}%
    }{O}{c}{F}{T}{S}%
  }
}

\newcommand{\bl}[1] {\boldsymbol{#1}}
\newcommand{\Wt}[1] {\stackrel{\sim}{\smash{#1}\rule{0pt}{1.1ex}}}
\newcommand{\wt}[1] {\widetilde{#1}}
\newcommand{\tf}[1] {\textbf{#1}}


%For boxed texts in align, use Aboxed{}
%otherwise use boxed{}

\DeclareMathSymbol{\widehatsym}{\mathord}{largesymbols}{"62}
\newcommand\lowerwidehatsym{%
  \text{\smash{\raisebox{-1.3ex}{%
    $\widehatsym$}}}}
\newcommand\fixwidehat[1]{%
  \mathchoice
    {\accentset{\displaystyle\lowerwidehatsym}{#1}}
    {\accentset{\textstyle\lowerwidehatsym}{#1}}
    {\accentset{\scriptstyle\lowerwidehatsym}{#1}}
    {\accentset{\scriptscriptstyle\lowerwidehatsym}{#1}}
}

\usepackage{graphicx}
    
% text on arrow for xRightarrow
\makeatletter
%\newcommand{\xRightarrow}[2][]{\ext@arrow 0359\Rightarrowfill@{#1}{#2}}
\makeatother


\newcommand{\dom}[1]{%
\mathrm{dom}{(#1)}
}

% Roman numerals
\makeatletter
\newcommand*{\rom}[1]{\expandafter\@slowromancap\romannumeral #1@}
\makeatother

\def \fR {\mathfrak{R}}
\def \bx {\boldsymbol{x}}
\def \bz {\boldsymbol{z}}
\def \ba {\boldsymbol{a}}
\def \bh {\boldsymbol{h}}
\def \bo {\boldsymbol{o}}
\def \bU {\boldsymbol{U}}
\def \bc {\boldsymbol{c}}
\def \bV {\boldsymbol{V}}
\def \bI {\boldsymbol{I}}
\def \bK {\boldsymbol{K}}
\def \bt {\boldsymbol{t}}
\def \bb {\boldsymbol{b}}
\def \bA {\boldsymbol{A}}
\def \bX {\boldsymbol{X}}
\def \bu {\boldsymbol{u}}
\def \bS {\boldsymbol{S}}
\def \bZ {\boldsymbol{Z}}
\def \bz {\boldsymbol{z}}
\def \by {\boldsymbol{y}}
\def \bw {\boldsymbol{w}}
\def \bT {\boldsymbol{T}}
\def \bF {\boldsymbol{F}}
\def \bS {\boldsymbol{S}}
\def \bm {\boldsymbol{m}}
\def \bW {\boldsymbol{W}}
\def \bR {\boldsymbol{R}}
\def \bQ {\boldsymbol{Q}}
\def \bS {\boldsymbol{S}}
\def \bP {\boldsymbol{P}}
\def \bT {\boldsymbol{T}}
\def \bY {\boldsymbol{Y}}
\def \bH {\boldsymbol{H}}
\def \bB {\boldsymbol{B}}
\def \blambda {\boldsymbol{\lambda}}
\def \bPhi {\boldsymbol{\Phi}}
\def \btheta {\boldsymbol{\theta}}
\def \bTheta {\boldsymbol{\Theta}}
\def \bmu {\boldsymbol{\mu}}
\def \bphi {\boldsymbol{\phi}}
\def \bSigma {\boldsymbol{\Sigma}}
\def \lb {\left\{}
\def \rb {\right\}}
\def \la {\langle}
\def \ra {\rangle}
\def \caln {\mathcal{N}}
\def \dissum {\displaystyle\Sigma}
\def \dispro {\displaystyle\prod}
\def \E {\mathbb{E}}
\def \Q {\mathbb{Q}}
\def \N {\mathbb{N}}
\def \V {\mathbb{V}}
\def \R {\mathbb{R}}
\def \P {\mathbb{P}}
\def \A {\mathbb{A}}
\def \Z {\mathbb{Z}}
\def \I {\mathbb{I}}
\def \C {\mathbb{C}}
\def \cala {\mathcal{A}}
\def \calb {\mathcal{B}}
\def \calq {\mathcal{Q}}
\def \calp {\mathcal{P}}
\def \cals {\mathcal{S}}
\def \calg {\mathcal{G}}
\def \caln {\mathcal{N}}
\def \calr {\mathcal{R}}
\def \calm {\mathcal{M}}
\def \calc {\mathcal{C}}
\def \calf {\mathcal{F}}
\def \calk {\mathcal{K}}
\def \call {\mathcal{L}}
\def \calu {\mathcal{U}}
\def \bcup {\bigcup}


\def \uin {\underline{\in}}
\def \oin {\overline{\in}}
\def \uR {\underline{R}}
\def \oR {\overline{R}}
\def \uP {\underline{P}}
\def \oP {\overline{P}}

\def \Ra {\Rightarrow}

\def \e {\enspace}

\def \sgn {\operatorname{sgn}}
\def \gen {\operatorname{gen}}
\def \ker {\operatorname{ker}}
\def \im {\operatorname{im}}

\def \tril {\triangleleft}

% \varprod
\DeclareSymbolFont{largesymbolsA}{U}{txexa}{m}{n}
\DeclareMathSymbol{\varprod}{\mathop}{largesymbolsA}{16}

% \bigtimes
\DeclareFontFamily{U}{mathx}{\hyphenchar\font45}
\DeclareFontShape{U}{mathx}{m}{n}{
      <5> <6> <7> <8> <9> <10>
      <10.95> <12> <14.4> <17.28> <20.74> <24.88>
      mathx10
      }{}
\DeclareSymbolFont{mathx}{U}{mathx}{m}{n}
\DeclareMathSymbol{\bigtimes}{1}{mathx}{"91}
% \odiv
\DeclareFontFamily{U}{matha}{\hyphenchar\font45}
\DeclareFontShape{U}{matha}{m}{n}{
      <5> <6> <7> <8> <9> <10> gen * matha
      <10.95> matha10 <12> <14.4> <17.28> <20.74> <24.88> matha12
      }{}
\DeclareSymbolFont{matha}{U}{matha}{m}{n}
\DeclareMathSymbol{\odiv}         {2}{matha}{"63}


\newcommand\subsetsim{\mathrel{%
  \ooalign{\raise0.2ex\hbox{\scalebox{0.9}{$\subset$}}\cr\hidewidth\raise-0.85ex\hbox{\scalebox{0.9}{$\sim$}}\hidewidth\cr}}}
\newcommand\simsubset{\mathrel{%
  \ooalign{\raise-0.2ex\hbox{\scalebox{0.9}{$\subset$}}\cr\hidewidth\raise0.75ex\hbox{\scalebox{0.9}{$\sim$}}\hidewidth\cr}}}

\newcommand\simsubsetsim{\mathrel{%
  \ooalign{\raise0ex\hbox{\scalebox{0.8}{$\subset$}}\cr\hidewidth\raise1ex\hbox{\scalebox{0.75}{$\sim$}}\hidewidth\cr\raise-0.95ex\hbox{\scalebox{0.8}{$\sim$}}\cr\hidewidth}}}
\newcommand{\stcomp}[1]{{#1}^{\mathsf{c}}}


\author{Qi'ao Chen}
\date{\today}
\title{Notes on Set Theory}
\hypersetup\{
 pdfauthor=\{Qi'ao Chen\},
 pdftitle=\{Notes on Set Theory\},
 pdfkeywords=\{\},
 pdfsubject=\{\},
 pdfcreator=\{Emacs 26.3 (Org mode 9.3)\}, 
 pdflang=\{English\}\}
\begin{document}

\maketitle
\tableofcontents \clearpage
\section{Foreword}
\label{sec:orgb74da21}
Notes for the entrance examination
\section{Models of Set - Sertraline}
\label{sec:org20aa1db}
\subsection{Some mathematical logic}
\label{sec:org1b1c148}
\begin{theorem}[Gödel’s second incompleteness theorem]
If a consistent recursive axiom set \(T\) contains \(\zfc\), then
\begin{equation*}
T\not\vdash\con{t}
\end{equation*}
especially, \(\zfc\not\vdash\con{\zfc}\)
\end{theorem}

\begin{definition}[]
Suppose \((M,E_M)\) and \((N,E_N)\) are two models of set theory, then
\begin{enumerate}
\item if for any formula \(\sigma\), \(M\models\sigma\) if and only if
\(N\models\sigma\), then \(M\) and \(N\) are \tf{elementary equivalent}, denoted
by \(M\equiv N\)
\item If bijection \(f:M\to N\) satisfies: for any \(a,b\in M\), \(aE_Mb\) iff
\(f(a)E_Nf(b)\), then \(f:M\cong N\) is an \tf{isomorphism}
\item If \(M\subseteq N\) and \(E_M=E_N\restriction M\), then \(M\) is \(N\)'s submodel
\item If \(M\) is isomorphic to a submodel of \(N\) by injection \(f\), and for any
formula \(\varphi(x_1,\dots,x_n)\), for any \(a_1,\dots,a_n\in M\), 
\(M\models\varphi[a_1,\dots,a_n]\) iff
\(N\models\varphi[f(a_1),\dots,f(a_n)]\), then \(f\) is called an
\tf{elementary embedding} from \(M\) to \(N\), written as \(f:M\prec N\)
\item If \(M\subseteq N\) and \(M\prec N\), then \(M\) is a \tf{elementary submodel}
of \(N\)
\end{enumerate}
\end{definition}

\begin{lemma}[]
Suppose \(N\models\zfc,M\subseteq N\), then \(M\prec N\) iff
\(\forall\varphi(x,x_1,\dots,x_n)\), \(\forall(a_1,\dots,a_n)\in M\), if 
\(\exists a\in N\) s.t. \(N\models\varphi[a,a_1,\dots,a_n]\), then \(\exists a\in
  M\) s.t. 
\(M\models\varphi[a,a_1,\dots,a_n]\)
\end{lemma}

\begin{definition}[]
Suppose \((M,E)\models\zfc\)
\begin{enumerate}
\item \(h_\varphi:M^n\to M\) is \(\varphi\)'s \tf{Skolem function} if 
\(\forall a_1,\dots,a_n\in M\), if \(\exists a\in M\) s.t.
\(M\models\varphi[a,a_1,\dots,a_n]\), then
\(M\models\varphi[h_\varphi(a_1,\dots,a_n),a_1,\dots,a_n]\) . requires \ac
\item Let \(\calh=\{h_\varphi\mid\varphi \text{is a formula on set theory}\}\). For
any \(S\subseteq M\), \tf{Skolem hull} \(\calh(S)\) is the smallest set
consisting of \(S\) and closed under \(\calh\)
\end{enumerate}
\end{definition}

\begin{lemma}[]
\(N\models\zfc,S\subseteq N\), if \(M=\calh(S)\), then \(M\prec N\)
\end{lemma}

\begin{theorem}[Löwenheim-Skolem theorem]
Suppose \(N\models\zfc\) and is infinite, then there is a model \(M\) s.t.
\(\abs{M}=\omega\) and \(M\prec N\)
\end{theorem}
\subsection{Cumulative Hierarchy}
\label{sec:org4cb1e41}
This section works in \zfm(a.k.a. \(\zf-\text{axiom of foundation}\))

\begin{definition}[]
For any \(\alpha\), define sequence \(V_\alpha\)
\begin{enumerate}
\item \(V_0=\emptyset\)
\item \(V_{\alpha+1}=\calp(V_\alpha)\)
\item For any limit ordinal \(\lambda\), \(V_\lambda=\bigcup_{\beta<\lambda}V_\beta\)
\end{enumerate}


And \(\wf=\displaystyle\bigcup_{\alpha\in\on}V_\alpha\)
\end{definition}

\begin{lemma}[]
For any ordinal \(\alpha\)
\begin{enumerate}
\item \(V_\alpha\) is transitive
\item if \(\xi\le\alpha\), then \(V_\xi\subseteq V_\alpha\)
\item if \(\kappa\) is inaccessible cardinal, then \(\abs{V_\kappa}=\kappa\)
\end{enumerate}
\end{lemma}

\begin{proof}
\begin{enumerate}
\item Obviously \(\kappa\le V_\kappa\). Since \(\kappa\) is inaccessible, then for any
\(\alpha<\kappa\), \(\abs{V_\alpha}<\kappa\).
\end{enumerate}
\end{proof}

\begin{definition}[]
For any set \(x\in\wf\), 
\begin{equation*}
\rank{x}=\min\{\beta\mid x\in V_{\beta+1}\}
\end{equation*}
\end{definition}

\begin{lemma}[]
\begin{enumerate}
\item \(V_\alpha=\{x\in\wf\mid\rank{x}<\alpha\}\)
\item \wf is transitive
\item For any \(x,y\in\wf\), if \(x\in y\), then \(\rank{x}<\rank{y}\)
\item for any \(y\in\wf\), \(\rank{y}=\sup\{\rank{x}+1\mid x\in y\}\)
\end{enumerate}
\end{lemma}

\begin{lemma}[]
Supoose \(\alpha\) is an ordinal
\begin{enumerate}
\item \(\alpha\in\wf\) and \(\rank{\alpha}=\alpha\)
\item \(V_\alpha\cap\on=\alpha\)
\end{enumerate}
\end{lemma}

\begin{lemma}[]
\begin{enumerate}
\item If \(x\in\wf\), then \(\bigcup x,\calp(x),\{x\}\in\wf\), and their ranks are
all less than \(\rank{x}+\omega\)
\item If \(x,y\in\wf\), then \(x\times y,x\cup y,x\cap y,\{x,y\},(x,y),x^y\in\wf\),
and their ranks are all less than \(\rank{x}+\rank{y}+\omega\)
\item \(\Z,\Q,\R\in V_{\omega+\omega}\)
\item for any set \(x\), \(x\in\wf\) iff \(x\subset\wf\)
\end{enumerate}
\end{lemma}

\begin{lemma}[]
Suppose \ac
\begin{enumerate}
\item for any group \(G\), there exists group \(G'\cong G\) in \wf
\item for any topological space \(T\), there exists \(T'\cong T\) in \wf
\end{enumerate}
\end{lemma}

\begin{definition}[]
Binary relation \(<\) on set \(A\) is \tf{well-founded} if for any nonempty
\(X\subseteq A\), \(X\) has minimal element under \(<\)
\end{definition}


\begin{theorem}[]
If \(A\in\wf\), then \(\in\) is a well-founded relation on \(A\)
\end{theorem}

\begin{lemma}[]
If set \(A\) is transitive and \(\in\) is well-founded on \(A\), then \(A\in\wf\)
\end{lemma}

\begin{lemma}[]
For any set \(x\), there is a smallest transitive set \(\trcl{x}\) s.t.
\(x\subseteq\trcl{x}\) 
\end{lemma}

\begin{proof}
\begin{align*}
x_0&=x\\
x_{n+1}&=\bigcup x_n\\
\trcl{x}&=\displaystyle\bigcup_{n<\omega}x_n
\end{align*}
\end{proof}

\(\trcl{x}\) is called \tf{transitive closure} of \(x\)


\begin{lemma}[]
Without axiom of power set
\begin{enumerate}
\item if \(x\) is transitive, then \(\trcl{x}=x\)
\item if \(y\in x\), then \(\trcl{y}\subseteq\trcl{x}\)
\item \(\trcl{x}=x\cup\bigcup\{\trcl{y}\mid y\in x\}\)
\end{enumerate}
\end{lemma}

\begin{theorem}[]
For any set \(X\), the following are equivalent
\begin{enumerate}
\item \(X\in\wf\)
\item \(\trcl{X}\in\wf\)
\item \(\in\) is a well-founded relation on \(\trcl{X}\)
\end{enumerate}
\end{theorem}

\begin{theorem}[]
The following propositions are equivalent
\begin{enumerate}
\item Axiom of foundation
\item For any set \(X\), \(\in\) is a well-founded relation on \(X\)
\item \(\tf{V}=\wf\)
\end{enumerate}
\end{theorem}
\subsection{Relativization}
\label{sec:org6c601b0}
\begin{definition}[]
Let \tf{M} be a class \(\varphi\) a formula, the \tf{relativization} of \(\varphi\)
to \tf{M} is \(\varphi^{\tf{M}}\) defined inductively
\begin{align*}
(x\in y)^{\cm}&\leftrightarrow x=y\\
(x\in y)^{\cm}&\leftrightarrow x\in y\\
(\varphi\to\psi)^{\cm}&\leftrightarrow \varphi^{\cm}\to\psi^\cm\\
(\neg\varphi)^\cm&\leftrightarrow\neg\varphi^\cm\\
(\forall x\varphi)^\cm&\leftrightarrow(\forall x\in\cm)\varphi^\cm
\end{align*}
\end{definition}

Note \(\varphi^\cv=\varphi\) and
\begin{equation*}
f^\cm=\{(x_1,\dots,x_n,x_{n+1})\in\cm\mid\varphi^\cm(x_1,\dots,x_n,x_{n+1})\}
\end{equation*}

\begin{definition}[]
For any theory \(T\), any class \(\cm\), \(\cm\models T\) iff for any axiom
\(\varphi\) of \(T\), \(\varphi^\cm\) holds
\end{definition}


\begin{theorem}[\zfm]
\(\wf\models\zf\)
\end{theorem}

\begin{proof}
\begin{itemize}
\item \tf{Axiom of existence}

\((\exists x(x=x))^\cm\leftrightarrow\exists x\in\cm(x=x)\), which is
equivalent to \cm being nonempty
\item \tf{Axiom of extensionality}

\begin{gather*}
\forall X\forall Y\forall u((u\in X\leftrightarrow u\in Y)\to X=Y)^\cm
\Leftrightarrow\\
\forall X\in\cm\forall Y\in\cm\forall u\in\cm
((u\in X\leftrightarrow u\in Y)\to X=Y)
\end{gather*}

\begin{lemma}
If $\cm$ is transitive, then axiom of extensionality holds in \cm
\end{lemma}

\item \tf{Axiom schema of specification}

\begin{equation*}
\forall X\in\cm\exists Y\in\cm\forall u\in\cm(u\in Y\leftrightarrow
u\in X\wedge\varphi^\cm(u))
\end{equation*}

Since for any \(X\in\wf\), \(\calp(X)\subseteq \wf\)
\item \tf{Axiom of paring}
\item \tf{Axiom of union}
\item \tf{Axiom of power set}

\begin{equation*}
\forall X\in\cm\exists Y\in\cm\forall u\in\cm(u\in Y\leftrightarrow(u\subseteq X)^\cm)
\end{equation*}
and 
\begin{equation*}
(u\subseteq X)^\cm\leftrightarrow\forall x\in\cm(x\in u\to x\in X)
\leftrightarrow u\cap\cm\subseteq X
\end{equation*}
\item \tf{Axiom of foundation}
\item \tf{Axiom schema of replacement}
\end{itemize}
\end{proof}
\subsection{Absoluteness}
\label{sec:org786b00c}
\begin{definition}[]
For any formula \(\psi(x_1,\dots,x_n)\) and any class \cm,\cn, 
\(\cm\subseteq \cn\), if
\begin{equation*}
\forall x_1\dots\forall x_n\in\cm(\psi^\cm(x_1,\dots,x_n)
\leftrightarrow\psi^\cn(x_1,\dots,x_n))
\end{equation*}
then \(\psi(x_1,\dots,x_n)\) is \tf{absolute} for \cm,cn. If \(\cn=\cv\), then
\(\psi\) is \tf{absolute} for \cm
\end{definition}

\begin{lemma}[]
Suppose \(\cm\subseteq\cn\) and \(\varphi\),\(\psi\) are formulas, then
\begin{enumerate}
\item if \(\varphi\),\(\psi\) are absolute for \cm,cn, then so are
\(\neg\varphi,\varphi\to\psi\)
\item if \(\varphi\) doesn't contain any quantifiers, then \(\varphi\) is absolute for
any \cm
\item if \cm,\cn  are transitive and \(\varphi\) is absolute for them, then so are
\(\forall x\in y\varphi\)
\end{enumerate}
\end{lemma}

\begin{definition}[]
\(\Delta_0\) formula
\begin{enumerate}
\item \(x=y,x\in y\) are \(\Delta_0\) formulas
\item if \(\varphi\),\(\psi\) are \(\Delta_0\), then so are \(\neg\varphi,\varphi\to\psi\)
\item if \(\varphi\) is \(\Delta_0\), \(y\) is any set, then \((\forall x\in y)\varphi\)
is \(\Delta_0\)
\end{enumerate}


If \(\varphi\) is \(\Delta_0\), then \(\exists x_1\dots\exists x_n\varphi\) is
\(\Sigma_1\) formula, \(\forall x_1\dots\forall x_n\varphi\) is \(\Pi_1\)
\end{definition}

\begin{lemma}[]
\(\cm\subseteq\cn\) are both transitive, \(\psi(x_0,\dots,x_n)\) is a formula,
then
\begin{enumerate}
\item if \(\psi\) is \(\Delta_0\), then it's absolute for \cm,cn
\item if \(\psi\) is \(\Sigma_1\), then
\begin{equation*}
\forall x_1\dots x_n(\psi^\cm(x_1,\dots,x_n)\to\psi^\cn(x_1,\dots,x_n))
\end{equation*}
\item if \(\psi\) is \(\Pi_1\), then
\begin{equation*}
\forall x_1\dots x_n(\psi^\cn(x_1,\dots,x_n)\to\psi^\cm(x_1,\dots,x_n))
\end{equation*}
\end{enumerate}
\end{lemma}

\begin{lemma}[]
If \(\cm\subseteq\cn\), \(\cm\models\Sigma,\cn\models\Sigma\) and
\begin{equation*}
\Sigma\vdash\forall x_1\dots\forall x_n(\varphi(x_1,\dots,x_n)\leftrightarrow
\psi(x_1,\dots,x_n))
\end{equation*}
then \(\varphi\) is absolute for \cm,\cn if and only if \(\psi\) is absolute for \cm,\cn
\end{lemma}


\begin{definition}[]
Suppose \(\cm\subseteq\cn\), \(f(x_1,\dots,x_n)\) is a function. \(f\) is
\tf{absolute} for \cm and \cn if and only if \(\varphi(x_1,\dots,x_n,x_{n+1})\)
defining \(f\) is absolute.
\end{definition}

\begin{theorem}[]
Following relations and functions can be defined in
\(\zfmm-\text{Pow}-\text{Inf}\) and are equivalent to some \(\Delta_0\) formulas.
So they are absolute for any transitive model \cm on 
\(\zfmm-\text{Pow}-\text{Inf}\)
\begin{enumerate}
\item \(x\in y\)
\item \(x=y\)
\item \(x\subset y\)
\item \(\{x,y\}\)
\item \(\{x\}\)

\item \((x,y)\)
\item \(\emptyset\)
\item \(x\cup y\)
\item \(x-y\)
\item \(x\cap y\)
\item \(x^+\)
\item \(x\) is a transitive set
\item \(\bigcup x\)
\item \(\bigcap x\) (\(\bigcap\emptyset=\emptyset\))
\end{enumerate}
\end{theorem}

\begin{lemma}[]
Absoluteness is closed under operation composition
\end{lemma}

\begin{theorem}[]
Following relations and functions are absolute for any transitive model \cm on 
\(\zfmm-\text{Pow}-\text{Inf}\)
\begin{enumerate}
\item \(z\) is an ordered pair
\item \(A\times B\)
\item \(R\) is a relation
\item \(\dom{R}\)
\item \(\ran{R}\)
\item \(f\) is a function
\item \(f(x)\)
\item \(f\) is injective
\end{enumerate}
\end{theorem}
\subsection{Relative consistence of the axiom of foundation}
\label{sec:orgabec2d6}
\begin{lemma}[]
Suppose transitive class \(\cm\models\zfmm-\text{Pow}-\text{inf}\) and
\(\omega\in\cm\), then the axiom of infinity is true in \cm. Hence the axiom of
infinity is true in \wf
\end{lemma}

\begin{theorem}[]
\label{7.5.2}
Let \(T\) be a theory of set theory language and \(\Sigma\) a set of sentences.
Suppose \cm is a class and \(T\vdash\cm\neq\emptyset\), then if
\(\cm\models_T\Sigma\), then
\begin{enumerate}
\item for any sentences \(\varphi\), if \(\Sigma\vdash\varphi\), then
\(T\vdash\varphi^\cm\)
\item if \(T\) is consistent, then so is \(\text{Cn}(\Sigma)\)
\end{enumerate}
\end{theorem}


\begin{theorem}[]
The axiom of foundation is consistent with \zfm.
\end{theorem}

\begin{proof}
By \ref{7.5.2}, let T be \zfm, \(\Sigma\) be \zf and \cm be \wf
\end{proof}

\begin{lemma}[\zfmm]
Suppose transitive model \(\cmm\models\zfmm-\text{Pow}-\text{Inf}\). If
\(X,R\in\cm\) and \(R\) is a well-order on \(X\), then
\begin{equation*}
(R\text{ is a well-order on }X)^\cmm
\end{equation*}
\end{lemma}
\end{document}
\end{document}