% Created 2019-12-11 三 14:47
% Intended LaTeX compiler: pdflatex
\documentclass[11pt]{article}
\usepackage[utf8]{inputenc}
\usepackage[T1]{fontenc}
\usepackage{graphicx}
\usepackage{grffile}
\usepackage{longtable}
\usepackage{wrapfig}
\usepackage{rotating}
\usepackage[normalem]{ulem}
\usepackage{amsmath}
\usepackage{textcomp}
\usepackage{amssymb}
\usepackage{capt-of}
\usepackage{hyperref}
\usepackage{minted}
% TIPS
% \substack{a\\b} for multiple lines text





% pdfplots will load xolor automatically without option
\usepackage[dvipsnames]{xcolor}

\usepackage{forest}
% two-line text in node by [two \\ lines]
% \begin{forest} qtree, [..] \end{forest}
\forestset{
  qtree/.style={
    baseline,
    for tree={
      parent anchor=south,
      child anchor=north,
      align=center,
      inner sep=1pt,
    }}}
%\usepackage{flexisym}
% load order of mathtools and mathabx, otherwise conflict overbrace

\usepackage{mathtools}
%\usepackage{fourier}
\usepackage{pgfplots}
\usepackage{amsthm}
\usepackage{amsmath}
%\usepackage{unicode-math}
%
\usepackage{commath}
%\usepackage{,  , }
\usepackage{amsfonts}
\usepackage{amssymb}
% importing symbols https://tex.stackexchange.com/questions/14386/importing-a-single-symbol-from-a-different-font
%mathabx change every symbol
% use instead stmaryrd
%\usepackage{mathabx}
\usepackage{stmaryrd}
\usepackage{empheq}
\usepackage{tikz}
\usepackage{tikz-cd}
%\usepackage[notextcomp]{stix}
\usetikzlibrary{arrows.meta}
\usepackage[most]{tcolorbox}
%\utilde
%\usepackage{../../latexpackage/undertilde/undertilde}
% left and right superscript and subscript
\usepackage{actuarialsymbol}
\usepackage{threeparttable}
\usepackage{scalerel,stackengine}
\usepackage{stackrel}
% \stackrel[a]{b}{c}
\usepackage{dsfont}
% text font
\usepackage{newpxtext}
%\usepackage{newpxmath}

%\newcounter{dummy} \numberwithin{dummy}{section}
\newtheorem{dummy}{dummy}[section]
\theoremstyle{definition}
\newtheorem{definition}[dummy]{Definition}
\newtheorem{corollary}[dummy]{Corollary}
\newtheorem{lemma}[dummy]{Lemma}
\newtheorem{proposition}[dummy]{Proposition}
\newtheorem{theorem}[dummy]{Theorem}
\theoremstyle{definition}
\newtheorem{example}[dummy]{Example}
\theoremstyle{remark}
\newtheorem*{remark}{Remark}


\newcommand\what[1]{\ThisStyle{%
    \setbox0=\hbox{$\SavedStyle#1$}%
    \stackengine{-1.0\ht0+.5pt}{$\SavedStyle#1$}{%
      \stretchto{\scaleto{\SavedStyle\mkern.15mu\char'136}{2.6\wd0}}{1.4\ht0}%
    }{O}{c}{F}{T}{S}%
  }
}

\newcommand\wtilde[1]{\ThisStyle{%
    \setbox0=\hbox{$\SavedStyle#1$}%
    \stackengine{-.1\LMpt}{$\SavedStyle#1$}{%
      \stretchto{\scaleto{\SavedStyle\mkern.2mu\AC}{.5150\wd0}}{.6\ht0}%
    }{O}{c}{F}{T}{S}%
  }
}

\newcommand\wbar[1]{\ThisStyle{%
    \setbox0=\hbox{$\SavedStyle#1$}%
    \stackengine{.5pt+\LMpt}{$\SavedStyle#1$}{%
      \rule{\wd0}{\dimexpr.3\LMpt+.3pt}%
    }{O}{c}{F}{T}{S}%
  }
}

\newcommand{\bl}[1] {\boldsymbol{#1}}
\newcommand{\Wt}[1] {\stackrel{\sim}{\smash{#1}\rule{0pt}{1.1ex}}}
\newcommand{\wt}[1] {\widetilde{#1}}
\newcommand{\tf}[1] {\textbf{#1}}


%For boxed texts in align, use Aboxed{}
%otherwise use boxed{}

\DeclareMathSymbol{\widehatsym}{\mathord}{largesymbols}{"62}
\newcommand\lowerwidehatsym{%
  \text{\smash{\raisebox{-1.3ex}{%
    $\widehatsym$}}}}
\newcommand\fixwidehat[1]{%
  \mathchoice
    {\accentset{\displaystyle\lowerwidehatsym}{#1}}
    {\accentset{\textstyle\lowerwidehatsym}{#1}}
    {\accentset{\scriptstyle\lowerwidehatsym}{#1}}
    {\accentset{\scriptscriptstyle\lowerwidehatsym}{#1}}
}

\usepackage{graphicx}
    
% text on arrow for xRightarrow
\makeatletter
%\newcommand{\xRightarrow}[2][]{\ext@arrow 0359\Rightarrowfill@{#1}{#2}}
\makeatother


\newcommand{\dom}[1]{%
\mathrm{dom}{(#1)}
}

% Roman numerals
\makeatletter
\newcommand*{\rom}[1]{\expandafter\@slowromancap\romannumeral #1@}
\makeatother

\def \fR {\mathfrak{R}}
\def \bx {\boldsymbol{x}}
\def \bz {\boldsymbol{z}}
\def \ba {\boldsymbol{a}}
\def \bh {\boldsymbol{h}}
\def \bo {\boldsymbol{o}}
\def \bU {\boldsymbol{U}}
\def \bc {\boldsymbol{c}}
\def \bV {\boldsymbol{V}}
\def \bI {\boldsymbol{I}}
\def \bK {\boldsymbol{K}}
\def \bt {\boldsymbol{t}}
\def \bb {\boldsymbol{b}}
\def \bA {\boldsymbol{A}}
\def \bX {\boldsymbol{X}}
\def \bu {\boldsymbol{u}}
\def \bS {\boldsymbol{S}}
\def \bZ {\boldsymbol{Z}}
\def \bz {\boldsymbol{z}}
\def \by {\boldsymbol{y}}
\def \bw {\boldsymbol{w}}
\def \bT {\boldsymbol{T}}
\def \bF {\boldsymbol{F}}
\def \bS {\boldsymbol{S}}
\def \bm {\boldsymbol{m}}
\def \bW {\boldsymbol{W}}
\def \bR {\boldsymbol{R}}
\def \bQ {\boldsymbol{Q}}
\def \bS {\boldsymbol{S}}
\def \bP {\boldsymbol{P}}
\def \bT {\boldsymbol{T}}
\def \bY {\boldsymbol{Y}}
\def \bH {\boldsymbol{H}}
\def \bB {\boldsymbol{B}}
\def \blambda {\boldsymbol{\lambda}}
\def \bPhi {\boldsymbol{\Phi}}
\def \btheta {\boldsymbol{\theta}}
\def \bTheta {\boldsymbol{\Theta}}
\def \bmu {\boldsymbol{\mu}}
\def \bphi {\boldsymbol{\phi}}
\def \bSigma {\boldsymbol{\Sigma}}
\def \lb {\left\{}
\def \rb {\right\}}
\def \la {\langle}
\def \ra {\rangle}
\def \caln {\mathcal{N}}
\def \dissum {\displaystyle\Sigma}
\def \dispro {\displaystyle\prod}
\def \E {\mathbb{E}}
\def \Q {\mathbb{Q}}
\def \N {\mathbb{N}}
\def \V {\mathbb{V}}
\def \R {\mathbb{R}}
\def \P {\mathbb{P}}
\def \A {\mathbb{A}}
\def \Z {\mathbb{Z}}
\def \I {\mathbb{I}}
\def \C {\mathbb{C}}
\def \cala {\mathcal{A}}
\def \calb {\mathcal{B}}
\def \calq {\mathcal{Q}}
\def \calp {\mathcal{P}}
\def \cals {\mathcal{S}}
\def \calg {\mathcal{G}}
\def \caln {\mathcal{N}}
\def \calr {\mathcal{R}}
\def \calm {\mathcal{M}}
\def \calc {\mathcal{C}}
\def \calf {\mathcal{F}}
\def \calk {\mathcal{K}}
\def \call {\mathcal{L}}
\def \calu {\mathcal{U}}
\def \bcup {\bigcup}


\def \uin {\underline{\in}}
\def \oin {\overline{\in}}
\def \uR {\underline{R}}
\def \oR {\overline{R}}
\def \uP {\underline{P}}
\def \oP {\overline{P}}

\def \Ra {\Rightarrow}

\def \e {\enspace}

\def \sgn {\operatorname{sgn}}
\def \gen {\operatorname{gen}}
\def \ker {\operatorname{ker}}
\def \im {\operatorname{im}}

\def \tril {\triangleleft}

% \varprod
\DeclareSymbolFont{largesymbolsA}{U}{txexa}{m}{n}
\DeclareMathSymbol{\varprod}{\mathop}{largesymbolsA}{16}

% \bigtimes
\DeclareFontFamily{U}{mathx}{\hyphenchar\font45}
\DeclareFontShape{U}{mathx}{m}{n}{
      <5> <6> <7> <8> <9> <10>
      <10.95> <12> <14.4> <17.28> <20.74> <24.88>
      mathx10
      }{}
\DeclareSymbolFont{mathx}{U}{mathx}{m}{n}
\DeclareMathSymbol{\bigtimes}{1}{mathx}{"91}
% \odiv
\DeclareFontFamily{U}{matha}{\hyphenchar\font45}
\DeclareFontShape{U}{matha}{m}{n}{
      <5> <6> <7> <8> <9> <10> gen * matha
      <10.95> matha10 <12> <14.4> <17.28> <20.74> <24.88> matha12
      }{}
\DeclareSymbolFont{matha}{U}{matha}{m}{n}
\DeclareMathSymbol{\odiv}         {2}{matha}{"63}


\newcommand\subsetsim{\mathrel{%
  \ooalign{\raise0.2ex\hbox{\scalebox{0.9}{$\subset$}}\cr\hidewidth\raise-0.85ex\hbox{\scalebox{0.9}{$\sim$}}\hidewidth\cr}}}
\newcommand\simsubset{\mathrel{%
  \ooalign{\raise-0.2ex\hbox{\scalebox{0.9}{$\subset$}}\cr\hidewidth\raise0.75ex\hbox{\scalebox{0.9}{$\sim$}}\hidewidth\cr}}}

\newcommand\simsubsetsim{\mathrel{%
  \ooalign{\raise0ex\hbox{\scalebox{0.8}{$\subset$}}\cr\hidewidth\raise1ex\hbox{\scalebox{0.75}{$\sim$}}\hidewidth\cr\raise-0.95ex\hbox{\scalebox{0.8}{$\sim$}}\cr\hidewidth}}}
\newcommand{\stcomp}[1]{{#1}^{\mathsf{c}}}


\author{Qi'ao Chen}
\date{\today}
\title{Notes on Set Theory}
\hypersetup{
 pdfauthor={Qi'ao Chen},
 pdftitle={Notes on Set Theory},
 pdfkeywords={},
 pdfsubject={},
 pdfcreator={Emacs 26.3 (Org mode 9.3)}, 
 pdflang={English}}
\begin{document}

\maketitle
\tableofcontents \clearpage
\section{Foreword}
\label{sec:orgb656399}
Notes for the entrance examination
\section{Models of Set - Sertraline}
\label{sec:orgb898be7}
\subsection{Some mathematical logic}
\label{sec:org963875a}
\begin{theorem}[Gödel’s second incompleteness theorem]
If a consistent recursive axiom set \(T\) contains \(\zfc\), then
\begin{equation*}
T\not\vdash\con{t}
\end{equation*}
especially, \(\zfc\not\vdash\con{\zfc}\)
\end{theorem}

\begin{definition}[]
Suppose \((M,E_M)\) and \((N,E_N)\) are two models of set theory, then
\begin{enumerate}
\item if for any formula \(\sigma\), \(M\models\sigma\) if and only if
\(N\models\sigma\), then \(M\) and \(N\) are \tf{elementary equivalent}, denoted
by \(M\equiv N\)
\item If bijection \(f:M\to N\) satisfies: for any \(a,b\in M\), \(aE_Mb\) iff
\(f(a)E_Nf(b)\), then \(f:M\cong N\) is an \tf{isomorphism}
\item If \(M\subseteq N\) and \(E_M=E_N\restriction M\), then \(M\) is \(N\)'s submodel
\item If \(M\) is isomorphic to a submodel of \(N\) by injection \(f\), and for any
formula \(\varphi(x_1,\dots,x_n)\), for any \(a_1,\dots,a_n\in M\), 
\(M\models\varphi[a_1,\dots,a_n]\) iff
\(N\models\varphi[f(a_1),\dots,f(a_n)]\), then \(f\) is called an
\tf{elementary embedding} from \(M\) to \(N\), written as \(f:M\prec N\)
\item If \(M\subseteq N\) and \(M\prec N\), then \(M\) is a \tf{elementary submodel}
of \(N\)
\end{enumerate}
\end{definition}

\begin{lemma}[]
Suppose \(N\models\zfc,M\subseteq N\), then \(M\prec N\) iff
\(\forall\varphi(x,x_1,\dots,x_n)\), \(\forall(a_1,\dots,a_n)\in M\), if 
\(\exists a\in N\) s.t. \(N\models\varphi[a,a_1,\dots,a_n]\), then \(\exists a\in
  M\) s.t. 
\(M\models\varphi[a,a_1,\dots,a_n]\)
\end{lemma}

\begin{definition}[]
Suppose \((M,E)\models\zfc\)
\begin{enumerate}
\item \(h_\varphi:M^n\to M\) is \(\varphi\)'s \tf{Skolem function} if 
\(\forall a_1,\dots,a_n\in M\), if \(\exists a\in M\) s.t.
\(M\models\varphi[a,a_1,\dots,a_n]\), then
\(M\models\varphi[h_\varphi(a_1,\dots,a_n),a_1,\dots,a_n]\) . requires \ac
\item Let \(\calh=\{h_\varphi\mid\varphi \text{is a formula on set theory}\}\). For
any \(S\subseteq M\), \tf{Skolem hull} \(\calh(S)\) is the smallest set
consisting of \(S\) and closed under \(\calh\)
\end{enumerate}
\end{definition}

\begin{lemma}[]
\(N\models\zfc,S\subseteq N\), if \(M=\calh(S)\), then \(M\prec N\)
\end{lemma}

\begin{theorem}[Löwenheim-Skolem theorem]
Suppose \(N\models\zfc\) and is infinite, then there is a model \(M\) s.t.
\(\abs{M}=\omega\) and \(M\prec N\)
\end{theorem}
\subsection{Cumulative Hierarchy}
\label{sec:org1ae1cdb}
This section works in \zfm(a.k.a. \(\zf-\text{axiom of foundation}\))

\begin{definition}[]
For any \(\alpha\), define sequence \(V_\alpha\)
\begin{enumerate}
\item \(V_0=\emptyset\)
\item \(V_{\alpha+1}=\calp(V_\alpha)\)
\item For any limit ordinal \(\lambda\), \(V_\lambda=\bigcup_{\beta<\lambda}V_\beta\)
\end{enumerate}


And \(\wf=\displaystyle\bigcup_{\alpha\in\on}V_\alpha\)
\end{definition}

\begin{lemma}[]
For any ordinal \(\alpha\)
\begin{enumerate}
\item \(V_\alpha\) is transitive
\item if \(\xi\le\alpha\), then \(V_\xi\subseteq V_\alpha\)
\item if \(\kappa\) is inaccessible cardinal, then \(\abs{V_\kappa}=\kappa\)
\end{enumerate}
\end{lemma}

\begin{proof}
\begin{enumerate}
\item Obviously \(\kappa\le V_\kappa\). Since \(\kappa\) is inaccessible, then for any
\(\alpha<\kappa\), \(\abs{V_\alpha}<\kappa\).
\end{enumerate}
\end{proof}

\begin{definition}[]
For any set \(x\in\wf\), 
\begin{equation*}
\rank(x)=\min\{\beta\mid x\in V_{\beta+1}\}
\end{equation*}
\end{definition}

\begin{lemma}[]
\begin{enumerate}
\item \(V_\alpha=\{x\in\wf\mid\rank(x)<\alpha\}\)
\item \wf is transitive
\item For any \(x,y\in\wf\), if \(x\in y\), then \(\rank(x)<\rank(y)\)
\item for any \(y\in\wf\), \(\rank(y)=\sup\{\rank(x)+1\mid x\in y\}\)
\end{enumerate}
\end{lemma}

\begin{lemma}[]
Supoose \(\alpha\) is an ordinal
\begin{enumerate}
\item \(\alpha\in\wf\) and \(\rank(\alpha)=\alpha\)
\item \(V_\alpha\cap\on=\alpha\)
\end{enumerate}
\end{lemma}

\begin{lemma}[]
\begin{enumerate}
\item If \(x\in\wf\), then \(\bigcup x,\calp(x),\{x\}\in\wf\), and their ranks are
all less than \(\rank(x)+\omega\)
\item If \(x,y\in\wf\), then \(x\times y,x\cup y,x\cap y,\{x,y\},(x,y),x^y\in\wf\),
and their ranks are all less than \(\rank(x)+\rank(y)+\omega\)
\item \(\Z,\Q,\R\in V_{\omega+\omega}\)
\item for any set \(x\), \(x\in\wf\) iff \(x\subset\wf\)
\end{enumerate}
\end{lemma}

\begin{lemma}[]
Suppose \ac
\begin{enumerate}
\item for any group \(G\), there exists group \(G'\cong G\) in \wf
\item for any topological space \(T\), there exists \(T'\cong T\) in \wf
\end{enumerate}
\end{lemma}

\begin{definition}[]
Binary relation \(<\) on set \(A\) is \tf{well-founded} if for any nonempty
\(X\subseteq A\), \(X\) has minimal element under \(<\)
\end{definition}


\begin{theorem}[]
If \(A\in\wf\), then \(\in\) is a well-founded relation on \(A\)
\end{theorem}

\begin{lemma}[]
If set \(A\) is transitive and \(\in\) is well-founded on \(A\), then \(A\in\wf\)
\end{lemma}

\begin{lemma}[]
For any set \(x\), there is a smallest transitive set \(\trcl{x}\) s.t.
\(x\subseteq\trcl{x}\) 
\end{lemma}

\begin{proof}
\begin{align*}
x_0&=x\\
x_{n+1}&=\bigcup x_n\\
\trcl{x}&=\displaystyle\bigcup_{n<\omega}x_n
\end{align*}
\end{proof}

\(\trcl{x}\) is called \tf{transitive closure} of \(x\)


\begin{lemma}[]
Without axiom of power set
\begin{enumerate}
\item if \(x\) is transitive, then \(\trcl{x}=x\)
\item if \(y\in x\), then \(\trcl{y}\subseteq\trcl{x}\)
\item \(\trcl{x}=x\cup\bigcup\{\trcl{y}\mid y\in x\}\)
\end{enumerate}
\end{lemma}

\begin{theorem}[]
For any set \(X\), the following are equivalent
\begin{enumerate}
\item \(X\in\wf\)
\item \(\trcl{X}\in\wf\)
\item \(\in\) is a well-founded relation on \(\trcl{X}\)
\end{enumerate}
\end{theorem}

\begin{theorem}[]
The following propositions are equivalent
\begin{enumerate}
\item Axiom of foundation
\item For any set \(X\), \(\in\) is a well-founded relation on \(X\)
\item \(\tf{V}=\wf\)
\end{enumerate}
\end{theorem}
\subsection{Relativization}
\label{sec:orge9709e2}
\begin{definition}[]
Let \tf{M} be a class \(\varphi\) a formula, the \tf{relativization} of \(\varphi\)
to \tf{M} is \(\varphi^{\tf{M}}\) defined inductively
\begin{align*}
(x\in y)^{\cm}&\leftrightarrow x=y\\
(x\in y)^{\cm}&\leftrightarrow x\in y\\
(\varphi\to\psi)^{\cm}&\leftrightarrow \varphi^{\cm}\to\psi^\cm\\
(\neg\varphi)^\cm&\leftrightarrow\neg\varphi^\cm\\
(\forall x\varphi)^\cm&\leftrightarrow(\forall x\in\cm)\varphi^\cm
\end{align*}
\end{definition}

Note \(\varphi^\cv=\varphi\) and
\begin{equation*}
f^\cm=\{(x_1,\dots,x_n,x_{n+1})\in\cm\mid\varphi^\cm(x_1,\dots,x_n,x_{n+1})\}
\end{equation*}

\begin{definition}[]
For any theory \(T\), any class \(\cm\), \(\cm\models T\) iff for any axiom
\(\varphi\) of \(T\), \(\varphi^\cm\) holds
\end{definition}


\begin{theorem}[\zfm]
\(\wf\models\zf\)
\end{theorem}

\begin{proof}
\begin{itemize}
\item \tf{Axiom of existence}

\((\exists x(x=x))^\cm\leftrightarrow\exists x\in\cm(x=x)\), which is
equivalent to \cm being nonempty
\item \tf{Axiom of extensionality}

\begin{gather*}
\forall X\forall Y\forall u((u\in X\leftrightarrow u\in Y)\to X=Y)^\cm
\Leftrightarrow\\
\forall X\in\cm\forall Y\in\cm\forall u\in\cm
((u\in X\leftrightarrow u\in Y)\to X=Y)
\end{gather*}

\begin{lemma}
If $\cm$ is transitive, then axiom of extensionality holds in \cm
\end{lemma}

\item \tf{Axiom schema of specification}

\begin{equation*}
\forall X\in\cm\exists Y\in\cm\forall u\in\cm(u\in Y\leftrightarrow
u\in X\wedge\varphi^\cm(u))
\end{equation*}

Since for any \(X\in\wf\), \(\calp(X)\subseteq \wf\)
\item \tf{Axiom of paring}
\item \tf{Axiom of union}
\item \tf{Axiom of power set}

\begin{equation*}
\forall X\in\cm\exists Y\in\cm\forall u\in\cm(u\in Y\leftrightarrow(u\subseteq X)^\cm)
\end{equation*}
and 
\begin{equation*}
(u\subseteq X)^\cm\leftrightarrow\forall x\in\cm(x\in u\to x\in X)
\leftrightarrow u\cap\cm\subseteq X
\end{equation*}
\item \tf{Axiom of foundation}
\item \tf{Axiom schema of replacement}
\end{itemize}
\end{proof}
\subsection{Absoluteness}
\label{sec:org6e343ed}
\begin{definition}[]
For any formula \(\psi(x_1,\dots,x_n)\) and any class \cm,\cn, 
\(\cm\subseteq \cn\), if
\begin{equation*}
\forall x_1\dots\forall x_n\in\cm(\psi^\cm(x_1,\dots,x_n)
\leftrightarrow\psi^\cn(x_1,\dots,x_n))
\end{equation*}
then \(\psi(x_1,\dots,x_n)\) is \tf{absolute} for \cm,cn. If \(\cn=\cv\), then
\(\psi\) is \tf{absolute} for \cm
\end{definition}

\begin{lemma}[]
Suppose \(\cm\subseteq\cn\) and \(\varphi\),\(\psi\) are formulas, then
\begin{enumerate}
\item if \(\varphi\),\(\psi\) are absolute for \cm,cn, then so are
\(\neg\varphi,\varphi\to\psi\)
\item if \(\varphi\) doesn't contain any quantifiers, then \(\varphi\) is absolute for
any \cm
\item if \cm,\cn  are transitive and \(\varphi\) is absolute for them, then so are
\(\forall x\in y\varphi\)
\end{enumerate}
\end{lemma}

\begin{definition}[]
\(\Delta_0\) formula
\begin{enumerate}
\item \(x=y,x\in y\) are \(\Delta_0\) formulas
\item if \(\varphi\),\(\psi\) are \(\Delta_0\), then so are \(\neg\varphi,\varphi\to\psi\)
\item if \(\varphi\) is \(\Delta_0\), \(y\) is any set, then \((\forall x\in y)\varphi\)
is \(\Delta_0\)
\end{enumerate}


If \(\varphi\) is \(\Delta_0\), then \(\exists x_1\dots\exists x_n\varphi\) is
\(\Sigma_1\) formula, \(\forall x_1\dots\forall x_n\varphi\) is \(\Pi_1\)
\end{definition}

\begin{lemma}[]
\(\cm\subseteq\cn\) are both transitive, \(\psi(x_0,\dots,x_n)\) is a formula,
then
\begin{enumerate}
\item if \(\psi\) is \(\Delta_0\), then it's absolute for \cm,cn
\item if \(\psi\) is \(\Sigma_1\), then
\begin{equation*}
\forall x_1\dots x_n(\psi^\cm(x_1,\dots,x_n)\to\psi^\cn(x_1,\dots,x_n))
\end{equation*}
\item if \(\psi\) is \(\Pi_1\), then
\begin{equation*}
\forall x_1\dots x_n(\psi^\cn(x_1,\dots,x_n)\to\psi^\cm(x_1,\dots,x_n))
\end{equation*}
\end{enumerate}
\end{lemma}

\begin{lemma}[]
If \(\cm\subseteq\cn\), \(\cm\models\Sigma,\cn\models\Sigma\) and
\begin{equation*}
\Sigma\vdash\forall x_1\dots\forall x_n(\varphi(x_1,\dots,x_n)\leftrightarrow
\psi(x_1,\dots,x_n))
\end{equation*}
then \(\varphi\) is absolute for \cm,\cn if and only if \(\psi\) is absolute for \cm,\cn
\end{lemma}


\begin{definition}[]
Suppose \(\cm\subseteq\cn\), \(f(x_1,\dots,x_n)\) is a function. \(f\) is
\tf{absolute} for \cm and \cn if and only if \(\varphi(x_1,\dots,x_n,x_{n+1})\)
defining \(f\) is absolute.
\end{definition}

\begin{theorem}[]
Following relations and functions can be defined in
\(\zfmm-\text{Pow}-\text{Inf}\) and are equivalent to some \(\Delta_0\) formulas.
So they are absolute for any transitive model \cm on 
\(\zfmm-\text{Pow}-\text{Inf}\)
\begin{enumerate}
\item \(x\in y\)
\item \(x=y\)
\item \(x\subset y\)
\item \(\{x,y\}\)
\item \(\{x\}\)

\item \((x,y)\)
\item \(\emptyset\)
\item \(x\cup y\)
\item \(x-y\)
\item \(x\cap y\)
\item \(x^+\)
\item \(x\) is a transitive set
\item \(\bigcup x\)
\item \(\bigcap x\) (\(\bigcap\emptyset=\emptyset\))
\end{enumerate}
\end{theorem}

\begin{lemma}[]
Absoluteness is closed under operation composition
\end{lemma}

\begin{theorem}[]
Following relations and functions are absolute for any transitive model \cm on 
\(\zfmm-\text{Pow}-\text{Inf}\)
\begin{enumerate}
\item \(z\) is an ordered pair
\item \(A\times B\)
\item \(R\) is a relation
\item \(\dom{R}\)
\item \(\ran{R}\)
\item \(f\) is a function
\item \(f(x)\)
\item \(f\) is injective
\end{enumerate}
\end{theorem}
\subsection{Relative consistence of the axiom of foundation}
\label{sec:org3377b0c}
\begin{lemma}[]
Suppose transitive class \(\cm\models\zfmm-\text{Pow}-\text{inf}\) and
\(\omega\in\cm\), then the axiom of infinity is true in \cm. Hence the axiom of
infinity is true in \wf
\end{lemma}

\begin{theorem}[]
\label{7.5.2}
Let \(T\) be a theory of set theory language and \(\Sigma\) a set of sentences.
Suppose \cm is a class and \(T\vdash\cm\neq\emptyset\), then if
\(\cm\models_T\Sigma\), then
\begin{enumerate}
\item for any sentences \(\varphi\), if \(\Sigma\vdash\varphi\), then
\(T\vdash\varphi^\cm\)
\item if \(T\) is consistent, then so is \(\text{Cn}(\Sigma)\)
\end{enumerate}
\end{theorem}


\begin{theorem}[]
The axiom of foundation is consistent with \zfm.
\end{theorem}

\begin{proof}
By \ref{7.5.2}, let T be \zfm, \(\Sigma\) be \zf and \cm be \wf
\end{proof}

\begin{lemma}[$\zfmm$]
Suppose transitive model \(\cmm\models\zfmm-\text{Pow}-\text{Inf}\). If
\(X,R\in\cm\) and \(R\) is a well-order on \(X\), then
\begin{equation*}
(R\text{ is a well-order on }X)^\cmm
\end{equation*}
\end{lemma}

\begin{theorem}[$\zfmm$]
\(V_\omega\models\zfc-\text{Inf}+\neg\text{Inf}\)
\end{theorem}
\begin{proof}
For any \(X\in V_\omega\), \(X\) is finite hence there is a well-ordering on \(X\)
\end{proof}

\begin{corollary}
$\con{\zfmm}\to\con{\zfc-\text{Inf}+\neg\text{Inf}}$
\end{corollary}
\subsection{Induction and recursion based on well-order relation}
\label{sec:org3ea4e84}
\begin{definition}[]
\(\bR\) is a well-founded relation on \(\bX\) if and only if
\begin{equation*}
\forall U\subset\bX(U\neq\emptyset\to\exists y\in U(\neg\exists z\in U(z\bR y)))
\end{equation*}
\end{definition}


\begin{definition}[]
Relation \(\bR\) is \tf{set-like} on \(\bX\) iff for any \(x\in\bX\),\par
\(\{y\in\bX\mid y\bR x\}\) is a set
\end{definition}

\begin{definition}[]
If \(\bR\) is a set-like relation on \(\bX\) and \(x\in \bX\), define
\begin{align*}
\pred^0(\bX,x,\bR)&=\{y\in\bX\mid y\bR x\}\\
\pred^{n+1}(\bX,x,bR)&=\bigcup\{\pred(\bX,y,\bR)\mid y\in\pred^n(\bX,x,\bR)\}\\
\cl(\bX,x,\bR)&=\displaystyle\bigcup_{n\in\omega}\pred^n(\bX,x,\bR)
\end{align*}
\end{definition}

\begin{lemma}[]
If \(\bR\) is a set-like relation on \(\bX\), then for any \(y\in\cl(\bX,x,\bR)\),
\(\pred(\bX,y,\bR)\subseteq\cl(\bX,x,\bR)\)
\end{lemma}

\begin{theorem}[Induction on well-founded set-like relation]
If \(\bR\) is a well-founded set-like relation on \(\bX\), then every nonempty 
\(\bY\subseteq\bX\) has minimal element under \(\bR\)
\end{theorem}

\begin{theorem}[]
Suppose \(\bR\) is a well-founded set-like relation on \(\bX\). If 
\(\bF:\bX\times\bV\to\bV\), then there is a unique \(\bG:\bX\to\bV\) s.t.
\begin{equation*}
\forall x\in\bX(\bG(x)=\bF(x,\bG\restriction\pred(\bX,x,\bR)))
\end{equation*}
\end{theorem}

\begin{definition}[]
If \(\bR\) is a set-like well-founded relation on \(\bX\), define 
\begin{equation*}
\rank(x,\bX,\bR)=\sup\{\rank(y,\bX,\bR)+1\mid y\bR x\wedge y\in\bX\}
\end{equation*}
\end{definition}

Note that
\begin{equation*}
\bF(x,h)=\sup\{\alpha+1\mid\alpha\in\ran{h}\}
\end{equation*}

\begin{lemma}[$\zfmm$]
If \(\bX\) is transitive and \(\in\) is well-founded on \(\bX\), then
\(\bX\subseteq\wf\) and for any \(x\in\bX\), \(\rank(x,\bX,\in)=\rank(x)\)
\end{lemma}

\begin{definition}[]
\(\bR\) is a set-like well-founded relation on \(\bX\), \tf{Mostowski function}
\(\bG\) on \((\bX,\bR)\) is 
\begin{equation*}
\bG(x)=\{\bG(y)\mid y\in\bX\wedge y\bR x\}
\end{equation*}
\(\cmm=\ran{\bG}\) is called the \tf{Mostowski collapse} of \((\bX,\bR)\)
\end{definition}

\begin{lemma}[]
\begin{enumerate}
\item \(\forall x,y\in\bX(x\bR y\to\bG(x)\in\bG(y))\)
\item \cm is transitive
\item If the axiom of power set holds, \(\cm\subseteq\wf\)
\item if the axiom of power set holds and \(x\in\bX\), then\par
\(\rank(x,\bX,\bR)=\rank(\bG(x))\)
\end{enumerate}
\end{lemma}

\begin{definition}[]
\(\bR\) is extensional on \(\bX\) iff
\begin{equation*}
\forall x,y\in\bX(\forall z\in\bX(z\bR x\leftrightarrow z\bR y)\to x=y)
\end{equation*}
\end{definition}

\begin{lemma}[]
If \(\bX\) is transitive then \(\in\) is extensional on \(\bX\)
\end{lemma}


\begin{lemma}[]
Let \(\bR\) be a set-like well-founded relation on \(\bX\), \(\bG\) is a Mostowski
function on it. If \(\bR\) is extensional, then \(\bG\) is an isomorphism
\end{lemma}

\begin{theorem}[Mostowski collapse theorem]
Suppose \(\bR\) is set-like well-founded extensional on \(\bX\), then there are
unique transitive class \cm and bijection \(\bG:\bX\to\cm\) s.t. 
\(\bG:(\bX,\bR)\cong(\cm,\in)\)
\end{theorem}
\subsection{Absoluteness under the axiom of foundation}
\label{sec:orgeb7f2d3}
\begin{theorem}[]
The following relations and functions can be defined by formulas in
\(\zf-\text{Pow}\) and are equivalent to some \(\Delta_0\) formulas
\begin{enumerate}
\item \(x\) is an ordinal
\item \(x\) is a limit ordinal
\item \(x\) is a successor ordinal
\item \(\omega\)
\item \(x\) is a finite ordinal
\item \(0,1,2,\dots,20,\dots\)
\end{enumerate}
\end{theorem}

\begin{theorem}[]


If transitive model \(\cm\models\zf-\text{Pow}\), then every finite subset of
\cm belongs to \cm
\end{theorem}

\begin{proof}
prove 
\begin{equation*}
\forall x\subset\cm(\abs{x}=n\to x\in\cm)
\end{equation*}
\end{proof}

\begin{theorem}[]
The following concepts are absolute for any transitive model of
\(\zf-\text{Pow}\) 
\begin{enumerate}
\item \(x\) is finite
\item \(X^n\)
\item \(X^{<\omega}\)
\item \(R\) is a well-ordering on \(X\)
\item \(\text{type}(X,R)\)
\item \(\alpha+1\)
\item \(\alpha-1\)
\item \(\alpha+\beta\)
\item \(\alpha\cdot\beta\)
\end{enumerate}
\end{theorem}


Class \(\bX\) is in fact a formula \(\bX(x)\). It's absolute for \cm if and only
if \(\forall x\in\cm(\bX^\cm(x)\leftrightarrow\bX(x))\), which is equivalent to
\(\{x\in\cm\mid\bX(x)\}=\{x\in\cm\mid\bX^\cm(x)\}\). Hence \(\bX\) is absolute
for \cm if and only if \(\bX^\cm=\cm\cap\bX\)

\begin{theorem}[]
Suppose \(\bR\) is a well-founded set-like relation on \(\bX\),
\(\bF:\bX\times\bV\to\bV\),
\begin{equation*}
\forall x\in\bX(\bG(x)=\bF(x,\bG\restriction(\bX,x,\bR)))
\end{equation*}
transitive model \(\cm\models\zf-\text{Pow}\) and
\begin{enumerate}
\item \(\bF\) is absolute for \cm
\item \(\bX,\bR\) are absolute for \cm, \((\bR\text{ is set-like on }\bX)^\cm\) and
\begin{equation*}
\forall x\in\cm(\pred(\bX,x,\bR)\subseteq\cm)
\end{equation*}
\end{enumerate}


then \(\bG\) is absolute for \(\cm\)
\end{theorem}

\begin{theorem}[]
The following concept is absolute for any transitive model of
\(\zf-\text{Pow}\)
\begin{enumerate}
\item \(\alpha^\beta\)
\item \(\rank(x)\)
\item \(\trcl{x}\)
\end{enumerate}
\end{theorem}

\begin{lemma}[]
transitive \(\cm\models\zf\)
\begin{enumerate}
\item if \(x\in\cm\), then \(\calp^\cm(x)=\calp(x)\cap\cm\)
\item if \(\alpha\in\cm\), then \(V_\alpha^\cm=V_\alpha\cap\cm\)
\end{enumerate}
\end{lemma}
\subsection{Unaccessible cardinal and models of \zfc}
\label{sec:org175eb1f}
\(\bZ=\zff-\text{Rep},\zfmm=\zfcm-\text{Rep}\)
\begin{theorem}[]
If \(\gamma>\omega\) is a limit ordinal, then \(V_\gamma\models_{\zff}\bZ\) and 
\(V_\gamma\models_{\zfcm}\zc\)
\end{theorem}

\begin{corollary}[]
\(V_{\omega+\omega}\) doesn't satisfies the axiom of replacement
\end{corollary}

\begin{proof}
   
\end{proof}

\begin{theorem}[]
\(\zcm\not\vdash\exists x(x=V_\omega),\zcm\not\vdash\forall x\exists y(\trcl{x}=y)\)
\end{theorem}

\begin{theorem}[]
If \(\kappa\) is an inaccessible cardinal, then \(V_\kappa\models_{\zfmm}\zff\),\par
\(V_\kappa\models_{\zfcc}\zfc\)
\end{theorem}

\begin{proof}
Since \(\kappa\) is inaccessible, \(\abs{V_\kappa}=\kappa\). For any \(A\in
   V_\kappa\), \(\abs{A}<\kappa\). Since \(\kappa\) is regular, any 
\(f:A\to V_\kappa\) is bounded. Hence there exists \(\alpha<\kappa\) s.t. 
\(\ran{f}\subseteq V_\alpha\)
\end{proof}

\begin{corollary}[]
We cannot prove "there is some inaccessible cardinals" in \zfc
\end{corollary}

\begin{proof}
Suppose we could. Then we have \(V_\kappa\models\zfc\), which contradicts
Gödel’s second incompleteness theorem 
\end{proof}

\begin{lemma}[]
Suppose \(\kappa\) is inaccessible. The following concepts are absolute for
\(V_\kappa\) 
\begin{enumerate}
\item \(x\) is a cardinal
\item \(x\) is a regular cardinal
\item \(x\) is an inaccessible cardinal
\end{enumerate}
\end{lemma}

\begin{lemma}[]
\(\con(\zfcm)\to\con(\zfcm+\text{"there is no inaccessible cardinal"})\)
\end{lemma}

\begin{proof}
If \(\kappa\) is the smallest inaccessible cardinal, then \par
\(V_\kappa\models\zfcm+\text{"there is no inaccessible cardinal"}\). Define
\begin{equation*}
\cm=\bigcap\{V_\kappa\mid\kappa\text{ is inaccessible}\}
\end{equation*}
\end{proof}
If there are, then \(\cm=V_\kappa\)

\begin{corollary}[]
\con(\zfcm)\textlnot{}\(\to\)\con(\zfcm+\text{"there are some inaccessible cardinals"})
\end{corollary}

\begin{definition}[]
For any infinite cardinal \(\kappa\), \(H_\kappa=\{x\mid\abs{\trcl{x}}<\kappa\}\)
is the collection of sets which \tf{hereditarily have size less than } \(\kappa\).
Element of \(H_\omega\) is called \tf{hereditarily finite set}. Element of
\(H_{\omega_1}\) is called \tf{hereditarily countable set}
\end{definition}

\begin{lemma}[]
For any infinite cardinal \(\kappa\), \(H_\kappa\subseteq V_\kappa\)
\end{lemma}

\begin{lemma}[]
If \(\kappa\) is regular, then \(H_\kappa=V_\kappa\) if and only if \(\kappa\) is
inaccessible
\end{lemma}

\begin{proof}
which implies \(\abs{V_\kappa}=\kappa\)
\end{proof}

\begin{lemma}[]
For any infinite cardinal \(\kappa\)
\begin{enumerate}
\item \(H_\kappa\) is transitive
\item \(H_\kappa\cap\on=\kappa\)
\item If \(x\in H_\kappa\), then \(\bigcup x\in H_\kappa\)
\item If \(x,y\in H_\kappa\), then \(\{x,y\}\in H_\kappa\)
\item If \(x\in H_\kappa,y\subseteq x\), then \(y\in H_\kappa\)
\item if \(\kappa\) is regular, then \(\forall x(x\in H_\kappa\leftrightarrow
      x\subset H_\kappa\wedge\abs{x}<\kappa)\)
\end{enumerate}
\end{lemma}

\begin{theorem}[]
If \(\kappa\) is uncountable regular cardinal, then
\(H_\kappa\models_{\zfcm}\zfcm-\text{Pow}\) 
\end{theorem}

\begin{theorem}[]
If \(\kappa\) is uncountable regular cardianl, then the following propositions
are equivalent
\begin{enumerate}
\item \(H_\kappa\models\zfcm\)
\item \(H_\kappa=V_\kappa\)
\item \(\kappa\) is inaccessible
\end{enumerate}
\end{theorem}

\begin{corollary}[]
\(\con(\zfcm)\to\con(\zfcm-\text{pow}+\forall x(x\text{ is countable}))\)
\end{corollary}
\subsection{Reflection theorem}
\label{sec:org66f54d9}
\begin{lemma}[]
\(\cm\subseteq\cn\) are classes. \(\varphi_1,\dots,\varphi_n\) is a sequence
closed under subformula, then the following propositions are equivalent
\begin{enumerate}
\item \(\varphi_1,\dots,\varphi_n\) are absolute for \cm and \cn
\item if \(\varphi_i=\exists\varphi_j(x,y_1,\dots,y_m)\), then
\begin{equation*}
\forall y_1,\dots,y_m\in\cm(\exists x\in\cn\varphi_j^\cn(x,y_1,\dots,y_m)
\to\exists x\in\cm\varphi_j^\cm(x,y_1,\dots,y_m))
\end{equation*}
\end{enumerate}
\end{lemma}

\begin{theorem}[reflection theorem(\zff)]
For any finite formula set \(F=\{\varphi_1,\dots,\varphi_n\}\), for any
\(V_\alpha\), there exists \(V_\beta\) s.t. \(V_\alpha\subseteq V_\beta\) and 
\(\varphi_1,\dots,\varphi_n\) are absolute for \(V_\beta\)
\end{theorem}

\begin{corollary}[\zff]
\(F=\{\sigma_1,\dots,\sigma_n\}\) are finite subsets of \zf, then
\begin{equation*}
\forall\alpha\exists\beta>\alpha(\sigma_1^{V_\beta}\wedge\dots\wedge\sigma_n^{V_\beta})
\end{equation*}
\end{corollary}

\begin{corollary}[]
\(F=\{\sigma_1,\dots,\sigma_n\}\) is a finite subset of \zf. Unless \zf is
unconsistent, \(F\) cannot prove all axioms of \zf
\end{corollary}

\begin{theorem}[\zfcm]
For any finite formula set \(F=\{\varphi_1,\dots,\varphi_n\}\), for any set
\(N\), there exists set \(M\) s.t.
\begin{enumerate}
\item \(N\subseteq M\)
\item \(\varphi_1,\dots,\varphi_n\) are absolute for \((M,\in)\)
\item \(\abs{M}\le\abs{N}\cdot\omega\)
\end{enumerate}
\end{theorem}

\begin{corollary}[\zfcm]
For any finite formula set \(F=\{\varphi_1,\dots,\varphi_n\}\), for any set
\(N\), there exists set \(M\) s.t.
\begin{enumerate}
\item \(N\subseteq M\)
\item \(\varphi_1,\dots,\varphi_n\) are absolute for \((M,\in)\)
\item \(\abs{M}\le\abs{N}\cdot\omega\)
\item \(M\) is transitive
\end{enumerate}
\end{corollary}
\newpage
\section{Constructable Set - Venlafaxine}
\label{sec:org1224083}
\subsection{Definablity and Gödel operation}
\label{sec:org35538ef}
\begin{definition}[]
\(M\) is a set, \(\psi(x_1,\dots,x_n,y_1,\dots,y_m)\) is a formula, 
\(X\subseteq M^n\) is \tf{definable in $M$ from parameters from $\psi$} if and
only if there are \(y_1,\dots,y_m\in M\) s.t.
\begin{equation*}
X=\{(x_1,\dots,x_n)\mid(\psi^M(x_1,\dots,x_n,y_1,\dots,y_m))\}
\end{equation*}
\begin{equation*}
\deff(M)=\{X\subseteq M\mid\exists\psi,X\text{ is definable in } 
M \text{ from } \psi\} 
\end{equation*}
\end{definition}

\begin{definition}[]
\tf{Gödel operation}
\begin{enumerate}
\item \(G_1(X,Y)=\{X,Y\}\)
\item \(G_2(X,Y)=X\times Y\)
\item \(G_3(X,Y)=\in\restriction X\times Y\)
\item \(G_4(X,Y)=X-Y\)
\item \(G_5(X,Y)=X\cap Y\)
\item \(G_6(X,Y)=\bigcap X\)
\item \(G_7(X,Y)=\dom{X}\)
\item \(G_8(X,Y)=\{(x,y)\mid(y,x)\in X\}\)
\item \(G_9(X,Y)=\{(x,y,z)\mid(x,z,y)\in X\}\)
\item \(G_{10}(X,Y)=\{(x,y,z)\mid(y,z,x)\in X\}\)
\end{enumerate}


Class \(C\) is closed under Gödel operation if for any \(X,Y\), X,Y\(\in\) C\$ implies
\(G_i(X,Y)\in C\). For any set \(M\), \(\cl_G(M)\) is the 
\tf{closure under Gödel operation}
\end{definition}

\begin{definition}[]
\(\psi\) is a \tf{normal form} if
\begin{enumerate}
\item only \(\neg,\wedge,\exists\) are logical symbol
\item = doesn't appear
\item if \(x_i\in x_j\) then \(i\neq j\)
\item \(\exists\) only shown as: \(\exists x_{m+1}\in
      x_i\varphi(x_1,\dots,x_{m+1})\), \(1\le i\le m\)
\end{enumerate}
\end{definition}

\begin{lemma}[]
Any \(\Delta_0\) formula can be transformed into normal form
\end{lemma}

\begin{theorem}[]
For any \(\Delta_0\) formula \(\psi(x_1,\dots,x_n)\), there is Gödel operations'
composition \(G\) s.t. for any \(X_1,\dots,X_n\)
\begin{align*}
G(X_1,\dots,X_n)=&\{(x_1,\dots,x_n)\mid\\
&x_1\in X_1\wedge\dots\wedge x_n\in X_n\wedge\psi(x_1,\dots,x_n)\}
\end{align*}
\end{theorem}

\begin{corollary}[]
If \(M\) is transitive and \(M=\cl_G(M)\), then for any \(\Delta_0\) formula
\(\psi(x,y_1,\dots,y_m)\), any set \(X\in M\), any \(y_1,\dots,y_m\in M\) if
\begin{equation*}
Y=\{x\in X\mid\psi(x,y_1,\dots,y_m)\}
\end{equation*}
then \(Y\in M\). Hence \(\Delta_0\) schema of specification holds in \(M\)
\end{corollary}

\begin{lemma}[]
If \(G(X_1,\dots,X_n)\) is Gödel operations' composition, then
\(Z=G(X_1,\dots,X_n)\) is equivalent to a \(\Delta_0\) formula
\end{lemma}

\begin{theorem}[]
For any transitive set \(M\), \(\deff(M)=\cl_G(M\cup\{M\})\cap\calp(M)\)
\end{theorem}

\begin{lemma}[]
If transitive \(\cm\models\zff\), then for any transitive set \(M\in\cm\),
\(\deff(M)\) is absolute for \cm
\end{lemma}

\begin{lemma}[]
For any transitive set \(M\)
\begin{enumerate}
\item \(\deff(M)\subseteq\calp(M)\)
\item \(M\subseteq \deff(M)\)
\item for any \(X\subseteq M\), if \(X\) is finite, then \(X\in\deff(M)\)
\item assume \(\ac\) and \(\abs{M}\ge\omega\), then \(\abs{\deff(M)}=\abs{M}\)
\end{enumerate}
\end{lemma}
\subsection{Gödel's L}
\label{sec:org8a3e572}
\begin{definition}[]
for any \(\alpha\)
\begin{enumerate}
\item \(L_0=\emptyset\)
\item \(L_{\alpha+1}=\deff(L_\alpha)\)
\item For any limit \(\alpha\), \(L_\alpha=\bigcup_{\beta<\alpha}L_\beta\)
\end{enumerate}


\(\textbf{L}=\displaystyle\bigcup_{\alpha\in\on}L_\alpha\). Element of \gl is
called constructible set
\end{definition}

\begin{lemma}[]
For any ordinal \(\alpha\)
\begin{enumerate}
\item \(L_\alpha\) is transitive
\item If \(\alpha<\beta\), then \(L_\alpha\subseteq L_\beta\)
\item \(L_\alpha\subseteq V_\alpha\)
\end{enumerate}
\end{lemma}

\begin{definition}[]
\(x\in\gll\)
\begin{equation*}
\rank_\gll(x)=\min\{\beta\mid x\in\gll_{\beta+1}\}
\end{equation*}
\end{definition}

\begin{lemma}[]
For any \(\alpha\)
\begin{equation*}
L_\alpha=\{x\in\gll\mid\rank_{\gll}(x)<\alpha\}
\end{equation*}
\end{lemma}

\begin{lemma}[]
For any ordinal \(\alpha\)
\begin{enumerate}
\item \(L_\alpha\cap\on=\alpha\)
\item \(\alpha\in\gll\cap\rank_{\gll}(\alpha)=\alpha\)
\end{enumerate}
\end{lemma}

\begin{proof}
since "\(\alpha\) is a cardinal" is absolute for any transitive set. 
\begin{align*}
\alpha&=L_\alpha\cap\on=\{\eta\in L_\alpha\mid\eta\text{ is a ordinal}\}\\
&=\{\eta\in L_\alpha\mid(\eta\text{ is an ordinal}^{L_\alpha})\}\in\deff(L_\alpha)
\end{align*}
\end{proof}

\begin{lemma}[]
for any ordinal \(\alpha\)
\begin{enumerate}
\item \(L_\alpha\in L_{\alpha+1}\)
\item any finite subset of \(L_\alpha\) belongs to \(L_{\alpha+1}\)
\end{enumerate}
\end{lemma}

\begin{lemma}[]
\begin{enumerate}
\item \(\forall n\in\omega(L_n=V_n)\)
\item \(L_\omega=V_\omega\)
\end{enumerate}
\end{lemma}

\begin{lemma}[]
If \ac, then for any \(\alpha\ge\omega,\abs{L_\alpha}=\abs{\alpha}\)
\end{lemma}

\begin{theorem}[]
\(\gll\models\zff\)
\end{theorem}

\subsection{Axiom of constructibility and relativization}
\label{sec:org712240e}
\begin{theorem}[Axiom of constructibility]
\(\cvm=\gll\)
\end{theorem}

\begin{lemma}[]
\label{8.3.2}
function \(\alpha\mapsto L_\alpha\) is absolute for any transitive model of \zf
\end{lemma}

\begin{theorem}[]
\(\gll\models\zff+\cv=\gll\)
\end{theorem}

\begin{proof}
\((\cv=\gll)^\gll\) is \(\forall x\in\gll\exists\alpha\in\gll(x\in
   L_\alpha)^\gll\). 
By \ref{8.3.2}, \((x\in L_\alpha)^\gll\Leftrightarrow x\in L_\alpha\). Hence
\(\gll\models\cv=\gll\)
\end{proof}
Hence
\begin{theorem}[]
\(\con(\zff)\to\con(\zff+\cv=\gll)\)
\end{theorem}

\begin{theorem}[]
Suppose transitive proper class \(\cmm\models\zff-\text{Pow}\), then\par
\(\gll=\gll^\cmm\subseteq\cmm\) 
\end{theorem}

\begin{proof}
For any ordinal \(\alpha\), since \cm is proper, \(\cmm\not\subseteq V_\alpha\).
Hence there is \(x\in\cmm\) s.t. \(\rank(x)\ge\alpha\). Since rank is absolute,
\(\rank(x)\in\cmm\). And \cm is transitive, hence \(\alpha\in\cmm\). By
\ref{8.3.2}, \(L_\alpha\in\cmm\)
\begin{align*}
 \gll^\cmm &=\{x\in\cmm\mid(\exists\alpha\in\on(x\in L_\alpha))^\cmm\}\\
 &=\{x\mid\exists\alpha\in\on\cap\cmm(x\in\ L_\alpha\cap\cmm)\}\\
 &=\{x\mid\exists\alpha\in\on(x\in L_\alpha)\}\\
 &=\gll
\end{align*}
\end{proof}

\begin{definition}[]
If transitive model \(\cm\models\zff\) contains all ordinals, then it's an
\tf{inner model}
\end{definition}

\begin{lemma}[]
there is a finite set of axioms \(\{\psi_1,\dots,\psi_n\}\) of
\(\zff-\text{Pow}\) s.t. ordinals,rank and \(L_\alpha\) are absolute for any
model of \(\{\psi_1,\dots,\psi_n\}\)
\end{lemma}

\begin{lemma}[]
If set \(M\) is transitive, then \(M\cap\on\) is a ordinall and is the least that
doesn't belong to \(M\), denoted by \(\alpha^M\)
\end{lemma}

\begin{theorem}[]
There is a finite subset \(\{\psi_1,\dots,\psi_n\}\) of axioms of
\(\zff-\text{Pow}\) satisfying
\begin{equation*}
\forall M(M\text{ is transitive }\wedge\psi_1^M\wedge\dots\wedge\psi_n^m\to
(L_{\alpha^M}=\gll^M\subseteq M))
\end{equation*}
\end{theorem}

\begin{theorem}[]
The is a finite subset \(\{\psi_1,\dots,\psi_{n+1}\}\) of axioms of
\(\zff-\text{Pow}+\cvm=\gll\) satisfying
\begin{enumerate}
\item If \cm is a transitive proper class and
\(\psi_1^\cm\wedge\dots\wedge\psi_{n+1}^\cm\), then \(\cm=\gll\)
\item \(\forall M(M\text{ is transitive
      }\wedge\psi_1^M\wedge\dots\wedge\psi_n^m\to (L_{\alpha^M}=M))\)
\end{enumerate}
\end{theorem}

\begin{theorem}[]
There is a well-ordering on \gll. Hence \(\cvm=\gll\to\ac\)
\end{theorem}

If \(\cvm=\gll\), hence \(\aleph_\alpha\subseteq L_{\aleph_{\alpha+1}}\). Because
\(\abs{L_{\alpha_{\alpha+1}}}=\aleph_{\alpha+1}\),
\(2^{\aleph_\alpha}\le\aleph_{\alpha+1}\) 
\begin{theorem}[]
If \(\cvm=\gll\), then for any infinite ordinal \(\alpha\),
\(\calp(L_\alpha)\subseteq L_{\abs{\alpha}^+}\)
\end{theorem}

\begin{corollary}[$\zff$]
\((\ac+\gchh)^\gll\)
\end{corollary}

\begin{theorem}[$\zff$]
\(\con(\zff)\to\con(\zfcm+\gchh)\)
\end{theorem}

\begin{theorem}[$\zff$]
Suppose \(S_0=\{\psi_1,\dots,\psi_n\}\subseteq\zff+\cvm=\gll\), then
\begin{equation*}
\zff\vdash\exists M(\abs{M}=\omega\wedge M\text{ is transitive}\wedge
(\psi_1^M\wedge\dots\wedge\psi_n^M))
\end{equation*}
\end{theorem}

\begin{lemma}[]
Suppose \(\cvm=\gll\). For any uncountable regular cardinal \(\kappa\),
\(L_\kappa=H_\kappa\) 
\end{lemma}

\begin{corollary}[]
If \(\kappa\) is a uncountable regular cardinal, then
\(\L_\kappa\models\zff-\text{Pow}+\cvm=\gll\). If \(\kappa\) is inaccessible, then 
\(L_\kappa\models\zff+\cvm=\gll\)
\end{corollary}

\newpage

\section{The end}
\label{sec:org6ea1888}
Learn and forget
\end{document}