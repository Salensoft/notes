% Created 2019-12-23 一 21:39
% Intended LaTeX compiler: pdflatex
\documentclass[11pt]{article}
\usepackage[utf8]{inputenc}
\usepackage[T1]{fontenc}
\usepackage{graphicx}
\usepackage{grffile}
\usepackage{longtable}
\usepackage{wrapfig}
\usepackage{rotating}
\usepackage[normalem]{ulem}
\usepackage{amsmath}
\usepackage{textcomp}
\usepackage{amssymb}
\usepackage{capt-of}
\usepackage{hyperref}
\usepackage{minted}
% TIPS
% \substack{a\\b} for multiple lines text





% pdfplots will load xolor automatically without option
\usepackage[dvipsnames]{xcolor}

\usepackage{forest}
% two-line text in node by [two \\ lines]
% \begin{forest} qtree, [..] \end{forest}
\forestset{
  qtree/.style={
    baseline,
    for tree={
      parent anchor=south,
      child anchor=north,
      align=center,
      inner sep=1pt,
    }}}
%\usepackage{flexisym}
% load order of mathtools and mathabx, otherwise conflict overbrace

\usepackage{mathtools}
%\usepackage{fourier}
\usepackage{pgfplots}
\usepackage{amsthm}
\usepackage{amsmath}
%\usepackage{unicode-math}
%
\usepackage{commath}
%\usepackage{,  , }
\usepackage{amsfonts}
\usepackage{amssymb}
% importing symbols https://tex.stackexchange.com/questions/14386/importing-a-single-symbol-from-a-different-font
%mathabx change every symbol
% use instead stmaryrd
%\usepackage{mathabx}
\usepackage{stmaryrd}
\usepackage{empheq}
\usepackage{tikz}
\usepackage{tikz-cd}
%\usepackage[notextcomp]{stix}
\usetikzlibrary{arrows.meta}
\usepackage[most]{tcolorbox}
%\utilde
%\usepackage{../../latexpackage/undertilde/undertilde}
% left and right superscript and subscript
\usepackage{actuarialsymbol}
\usepackage{threeparttable}
\usepackage{scalerel,stackengine}
\usepackage{stackrel}
% \stackrel[a]{b}{c}
\usepackage{dsfont}
% text font
\usepackage{newpxtext}
%\usepackage{newpxmath}

%\newcounter{dummy} \numberwithin{dummy}{section}
\newtheorem{dummy}{dummy}[section]
\theoremstyle{definition}
\newtheorem{definition}[dummy]{Definition}
\newtheorem{corollary}[dummy]{Corollary}
\newtheorem{lemma}[dummy]{Lemma}
\newtheorem{proposition}[dummy]{Proposition}
\newtheorem{theorem}[dummy]{Theorem}
\theoremstyle{definition}
\newtheorem{example}[dummy]{Example}
\theoremstyle{remark}
\newtheorem*{remark}{Remark}


\newcommand\what[1]{\ThisStyle{%
    \setbox0=\hbox{$\SavedStyle#1$}%
    \stackengine{-1.0\ht0+.5pt}{$\SavedStyle#1$}{%
      \stretchto{\scaleto{\SavedStyle\mkern.15mu\char'136}{2.6\wd0}}{1.4\ht0}%
    }{O}{c}{F}{T}{S}%
  }
}

\newcommand\wtilde[1]{\ThisStyle{%
    \setbox0=\hbox{$\SavedStyle#1$}%
    \stackengine{-.1\LMpt}{$\SavedStyle#1$}{%
      \stretchto{\scaleto{\SavedStyle\mkern.2mu\AC}{.5150\wd0}}{.6\ht0}%
    }{O}{c}{F}{T}{S}%
  }
}

\newcommand\wbar[1]{\ThisStyle{%
    \setbox0=\hbox{$\SavedStyle#1$}%
    \stackengine{.5pt+\LMpt}{$\SavedStyle#1$}{%
      \rule{\wd0}{\dimexpr.3\LMpt+.3pt}%
    }{O}{c}{F}{T}{S}%
  }
}

\newcommand{\bl}[1] {\boldsymbol{#1}}
\newcommand{\Wt}[1] {\stackrel{\sim}{\smash{#1}\rule{0pt}{1.1ex}}}
\newcommand{\wt}[1] {\widetilde{#1}}
\newcommand{\tf}[1] {\textbf{#1}}


%For boxed texts in align, use Aboxed{}
%otherwise use boxed{}

\DeclareMathSymbol{\widehatsym}{\mathord}{largesymbols}{"62}
\newcommand\lowerwidehatsym{%
  \text{\smash{\raisebox{-1.3ex}{%
    $\widehatsym$}}}}
\newcommand\fixwidehat[1]{%
  \mathchoice
    {\accentset{\displaystyle\lowerwidehatsym}{#1}}
    {\accentset{\textstyle\lowerwidehatsym}{#1}}
    {\accentset{\scriptstyle\lowerwidehatsym}{#1}}
    {\accentset{\scriptscriptstyle\lowerwidehatsym}{#1}}
}

\usepackage{graphicx}
    
% text on arrow for xRightarrow
\makeatletter
%\newcommand{\xRightarrow}[2][]{\ext@arrow 0359\Rightarrowfill@{#1}{#2}}
\makeatother


\newcommand{\dom}[1]{%
\mathrm{dom}{(#1)}
}

% Roman numerals
\makeatletter
\newcommand*{\rom}[1]{\expandafter\@slowromancap\romannumeral #1@}
\makeatother

\def \fR {\mathfrak{R}}
\def \bx {\boldsymbol{x}}
\def \bz {\boldsymbol{z}}
\def \ba {\boldsymbol{a}}
\def \bh {\boldsymbol{h}}
\def \bo {\boldsymbol{o}}
\def \bU {\boldsymbol{U}}
\def \bc {\boldsymbol{c}}
\def \bV {\boldsymbol{V}}
\def \bI {\boldsymbol{I}}
\def \bK {\boldsymbol{K}}
\def \bt {\boldsymbol{t}}
\def \bb {\boldsymbol{b}}
\def \bA {\boldsymbol{A}}
\def \bX {\boldsymbol{X}}
\def \bu {\boldsymbol{u}}
\def \bS {\boldsymbol{S}}
\def \bZ {\boldsymbol{Z}}
\def \bz {\boldsymbol{z}}
\def \by {\boldsymbol{y}}
\def \bw {\boldsymbol{w}}
\def \bT {\boldsymbol{T}}
\def \bF {\boldsymbol{F}}
\def \bS {\boldsymbol{S}}
\def \bm {\boldsymbol{m}}
\def \bW {\boldsymbol{W}}
\def \bR {\boldsymbol{R}}
\def \bQ {\boldsymbol{Q}}
\def \bS {\boldsymbol{S}}
\def \bP {\boldsymbol{P}}
\def \bT {\boldsymbol{T}}
\def \bY {\boldsymbol{Y}}
\def \bH {\boldsymbol{H}}
\def \bB {\boldsymbol{B}}
\def \blambda {\boldsymbol{\lambda}}
\def \bPhi {\boldsymbol{\Phi}}
\def \btheta {\boldsymbol{\theta}}
\def \bTheta {\boldsymbol{\Theta}}
\def \bmu {\boldsymbol{\mu}}
\def \bphi {\boldsymbol{\phi}}
\def \bSigma {\boldsymbol{\Sigma}}
\def \lb {\left\{}
\def \rb {\right\}}
\def \la {\langle}
\def \ra {\rangle}
\def \caln {\mathcal{N}}
\def \dissum {\displaystyle\Sigma}
\def \dispro {\displaystyle\prod}
\def \E {\mathbb{E}}
\def \Q {\mathbb{Q}}
\def \N {\mathbb{N}}
\def \V {\mathbb{V}}
\def \R {\mathbb{R}}
\def \P {\mathbb{P}}
\def \A {\mathbb{A}}
\def \Z {\mathbb{Z}}
\def \I {\mathbb{I}}
\def \C {\mathbb{C}}
\def \cala {\mathcal{A}}
\def \calb {\mathcal{B}}
\def \calq {\mathcal{Q}}
\def \calp {\mathcal{P}}
\def \cals {\mathcal{S}}
\def \calg {\mathcal{G}}
\def \caln {\mathcal{N}}
\def \calr {\mathcal{R}}
\def \calm {\mathcal{M}}
\def \calc {\mathcal{C}}
\def \calf {\mathcal{F}}
\def \calk {\mathcal{K}}
\def \call {\mathcal{L}}
\def \calu {\mathcal{U}}
\def \bcup {\bigcup}


\def \uin {\underline{\in}}
\def \oin {\overline{\in}}
\def \uR {\underline{R}}
\def \oR {\overline{R}}
\def \uP {\underline{P}}
\def \oP {\overline{P}}

\def \Ra {\Rightarrow}

\def \e {\enspace}

\def \sgn {\operatorname{sgn}}
\def \gen {\operatorname{gen}}
\def \ker {\operatorname{ker}}
\def \im {\operatorname{im}}

\def \tril {\triangleleft}

% \varprod
\DeclareSymbolFont{largesymbolsA}{U}{txexa}{m}{n}
\DeclareMathSymbol{\varprod}{\mathop}{largesymbolsA}{16}

% \bigtimes
\DeclareFontFamily{U}{mathx}{\hyphenchar\font45}
\DeclareFontShape{U}{mathx}{m}{n}{
      <5> <6> <7> <8> <9> <10>
      <10.95> <12> <14.4> <17.28> <20.74> <24.88>
      mathx10
      }{}
\DeclareSymbolFont{mathx}{U}{mathx}{m}{n}
\DeclareMathSymbol{\bigtimes}{1}{mathx}{"91}
% \odiv
\DeclareFontFamily{U}{matha}{\hyphenchar\font45}
\DeclareFontShape{U}{matha}{m}{n}{
      <5> <6> <7> <8> <9> <10> gen * matha
      <10.95> matha10 <12> <14.4> <17.28> <20.74> <24.88> matha12
      }{}
\DeclareSymbolFont{matha}{U}{matha}{m}{n}
\DeclareMathSymbol{\odiv}         {2}{matha}{"63}


\newcommand\subsetsim{\mathrel{%
  \ooalign{\raise0.2ex\hbox{\scalebox{0.9}{$\subset$}}\cr\hidewidth\raise-0.85ex\hbox{\scalebox{0.9}{$\sim$}}\hidewidth\cr}}}
\newcommand\simsubset{\mathrel{%
  \ooalign{\raise-0.2ex\hbox{\scalebox{0.9}{$\subset$}}\cr\hidewidth\raise0.75ex\hbox{\scalebox{0.9}{$\sim$}}\hidewidth\cr}}}

\newcommand\simsubsetsim{\mathrel{%
  \ooalign{\raise0ex\hbox{\scalebox{0.8}{$\subset$}}\cr\hidewidth\raise1ex\hbox{\scalebox{0.75}{$\sim$}}\hidewidth\cr\raise-0.95ex\hbox{\scalebox{0.8}{$\sim$}}\cr\hidewidth}}}
\newcommand{\stcomp}[1]{{#1}^{\mathsf{c}}}


\author{Thomas Jech}
\date{\today}
\title{Notes on Set Theory}
\hypersetup{
 pdfauthor={Thomas Jech},
 pdftitle={Notes on Set Theory},
 pdfkeywords={},
 pdfsubject={},
 pdfcreator={Emacs 26.3 (Org mode 9.3)}, 
 pdflang={English}}
\begin{document}

\maketitle
\tableofcontents \clearpage
\section{Ordinal}
\label{sec:orge36a849}
\subsection{Linear and partial ordering}
\label{sec:org79321a7}
options []
\begin{definition}
A binary relation \(<\) on a set \(P\) is a \tf{partial ordering} of \(P\) if:
\begin{enumerate}
\item \(p\not< p\) for any \(p\in P\)
\item if \(p<q\) and \(q<r\) then \(p<r\)

\((P,<)\) is called a \tf{partial ordered set}. A partial ordering \(<\) of
\(P\) is a \tf{linear ordering} if moreover
\item \(p<q\) or \(q<p\) or \(p=q\) for all \(p,q\in P\)
\end{enumerate}
\end{definition}


If \((P,<)\) and \((Q,<)\) are poset and \(f:P\to Q\), then \(f\) is
\tf{order-preserving} if \(x<y\) implies \(f(x)<f(y)\). If \(P\) and \(Q\) are
linearly ordered, then \(f\) is also called \tf{increasing}
\subsection{Well-Ordering}
\label{sec:org99304a3}
\begin{definition}[]
A linear ordering \(<\) of a set \(P\) is a \tf{well-ordering} if every nonempty
subset of \(P\) has a least element
\end{definition}

\begin{lemma}[]
\label{lemma1}
If \((W,<)\) is a well-ordering set and \(f:W\to W\) is an increasing function,
then \(f(x)\ge x\) for each \(x\in W\)
\end{lemma}
\begin{proof}
Assume that the set \(X=\{x\in W\mid f(x)<x\}\) is nonempty and let \(z\) be the
least element of \(X\). Hence \(f(f(x))<f(x)\) and \(f(x)\in X\), a contradiction.
\end{proof}

\begin{corollary}[]
The only automorphism of a well-ordered set is the identity
\end{corollary}

\begin{corollary}[]
If two well-ordered sets \(W_1,W_2\) are isomorphic, then the isomorphism of
\(W_1\) onto \(W_2\) is unique
\end{corollary}

If \(W\) is a well-ordered set and \(u\in W\), then \(\{x\in W:x<u\}\) is an
\tf{initial segment} of \(W\)
\begin{lemma}[]
\label{lemma2}
No well-ordered set is isomorphic to an initial segment of itself
\end{lemma}
\begin{proof}
If \(\ran{f}=\{x:x<u\}\), then \(f(u)<u\), contrary to lemma \ref{lemma1}
\end{proof}

\begin{theorem}[]
If \(W_1\) and \(W_2\) are well-ordered sets, then exactly one of the following
three cases holds:
\begin{enumerate}
\item \(W_1\cong W_2\)
\item \(W_1\) is isomorphic to an initial segment of \(W_2\)
\item \(W_2\) is isomorphic to an initial segment of \(W_1\)
\end{enumerate}
\end{theorem}
\begin{proof}
For \(u\in W_i,(i=1,2)\), let \(W_i(u)\) denote the initial segment of \(W_i\)
given by \(u\). Let
\begin{equation*}
f=\{(x,y)\in W_1\times W_2\mid W_1(x)\cong W_2(y)\}
\end{equation*}

If \(W_1(x)\cong W_w(y)\) and \(W_1(x)\cong W_2(y')\), then \(W_2(y)\cong
   W_1(y')\). According to lemma \ref{lemma2}, \(y=y'\). Hence it's easy to see that
\(f\) is a one-to-one function.

If \(h\) is an isomorphism between \(W_1(x)\) and \(W_2(y)\) and \(x'<x\), then
\(W_1(x')\cong W_2(h(x'))\). It follows that \(f\) is order-preserving.

If \(\dom{f}=W_1\) and \(\ran{f}=W_2\), then case 1 holds.

If \(y_1<y_2\) and \(y_2\in \ran{f}\), then \(y_1\in\ran{f}\). If there is some
\(y<y_2\) and \(y\not\in\ran{f}\). Consider the least element \(y'\) of \(\{y\in
   W_2\mid y<y_2\wedge y\not\in\ran{f}\}\). Let \(x'=\sup\{x\in W_1\mid\exists
   y\in W_2(W_1(x)\cong W_2(y)\wedge y<y')\}\), then \(W_1(x')\cong W_2(y')\), a
contradiction. 

If \(\ran{f}\neg W_2\) and \(y_0\) is the least element of \(W_2-\ran{f}\). We have
\(\ran{f}=W_2(x_0)\). Necessarily, \(\dom{f}=W_1\), for otherwise we could have
\((x_0,y_0)\in f\) where \(x_0=\)least element of \(W_1-\dom{f}\). Thus case 2
holds.

Similarly, case 3 holds.
\end{proof}

If \(W_1\cong W_2\), we say that they have the same \tf{order-type}


\subsection{Ordinal Numbers}
\label{sec:org4a3aad8}
The idea is to define ordinal numbers so that
\begin{equation*}
\alpha<\beta\Leftrightarrow\alpha\in\beta\wedge\alpha=\{\beta:\beta<\alpha\}
\end{equation*}
\begin{definition}[]
A set \(T\) is \tf{transitive} if every element of \(T\) is a subset of \(T\)
\end{definition}
\begin{definition}[]
A set is an \tf{ordinal number} (an \tf{ordinal}) if it's transitive and
well-ordered by \(\in\)
\end{definition}
The class of all ordinals is denoted by \(Ord\)

We define
\begin{equation*}
\alpha<\beta\Leftrightarrow\alpha\in\beta
\end{equation*}
\begin{lemma}[]
\label{lemma3}
\begin{enumerate}
\item \(0=\emptyset\) is an ordinal
\item If \(\alpha\) is an ordinal and \(\beta\in\alpha\), then \(\beta\) is an ordinal
\item If \(\alpha\neq\beta\) are ordinals and \(\alpha\subset\beta\), then
\(\alpha\in\beta\)
\item If \(\alpha\),\(\beta\) are ordinals, then either \(\alpha\subset\beta\) or
\(\beta\subset\alpha\)
\end{enumerate}
\end{lemma}
\begin{proof}
\begin{enumerate}
\item definition
\item definition
\item If \(\alpha\subset\beta\), let \(\gamma\) be the least element of the set
\(\beta-\alpha\). Since \(\alpha\) is transitive, it follows that \(\alpha\) is the
initial segment of \(\beta\) given by \(\gamma\). Thus
\(\alpha=\{\xi\in\beta\mid\xi<\gamma\}=\gamma\in\beta\)
\item Clearly \(\alpha\cap\beta\) is an ordinal \(\gamma\). We have \(\gamma=\alpha\) or
\(\gamma=\beta\), for otherwise \(\gamma\in\alpha\) and \(\gamma\in\beta\) by 3.
Then \(\gamma\in\gamma\) which contradicts the definition of an ordinal
\end{enumerate}
\end{proof}
Using lemma \ref{lemma3} one gets the following facts about ordinal numbers
\begin{enumerate}
\item \(<\) is a linear ordering of the class \(Ord\)
\item For each \(\alpha\), \(\alpha=\{\beta:\beta<\alpha\}\)
\item If \(C\) is a nonempty class of ordinals, then \(\bigcap C\) is an ordinal,
\(\bigcap C\in C\) and \(\bigcap C=\inf C\)
\item If \(X\) is a nonempty set of ordinals, then \(\bigcup X\) is an ordinal and
\(\bigcup X=\sup X\)
\item For every \(\alpha\), \(\alpha\cup\{\alpha\}\) is an ordinal and
\(\alpha\cup\{\alpha\}=\inf\{\beta:\beta>\alpha\}\)
\end{enumerate}


We thus define \(\alpha+1=\alpha\cup\{\alpha\}\)(the \tf{succesor} of \(\alpha\)) 

\begin{theorem}[]
Every well-ordered set is isomorphic to a unique ordinal number
\end{theorem}

\begin{proof}
The uniqueness follows from lemma \ref{lemma2}. Given a well-ordered set \(W\),
we find an isomorphic ordinal as follows: Define \(F(x)=\alpha\) if \(\alpha\) is
isomorphic to the initial segment of \(W\) given by \(x\). If such an \(\alpha\)
exists, then it's unique. By the replacement axiom, \(F(W)\) is a set. For each
\(x\in W\), such an \(\alpha\) exists. Otherwise consider the least \(x\) such that
\(\alpha\) doesn't exist. Let \(\alpha=\sup\{F(x')\mid x'\in W\wedge x' <x\}\) and
\(F(x)=\alpha\). If \(\gamma\) is the least \(\gamma\not\in F(W)\), then
\(F(W)=\gamma\) and we have an isomorphism of \(W\) onto \(\gamma\)
\end{proof}

If \(\alpha=\beta+1\), then \(\alpha\) is a \tf{succesor ordinal}. If \(\alpha\) is not
a succesor ordinal then \(\alpha=\sup\{\beta:\beta<\alpha\}=\bigcup\alpha\) is
called a \tf{limit ordinal}. We also consider 0 a limit ordinal and define
\(\sup\emptyset=0\).

\subsection{Induction and Recursion}
\label{sec:orgc895cac}
\begin{theorem}[Transfinite Induction]
Let \(C\) be a class of ordinals and assume
\begin{enumerate}
\item \(0\in C\)
\item if \(\alpha\in C\), then \(\alpha+1\in C\)
\item if \(\alpha\) is a nonzero limit ordinal and \(\beta\in C\) for all
\(\beta<\alpha\), then \(\alpha\in C\)
\end{enumerate}


Then \(C\) is the class of all ordinals
\end{theorem}

\begin{proof}
Otherwise let \(\alpha\) be the least ordinal \(\alpha\not\in C\) and apply 1, 2 or 3
\end{proof}

A function whose domain is the set \(\N\) is called an \tf\{(infinite)
sequence\} (A \tf{sequence} in \(X\) is a function \(f:\N\to X\)). The standard
notation for a sequence is
\begin{equation*}
\la a_n:n<\omega\ra
\end{equation*}
A \tf{finite sequence} is a function \(s\) s.t. \(\dom{s}=\{i:i<n\}\) for some
\(n\in\N\); then \(s\) is a \tf{sequence of length} \(n\)

A \tf{transfinite sequence} is a function whose domain is an ordinal
\begin{equation*}
\la a_\xi:\xi<\alpha\ra
\end{equation*}
It is also called an \(\alpha\)-\tf{sequence} or a \tf{sequence of length}
\(\alpha\). We also say that a sequence \(\la a_\xi:\xi<alpha\ra\) is an
\tf{enumeration} of its range \(\{a_\xi:\xi<\alpha\}\). If \(s\) is a sequence of
length \(\alpha\), then \(s^\smallfrown x\) or simply \(sx\) denotes the sequence of length
\(\alpha+1\) that extends \(s\) and whose \(\alpha\)th term is \(x\):
\begin{equation*}
s^\smallfrown x=sx=s\cap\{(\alpha,x)\}
\end{equation*}

\begin{theorem}[Transfinite Recursion]
Let \(G\) be a function, then \ref{align1} below defines a unique function \(F\) on
\(Ord\) s.t.
\begin{equation*}
F(\alpha)=G(F\restriction\alpha)
\end{equation*}
for each \(\alpha\)
\end{theorem}
In other words, if we let \(a_\alpha=F(\alpha)\), then for each \(\alpha\)
\begin{equation*}
a_\alpha=G(\la a_\xi:\xi<\alpha\ra)
\end{equation*}

\begin{corollary}[]
Let \(X\) be a set and \(\theta\) be an ordinal number. For every function \(G\) on
the set of all transfinite sequences in \(X\) of length \(<\theta\) s.t.
\(\ran{G}\subset X\) there exists a unique \(\theta\)-sequence in \(X\) s.t. 
\(a_\alpha=G(\la a_\xi:\xi<\theta)\) for every \(\alpha<\theta\)
\end{corollary}
\begin{proof}

Let
\begin{align}
\label{align1}
F(\alpha)=x\leftrightarrow&\text{ there is a sequence }
\la a_\xi:\xi<\alpha\ra \text{ such that }\\
&1.\;(\forall \xi<\alpha)a_\xi=G(\la a_n\eta:\eta<\xi\ra)\nonumber \\
&2.\; x=G(\la a_\xi:\xi<\alpha\ra)\nonumber
\end{align}

For every \(\alpha\), if there is an \(\alpha\)-sequence that satisfying 1, then such
a sequence is unique. Thus \(F(\alpha)\) is determined uniquely by 2 and
therefore \(F\) is a function. 
\end{proof}

\begin{definition}[]
Let \(\alpha>0\) be a limit ordinal and let \(\la\gamma_\xi:\xi<\alpha\ra\) be a
\tf{nondecreasing} sequence of ordinals (i.e., \(\xi<\eta\) implies
\(\gamma_\xi\le\gamma_eta\)). We define the \tf{limit} of the sequence by
\begin{equation*}
\lim_{\xi\to\alpha}\gamma_\xi=\sup\{\gamma_\xi:\xi<\alpha\}
\end{equation*}

A sequence of ordinals \(\la\gamma_\alpha:\alpha\in Ord\ra\) is \tf{normal} if
it's increasing and \tf{continuous}, i.e., for every limit \(\alpha\),
\(\gamma_\alpha=\lim_{\xi\to\alpha}\gamma_\xi\) 
\end{definition}


\subsection{Ordinal Arithmetic}
\label{sec:orge7b976f}
\begin{definition}[Addition]
For all ordinal numbers \(\alpha\)
\begin{enumerate}
\item \(\alpha+0=\alpha\)
\item \(\alpha+(\beta+1)=(\alpha+\beta)+1\), for all \(\beta\)
\item \(\alpha+\beta=\lim_{\xi\to\beta}(\alpha+\xi)\) for all limit \(\beta>0\)
\end{enumerate}
\end{definition}

\begin{definition}[Multiplication]
For all ordinal numbers \(\alpha\)
\begin{enumerate}
\item \(\alpha\cdot 0=0\)
\item \(\alpha\cdot(\beta+1)=(\alpha\cdot\beta)+\alpha\), for all \(\beta\)
\item \(\alpha\cdot\beta=\lim_{\xi\to\beta}(\alpha\cdot\xi)\) for all limit \(\beta>0\)
\end{enumerate}
\end{definition}

\begin{definition}[Exponentiation]
For all ordinal numbers \(\alpha\)
\begin{enumerate}
\item \(\alpha^0=1\)
\item \(\alpha^{\beta+1}=\alpha^\beta\cdot\alpha\), for all \(\beta\)
\item \(\alpha^\beta=\lim_{\xi\to\beta}\alpha^\xi\) for all limit \(\beta>0\)
\end{enumerate}
\end{definition}

\begin{lemma}[]
For all ordinals \(\alpha\), \(\beta\) and \(\gamma\)
\begin{enumerate}
\item \(\alpha+(\beta+\gamma)=(\alpha+\beta)+\gamma\)
\item \(\alpha\cdot(\beta\cdot\gamma)=(\alpha\cdot\beta)\cdot\gamma\)
\end{enumerate}
\end{lemma}
Neither \(+\) nor \(\cdot\) are commutative
\begin{equation*}
1+\omega=\omega\neq \omega+1,\e 2\cdot\omega=\omega\neq\omega\cdot 2
\end{equation*}

\begin{definition}[]
Let \((A,<_A)\) and \((B,<_B)\) be disjoint linearly ordered sets. The \tf{sum}
of these linear orders is the set \(A\cup B\) with the ordering defined as
follows:
\(x<y\) if and only if
\begin{enumerate}
\item \(x,y\in A\) and \(x<_A y\)
\item \(x,y\in B\) and \(x<_B y\)
\item \(x\in A\) and \(y\in B\)
\end{enumerate}
\end{definition}

\begin{definition}[]
Let \((A,<)\) and \((B,<)\) be linearly ordered sets. The \tf{product} of these
linear orders is the set \(A\times B\) with the ordering defined by
\begin{equation*}
(a_1,b_1)<(a_2,b_2)\Leftrightarrow b_1<b_2\text{ or } (b_1=b_2\wedge a_1<a_2)
\end{equation*}
\end{definition}
\begin{lemma}[]
For all ordinals \(\alpha\) and \(\beta\), \(\alpha+\beta\) and \(\alpha\cdot\beta\) are
respectively isomorphic to the sum and to the product of \(\alpha\) and \(\beta\)
\end{lemma}

\begin{proof}
Suppose \((A,<_A)\cong\alpha\) and \((B,<_B)\cong\beta\). 
\begin{enumerate}
\item if \(\beta=0\), then \(B=\emptyset, A\cup B=A\)
\item if \((A\cup B,<_{A\cup B})\cong \alpha+\beta\), let \(B'\equal B\cup\{c\}\) s.t.
\(\{c\}\cap A=\{c\}\cap B=\emptyset\) all for all \(b\in B\), \(b<c\). Hence
\begin{equation*}
\alpha+(\beta+1)=(\alpha+\beta)+1\cong(A\cup B)\cup\{c\}=A\cup B'
\end{equation*}

\item if \(\beta\) is a limit ordinal and for all \(\xi<\beta\) and \(B_\xi\cong\xi\),\par
\((A\cup B_\xi,<_{A\cup B_\xi})\cong\alpha+\xi\),
\begin{equation*}
A\cup B=A\cup\sup{B_\xi}=\sup(A\cup B_\xi)\cong\sup(\alpha+\xi)=\alpha+\beta
\end{equation*}
\end{enumerate}
\end{proof}

\begin{lemma}[]
\begin{enumerate}
\item If \(\beta<\gamma\) then \(\alpha+\beta<\alpha+\gamma\)
\item If \(\alpha<\beta\) then there exists a unique \(\delta\) s.t.
\(\alpha+\delta=\beta\)
\item If \(\beta < \gamma\) and \(\alpha>0\), then
\(\alpha\cdot\beta<\alpha\cdot\gamma\)
\item If \(\alpha>0\) and \(\gamma\) is arbitrary, then there exist a unique \(\beta\) and
a unique \(\rho<\alpha\) s.t. \(\gamma=\alpha\cdot\beta+\rho\)
\item If \(\beta<\gamma\) and \(\alpha>1\), then \(\alpha^\beta<\alpha^\gamma\)
\end{enumerate}
\end{lemma}
\begin{proof}
\begin{enumerate}
\setcounter{enumi}{1}
\item Let \(\delta\) be the order-type of the set \(\{\xi:\alpha\le\xi<\beta\}\)
\setcounter{enumi}{3}
\item Let \(\beta\) be the greatest ordinal s.t. \(\alpha\cdot\beta\le\gamma\)
\end{enumerate}
\end{proof}


\begin{theorem}[Cantor's Normal Form Theorem]
Every ordinal \(\alpha>0\) can be represented uniquely in the form
\begin{equation*}
\alpha=\omega^{\beta_1}\cdot k_1+\dots+\omega^{\beta_n}\cdot k_n
\end{equation*}
where \(n\ge 1\), \(\alpha\ge\beta_1>\dots>\beta_n\) and \(k_1,\dots,k_n\) are
nonzero natural numbers.
\end{theorem}
\begin{proof}
By induction on \(\alpha\). For \(\alpha=1\) we have \(1=\omega^0+1\); for arbitrary
\(\alpha>0\), let \(\beta\) be the greatest ordinal s.t. \(\omega^\beta\le
   \alpha\).
The uniqueness of the normal form is proved by induction
\end{proof}


\subsection{Well-Founded Relations}
\label{sec:org67a010e}
A binary relation \(E\) on a set \(P\) is \tf{well-founded} if every nonempty
\(X\subset P\) has an \(E\)-\tf{minimal} element.

Given a well-founded relation \(E\) on a set \(P\), we can define the \tf{height}
of \(E\) and assign to each \(x\in P\) and ordinal number, the \tf{rank} of \(x\)
in \(E\)

\begin{theorem}[]
If \(E\) is a well-founded relation on \(P\), then there exists a unique function
\(\rho\) from \(P\) into the ordinals s.t. for all \(x\in P\)
\begin{equation*}
\rho(x)=\sup\{\rho(y)+1:yEx\}
\end{equation*}
\end{theorem}
The range of \(\rho\) is an initial segment of the ordinals, thus an ordinal
number. This ordinal is called the \tf{height} of \(E\)

\begin{proof}
By induction, let
\begin{align*}
&P_0=\emptyset\\
&P_{\alpha+1}=\{x\in P:\forall y(yEx\to y\in P_\alpha)\}\cup P_\alpha\\
&P_\alpha=\displaystyle\bigcup_{\xi<\alpha}P_\xi \e\text{if } \alpha 
\text{ is a limit ordinal}
\end{align*}
Let \(\theta\) be the least ordinal s.t. \(P_{\theta+1}=P_\theta\). We claim that
\(P_\theta=P\) 
\end{proof}

\subsection{Exercise}
\label{sec:org05f1225}
\begin{enumerate}
\item Every normal sequence \(\la\gamma_\alpha:\alpha\in Ord\ra\) has arbitrarily
large \tf{fixed points}, i.e., \(\alpha\) s.t. \(\gamma_\alpha=\alpha\)

\begin{proof}
From
\href{https://math.stackexchange.com/questions/1808103/show-that-there-exists-a-fixed-point-for-this-set-theoretic-class-function}{StackExchange}.
\end{proof}
\end{enumerate}


A limit ordinal \(\gamma>0\) is called \tf{indecomposable} if there exist no
\(\alpha<\gamma\) and \(beta<\gamma\) s.t. \(\alpha+\beta=\gamma\)
\begin{enumerate}
\setcounter{enumi}{1}
\item A limit ordinal \(\gamma>0\) is indecomposable if and only if
\(\alpha+\gamma=\gamma\) for all \(\alpha<\gamma\) if and only if
\(\gamma=\omega^\alpha\) for some \(\alpha\)
\begin{proof}
\begin{enumerate}
\item \((3)\to(1)\). Assume \(\gamma_1,\gamma_2<\gamma=\omega^\alpha\). By
Cantor's normal form theorem, there exist \(\alpha'\) and \(k\) s.t. 
\(\gamma_1,\gamma_2<\omega^{\alpha'}\cdot k\)
\item \((2)\to(3)\). Assume that \(\gamma\) can't be written as \(\omega^\alpha\).
Then by Cantor's theorem, \(\gamma=\omega^{\beta_1}\cdot
         k_1+\dots+\omega^{\beta_n}\cdot k_n\). But then
\(\omega^{\beta_1}<\gamma\) and \(\omega^{\beta_1}+\gamma>\gamma\)
\end{enumerate}
\end{proof}
\end{enumerate}
\section{Models of Set - Sertraline}
\label{sec:org3a0162d}
\subsection{Some mathematical logic}
\label{sec:orgc6b013d}
\begin{theorem}[Gödel’s second incompleteness theorem]
If a consistent recursive axiom set \(T\) contains \(\zfc\), then
\begin{equation*}
T\not\vdash\con(t)
\end{equation*}
especially, \(\zfc\not\vdash\con(\zfc)\)
\end{theorem}

\begin{definition}[]
Suppose \((M,E_M)\) and \((N,E_N)\) are two models of set theory, then
\begin{enumerate}
\item if for any formula \(\sigma\), \(M\models\sigma\) if and only if
\(N\models\sigma\), then \(M\) and \(N\) are \tf{elementary equivalent}, denoted
by \(M\equiv N\)
\item If bijection \(f:M\to N\) satisfies: for any \(a,b\in M\), \(aE_Mb\) iff
\(f(a)E_Nf(b)\), then \(f:M\cong N\) is an \tf{isomorphism}
\item If \(M\subseteq N\) and \(E_M=E_N\restriction M\), then \(M\) is \(N\)'s submodel
\item If \(M\) is isomorphic to a submodel of \(N\) by injection \(f\), and for any
formula \(\varphi(x_1,\dots,x_n)\), for any \(a_1,\dots,a_n\in M\), 
\(M\models\varphi[a_1,\dots,a_n]\) iff
\(N\models\varphi[f(a_1),\dots,f(a_n)]\), then \(f\) is called an
\tf{elementary embedding} from \(M\) to \(N\), written as \(f:M\prec N\)
\item If \(M\subseteq N\) and \(M\prec N\), then \(M\) is a \tf{elementary submodel}
of \(N\)
\end{enumerate}
\end{definition}

\begin{lemma}[]
Suppose \(N\models\zfc,M\subseteq N\), then \(M\prec N\) iff
\(\forall\varphi(x,x_1,\dots,x_n)\), \(\forall(a_1,\dots,a_n)\in M\), if 
\(\exists a\in N\) s.t. \(N\models\varphi[a,a_1,\dots,a_n]\), then \(\exists a\in
  M\) s.t. 
\(M\models\varphi[a,a_1,\dots,a_n]\)
\end{lemma}

\begin{definition}[]
Suppose \((M,E)\models\zfc\)
\begin{enumerate}
\item \(h_\varphi:M^n\to M\) is \(\varphi\)'s \tf{Skolem function} if 
\(\forall a_1,\dots,a_n\in M\), if \(\exists a\in M\) s.t.
\(M\models\varphi[a,a_1,\dots,a_n]\), then
\(M\models\varphi[h_\varphi(a_1,\dots,a_n),a_1,\dots,a_n]\) . requires \ac
\item Let \(\calh=\{h_\varphi\mid\varphi \text{is a formula on set theory}\}\). For
any \(S\subseteq M\), \tf{Skolem hull} \(\calh(S)\) is the smallest set
consisting of \(S\) and closed under \(\calh\)
\end{enumerate}
\end{definition}

\begin{lemma}[]
\(N\models\zfc,S\subseteq N\), if \(M=\calh(S)\), then \(M\prec N\)
\end{lemma}

\begin{theorem}[Löwenheim-Skolem theorem]
Suppose \(N\models\zfc\) and is infinite, then there is a model \(M\) s.t.
\(\abs{M}=\omega\) and \(M\prec N\)
\end{theorem}
\subsection{Cumulative Hierarchy}
\label{sec:orgd39a292}
This section works in \zfm(a.k.a. \(\zf-\text{axiom of foundation}\))

\begin{definition}[]
For any \(\alpha\), define sequence \(V_\alpha\)
\begin{enumerate}
\item \(V_0=\emptyset\)
\item \(V_{\alpha+1}=\calp(V_\alpha)\)
\item For any limit ordinal \(\lambda\), \(V_\lambda=\bigcup_{\beta<\lambda}V_\beta\)
\end{enumerate}


And \(\wf=\displaystyle\bigcup_{\alpha\in\on}V_\alpha\)
\end{definition}

\begin{lemma}[]
For any ordinal \(\alpha\)
\begin{enumerate}
\item \(V_\alpha\) is transitive
\item if \(\xi\le\alpha\), then \(V_\xi\subseteq V_\alpha\)
\item if \(\kappa\) is inaccessible cardinal, then \(\abs{V_\kappa}=\kappa\)
\end{enumerate}
\end{lemma}

\begin{proof}
\begin{enumerate}
\item Obviously \(\kappa\le V_\kappa\). Since \(\kappa\) is inaccessible, then for any
\(\alpha<\kappa\), \(\abs{V_\alpha}<\kappa\).
\end{enumerate}
\end{proof}

\begin{definition}[]
For any set \(x\in\wf\), 
\begin{equation*}
\rank(x)=\min\{\beta\mid x\in V_{\beta+1}\}
\end{equation*}
\end{definition}

\begin{lemma}[]
\begin{enumerate}
\item \(V_\alpha=\{x\in\wf\mid\rank(x)<\alpha\}\)
\item \wf is transitive
\item For any \(x,y\in\wf\), if \(x\in y\), then \(\rank(x)<\rank(y)\)
\item for any \(y\in\wf\), \(\rank(y)=\sup\{\rank(x)+1\mid x\in y\}\)
\end{enumerate}
\end{lemma}

\begin{lemma}[]
Supoose \(\alpha\) is an ordinal
\begin{enumerate}
\item \(\alpha\in\wf\) and \(\rank(\alpha)=\alpha\)
\item \(V_\alpha\cap\on=\alpha\)
\end{enumerate}
\end{lemma}

\begin{lemma}[]
\begin{enumerate}
\item If \(x\in\wf\), then \(\bigcup x,\calp(x),\{x\}\in\wf\), and their ranks are
all less than \(\rank(x)+\omega\)
\item If \(x,y\in\wf\), then \(x\times y,x\cup y,x\cap y,\{x,y\},(x,y),x^y\in\wf\),
and their ranks are all less than \(\rank(x)+\rank(y)+\omega\)
\item \(\Z,\Q,\R\in V_{\omega+\omega}\)
\item for any set \(x\), \(x\in\wf\) iff \(x\subset\wf\)
\end{enumerate}
\end{lemma}

\begin{lemma}[]
Suppose \ac
\begin{enumerate}
\item for any group \(G\), there exists group \(G'\cong G\) in \wf
\item for any topological space \(T\), there exists \(T'\cong T\) in \wf
\end{enumerate}
\end{lemma}

\begin{definition}[]
Binary relation \(<\) on set \(A\) is \tf{well-founded} if for any nonempty
\(X\subseteq A\), \(X\) has minimal element under \(<\)
\end{definition}


\begin{theorem}[]
If \(A\in\wf\), then \(\in\) is a well-founded relation on \(A\)
\end{theorem}

\begin{lemma}[]
If set \(A\) is transitive and \(\in\) is well-founded on \(A\), then \(A\in\wf\)
\end{lemma}

\begin{lemma}[]
For any set \(x\), there is a smallest transitive set \(\trcl{x}\) s.t.
\(x\subseteq\trcl{x}\) 
\end{lemma}

\begin{proof}
\begin{align*}
x_0&=x\\
x_{n+1}&=\bigcup x_n\\
\trcl{x}&=\displaystyle\bigcup_{n<\omega}x_n
\end{align*}
\end{proof}

\(\trcl{x}\) is called \tf{transitive closure} of \(x\)


\begin{lemma}[]
Without axiom of power set
\begin{enumerate}
\item if \(x\) is transitive, then \(\trcl{x}=x\)
\item if \(y\in x\), then \(\trcl{y}\subseteq\trcl{x}\)
\item \(\trcl{x}=x\cup\bigcup\{\trcl{y}\mid y\in x\}\)
\end{enumerate}
\end{lemma}

\begin{theorem}[]
For any set \(X\), the following are equivalent
\begin{enumerate}
\item \(X\in\wf\)
\item \(\trcl{X}\in\wf\)
\item \(\in\) is a well-founded relation on \(\trcl{X}\)
\end{enumerate}
\end{theorem}

\begin{theorem}[]
The following propositions are equivalent
\begin{enumerate}
\item Axiom of foundation
\item For any set \(X\), \(\in\) is a well-founded relation on \(X\)
\item \(\tf{V}=\wf\)
\end{enumerate}
\end{theorem}
\subsection{Relativization}
\label{sec:orge1e321a}
\begin{definition}[]
Let \tf{M} be a class \(\varphi\) a formula, the \tf{relativization} of \(\varphi\)
to \tf{M} is \(\varphi^{\tf{M}}\) defined inductively
\begin{align*}
(x\in y)^{\cm}&\leftrightarrow x=y\\
(x\in y)^{\cm}&\leftrightarrow x\in y\\
(\varphi\to\psi)^{\cm}&\leftrightarrow \varphi^{\cm}\to\psi^\cm\\
(\neg\varphi)^\cm&\leftrightarrow\neg\varphi^\cm\\
(\forall x\varphi)^\cm&\leftrightarrow(\forall x\in\cm)\varphi^\cm
\end{align*}
\end{definition}

Note \(\varphi^\cv=\varphi\) and
\begin{equation*}
f^\cm=\{(x_1,\dots,x_n,x_{n+1})\in\cm\mid\varphi^\cm(x_1,\dots,x_n,x_{n+1})\}
\end{equation*}

\begin{definition}[]
For any theory \(T\), any class \(\cm\), \(\cm\models T\) iff for any axiom
\(\varphi\) of \(T\), \(\varphi^\cm\) holds
\end{definition}


\begin{theorem}[\zfm]
\(\wf\models\zf\)
\end{theorem}

\begin{proof}
\begin{itemize}
\item \tf{Axiom of existence}

\((\exists x(x=x))^\cm\leftrightarrow\exists x\in\cm(x=x)\), which is
equivalent to \cm being nonempty
\item \tf{Axiom of extensionality}

\begin{gather*}
\forall X\forall Y\forall u((u\in X\leftrightarrow u\in Y)\to X=Y)^\cm
\Leftrightarrow\\
\forall X\in\cm\forall Y\in\cm\forall u\in\cm
((u\in X\leftrightarrow u\in Y)\to X=Y)
\end{gather*}

\begin{lemma}
If $\cm$ is transitive, then axiom of extensionality holds in \cm
\end{lemma}

\item \tf{Axiom schema of specification}

\begin{equation*}
\forall X\in\cm\exists Y\in\cm\forall u\in\cm(u\in Y\leftrightarrow
u\in X\wedge\varphi^\cm(u))
\end{equation*}

Since for any \(X\in\wf\), \(\calp(X)\subseteq \wf\)
\item \tf{Axiom of paring}
\item \tf{Axiom of union}
\item \tf{Axiom of power set}

\begin{equation*}
\forall X\in\cm\exists Y\in\cm\forall u\in\cm(u\in Y\leftrightarrow(u\subseteq X)^\cm)
\end{equation*}
and 
\begin{equation*}
(u\subseteq X)^\cm\leftrightarrow\forall x\in\cm(x\in u\to x\in X)
\leftrightarrow u\cap\cm\subseteq X
\end{equation*}
\item \tf{Axiom of foundation}
\item \tf{Axiom schema of replacement}
\end{itemize}
\end{proof}
\subsection{Absoluteness}
\label{sec:orga70fd07}
\begin{definition}[]
For any formula \(\psi(x_1,\dots,x_n)\) and any class \cm,\cn, 
\(\cm\subseteq \cn\), if
\begin{equation*}
\forall x_1\dots\forall x_n\in\cm(\psi^\cm(x_1,\dots,x_n)
\leftrightarrow\psi^\cn(x_1,\dots,x_n))
\end{equation*}
then \(\psi(x_1,\dots,x_n)\) is \tf{absolute} for \cm,cn. If \(\cn=\cv\), then
\(\psi\) is \tf{absolute} for \cm
\end{definition}

\begin{lemma}[]
Suppose \(\cm\subseteq\cn\) and \(\varphi\),\(\psi\) are formulas, then
\begin{enumerate}
\item if \(\varphi\),\(\psi\) are absolute for \cm,cn, then so are
\(\neg\varphi,\varphi\to\psi\)
\item if \(\varphi\) doesn't contain any quantifiers, then \(\varphi\) is absolute for
any \cm
\item if \cm,\cn  are transitive and \(\varphi\) is absolute for them, then so are
\(\forall x\in y\varphi\)
\end{enumerate}
\end{lemma}

\begin{definition}[]
\(\Delta_0\) formula
\begin{enumerate}
\item \(x=y,x\in y\) are \(\Delta_0\) formulas
\item if \(\varphi\),\(\psi\) are \(\Delta_0\), then so are \(\neg\varphi,\varphi\to\psi\)
\item if \(\varphi\) is \(\Delta_0\), \(y\) is any set, then \((\forall x\in y)\varphi\)
is \(\Delta_0\)
\end{enumerate}


If \(\varphi\) is \(\Delta_0\), then \(\exists x_1\dots\exists x_n\varphi\) is
\(\Sigma_1\) formula, \(\forall x_1\dots\forall x_n\varphi\) is \(\Pi_1\)
\end{definition}

\begin{lemma}[]
\(\cm\subseteq\cn\) are both transitive, \(\psi(x_0,\dots,x_n)\) is a formula,
then
\begin{enumerate}
\item if \(\psi\) is \(\Delta_0\), then it's absolute for \cm,cn
\item if \(\psi\) is \(\Sigma_1\), then
\begin{equation*}
\forall x_1\dots x_n(\psi^\cm(x_1,\dots,x_n)\to\psi^\cn(x_1,\dots,x_n))
\end{equation*}
\item if \(\psi\) is \(\Pi_1\), then
\begin{equation*}
\forall x_1\dots x_n(\psi^\cn(x_1,\dots,x_n)\to\psi^\cm(x_1,\dots,x_n))
\end{equation*}
\end{enumerate}
\end{lemma}

\begin{lemma}[]
If \(\cm\subseteq\cn\), \(\cm\models\Sigma,\cn\models\Sigma\) and
\begin{equation*}
\Sigma\vdash\forall x_1\dots\forall x_n(\varphi(x_1,\dots,x_n)\leftrightarrow
\psi(x_1,\dots,x_n))
\end{equation*}
then \(\varphi\) is absolute for \cm,\cn if and only if \(\psi\) is absolute for \cm,\cn
\end{lemma}


\begin{definition}[]
Suppose \(\cm\subseteq\cn\), \(f(x_1,\dots,x_n)\) is a function. \(f\) is
\tf{absolute} for \cm and \cn if and only if \(\varphi(x_1,\dots,x_n,x_{n+1})\)
defining \(f\) is absolute.
\end{definition}

\begin{theorem}[]
Following relations and functions can be defined in
\(\zfmm-\text{Pow}-\text{Inf}\) and are equivalent to some \(\Delta_0\) formulas.
So they are absolute for any transitive model \cm on 
\(\zfmm-\text{Pow}-\text{Inf}\)
\begin{enumerate}
\item \(x\in y\)
\item \(x=y\)
\item \(x\subset y\)
\item \(\{x,y\}\)
\item \(\{x\}\)

\item \((x,y)\)
\item \(\emptyset\)
\item \(x\cup y\)
\item \(x-y\)
\item \(x\cap y\)
\item \(x^+\)
\item \(x\) is a transitive set
\item \(\bigcup x\)
\item \(\bigcap x\) (\(\bigcap\emptyset=\emptyset\))
\end{enumerate}
\end{theorem}

\begin{lemma}[]
Absoluteness is closed under operation composition
\end{lemma}

\begin{theorem}[]
Following relations and functions are absolute for any transitive model \cm on 
\(\zfmm-\text{Pow}-\text{Inf}\)
\begin{enumerate}
\item \(z\) is an ordered pair
\item \(A\times B\)
\item \(R\) is a relation
\item \(\dom{R}\)
\item \(\ran{R}\)
\item \(f\) is a function
\item \(f(x)\)
\item \(f\) is injective
\end{enumerate}
\end{theorem}
\subsection{Relative consistence of the axiom of foundation}
\label{sec:org0b843c6}
\begin{lemma}[]
Suppose transitive class \(\cm\models\zfmm-\text{Pow}-\text{inf}\) and
\(\omega\in\cm\), then the axiom of infinity is true in \cm. Hence the axiom of
infinity is true in \wf
\end{lemma}

\begin{theorem}[]
\label{7.5.2}
Let \(T\) be a theory of set theory language and \(\Sigma\) a set of sentences.
Suppose \cm is a class and \(T\vdash\cm\neq\emptyset\), then if
\(\cm\models_T\Sigma\), then
\begin{enumerate}
\item for any sentences \(\varphi\), if \(\Sigma\vdash\varphi\), then
\(T\vdash\varphi^\cm\)
\item if \(T\) is consistent, then so is \(\text{Cn}(\Sigma)\)
\end{enumerate}
\end{theorem}


\begin{theorem}[]
The axiom of foundation is consistent with \zfm.
\end{theorem}

\begin{proof}
By \ref{7.5.2}, let T be \zfm, \(\Sigma\) be \zf and \cm be \wf
\end{proof}

\begin{lemma}[$\zfmm$]
Suppose transitive model \(\cmm\models\zfmm-\text{Pow}-\text{Inf}\). If
\(X,R\in\cm\) and \(R\) is a well-order on \(X\), then
\begin{equation*}
(R\text{ is a well-order on }X)^\cmm
\end{equation*}
\end{lemma}

\begin{theorem}[$\zfmm$]
\(V_\omega\models\zfc-\text{Inf}+\neg\text{Inf}\)
\end{theorem}
\begin{proof}
For any \(X\in V_\omega\), \(X\) is finite hence there is a well-ordering on \(X\)
\end{proof}

\begin{corollary}
$\con{\zfmm}\to\con{\zfc-\text{Inf}+\neg\text{Inf}}$
\end{corollary}
\subsection{Induction and recursion based on well-order relation}
\label{sec:orgf6d309a}
\begin{definition}[]
\(\bR\) is a well-founded relation on \(\bX\) if and only if
\begin{equation*}
\forall U\subset\bX(U\neq\emptyset\to\exists y\in U(\neg\exists z\in U(z\bR y)))
\end{equation*}
\end{definition}


\begin{definition}[]
Relation \(\bR\) is \tf{set-like} on \(\bX\) iff for any \(x\in\bX\),\par
\(\{y\in\bX\mid y\bR x\}\) is a set
\end{definition}

\begin{definition}[]
If \(\bR\) is a set-like relation on \(\bX\) and \(x\in \bX\), define
\begin{align*}
\pred^0(\bX,x,\bR)&=\{y\in\bX\mid y\bR x\}\\
\pred^{n+1}(\bX,x,bR)&=\bigcup\{\pred(\bX,y,\bR)\mid y\in\pred^n(\bX,x,\bR)\}\\
\cl(\bX,x,\bR)&=\displaystyle\bigcup_{n\in\omega}\pred^n(\bX,x,\bR)
\end{align*}
\end{definition}

\begin{lemma}[]
If \(\bR\) is a set-like relation on \(\bX\), then for any \(y\in\cl(\bX,x,\bR)\),
\(\pred(\bX,y,\bR)\subseteq\cl(\bX,x,\bR)\)
\end{lemma}

\begin{theorem}[Induction on well-founded set-like relation]
If \(\bR\) is a well-founded set-like relation on \(\bX\), then every nonempty 
\(\bY\subseteq\bX\) has minimal element under \(\bR\)
\end{theorem}

\begin{theorem}[]
Suppose \(\bR\) is a well-founded set-like relation on \(\bX\). If 
\(\bF:\bX\times\bV\to\bV\), then there is a unique \(\bG:\bX\to\bV\) s.t.
\begin{equation*}
\forall x\in\bX(\bG(x)=\bF(x,\bG\restriction\pred(\bX,x,\bR)))
\end{equation*}
\end{theorem}

\begin{definition}[]
If \(\bR\) is a set-like well-founded relation on \(\bX\), define 
\begin{equation*}
\rank(x,\bX,\bR)=\sup\{\rank(y,\bX,\bR)+1\mid y\bR x\wedge y\in\bX\}
\end{equation*}
\end{definition}

Note that
\begin{equation*}
\bF(x,h)=\sup\{\alpha+1\mid\alpha\in\ran{h}\}
\end{equation*}

\begin{lemma}[$\zfmm$]
If \(\bX\) is transitive and \(\in\) is well-founded on \(\bX\), then
\(\bX\subseteq\wf\) and for any \(x\in\bX\), \(\rank(x,\bX,\in)=\rank(x)\)
\end{lemma}

\begin{definition}[]
\(\bR\) is a set-like well-founded relation on \(\bX\), \tf{Mostowski function}
\(\bG\) on \((\bX,\bR)\) is 
\begin{equation*}
\bG(x)=\{\bG(y)\mid y\in\bX\wedge y\bR x\}
\end{equation*}
\(\cmm=\ran{\bG}\) is called the \tf{Mostowski collapse} of \((\bX,\bR)\)
\end{definition}

\begin{lemma}[]
\begin{enumerate}
\item \(\forall x,y\in\bX(x\bR y\to\bG(x)\in\bG(y))\)
\item \cm is transitive
\item If the axiom of power set holds, \(\cm\subseteq\wf\)
\item if the axiom of power set holds and \(x\in\bX\), then\par
\(\rank(x,\bX,\bR)=\rank(\bG(x))\)
\end{enumerate}
\end{lemma}

\begin{definition}[]
\(\bR\) is extensional on \(\bX\) iff
\begin{equation*}
\forall x,y\in\bX(\forall z\in\bX(z\bR x\leftrightarrow z\bR y)\to x=y)
\end{equation*}
\end{definition}

\begin{lemma}[]
If \(\bX\) is transitive then \(\in\) is extensional on \(\bX\)
\end{lemma}


\begin{lemma}[]
Let \(\bR\) be a set-like well-founded relation on \(\bX\), \(\bG\) is a Mostowski
function on it. If \(\bR\) is extensional, then \(\bG\) is an isomorphism
\end{lemma}

\begin{theorem}[Mostowski collapse theorem]
Suppose \(\bR\) is set-like well-founded extensional on \(\bX\), then there are
unique transitive class \cm and bijection \(\bG:\bX\to\cm\) s.t. 
\(\bG:(\bX,\bR)\cong(\cm,\in)\)
\end{theorem}
\subsection{Absoluteness under the axiom of foundation}
\label{sec:org2c7dc70}
\begin{theorem}[]
The following relations and functions can be defined by formulas in
\(\zf-\text{Pow}\) and are equivalent to some \(\Delta_0\) formulas
\begin{enumerate}
\item \(x\) is an ordinal
\item \(x\) is a limit ordinal
\item \(x\) is a successor ordinal
\item \(\omega\)
\item \(x\) is a finite ordinal
\item \(0,1,2,\dots,20,\dots\)
\end{enumerate}
\end{theorem}

\begin{theorem}[]


If transitive model \(\cm\models\zf-\text{Pow}\), then every finite subset of
\cm belongs to \cm
\end{theorem}

\begin{proof}
prove 
\begin{equation*}
\forall x\subset\cm(\abs{x}=n\to x\in\cm)
\end{equation*}
\end{proof}

\begin{theorem}[]
The following concepts are absolute for any transitive model of
\(\zf-\text{Pow}\) 
\begin{enumerate}
\item \(x\) is finite
\item \(X^n\)
\item \(X^{<\omega}\)
\item \(R\) is a well-ordering on \(X\)
\item \(\text{type}(X,R)\)
\item \(\alpha+1\)
\item \(\alpha-1\)
\item \(\alpha+\beta\)
\item \(\alpha\cdot\beta\)
\end{enumerate}
\end{theorem}


Class \(\bX\) is in fact a formula \(\bX(x)\). It's absolute for \cm if and only
if \(\forall x\in\cm(\bX^\cm(x)\leftrightarrow\bX(x))\), which is equivalent to
\(\{x\in\cm\mid\bX(x)\}=\{x\in\cm\mid\bX^\cm(x)\}\). Hence \(\bX\) is absolute
for \cm if and only if \(\bX^\cm=\cm\cap\bX\)

\begin{theorem}[]
Suppose \(\bR\) is a well-founded set-like relation on \(\bX\),
\(\bF:\bX\times\bV\to\bV\),
\begin{equation*}
\forall x\in\bX(\bG(x)=\bF(x,\bG\restriction(\bX,x,\bR)))
\end{equation*}
transitive model \(\cm\models\zf-\text{Pow}\) and
\begin{enumerate}
\item \(\bF\) is absolute for \cm
\item \(\bX,\bR\) are absolute for \cm, \((\bR\text{ is set-like on }\bX)^\cm\) and
\begin{equation*}
\forall x\in\cm(\pred(\bX,x,\bR)\subseteq\cm)
\end{equation*}
\end{enumerate}


then \(\bG\) is absolute for \(\cm\)
\end{theorem}

\begin{theorem}[]
The following concept is absolute for any transitive model of
\(\zf-\text{Pow}\)
\begin{enumerate}
\item \(\alpha^\beta\)
\item \(\rank(x)\)
\item \(\trcl{x}\)
\end{enumerate}
\end{theorem}

\begin{lemma}[]
transitive \(\cm\models\zf\)
\begin{enumerate}
\item if \(x\in\cm\), then \(\calp^\cm(x)=\calp(x)\cap\cm\)
\item if \(\alpha\in\cm\), then \(V_\alpha^\cm=V_\alpha\cap\cm\)
\end{enumerate}
\end{lemma}
\subsection{Unaccessible cardinal and models of \zfc}
\label{sec:org4686132}
\(\bZ=\zff-\text{Rep},\zfmm=\zfcm-\text{Rep}\)
\begin{theorem}[]
If \(\gamma>\omega\) is a limit ordinal, then \(V_\gamma\models_{\zff}\bZ\) and 
\(V_\gamma\models_{\zfcm}\zc\)
\end{theorem}

\begin{corollary}[]
\(V_{\omega+\omega}\) doesn't satisfies the axiom of replacement
\end{corollary}

\begin{proof}
   
\end{proof}

\begin{theorem}[]
\(\zcm\not\vdash\exists x(x=V_\omega),\zcm\not\vdash\forall x\exists y(\trcl{x}=y)\)
\end{theorem}

\begin{theorem}[]
If \(\kappa\) is an inaccessible cardinal, then \(V_\kappa\models_{\zfmm}\zff\),\par
\(V_\kappa\models_{\zfcc}\zfc\)
\end{theorem}

\begin{proof}
Since \(\kappa\) is inaccessible, \(\abs{V_\kappa}=\kappa\). For any \(A\in
   V_\kappa\), \(\abs{A}<\kappa\). Since \(\kappa\) is regular, any 
\(f:A\to V_\kappa\) is bounded. Hence there exists \(\alpha<\kappa\) s.t. 
\(\ran{f}\subseteq V_\alpha\)
\end{proof}

\begin{corollary}[]
We cannot prove "there is some inaccessible cardinals" in \zfc
\end{corollary}

\begin{proof}
Suppose we could. Then we have \(V_\kappa\models\zfc\), which contradicts
Gödel’s second incompleteness theorem 
\end{proof}

\begin{lemma}[]
Suppose \(\kappa\) is inaccessible. The following concepts are absolute for
\(V_\kappa\) 
\begin{enumerate}
\item \(x\) is a cardinal
\item \(x\) is a regular cardinal
\item \(x\) is an inaccessible cardinal
\end{enumerate}
\end{lemma}

\begin{lemma}[]
\(\con(\zfcm)\to\con(\zfcm+\text{"there is no inaccessible cardinal"})\)
\end{lemma}

\begin{proof}
If \(\kappa\) is the smallest inaccessible cardinal, then \par
\(V_\kappa\models\zfcm+\text{"there is no inaccessible cardinal"}\). Define
\begin{equation*}
\cm=\bigcap\{V_\kappa\mid\kappa\text{ is inaccessible}\}
\end{equation*}
\end{proof}
If there are, then \(\cm=V_\kappa\)

\begin{corollary}[]
\con(\zfcm)\textlnot{}\(\to\)\con(\zfcm+\text{"there are some inaccessible cardinals"})
\end{corollary}

\begin{definition}[]
For any infinite cardinal \(\kappa\), \(H_\kappa=\{x\mid\abs{\trcl{x}}<\kappa\}\)
is the collection of sets which \tf{hereditarily have size less than } \(\kappa\).
Element of \(H_\omega\) is called \tf{hereditarily finite set}. Element of
\(H_{\omega_1}\) is called \tf{hereditarily countable set}
\end{definition}

\begin{lemma}[]
For any infinite cardinal \(\kappa\), \(H_\kappa\subseteq V_\kappa\)
\end{lemma}

\begin{lemma}[]
If \(\kappa\) is regular, then \(H_\kappa=V_\kappa\) if and only if \(\kappa\) is
inaccessible
\end{lemma}

\begin{proof}
which implies \(\abs{V_\kappa}=\kappa\)
\end{proof}

\begin{lemma}[]
For any infinite cardinal \(\kappa\)
\begin{enumerate}
\item \(H_\kappa\) is transitive
\item \(H_\kappa\cap\on=\kappa\)
\item If \(x\in H_\kappa\), then \(\bigcup x\in H_\kappa\)
\item If \(x,y\in H_\kappa\), then \(\{x,y\}\in H_\kappa\)
\item If \(x\in H_\kappa,y\subseteq x\), then \(y\in H_\kappa\)
\item if \(\kappa\) is regular, then \(\forall x(x\in H_\kappa\leftrightarrow
      x\subset H_\kappa\wedge\abs{x}<\kappa)\)
\end{enumerate}
\end{lemma}

\begin{theorem}[]
If \(\kappa\) is uncountable regular cardinal, then
\(H_\kappa\models_{\zfcm}\zfcm-\text{Pow}\) 
\end{theorem}

\begin{theorem}[]
If \(\kappa\) is uncountable regular cardianl, then the following propositions
are equivalent
\begin{enumerate}
\item \(H_\kappa\models\zfcm\)
\item \(H_\kappa=V_\kappa\)
\item \(\kappa\) is inaccessible
\end{enumerate}
\end{theorem}

\begin{corollary}[]
\(\con(\zfcm)\to\con(\zfcm-\text{pow}+\forall x(x\text{ is countable}))\)
\end{corollary}
\subsection{Reflection theorem}
\label{sec:org62f3e4f}
\begin{lemma}[]
\(\cm\subseteq\cn\) are classes. \(\varphi_1,\dots,\varphi_n\) is a sequence
closed under subformula, then the following propositions are equivalent
\begin{enumerate}
\item \(\varphi_1,\dots,\varphi_n\) are absolute for \cm and \cn
\item if \(\varphi_i=\exists\varphi_j(x,y_1,\dots,y_m)\), then
\begin{equation*}
\forall y_1,\dots,y_m\in\cm(\exists x\in\cn\varphi_j^\cn(x,y_1,\dots,y_m)
\to\exists x\in\cm\varphi_j^\cm(x,y_1,\dots,y_m))
\end{equation*}
\end{enumerate}
\end{lemma}

\begin{theorem}[reflection theorem(\zff)]
For any finite formula set \(F=\{\varphi_1,\dots,\varphi_n\}\), for any
\(V_\alpha\), there exists \(V_\beta\) s.t. \(V_\alpha\subseteq V_\beta\) and 
\(\varphi_1,\dots,\varphi_n\) are absolute for \(V_\beta\)
\end{theorem}

\begin{corollary}[\zff]
\(F=\{\sigma_1,\dots,\sigma_n\}\) are finite subsets of \zf, then
\begin{equation*}
\forall\alpha\exists\beta>\alpha(\sigma_1^{V_\beta}\wedge\dots\wedge\sigma_n^{V_\beta})
\end{equation*}
\end{corollary}

\begin{corollary}[]
\(F=\{\sigma_1,\dots,\sigma_n\}\) is a finite subset of \zf. Unless \zf is
unconsistent, \(F\) cannot prove all axioms of \zf
\end{corollary}

\begin{theorem}[\zfcm]
For any finite formula set \(F=\{\varphi_1,\dots,\varphi_n\}\), for any set
\(N\), there exists set \(M\) s.t.
\begin{enumerate}
\item \(N\subseteq M\)
\item \(\varphi_1,\dots,\varphi_n\) are absolute for \((M,\in)\)
\item \(\abs{M}\le\abs{N}\cdot\omega\)
\end{enumerate}
\end{theorem}

\begin{corollary}[\zfcm]
For any finite formula set \(F=\{\varphi_1,\dots,\varphi_n\}\), for any set
\(N\), there exists set \(M\) s.t.
\begin{enumerate}
\item \(N\subseteq M\)
\item \(\varphi_1,\dots,\varphi_n\) are absolute for \((M,\in)\)
\item \(\abs{M}\le\abs{N}\cdot\omega\)
\item \(M\) is transitive
\end{enumerate}
\end{corollary}
\newpage
\section{Constructable Set - Venlafaxine}
\label{sec:orge72ff85}
\subsection{Definablity and Gödel operation}
\label{sec:org6939a1a}
\begin{definition}[]
\(M\) is a set, \(\psi(x_1,\dots,x_n,y_1,\dots,y_m)\) is a formula, 
\(X\subseteq M^n\) is \tf{definable in $M$ from parameters from $\psi$} if and
only if there are \(y_1,\dots,y_m\in M\) s.t.
\begin{equation*}
X=\{(x_1,\dots,x_n)\mid(\psi^M(x_1,\dots,x_n,y_1,\dots,y_m))\}
\end{equation*}
\begin{equation*}
\deff(M)=\{X\subseteq M\mid\exists\psi,X\text{ is definable in } 
M \text{ from } \psi\} 
\end{equation*}
\end{definition}

\begin{definition}[]
\tf{Gödel operation}
\begin{enumerate}
\item \(G_1(X,Y)=\{X,Y\}\)
\item \(G_2(X,Y)=X\times Y\)
\item \(G_3(X,Y)=\in\restriction X\times Y\)
\item \(G_4(X,Y)=X-Y\)
\item \(G_5(X,Y)=X\cap Y\)
\item \(G_6(X,Y)=\bigcap X\)
\item \(G_7(X,Y)=\dom{X}\)
\item \(G_8(X,Y)=\{(x,y)\mid(y,x)\in X\}\)
\item \(G_9(X,Y)=\{(x,y,z)\mid(x,z,y)\in X\}\)
\item \(G_{10}(X,Y)=\{(x,y,z)\mid(y,z,x)\in X\}\)
\end{enumerate}


Class \(C\) is closed under Gödel operation if for any \(X,Y\), X,Y\(\in\) C\$ implies
\(G_i(X,Y)\in C\). For any set \(M\), \(\cl_G(M)\) is the 
\tf{closure under Gödel operation}
\end{definition}

\begin{definition}[]
\(\psi\) is a \tf{normal form} if
\begin{enumerate}
\item only \(\neg,\wedge,\exists\) are logical symbol
\item = doesn't appear
\item if \(x_i\in x_j\) then \(i\neq j\)
\item \(\exists\) only shown as: \(\exists x_{m+1}\in
      x_i\varphi(x_1,\dots,x_{m+1})\), \(1\le i\le m\)
\end{enumerate}
\end{definition}

\begin{lemma}[]
Any \(\Delta_0\) formula can be transformed into normal form
\end{lemma}

\begin{theorem}[]
For any \(\Delta_0\) formula \(\psi(x_1,\dots,x_n)\), there is Gödel operations'
composition \(G\) s.t. for any \(X_1,\dots,X_n\)
\begin{align*}
G(X_1,\dots,X_n)=&\{(x_1,\dots,x_n)\mid\\
&x_1\in X_1\wedge\dots\wedge x_n\in X_n\wedge\psi(x_1,\dots,x_n)\}
\end{align*}
\end{theorem}

\begin{corollary}[]
If \(M\) is transitive and \(M=\cl_G(M)\), then for any \(\Delta_0\) formula
\(\psi(x,y_1,\dots,y_m)\), any set \(X\in M\), any \(y_1,\dots,y_m\in M\) if
\begin{equation*}
Y=\{x\in X\mid\psi(x,y_1,\dots,y_m)\}
\end{equation*}
then \(Y\in M\). Hence \(\Delta_0\) schema of specification holds in \(M\)
\end{corollary}

\begin{lemma}[]
If \(G(X_1,\dots,X_n)\) is Gödel operations' composition, then
\(Z=G(X_1,\dots,X_n)\) is equivalent to a \(\Delta_0\) formula
\end{lemma}

\begin{theorem}[]
For any transitive set \(M\), \(\deff(M)=\cl_G(M\cup\{M\})\cap\calp(M)\)
\end{theorem}

\begin{lemma}[]
If transitive \(\cm\models\zff\), then for any transitive set \(M\in\cm\),
\(\deff(M)\) is absolute for \cm
\end{lemma}

\begin{lemma}[]
For any transitive set \(M\)
\begin{enumerate}
\item \(\deff(M)\subseteq\calp(M)\)
\item \(M\subseteq \deff(M)\)
\item for any \(X\subseteq M\), if \(X\) is finite, then \(X\in\deff(M)\)
\item assume \(\ac\) and \(\abs{M}\ge\omega\), then \(\abs{\deff(M)}=\abs{M}\)
\end{enumerate}
\end{lemma}
\subsection{Gödel's L}
\label{sec:org227a399}
\begin{definition}[]
for any \(\alpha\)
\begin{enumerate}
\item \(L_0=\emptyset\)
\item \(L_{\alpha+1}=\deff(L_\alpha)\)
\item For any limit \(\alpha\), \(L_\alpha=\bigcup_{\beta<\alpha}L_\beta\)
\end{enumerate}


\(\textbf{L}=\displaystyle\bigcup_{\alpha\in\on}L_\alpha\). Element of \gl is
called constructible set
\end{definition}

\begin{lemma}[]
For any ordinal \(\alpha\)
\begin{enumerate}
\item \(L_\alpha\) is transitive
\item If \(\alpha<\beta\), then \(L_\alpha\subseteq L_\beta\)
\item \(L_\alpha\subseteq V_\alpha\)
\end{enumerate}
\end{lemma}

\begin{definition}[]
\(x\in\gll\)
\begin{equation*}
\rank_\gll(x)=\min\{\beta\mid x\in\gll_{\beta+1}\}
\end{equation*}
\end{definition}

\begin{lemma}[]
For any \(\alpha\)
\begin{equation*}
L_\alpha=\{x\in\gll\mid\rank_{\gll}(x)<\alpha\}
\end{equation*}
\end{lemma}

\begin{lemma}[]
For any ordinal \(\alpha\)
\begin{enumerate}
\item \(L_\alpha\cap\on=\alpha\)
\item \(\alpha\in\gll\cap\rank_{\gll}(\alpha)=\alpha\)
\end{enumerate}
\end{lemma}

\begin{proof}
since "\(\alpha\) is a cardinal" is absolute for any transitive set. 
\begin{align*}
\alpha&=L_\alpha\cap\on=\{\eta\in L_\alpha\mid\eta\text{ is a ordinal}\}\\
&=\{\eta\in L_\alpha\mid(\eta\text{ is an ordinal}^{L_\alpha})\}\in\deff(L_\alpha)
\end{align*}
\end{proof}

\begin{lemma}[]
for any ordinal \(\alpha\)
\begin{enumerate}
\item \(L_\alpha\in L_{\alpha+1}\)
\item any finite subset of \(L_\alpha\) belongs to \(L_{\alpha+1}\)
\end{enumerate}
\end{lemma}

\begin{lemma}[]
\begin{enumerate}
\item \(\forall n\in\omega(L_n=V_n)\)
\item \(L_\omega=V_\omega\)
\end{enumerate}
\end{lemma}

\begin{lemma}[]
If \ac, then for any \(\alpha\ge\omega,\abs{L_\alpha}=\abs{\alpha}\)
\end{lemma}

\begin{theorem}[]
\(\gll\models\zff\)
\end{theorem}

\subsection{Axiom of constructibility and relativization}
\label{sec:orgdf77948}
\begin{theorem}[Axiom of constructibility]
\(\cvm=\gll\)
\end{theorem}

\begin{lemma}[]
\label{8.3.2}
function \(\alpha\mapsto L_\alpha\) is absolute for any transitive model of \zf
\end{lemma}

\begin{theorem}[]
\(\gll\models\zff+\cv=\gll\)
\end{theorem}

\begin{proof}
\((\cv=\gll)^\gll\) is \(\forall x\in\gll\exists\alpha\in\gll(x\in
   L_\alpha)^\gll\). 
By \ref{8.3.2}, \((x\in L_\alpha)^\gll\Leftrightarrow x\in L_\alpha\). Hence
\(\gll\models\cv=\gll\)
\end{proof}
Hence
\begin{theorem}[]
\(\con(\zff)\to\con(\zff+\cv=\gll)\)
\end{theorem}

\begin{theorem}[]
Suppose transitive proper class \(\cmm\models\zff-\text{Pow}\), then\par
\(\gll=\gll^\cmm\subseteq\cmm\) 
\end{theorem}

\begin{proof}
For any ordinal \(\alpha\), since \cm is proper, \(\cmm\not\subseteq V_\alpha\).
Hence there is \(x\in\cmm\) s.t. \(\rank(x)\ge\alpha\). Since rank is absolute,
\(\rank(x)\in\cmm\). And \cm is transitive, hence \(\alpha\in\cmm\). By
\ref{8.3.2}, \(L_\alpha\in\cmm\)
\begin{align*}
 \gll^\cmm &=\{x\in\cmm\mid(\exists\alpha\in\on(x\in L_\alpha))^\cmm\}\\
 &=\{x\mid\exists\alpha\in\on\cap\cmm(x\in\ L_\alpha\cap\cmm)\}\\
 &=\{x\mid\exists\alpha\in\on(x\in L_\alpha)\}\\
 &=\gll
\end{align*}
\end{proof}

\begin{definition}[]
If transitive model \(\cm\models\zff\) contains all ordinals, then it's an
\tf{inner model}
\end{definition}

\begin{lemma}[]
there is a finite set of axioms \(\{\psi_1,\dots,\psi_n\}\) of
\(\zff-\text{Pow}\) s.t. ordinals,rank and \(L_\alpha\) are absolute for any
model of \(\{\psi_1,\dots,\psi_n\}\)
\end{lemma}

\begin{lemma}[]
If set \(M\) is transitive, then \(M\cap\on\) is a ordinall and is the least that
doesn't belong to \(M\), denoted by \(\alpha^M\)
\end{lemma}

\begin{theorem}[]
There is a finite subset \(\{\psi_1,\dots,\psi_n\}\) of axioms of
\(\zff-\text{Pow}\) satisfying
\begin{equation*}
\forall M(M\text{ is transitive }\wedge\psi_1^M\wedge\dots\wedge\psi_n^m\to
(L_{\alpha^M}=\gll^M\subseteq M))
\end{equation*}
\end{theorem}

\begin{theorem}[]
The is a finite subset \(\{\psi_1,\dots,\psi_{n+1}\}\) of axioms of
\(\zff-\text{Pow}+\cvm=\gll\) satisfying
\begin{enumerate}
\item If \cm is a transitive proper class and
\(\psi_1^\cm\wedge\dots\wedge\psi_{n+1}^\cm\), then \(\cm=\gll\)
\item \(\forall M(M\text{ is transitive
      }\wedge\psi_1^M\wedge\dots\wedge\psi_n^m\to (L_{\alpha^M}=M))\)
\end{enumerate}
\end{theorem}

\begin{theorem}[]
There is a well-ordering on \gll. Hence \(\cvm=\gll\to\ac\)
\end{theorem}

If \(\cvm=\gll\), hence \(\aleph_\alpha\subseteq L_{\aleph_{\alpha+1}}\). Because
\(\abs{L_{\alpha_{\alpha+1}}}=\aleph_{\alpha+1}\),
\(2^{\aleph_\alpha}\le\aleph_{\alpha+1}\) 
\begin{theorem}[]
If \(\cvm=\gll\), then for any infinite ordinal \(\alpha\),
\(\calp(L_\alpha)\subseteq L_{\abs{\alpha}^+}\)
\end{theorem}

\begin{corollary}[$\zff$]
\((\ac+\gchh)^\gll\)
\end{corollary}

\begin{theorem}[$\zff$]
\(\con(\zff)\to\con(\zfcm+\gchh)\)
\end{theorem}

\begin{theorem}[$\zff$]
Suppose \(S_0=\{\psi_1,\dots,\psi_n\}\subseteq\zff+\cvm=\gll\), then
\begin{equation*}
\zff\vdash\exists M(\abs{M}=\omega\wedge M\text{ is transitive}\wedge
(\psi_1^M\wedge\dots\wedge\psi_n^M))
\end{equation*}
\end{theorem}

\begin{lemma}[]
Suppose \(\cvm=\gll\). For any uncountable regular cardinal \(\kappa\),
\(L_\kappa=H_\kappa\) 
\end{lemma}

\begin{corollary}[]
If \(\kappa\) is a uncountable regular cardinal, then
\(\L_\kappa\models\zff-\text{Pow}+\cvm=\gll\). If \(\kappa\) is inaccessible, then 
\(L_\kappa\models\zff+\cvm=\gll\)
\end{corollary}

\newpage

\section{The end}
\label{sec:org2235f57}
Learn and forget
\end{document}