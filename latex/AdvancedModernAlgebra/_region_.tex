\message{ !name(AdvancedModernAlgebra.tex)}% Created 2020-03-10 二 09:16
% Intended LaTeX compiler: pdflatex
\documentclass[11pt]{article}
\usepackage[utf8]{inputenc}
\usepackage[T1]{fontenc}
\usepackage{graphicx}
\usepackage{grffile}
\usepackage{longtable}
\usepackage{wrapfig}
\usepackage{rotating}
\usepackage[normalem]{ulem}
\usepackage{amsmath}
\usepackage{textcomp}
\usepackage{amssymb}
\usepackage{capt-of}
\usepackage{imakeidx}
\usepackage{hyperref}
\usepackage{minted}
% TIPS
% \substack{a\\b} for multiple lines text





% pdfplots will load xolor automatically without option
\usepackage[dvipsnames]{xcolor}

\usepackage{forest}
% two-line text in node by [two \\ lines]
% \begin{forest} qtree, [..] \end{forest}
\forestset{
  qtree/.style={
    baseline,
    for tree={
      parent anchor=south,
      child anchor=north,
      align=center,
      inner sep=1pt,
    }}}
%\usepackage{flexisym}
% load order of mathtools and mathabx, otherwise conflict overbrace

\usepackage{mathtools}
%\usepackage{fourier}
\usepackage{pgfplots}
\usepackage{amsthm}
\usepackage{amsmath}
%\usepackage{unicode-math}
%
\usepackage{commath}
%\usepackage{,  , }
\usepackage{amsfonts}
\usepackage{amssymb}
% importing symbols https://tex.stackexchange.com/questions/14386/importing-a-single-symbol-from-a-different-font
%mathabx change every symbol
% use instead stmaryrd
%\usepackage{mathabx}
\usepackage{stmaryrd}
\usepackage{empheq}
\usepackage{tikz}
\usepackage{tikz-cd}
%\usepackage[notextcomp]{stix}
\usetikzlibrary{arrows.meta}
\usepackage[most]{tcolorbox}
%\utilde
%\usepackage{../../latexpackage/undertilde/undertilde}
% left and right superscript and subscript
\usepackage{actuarialsymbol}
\usepackage{threeparttable}
\usepackage{scalerel,stackengine}
\usepackage{stackrel}
% \stackrel[a]{b}{c}
\usepackage{dsfont}
% text font
\usepackage{newpxtext}
%\usepackage{newpxmath}

%\newcounter{dummy} \numberwithin{dummy}{section}
\newtheorem{dummy}{dummy}[section]
\theoremstyle{definition}
\newtheorem{definition}[dummy]{Definition}
\newtheorem{corollary}[dummy]{Corollary}
\newtheorem{lemma}[dummy]{Lemma}
\newtheorem{proposition}[dummy]{Proposition}
\newtheorem{theorem}[dummy]{Theorem}
\theoremstyle{definition}
\newtheorem{example}[dummy]{Example}
\theoremstyle{remark}
\newtheorem*{remark}{Remark}


\newcommand\what[1]{\ThisStyle{%
    \setbox0=\hbox{$\SavedStyle#1$}%
    \stackengine{-1.0\ht0+.5pt}{$\SavedStyle#1$}{%
      \stretchto{\scaleto{\SavedStyle\mkern.15mu\char'136}{2.6\wd0}}{1.4\ht0}%
    }{O}{c}{F}{T}{S}%
  }
}

\newcommand\wtilde[1]{\ThisStyle{%
    \setbox0=\hbox{$\SavedStyle#1$}%
    \stackengine{-.1\LMpt}{$\SavedStyle#1$}{%
      \stretchto{\scaleto{\SavedStyle\mkern.2mu\AC}{.5150\wd0}}{.6\ht0}%
    }{O}{c}{F}{T}{S}%
  }
}

\newcommand\wbar[1]{\ThisStyle{%
    \setbox0=\hbox{$\SavedStyle#1$}%
    \stackengine{.5pt+\LMpt}{$\SavedStyle#1$}{%
      \rule{\wd0}{\dimexpr.3\LMpt+.3pt}%
    }{O}{c}{F}{T}{S}%
  }
}

\newcommand{\bl}[1] {\boldsymbol{#1}}
\newcommand{\Wt}[1] {\stackrel{\sim}{\smash{#1}\rule{0pt}{1.1ex}}}
\newcommand{\wt}[1] {\widetilde{#1}}
\newcommand{\tf}[1] {\textbf{#1}}


%For boxed texts in align, use Aboxed{}
%otherwise use boxed{}

\DeclareMathSymbol{\widehatsym}{\mathord}{largesymbols}{"62}
\newcommand\lowerwidehatsym{%
  \text{\smash{\raisebox{-1.3ex}{%
    $\widehatsym$}}}}
\newcommand\fixwidehat[1]{%
  \mathchoice
    {\accentset{\displaystyle\lowerwidehatsym}{#1}}
    {\accentset{\textstyle\lowerwidehatsym}{#1}}
    {\accentset{\scriptstyle\lowerwidehatsym}{#1}}
    {\accentset{\scriptscriptstyle\lowerwidehatsym}{#1}}
}

\usepackage{graphicx}
    
% text on arrow for xRightarrow
\makeatletter
%\newcommand{\xRightarrow}[2][]{\ext@arrow 0359\Rightarrowfill@{#1}{#2}}
\makeatother


\newcommand{\dom}[1]{%
\mathrm{dom}{(#1)}
}

% Roman numerals
\makeatletter
\newcommand*{\rom}[1]{\expandafter\@slowromancap\romannumeral #1@}
\makeatother

\def \fR {\mathfrak{R}}
\def \bx {\boldsymbol{x}}
\def \bz {\boldsymbol{z}}
\def \ba {\boldsymbol{a}}
\def \bh {\boldsymbol{h}}
\def \bo {\boldsymbol{o}}
\def \bU {\boldsymbol{U}}
\def \bc {\boldsymbol{c}}
\def \bV {\boldsymbol{V}}
\def \bI {\boldsymbol{I}}
\def \bK {\boldsymbol{K}}
\def \bt {\boldsymbol{t}}
\def \bb {\boldsymbol{b}}
\def \bA {\boldsymbol{A}}
\def \bX {\boldsymbol{X}}
\def \bu {\boldsymbol{u}}
\def \bS {\boldsymbol{S}}
\def \bZ {\boldsymbol{Z}}
\def \bz {\boldsymbol{z}}
\def \by {\boldsymbol{y}}
\def \bw {\boldsymbol{w}}
\def \bT {\boldsymbol{T}}
\def \bF {\boldsymbol{F}}
\def \bS {\boldsymbol{S}}
\def \bm {\boldsymbol{m}}
\def \bW {\boldsymbol{W}}
\def \bR {\boldsymbol{R}}
\def \bQ {\boldsymbol{Q}}
\def \bS {\boldsymbol{S}}
\def \bP {\boldsymbol{P}}
\def \bT {\boldsymbol{T}}
\def \bY {\boldsymbol{Y}}
\def \bH {\boldsymbol{H}}
\def \bB {\boldsymbol{B}}
\def \blambda {\boldsymbol{\lambda}}
\def \bPhi {\boldsymbol{\Phi}}
\def \btheta {\boldsymbol{\theta}}
\def \bTheta {\boldsymbol{\Theta}}
\def \bmu {\boldsymbol{\mu}}
\def \bphi {\boldsymbol{\phi}}
\def \bSigma {\boldsymbol{\Sigma}}
\def \lb {\left\{}
\def \rb {\right\}}
\def \la {\langle}
\def \ra {\rangle}
\def \caln {\mathcal{N}}
\def \dissum {\displaystyle\Sigma}
\def \dispro {\displaystyle\prod}
\def \E {\mathbb{E}}
\def \Q {\mathbb{Q}}
\def \N {\mathbb{N}}
\def \V {\mathbb{V}}
\def \R {\mathbb{R}}
\def \P {\mathbb{P}}
\def \A {\mathbb{A}}
\def \Z {\mathbb{Z}}
\def \I {\mathbb{I}}
\def \C {\mathbb{C}}
\def \cala {\mathcal{A}}
\def \calb {\mathcal{B}}
\def \calq {\mathcal{Q}}
\def \calp {\mathcal{P}}
\def \cals {\mathcal{S}}
\def \calg {\mathcal{G}}
\def \caln {\mathcal{N}}
\def \calr {\mathcal{R}}
\def \calm {\mathcal{M}}
\def \calc {\mathcal{C}}
\def \calf {\mathcal{F}}
\def \calk {\mathcal{K}}
\def \call {\mathcal{L}}
\def \calu {\mathcal{U}}
\def \bcup {\bigcup}


\def \uin {\underline{\in}}
\def \oin {\overline{\in}}
\def \uR {\underline{R}}
\def \oR {\overline{R}}
\def \uP {\underline{P}}
\def \oP {\overline{P}}

\def \Ra {\Rightarrow}

\def \e {\enspace}

\def \sgn {\operatorname{sgn}}
\def \gen {\operatorname{gen}}
\def \ker {\operatorname{ker}}
\def \im {\operatorname{im}}

\def \tril {\triangleleft}

% \varprod
\DeclareSymbolFont{largesymbolsA}{U}{txexa}{m}{n}
\DeclareMathSymbol{\varprod}{\mathop}{largesymbolsA}{16}

% \bigtimes
\DeclareFontFamily{U}{mathx}{\hyphenchar\font45}
\DeclareFontShape{U}{mathx}{m}{n}{
      <5> <6> <7> <8> <9> <10>
      <10.95> <12> <14.4> <17.28> <20.74> <24.88>
      mathx10
      }{}
\DeclareSymbolFont{mathx}{U}{mathx}{m}{n}
\DeclareMathSymbol{\bigtimes}{1}{mathx}{"91}
% \odiv
\DeclareFontFamily{U}{matha}{\hyphenchar\font45}
\DeclareFontShape{U}{matha}{m}{n}{
      <5> <6> <7> <8> <9> <10> gen * matha
      <10.95> matha10 <12> <14.4> <17.28> <20.74> <24.88> matha12
      }{}
\DeclareSymbolFont{matha}{U}{matha}{m}{n}
\DeclareMathSymbol{\odiv}         {2}{matha}{"63}


\newcommand\subsetsim{\mathrel{%
  \ooalign{\raise0.2ex\hbox{\scalebox{0.9}{$\subset$}}\cr\hidewidth\raise-0.85ex\hbox{\scalebox{0.9}{$\sim$}}\hidewidth\cr}}}
\newcommand\simsubset{\mathrel{%
  \ooalign{\raise-0.2ex\hbox{\scalebox{0.9}{$\subset$}}\cr\hidewidth\raise0.75ex\hbox{\scalebox{0.9}{$\sim$}}\hidewidth\cr}}}

\newcommand\simsubsetsim{\mathrel{%
  \ooalign{\raise0ex\hbox{\scalebox{0.8}{$\subset$}}\cr\hidewidth\raise1ex\hbox{\scalebox{0.75}{$\sim$}}\hidewidth\cr\raise-0.95ex\hbox{\scalebox{0.8}{$\sim$}}\cr\hidewidth}}}
\newcommand{\stcomp}[1]{{#1}^{\mathsf{c}}}


\setcounter{secnumdepth}{2}
\setcounter{tocdepth}{2}
\DeclareMathOperator{\Frac}{Frac}
\author{Joseph J. Rotman}
\date{\today}
\title{Advanced Modern Algebra}
\hypersetup{
 pdfauthor={Joseph J. Rotman},
 pdftitle={Advanced Modern Algebra},
 pdfkeywords={},
 pdfsubject={},
 pdfcreator={Emacs 26.3 (Org mode 9.3.6)}, 
 pdflang={English}}
\begin{document}

\message{ !name(AdvancedModernAlgebra.tex) !offset(-3) }


\maketitle
\setcounter{tocdepth}{2}
\tableofcontents \clearpage

\section{Things Past}
\label{sec:org6f08532}
\subsection{Some Number Theory}
\label{sec:org5ae0ba8}
\textbf{Least Integer Axiom} (\textbf{Well-ordering Principle}). There is a smallest integer in
every nonempty subset \(C\) of \(\N\)
\subsection{Roots of Unity}
\label{sec:org401d7b7}
\begin{proposition}[Polar Decomposition]
Every complex number \(z\) has a factorization
\begin{equation*}
z=r(\cos\theta+i\sin\theta)
\end{equation*}
where \(r=\abs{z}\ge0\) and \(0\le\theta\le 2\pi\)
\end{proposition}

\begin{proposition}[Addition Theorem]
If \(z=\cos\theta+i\sin\theta\) and \(w=\cos\psi+i\sin\psi\), then
\begin{equation*}
zw=\cos(\theta+\psi)+i\sin(\theta+\psi)
\end{equation*}
\end{proposition}

\begin{theorem}[De Moivre]
\(\forall x\in\R,n\in\N\)
\begin{equation*}
\cos(nx)+i\sin(nx)=(\cos x+i\sin x)^n
\end{equation*}
\end{theorem}

\begin{theorem}[Euler]
\(e^{ix}=\cos x+i\sin x\)
\end{theorem}

\begin{definition}[]
If \(n\in\N\ge 1\) , an \textbf{nth root of unity} is a complex number \(\xi\) with
\(\xi^n=1\)
\end{definition}

\begin{corollary}[]
Every nth root of unity is equal to
\begin{equation*}
e^{2\pi ik/n}=\cos(\frac{2\pi k}{n})+i\sin(\frac{2\pi k}{n})
\end{equation*}
for \(k=0,1,\dots,n-1\)
\end{corollary}

\begin{equation*}
x^n-1=\displaystyle\prod_{\xi^n=1}(x-\xi)
\end{equation*}

If \(\xi\) is an nth root of unity and if \(n\) is the smallest, then \(\xi\) is a
\textbf{primitive} \(n\)\tf{th root of unity}

\begin{definition}[]
If \(d\in\N^+\) , then the \$d\$th \textbf{cyclotomic polynomial} is 
\begin{equation*}
\Phi_d(x)=\displaystyle\prod(x-\xi)
\end{equation*}
where \(\xi\) ranges over all the \emph{primitive dth} roots of unity
\end{definition}

\begin{proposition}[]
For every integer \(n\ge 1\)
\begin{equation*}
x^n-1=\displaystyle\prod_{d|n}\Phi_d(x)
\end{equation*}
\end{proposition}

\begin{definition}[]
Define \textbf{Euler \(\phi\)-function} as the degree of the nth cyclotomic
polynomial
\begin{equation*}
\phi(n)=\deg(\Phi_n(x))
\end{equation*}
\end{definition}

\begin{proposition}[]
If \(n\ge1\) is an integer, then \(\phi(n)\) is the number of integers \(k\) with
\(1\le k\le n\) and \((k,n)=1\)
\end{proposition}

\begin{proof}
Suffice to prove \(e^{2\pi ik/n}\) is a primitive nth root of unity if and only
if \(k\) and \(n\) are relatively prime
\end{proof}

\begin{corollary}[]
For every integer \(n\ge 1\), we have
\begin{equation*}
n=\displaystyle\sum_{d|n}\phi(d)
\end{equation*}
\end{corollary}
\subsection{Some Set Theory}
\label{sec:org52d6ebd}
\begin{proposition}[]
\label{prop1.47}
\begin{enumerate}
\item If \(f:X\to Y\) and \(g:Y\to X\) are functions s.t. \(g\circ f=1_X\), then
\(f\) is injective and \(g\) is surjective
\item A function \(f:X\to Y\) has an inverse \(g:Y\to X\) if and only if \(f\) is a bijection
\end{enumerate}
\end{proposition}
\section{Group \rom{1}}
\label{sec:org172da78}
\subsection{Permutations}
\label{sec:org4008a71}
\begin{definition}[]
A \textbf{permutation} of a set \(X\) is a bijection from \(X\) to itself.
\end{definition}


\begin{definition}[]
The family of all the permutations of a set \(X\), denoted by \(S_X\) is called
the \textbf{symmetric group} on \(X\). When \(X=\lb 1,2,\dots,n\rb\), \(S_X\) is
usually denoted by \(X_n\) and is called the \textbf{symmetric group on } \(n\)
\textbf{letters} 
\end{definition}

\begin{definition}[]
Let \(i_1,i_2,\dots,i_r\) be distinct integers in \(\lb 1,2,\dots,n\rb\). If
\(\alpha\in S_n\) fixes the other integers and if
\begin{equation*}
\alpha(i_1)=i_2,\alpha(i_2)=i_3,\dots,\alpha(i_{r-1})=i_r,\alpha(i_r)=i_1
\end{equation*}
then \(\alpha\) is called an textbf\{r-cycle\}. \(\alpha\) is a cycle of
\textbf{length} \(r\) and denoted by
\begin{equation*}
\alpha=(i_1\; i_2\;\dots\; i_r)
\end{equation*}
\end{definition}

2-cycles are also called the \textbf{transpositions}.

\begin{definition}[]
Two permutations \(\alpha,\beta\in S_n\) are \textbf{disjoint} if every \(i\)
moved by one is fixed by the other.
\end{definition}

\begin{lemma}[]
Disjoint permutations \(\alpha,\beta\in S_n\) commute
\end{lemma}

\begin{proposition}[]
Every permutation \(\alpha\in S_n\) is either a cycle or a product of disjoint cycles.
\end{proposition}

\begin{proof}
Induction on the number \(k\) of points moved by \(\alpha\)
\end{proof}

\begin{definition}[]
A \textbf{complete factorization} of a permutation \(\alpha\) is a
factorization of \(\alpha\) into disjoint cycles that contains exactly one
1-cycle \((i)\) for every \(i\) fixed by \(\alpha\)
\end{definition}

\begin{theorem}[]
Let \(\alpha\in S_n\) and let \(\alpha=\beta_1\dots\beta_t\) be a complete
factorization into disjoint cycles. This factorization is unique except for
the order in which the cycles occur
\end{theorem}

\begin{proof}
for all \(i\), if \(\beta_t(i)\neq i\), then \(\beta_t^k(i)\neq\beta_t^{k-1}(i)\)
for any \(k\ge 1\)
\end{proof}

\begin{lemma}[]
If \(\gamma,\alpha\in S_n\), then \(\alpha\gamma\alpha^{-1}\) has the same cycle
structure as \(\gamma\). In more detail, if the complete factorization of
\(\gamma\) is
\begin{equation*}
\gamma=\beta_1\beta_2\dots(i_1\; i_2\;\dots)\dots\beta_t
\end{equation*}
then \(\alpha\gamma\alpha^{-1}\) is permutation that is obtained from \(\gamma\)
by applying \(\alpha\) to the symbols in the cycles of \(\gamma\)
\end{lemma}

\begin{examplle}[]
\label{example2.8}
Suppose
\begin{gather*}
\beta=(1\;2\;3)(4)(5)\\
\gamma=(5\;2\;4)(1)(3)
\end{gather*}
then we can easily find the \(\alpha\)
\begin{equation*}
\alpha=
\begin{pmatrix}
1&2&3&4&5\\
5&2&4&1&3
\end{pmatrix}
\end{equation*}
and so \(\alpha=(1\;5\;3\;4)\). Now \(\alpha\in S_5\) and \(\gamma=(\alpha 1\;\alpha
   2\;\alpha 3)\)
\end{examplle}
\begin{theorem}[]
Permutations \(\gamma\) and \(\sigma\) in \(S_n\) has the same cycle structure if
and only if there exists \(\alpha\in S_n\) with \(\sigma=\alpha\gamma\alpha^{-1}\)
\end{theorem}


\begin{proposition}[]
If \(n\ge 2\) then every \(\alpha\in S_n\) is a product of tranpositions
\end{proposition}
\begin{proof}
\((1\;2\;\dots\; r)=(1\; r)(1\; r-1)\dots(1\; 2)\)
\end{proof}

\begin{examplle}[]
The \textbf{15-puzzle} has a \textbf{starting position} that is a \(4\times 4\) array of the
numbers between 1 and 15 and a symbol \#, which we interpret as "blank". For
example, consider the following starting position

\begin{center}
\begin{tabular}{|c|c|c|c|}
\hline
3 & 15 & 4 & 8\\
\hline
10 & 11 & 1 & 9\\
\hline
2 & 5 & 13 & 12\\
\hline
6 & 7 & 14 & \#\\
\hline
\end{tabular}
\end{center}

A \textbf{simple move} interchanges the blank with a symbol adjacent to it. We win the
game if after a sequence of simple moves, the starting position is
transformed into the standard array \(1,2,\dots,15,\#\). 

To analyze this game, note that the given array is really a permutation
\(\alpha\in S_{16}\). For example, the given starting position is
\[
\begin{pmatrix}
 1 & 2 & 3 & 4 & 5 & 6 & 7 & 8 & 9 & 10 & 11 & 12 & 13 & 14 & 15 & 16 \\
 3 & 15 & 4 & 8 & 10 & 11 & 1 & 9 & 2 & 5 & 13 & 12 & 6 & 7 & 14 & 16 \\
\end{pmatrix}
\]

To win the game, we need special transpositions \(\tau_1,\dots,\tau_m\) sot
that
\begin{equation*}
\tau_m\dots\tau_1\alpha=(1)
\end{equation*}
\end{examplle}

\begin{definition}[]
A permutation \(\alpha\in S_n\) is \textbf{even} if it can be factored into a
product of an even number of transpositions. Otherwise \textbf{odd}
\end{definition}

\begin{definition}[]
If \(\alpha\in S_n\) and \(\alpha=\beta_1\dots\beta_t\) is a complete
factorization, then \textbf{signum} \(\alpha\) is defined by
\begin{equation*}
\sgn(\alpha)=(-1)^{n-t}
\end{equation*}
\end{definition}

\begin{theorem}[]
For all \(\alpha,\beta\in S_n\)
\begin{equation*}
\sgn(\alpha\beta)=\sgn(\alpha)\sgn(\beta)
\end{equation*}
\end{theorem}

\begin{theorem}[]
\begin{enumerate}
\item Let \(\alpha\in S_n\); if \(\sgn(\alpha)=1\) then \(\alpha\) is even. otherwise
odd
\item A permutation \(\alpha\) is odd if and only if it's a product of an odd
number of transpositions
\end{enumerate}
\end{theorem}

\begin{corollary}[]
Let \(\alpha,\beta\in S_n\). If \(\alpha\) and \(\beta\) have the same parity, then
\(\alpha\beta\) is even while if \(\alpha\) and \(\beta\) have distinct parity,
\(\alpha\beta\) is odd
\end{corollary}

\begin{examplle}[]
An analysis of the 15-puzzle shows that if \(\alpha\in S_{16}\) is the starting
position, then the game can be won if and only if \(\alpha\) is an even permutation
that fixes 16.

The blank 16 starts in position 16. Each simple move takes 16 up, down, left
or right. Thus the total number \(m\) of moves is \(u+d+l+r\). If 16 is to return
home, each one of these must be undone. Thus the total number of moves is
even: \(m=2u+2r\). Hence \(\alpha=\tau_1\dots\tau_m\) and so \(\alpha\) is an even
permutation. In example
\begin{equation*}
\alpha=(1\;3\;4\;8\;9\;2\;15\;14\;7)(5\;10)(6\;11\;13)(12)(16)
\end{equation*}
Now \(\sgn(\alpha)=(-1)^{16-5}=-1\).
\end{examplle}
\subsection{Groups}
\label{sec:org78ea59c}
\begin{definition}[]
A \textbf{binary operation} on a set \(G\) is a function
\begin{equation*}
*:G\times G\to G
\end{equation*}
\end{definition}

\begin{definition}[]
A \textbf{group} is a set \(G\) equipped with a binary operation * s.t.
\begin{enumerate}
\item the \textbf{associative law} holds
\item \textbf{identity}
\item every \(x\in G\) has an \textbf{inverse}, there is a \(x'\in G\)  with 
\(x*x'=e=x'*x\)
\end{enumerate}
\end{definition}

\begin{definition}[]
A group \(G\) is called \textbf{abelian} if it satisfies the
\textbf{commutative law}
\end{definition}

\begin{lemma}[]
Let \(G\) be a group
\begin{enumerate}
\item The \textbf{cancellation laws} holds: if either \(x*a=x*b\) or \(a*x=b*x\), then
\(a=b\)
\item \(e\) is unique
\item Each \(x\in G\) has a unique inverse
\item \((x^{-1})^{-1}=x\)
\end{enumerate}
\end{lemma}

\begin{definition}[]
An expression \(a_1a_2\dots a_n\) \textbf{needs no parentheses} if all the ultimate
products it yields are equal
\end{definition}

\begin{theorem}[Generalized Associativity]
If \(G\) is a group and \(a_1,a_2,\dots,a_n\in G\) then the expression
\(a_1a_2\dots a_n\) needs no parentheses
\end{theorem}

\begin{definition}[]
Let \(G\) be a group and let \(a\in G\). If \(a^k=1\) for some \(k>1\) then the
smallest such exponent \(k\ge 1\) is called the \textbf{order} or \(a\); if no such
power exists, then one says that \(a\) has \textbf{infinite order}
\end{definition}

\begin{proposition}[]
If \(G\) is a finite group, then every \(x\in G\) has finite order
\end{proposition}

\begin{definition}[]
A \textbf{motion} is a distance preserving bijection \(\varphi:\R^2\to\R^2\). If
\(\pi\) is a polygon in the plane, then its \textbf{symmetry group} \(\Sigma(\pi)\)
consists of all the motions \(\varphi\) for which \(\varphi(\pi)=\pi\). The
elements of \(\Sigma(\pi)\) are called the \textbf{symmetries} of \(\pi\)
\end{definition}

Let \(\pi_4\) be a square. Then the group \(\Sigma(\pi_4)\) is called the
\textbf{dihedral group} with 8 elements, denoted by \(D_8\)

\begin{definition}[]
If \(\pi_n\) is a regular polygon with \(n\) vertices \(v_1,\dots,v_n\) and center
\(O\), then the symmetry group \(\Sigma(\pi_n)\) is called the \tf\{dihedral
group\} with \(2n\) elements, and it's denoted by \(D_{2n}\)
\end{definition}

\begin{exercise}
\label{ex2.26}
If \(G\) is a group in which \(x^2=1\) for every \(x\in G\), prove that \(G\)
must be abelian
\end{exercise}

\begin{exercise}
\label{ex2.27}
If \(G\) is a group with an even number of elements, prove that the number of
elements in \(G\) of order 2 is odd. In particular, \(G\) must contain an element of
order 2.
\end{exercise}

\begin{proof}
1 is an element of order 1.
\end{proof}
\subsection{Lagrange's Theorem}
\label{sec:orgbb2f9ed}
\begin{theorem}[]

\end{theorem}

\begin{definition}[]
A subset \(H\) of a group \(G\) is a \textbf{subgroup} if
\begin{enumerate}
\item \(1\in H\)
\item if \(x,y\in H\), then \(xy\in H\)
\item if \(x\in H\), then \(x^{-1}\in H\)
\end{enumerate}
\end{definition}

If \(H\) is a subgroup of \(G\), we write \(H\le G\). If \(H\) is a proper subgroup,
then we write \(H<G\)

The four permutations
\begin{equation*}
\bV=\{(1),(1 2)(3 4),(1 3)(2 4),(1 4)(2 3)\}
\end{equation*}
form a group because \(\bV\le S_4\)

\begin{proposition}[]
A subset \(H\) of a group \(G\) is a subgroup if and only if \(H\) is nonempty and
whenever \(x,y\in H\), \(xy^{-1}\in H\)
\end{proposition}

\begin{proposition}[]
A nonempty subset \(H\) of a finite group \(G\) is a subgroup if and only if \(H\)
is closed; that is, if \(a,b\in H\), then \(ab\in H\)
\end{proposition}

\index{alternating group}
\begin{examplle}[]
The subset \(A_n\) of \(S_n\), consisting of all the even permutations, is a
subgroup called the \textbf{alternating group} on \(n\) letters
\end{examplle}

\index{cyclic group}
\begin{definition}[]
If \(G\) is a group and \(a\in G\)
\begin{equation*}
\langle a\rangle=\{a^n:n\in\Z\}=\{\text{all powers of } a\}
\end{equation*}
\(\la a\ra\) is called the \textbf{cyclic subgroup} of \(G\) \textbf{generated} by \(a\). A
group \(G\) is called \textbf{cyclic} if there exists \(a\in G\) s.t. \(G=\la a\ra\),
in which case \(a\) is called the \textbf{generator}
\end{definition}

\begin{definition}[]
The \textbf{integers mod \(m\)}, denoted by \(\I_m\) is the family of all congruence
classes mod \(m\)
\end{definition}


\begin{proposition}[]
Let \(m\ge 2\) be a fixed integer
\begin{enumerate}
\item If \(a\in \Z\), then \([a]=[r]\) for some \(r\) with \(0\le r<m\)
\item If \(0\le r'<r<m\), then \([r']\neq[r]\)
\item \(\I_m\) has exactly \(m\) elements
\end{enumerate}
\end{proposition}

\begin{theorem}[]
\begin{enumerate}
\item If \(G=\la a\ra\) is a cyclic group of order \(n\), then \(a^k\) is a generator
of \(G\) if and only if \((k,n)=1\)
\item If \(G\) is a cyclic group of order \(n\) and \(\gen(G)=\{\text{all generators
      of } G\}\), then
\begin{equation*}
\abs{\gen(G)}=\phi(n)
\end{equation*}
where \(\phi\) is the Euler \(\phi\)-function
\end{enumerate}
\end{theorem}
\begin{proof}
\begin{enumerate}
\item there is \(t\in\N\) s.t. \(a^{kt}=a\) hence \(a^{kt-1}=1\) and \(n\mid kt-1\)
\end{enumerate}
\end{proof}

\begin{proposition}[]
Let \(G\) be a finite group and let \(a\in G\). Then the order of \(a\) is
\(\abs{\la a\ra}\).
\end{proposition}

\begin{definition}[]
If \(G\) is a finite group, then the number of elements in \(G\), denoted by
\(\abs{G}\) is called the \textbf{order} of \(G\)
\end{definition}


\begin{proposition}[]
The intersection \(\bigcap_{i\in I}H_i\) of any family of subgroups of a group
\(G\) is again a subgroup of \(G\)
\end{proposition}


\begin{corollary}[]
If \(X\) is a subset of a group \(G\), then there is a subgroup \(\la X\ra\) of \(G\)
containing \(X\) tHhat is \textbf{smallest} in the sense that \(\la X\ra\le H\) for
every subgroup \(H\) 
of \(G\) that contains \(X\)
\end{corollary}


\begin{definition}[]
If \(X\) is a subset of a group \(G\), then \(\la X\ra\) is called the \textbf{subgroup}
\textbf{generated by} \(X\)
\end{definition}

A \textbf{word} on \(X\) is an element \(g\in G\) of the form \(g=x_1^{e_1}\dots
   x_n^{e_n}\) where \(x_i\in X\) and \(e_i=\pm 1\) for all \(i\)

\begin{proposition}[]
If \(X\) is a nonempty subset of a group \(G\), then \(\la X\ra\) is the set of all
words on \(X\)
\end{proposition}

\index{coset}
\begin{definition}[]
If \(H\le G\) and \(a\in G\), then the \textbf{coset} \(aH\) is the subset \(aH\) of \(G\),
where
\begin{equation*}
aH=\{ah:h\in H\}
\end{equation*}
\end{definition}
\(aH\) \textbf{left coset}, \(Ha\) \textbf{right coset}

\begin{lemma}[]
\(H\le G,a,b\in G\)
\begin{enumerate}
\item \(aH=bH\) if and only if \(b^{-1}a\in H\)
\item if \(aH\cap bH\neq\emptyset\), then \(aH=bH\)
\item \(\abs{aH}=\abs{H}\) for all \(a\in G\)
\end{enumerate}
\end{lemma}
\begin{proof}
define a relation \(a\equiv b\) if \(b^{-1}a\in H\)
\end{proof}


\begin{theorem}[Lagrange's Theorem]
If \(H\) is a subgroup of a finite group \(G\), then \(\abs{H}\) is a divisor of \(\abs{G}\)
\end{theorem}

\begin{proof}
Let \(\{a_1H,a_2H,\dots,a_tH\}\) be the family of all the distinct cosets of
\(H\) in \(G\). Then
\begin{equation*}
G=a_1H\cup a_2H\cup\dots\cup a_tH
\end{equation*}
hence
\begin{equation*}
\abs{G}=\abs{a_1H}+\dots+\abs{a_tH}
\end{equation*}
But \(\abs{a_iH}=\abs{H}\) for all \(i\). Hence \(\abs{G}=t\abs{H}\)
\end{proof}

\index{index}
\begin{definition}[]
The \textbf{index} of a subgroup \(H\) in \(G\) denoted by \([G:H]\), is the number of
left cosets of \(H\) in \(G\)
\end{definition}

Note that \(\abs{G}=[G:H]\abs{H}\)

\begin{corollary}[]
If \(G\) is a finite group and \(a\in G\), then the order of \(a\) is a divisor of
\(\abs{G}\) 
\end{corollary}

\begin{corollary}[]
If \(G\) is a finite group, then \(a^{\abs{G}}=1\) for all \(a\in G\)
\end{corollary}

\begin{corollary}[]
If \(p\) is a prime, then every group \(G\) of order \(p\) is cyclic
\end{corollary}

\begin{proposition}[]
The set \(U(\I_m)\), defined by
\begin{equation*}
 U(\I_m)=\{[r]\in\I_m:(r,m)=1\}
\end{equation*}
is a multiplicative group of order \(\phi(m)\). If \(p\) is a prime, then
\(U(\I_p)=\I_p^{\times}\), the nonzero elements of \(\I_p\).
\end{proposition}

\begin{proof}
\((r,m)=1=(r',m)\) implies \((rr',m)=1\). Hence \(U(\I_m)\) is closed under
multiplication. If \((x,m)=1\), then \(rs+sm=1\). There fore \((r,m)=1\). Each of
them have inverse.
\end{proof}

\index{Fermat's Theorem}
\begin{corollary}[Fermat]
\label{Fermat}
If \(p\) is a prime and \(a\in\Z\), then
\begin{equation*}
a^p\equiv a\mod p
\end{equation*}
\end{corollary}

\begin{proof}
suffices to show \([a^p]=[a]\) in \(\I_p\). If \([a]=[0]\), then \([a^p]=[a]^p=[0]\).
Else, since \(\abs{\I_p^\times}=p-1\), \([a]^{p-1}=[1]\)
\end{proof}


\begin{theorem}[Euler]
If \((r,m)=1\), then
\begin{equation*}
r^{\phi(m)}\equiv 1\mod m
\end{equation*}
\end{theorem}
\begin{proof}
Since \(\abs{U(\I_m)}=\phi(m)\). Lagrange's theorem gives
\([r]^{\phi(m)}=[1]\) for all \([r]\in U(\I_m)\).

In fact we construct a group to prove this.
\end{proof}

\begin{theorem}[Wilson's Theorem]
An integer \(p\) is a prime if and only if
\begin{equation*}
(p-1)!\equiv -1\mod p
\end{equation*}
\end{theorem}

\begin{proof}
Assume that \(p\) is a prime. If \(a_1,\dots,a_n\) is a list of all the elements
of finite abelian group, then product \(a_1a_2\dots a_n\) is the same as the
product of all elements \(a\) with \(a^2=1\). Since \(p\) is prime, \(\I_p^\times\)
has only one element of order 2, namely \([-1]\). It follows that the product
of all the elements in \(\I_p^\times\) namely \([(p-1)!]\) is equal to \([-1]\).

Conversly assume that \(m\) is composite: there are integers \(a\) and \(b\) with
\(m=ab\) and \(1<a\le b<m\). If \(a<b\) then \(m=ab\) is a divisor of \((m-1)!\). If
\(a=b\), then \(m=a^2\). if \(a=2\), then \((a^2-1)!\equiv 2\mod 4\). If \(2<a\), then
\(2a<a^2\) and so \(a\) and \(2a\) are factors of \((a^2-1)!\)
\end{proof}

\begin{exercise}
\label{ex2.36}
Let \(G\) be a group of order 4. Prove that either \(G\) is cyclic or \(x^2=1\)
for every \(x\in G\). Conclude, using Exercise \ref{ex2.26} that \(G\) must be abelian.
\end{exercise}

\begin{proof}
   
\end{proof}
\subsection{Homomorphisms}
\label{sec:org1bbe687}
\begin{definition}[]
If \((G,*)\) and \((H,\circ)\) are groups, then a function \(f:G\to H\) is a
\textbf{homomorphism} if
\begin{equation*}
f(x*y)=f(x)\circ f(y)
\end{equation*}
for all \(x,y\in G\). If \(f\) is also a bijection, then \(f\) is called an
\textbf{isomorphism}. \(G\) and \(H\) are called \textbf{isomorphic}, denoted by \(G\cong H\)
\end{definition}

\begin{lemma}[]
Let \(f:G\to H\) be a homomorphism
\begin{enumerate}
\item \(f(1)=1\)
\item \(f(x^{-1})=f(x)^{-1}\)
\item \(f(x^n)=f(x)^n\) for all \(n\in\Z\)
\end{enumerate}
\end{lemma}



\begin{definition}[]
If \(f:G\to H\) is a homomorphism, define
\begin{equation*}
\ker f=\{x\in G:f(x)=1
\end{equation*}
and
\begin{equation*}
\im f=\{h\in H:h=f(x)\text{ for some } x\in G\
\end{equation*}
\end{definition}

\begin{proposition}[]
Let \(f:G\to H\) be a homomorphism
\begin{enumerate}
\item \(\ker f\) is a subgroup of \(G\) and \(\im f\) is a subgroup of \(H\)
\item if \(x\in\ker f\) and if \(a\in G\), then \(axa^{-1}\in\ker f\)
\item \(f\) is an injection if and only if \(\ker f=\{1\}\)
\end{enumerate}
\end{proposition}

\begin{proof}
\begin{enumerate}
\setcounter{enumi}{2}
\item \(f(a)=f(b)\Leftrightarrow f(ab^{-1})=1\)
\end{enumerate}
\end{proof}

\begin{definition}[]
A subgroup \(K\) of a group \(G\) is called a \textbf{normal subgroup} if \(k\in K\)
and \(g\in G\) imply \(gkg^{-1}\in K\), denoted by \(K\triangleleft G\)
\end{definition}

\begin{definition}[]
If \(G\) is a group and \(a\in G\), then a \textbf{conjugate} of \(a\) is any element
in \(G\) of the form
\begin{equation*}
gag^{-1}
\end{equation*}
where \(g\in G\)
\end{definition}

\begin{definition}[]
If \(G\) is a group and \(g\in G\), define \textbf{conjugation} \(\gamma_g:G\to G\) by
\begin{equation*}
\gamma_g(a)=gag^{-1}
\end{equation*}
for all \(a\in G\)
\end{definition}

\begin{proposition}[]
\begin{enumerate}
\item If \(G\) is a group and \(g\in G\), then conjugation \(\gamma_g:G\to G\) is an
isomorphism
\item Conjugate elements have the same order
\end{enumerate}
\end{proposition}

\begin{proof}
\begin{enumerate}
\item bijection: \(\gamma_g\circ\gamma_{g^{-1}}=1=\gamma_{g^{-1}}\circ\gamma_g\).
\end{enumerate}
\end{proof}

\begin{examplle}[]
Define the \textbf{center} of a group \(G\), denoted by \(Z(G)\), to be
\begin{equation*}
Z(G)=\{z\in G:zg=gz\text{ for all }g\in G\}
\end{equation*}
\end{examplle}

\begin{examplle}[]
If \(G\) is a group, then an \textbf{automorphism} of \(G\) is an isomorphism \(f:G\to G\).
For example, every conjugation \(\gamma_g\) is an automorphism of \(G\) (it is
called an \textbf{inner automorphism}), for its inverse is conjugation by \(g^{-1}\).
The set \(\aut(G)\) of all the automorphism of \(G\) is itself a group.
\begin{equation*}
\inn(G)=\{\gamma_g:g\in G\}
\end{equation*}
is a subgroup of \(\aut(G)\)
\end{examplle}
\begin{proposition}[]
\begin{enumerate}
\item If \(H\) is a subgroup of index 2 in a group \(G\), then \(g^2\in H\) for every
\(g\in G\)
\item If \(H\) is a subgroup of index 2 in a group \(G\), then \(H\) is a normal
subgroup of \(G\)
\end{enumerate}
\end{proposition}


\begin{definition}[]
The group of \textbf{quaternions} is the group \(\bQ\) of order 8 consisting of the
following matrices in \(GL(2, \C)\)
\begin{equation*}
\bQ=\{I,A,A^2,A^3,B,BA,BA^2,BA^3\}
\end{equation*}
where \(I\) is the identity matrix
\begin{equation*}
A=
\begin{pmatrix}
0&1\\
-1&0
\end{pmatrix}, \text{ and }
B=\begin{pmatrix}
0&i\\
i&0
  \end{pmatrix}
\end{equation*}
\end{definition}

\begin{examplle}[]
\(\bQ\) is normal. By Lagrange's theorem the only possible orders of subgroups
are 1,2,4 or 8. The only subgroup of order 2 is \(\la -I\ra\) since \(-I\) is the
only element of order 2
\end{examplle}
\begin{proposition}[]
The alternating group \(A_4\) is a group of order 12 having no subgroup of
order 6
\end{proposition}

\begin{exercise}
Show that if there is a bijection \(f:X\to Y\), then there is an isomorphism
\(\varphi:S_X\to S_Y\)
\end{exercise}
\begin{proof}
If \(\alpha\in S_X\), define \(\varphi(\alpha)=f\circ\alpha\circ f^{-1}\). Since
\(f,\alpha,f^{-1}\) are bijections, \(\varphi(\alpha)\) is an bijection. \(\varphi\) is a
homomorphism. \(\forall \beta\in S_Y\), we have \(\alpha=f^{-1}\circ\beta\circ f\)
\end{proof}
\subsection{Quotient group}
\label{sec:org4d18282}
\(\cals(G)\) is the set of all nonempty subsets of a group \(G\). If
\(X,Y\in\cals(G)\), define
\begin{equation*}
XY=\{xy:x\in X\text{ and } y\in Y\}
\end{equation*}

\begin{lemma}[]
\(K\le G\) is normal if and only if
\begin{equation*}
gK=Kg
\end{equation*}
\end{lemma}

A natural question is that whether \(HK\) is a subgroup when \(H\) and \(K\) are
subgroups. The answer is no. Let \(G=S_3,H=\la(1\;2)\ra,K=\la(1\;3)\ra\)


\begin{proposition}[]
\begin{enumerate}
\item If \(H\) and \(K\) are subgroups of a group \(G\), and if one of them is normal,
then \(HK\le G\) and \(HK=KH\)
\item If \(H,K\tril G\), then \(HK\tril G\)
\end{enumerate}
\end{proposition}

\begin{theorem}[]
Let \(G/K\) denote the family of all the left cosets of a subgroup \(K\) of \(G\).
If \(K\tril G\), then
\begin{equation*}
aKbK=abK
\end{equation*}
for all \(a,b\in G\) and \(G/K\) is a group under this operation
\end{theorem}

\begin{proof}
\(aKbK=abKK=abK\)
\end{proof}

\(G/K\) is called the \textbf{quotient group} \(G\) mod \(K\)

\begin{corollary}[]
Every \(K\tril G\) is the kernel of some homomorphism
\end{corollary}

\begin{proof}
Define the \textbf{natural map} \(\pi:G\to G/K\), \(a\mapsto aK\)
\end{proof}

\begin{theorem}[First Isomorphism Theorem]
If \(f:G\to H\) is a homomorphism, then
\begin{equation*}
\ker f\tril G\quad\text{ and }\quad G/\ker f\cong\im f
\end{equation*}
If \(\ker f=K\) and \(\varphi:G/K\to\im f\le H,aK\mapsto f(a)\), then \(\varphi\)
is an isomorphism
\end{theorem}

\begin{remark}
\begin{center}
\begin{tikzcd}
G \arrow[rr,"f"] \arrow[dr,"\pi"]& &
H\\ 
&G/K \arrow[ur,"\varphi"]&
\end{tikzcd}
\end{center}
\end{remark}

\begin{examplle}[]
What's the quotient group \(\R/\Z\)? Define \(f:\R\to S^1\) where \(S^1\) is the
circle group by
\begin{equation*}
f:x\mapsto e^{2\pi ix}
\end{equation*}
\(\R/\Z\cong S^1\)
\end{examplle}

\begin{proposition}[Product Formula]
If \(H\) and \(K\) are subgroups of a finite group \(G\), then
\begin{equation*}
\abs{HK}\abs{H\cap K}=\abs{H}\abs{K}
\end{equation*}
\end{proposition}

\begin{proof}
Define a function \(f:H\times K\to HK,(h,k)\mapsto hk\). Show that
\(\abs{f^{-1}(x)}=\abs{H\cap K}\). 

Claim that if \(x=hk\), then
\begin{equation*}
f^{-1}(x)=\{(hd,d^{-1}k):d\in H\cap K\}
\end{equation*}
\end{proof}

\begin{theorem}[Second Isomorphism Theorem]
If \(H\tril G, K\le G\), then \(HK\le G,H\cap K\tril G\) and
\begin{equation*}
K/(H\cap K)\cong HK/H
\end{equation*}
\end{theorem}

\begin{proof}
\(hkH=kk^{-1}hkH=kh'H=kH\)
\end{proof}

\begin{theorem}[Third Isomorphism Theorem]
If \(H,K\tril G\) with \(K\le H\), then \(H/K\tril G/K\) and
\begin{equation*}
(G/K)/(H/K)\cong G/H
\end{equation*}
\end{theorem}

\begin{theorem}[Correspondence Theorem]
If \(K\tril G, \pi:G\to G/K\) is the natural map, then
\begin{equation*}
S\mapsto \pi(S)=S/K
\end{equation*}
is a bijection between \(Sub(G;K)\), the family of all those subgroups \(S\) of
\(G\) that contain \(K\), and \(Sub(G/K)\), the family of all the subgroups of
\(G/K\). If we denote \(S/K\) by \(S^*\), then
\begin{enumerate}
\item \(T\le S\le G\) if and only if \(T^*\le S^*\), in which case \([S:T]=[S^*:T^*]\)
\item \(T\tril S\) if and only if \(T^*\tril S^*\), in which case \(S/T\cong S^*/T^*\)
\end{enumerate}
\end{theorem}

\begin{center}
\begin{tikzcd}
G \arrow[d,dash] \arrow[rd]&\\
S \arrow[d,dash] \arrow[rd] & G/K \arrow[d,dash]\\
T \arrow[d,dash] \arrow[rd] & S/K=S^* \arrow[d,dash]\\
K  \arrow[rd] & T/K=T^* \arrow[d,dash]\\
& \{1\}
\end{tikzcd}
\end{center}

\begin{proof}
Use \(\pi^{-1}\pi=1\) and \(\pi\pi^{-1}=1\) to prove injectivity and surjectivity
respectively. 

For \([S:T]=[S^*:T^*]\), show there is a bijection between the family of all
cosets of the form \(sT\) and the family of all the cosets of the form
\(s^*T^*\).

injective:
\begin{align*}
\pi(m)T^*=\pi(n)T^*&\Leftrightarrow \pi(m)\pi(n)^{-1}\in T^*\\
&\Leftrightarrow mn^{-1}K\in T/K\\
&\Rightarrow mn^{-1}t^{-1}\in K\\
&\Rightarrow mn^{-1}=tk\in T\\
&\Leftrightarrow mT=nT\\
\end{align*}

surjective:


If \(G\) is finite, then
\begin{align*}
[S^*:T^*]&=\abs{S^*}/\abs{T^*}\\
&=\abs{S/K}/\abs{T/K}\\
&=(\abs{S}/\abs{K})/(\abs{T}/\abs{K})\\
&=\abs{S}/\abs{T}\\
&=[S:T]
\end{align*}

If \(T\tril S\), by third isomorphism theorem, \(T/S\cong (T/K)/(S/K)=T^*/S^*\)

If \(T^*\tril S^*\), 
\begin{equation*}
\pi(sts^{-1})\in \pi(s)T^*\pi(s)^{-1}=T^*
\end{equation*}
so that \(sts^{-1}\in \pi^{-1}(T^*)=T\)
\end{proof}


\begin{proposition}[]
\label{prop2.78}
If \(G\) is a finite abelian group and \(d\) is a divisor of \(\abs{G}\), then \(G\)
contains a subgroup of order \(d\)
\end{proposition}

\begin{proof}
Abelian group's subgroup is normal and hence we can build quotient groups.
p90 for proof. Use the correspondence theorem
\end{proof}

\begin{definition}[]
If \(H\) and \(K\) are grops, then their \textbf{direct product}, denoted by 
\(H\times K\) 
, is the set of all ordered pairs \((h,k)\) with the operation
\begin{equation*}
(h,k)(h',k')=(hh',kk')
\end{equation*}
\end{definition}

\begin{proposition}[]
Let \(G\) and \(G'\) be groups and \(K\tril G, K'\tril G'\). Then \(K\times K'\tril
   G\times G'\) and
\begin{equation*}
(G\times G')/(K\times K')\cong (G/K)\times(G'/K')
\end{equation*}
\end{proposition}

\begin{proof}
   
\end{proof}

\begin{proposition}[]
If \(G\) is a group containing normal subgroups \(H\) and \(K\) and \(H\cap K=\{1\}\)
and \(HK=G\), then \(G\cong H\times K\)
\end{proposition}

\begin{proof}
Note \(\abs{HK}\abs{H\cap K}=\abs{H}\abs{K}\). Consider \(\varphi:G\to H\times
   K\). Show it's homo and bijective.
\end{proof}

\begin{theorem}[]
\label{thm2.81}
If \(m,n\) are relatively prime, then
\begin{equation*}
\I_{mn}\cong\I_m\times\I_n
\end{equation*}
\end{theorem}

\begin{proof}
\begin{align*}
f:&\Z\to\I_m\times\I_n\\
&a\mapsto([a]_m,[a]_n)
\end{align*}
is a homo.
\(\Z/\la mn\ra\cong\I_m\times\I_n\)
\end{proof}

\begin{proposition}[]
Let \(G\) be a group, and \(a,b\in G\) be commuting elements of orders \(m,n\). If
\((m,n)=1\), then \(ab\) has order \(mn\)
\end{proposition}

\begin{corollary}[]
If \((m,n)=1\), then \(\phi(mn)=\phi(m)\phi(n)\)
\end{corollary}

\begin{proof}
Theorem \ref{thm2.81} shows that \(f:\I_{mn}\cong\I_m\times\I_n\). The result will
follow if we prove that \(f(U(\I_{mn}))=U(\I_m)\times U(\I_n)\), for then
\begin{align*}
\phi(mn)&=\abs{U(\I_{mn})}=\abs{f(U(\I_{mn}))}\\
&=\abs{U(\I_m)\times U(\I_n)}=\abs{U(\I_m)}\cdot\abs{U(\I_n)}
\end{align*}
If \([a]\in U(\I_{mn})\), then \([a][b]=[1]\) for some \([b]\in\I_{mn}\) and
\begin{align*}
f([ab])&=([ab]_m,[ab]_n)=([a]_m[b]_m,[a]_n[b]_n)\\
&=([a]_m,[a]_n)([b]_m,[b]_n)=([1]_m,[1]_n)
\end{align*}
Hence \(f([a])=([a]_m,[a]_n)\in U(\I_m)\times U(\I_n)\)

For the reverse inclusion, if \(f([c])=([c]_m,[c]_n)\in U(\I_m)\times
   U(\I_n)\), then we must show that \([c]\in U(\I_{mn})\). There is \([d]_m\in\I_m\)
with \([c]_m[d]_m=[1]_m\), and there is \([e]_n\I_n\) with \([c]_n[e]_n=[1]_n\).
Since \(f\) is surjective, there is \(b\in\Z\) with
\(([b]_m,[b]_n)=([d]_m,[e]_n)\), so that
\begin{equation*}
f([1])=([1]_m,[1]_n)=([c]_m[b]_m,[c]_n[b]_n)=f([c][b])
\end{equation*}
Since \(f\) is an injection, \([1]=[c][b]\) and \([c]\in U(\I_{mn})\)
\end{proof}

\begin{corollary}[]
\begin{enumerate}
\item If \(p\) is a prime, then \(\phi(p^e)=p^e-p^{e-1}=p^e(1-\frac{1}{p})\)
\item If \(n=p_1^{e_1}\dots p_t^{e_t}\), then
\begin{equation*}
\phi(n)=n(1-\frac{1}{p_1})\dots(1-\frac{1}{p_t})
\end{equation*}
\end{enumerate}
\end{corollary}

\begin{lemma}[]
A cyclic group of order \(n\) has a unique subgroup of order \(d\), for each
divisor \(d\) of \(n\), and this subgroup is cyclic.
\end{lemma}

Define an equivalence relation on a group \(G\) by \(x\equiv y\) if \(\la x\ra=\la
   y\ra\). Denote the equivalence class containing \(x\) by \(\gen(C)\), where \(C=\la
   x\ra\). Equivalence classes form a partition and we get
\begin{equation*}
G=\displaystyle\prod_{C}\gen(C)
\end{equation*}
where \(C\) ranges over all cyclic subgroups of \(G\). Note \(\abs{\gen(C)}=\phi(n)\)

\begin{theorem}[]
\label{thm2.86}
A group \(G\) of order \(n\) is cyclic if and only if for each divisor \(d\) of
\(n\), there is at most one cyclic subgroup of order \(d\)
\end{theorem}

\begin{theorem}[]
If \(G\) is an abelian group of order \(n\) having at most one cyclic subgroup of
order \(p\) for each prime divisor \(p\) of \(n\), then \(G\) is cyclic
\end{theorem}

Exercise:
\begin{itemize}
\item 2.71 Suppose \(H\le G, \abs{H}=\abs{K}\). Since \(\abs{H}=[H:K]\abs{K}\),
\([H:K]=1\). Hence \(H=K\)
\item 2.67 1. \(\inn(S_3)\cong S_3/Z(S_3)\cong S_3\) and \(\abs{\aut(S_3)}\le 6\).
Hence \(\aut(S_3)=\inn(S_3)\)
\end{itemize}


\begin{exercise}
\label{ex2.69}
Prove that if \(G\) is a group for which \(G/Z(G)\) is cyclic, then \(G\) is abelian
\end{exercise}
\begin{proof}
Suppose \(G/Z(G)=\la a\ra\), let \(g=a^kz^{-1},g'=a^{k'}z'^{-1}\), then 
\(gg'=a^kz^{-1}z^{k'}z'^{-1}=a^{k+k'}z'^{-1}z^{-1}=g'g\). Hence \(G\) is abelian.
\end{proof}
\subsection{Group Actions}
\label{sec:org05a8458}
\begin{theorem}[Cayley]
Every group \(G\) is isomorphic to a subgroup of the symmetric group \(S_G\). In
particular, if \(\abs{G}=n\), then \(G\) is isomorphic to a subgroup of \(S_n\)
\end{theorem}

\begin{proof}
For each \(a\in G\), define \(\tau_a(x)=ax\) for every \(x\in G\). \(\tau_a\) is a
bijection for its inverse is \(\tau_{a^{-1}}\)
\begin{equation*}
\tau_a\tau_{a^{-1}}=\tau_1=\tau_{a^{-1}}\tau_a
\end{equation*}
\end{proof}

\begin{theorem}[Representation on Cosets]
Let \(G\) be a group and \(H\le G\) having finite index \(n\). Then there exists a
homomorphism \(\varphi:G\to S_n\) with \(\ker\varphi\le H\)
\end{theorem}

\begin{proof}
We still denote the family of all the cosets of \(H\) in \(G\) by \(G/H\)

For each \(a\in G\), define "translation" \(\tau_a:G/H\to G/H\) by \(\tau_a(xH)=axH\)
for every \(x\in G\). For \(a,b\in G\)
\begin{equation*}
(\tau_a\circ\tau_b)(xH)=a(bxH)=(ab)xH
\end{equation*}
so that 
\begin{equation*}
\tau_a\tau_b=\tau_{ab}
\end{equation*}
It follows that each \(\tau_a\) is a bijection and so \(\tau_a\in S_{G/H}\). Define
\(\varphi:G\to S_{G/H}\) by \(\varphi(a)=\tau_a\). Rewriting
\begin{equation*}
\varphi(a)\varphi(b)=\tau_a\tau_b=\tau_{ab}=\varphi(ab)
\end{equation*}
so that \(\varphi\) is a homomorphism. Finally if \(a\in\ker\varphi\), then
\(\varphi(a)=1_{G/H}\), so that \(\tau_a(xH)=xH\), in particular, when \(x=1\), this
gives \(aH=H\) and \(a\in H\). And \(S_{G/H}\cong S_n\)
\end{proof}

When \(H=\{1\}\), this is the Cayley theorem.

Four-group \(\bV=\{1,(1 2)(3 4),(1 3)(2 4), (1 4)(2 3)\}\)
\begin{proposition}[]
Every group \(G\) of order 4 is isomorphic to either \(\I_4\) or the four-group
\(\bV\). And \(\I_4\not\cong\bV\)
\end{proposition}

\begin{proof}
By lagrange's theorem, every element in \(G\) other than 1 has order 2 or 4. If
4, then \(G\) is cyclic.

Suppose \(x,y\neq 1\), then \(xy\neq x,y\). Hence \(G=\{1,x,y,xy\}\).
\end{proof}


\begin{proposition}[]
If \(G\) is a group of order 6, then \(G\) is isomorphic to either \(\I_6\) or
\(S_3\). Moreover \(\I_6\not\cong S_3\)
\end{proposition}

\begin{proof}


If \(G\) is not cyclic, since \(\abs{G}\) is even, it has some elements having
order 2, say \(t\) by exercise \ref{ex2.27}

If \(G\) is abelian. Suppose it has another different element \(a\) with order 2.
Then \(H=\{1,a,t,at\}\) is a subgroup which contradict. Hence it must contain
an element \(b\) of order 3. Then \(bt\) has order 6 and \(G\) is cyclic.

If \(G\) is not abelian. If \(G\) doesn't have elements of order 3, then it's
abelian. Hence \(G\) has an element \(s\) of order 3.

Now \(\abs{\la s\ra}=3\), so \([G:\la s\ra]=\abs{G}/\abs{\la s\ra}=2\) and \(\la
   s\ra\) is normal. 
Since \(t=t^{-1}\), \(tst\in\la s\ra\). If \(tst=s^0=1\), \(s=1\).
If \(tst=s\), \(\abs{\la st\ra}=6\). Therefore \(tst=s^2=s^{-1}\).

Let \(H=\la t\ra\), \(\varphi:G\to S_{G/\la t\ra}\) given by
\begin{equation*}
\varphi(g):x\la t\ra\mapsto gx\la t\ra
\end{equation*}
By representation on cosets, \(\ker\varphi\le\la t\ra\). Hence
\(\ker\varphi=\{1\}\) or \(\ker\varphi=\la t\ra\). Since
\begin{equation*}
\varphi(t)=
\begin{pmatrix}
\la t\ra&s\la t\ra&s^2\la t\ra\\
t\la t\ra&ts\la t\ra&ts^2\la t \ra
\end{pmatrix}
\end{equation*}
If \(\varphi(t)\) is the identity permutation, then \(ts\la t\ra=s\la t\ra\), so
that \(s^{-1}ts\in\la t\ra=\{1,t\}\). But now \(s^{-1}ts=t\). Therefore
\(t\not\in\ker\varphi\) and \(\ker\varphi=\{1\}\). Therefore \(\varphi\) is
injective. Because \(\abs{G}=\abs{S_3}\), \(G\cong S_3\)
\end{proof}


\begin{definition}[]
If \(X\) is a set and \(G\) is a group, then \(G\) \textbf{acts} on \(X\) if there is a
function \(G\times X\to X\), denoted by \((g,x)\to gx\) s.t.
\begin{enumerate}
\item (gh)x=g(hx) for all \(g,h\in G\) and \(x\in X\)
\item \(1x=x\) for all \(x\in X\)
\end{enumerate}


\(X\) is a \textbf{\(G\)-set} if \(G\) acts on \(X\)
\end{definition}

If a group \(G\) acts on a set \(X\), then fixing the first variable, say \(g\),
gives a function \(\alpha_g:X\to X\), namely, \(\alpha_g:x\mapsto gx\). This
function is a permutation of \(X\), for its inverse is \(\alpha_{g^{-1}}\)
\begin{equation*}
\alpha_g\alpha_{g^{-1}}=1=\alpha_{g^{-1}}\alpha_g
\end{equation*}
If's easy to see that \(\alpha:G\to S_X\) defined by \(\alpha:g\mapsto\alpha_g\)
is a homomorphism. Conversely, given any homomorphism \(\varphi:G\to S_X\),
define \(gx=\varphi(g)(x)\). Thus an action of a group \(G\) on a set \(X\) is
another way of viewing a homomorphism.

\begin{definition}[]
If \(G\) acts on \(X\) and \(x\in X\), then the \textbf{orbit} of \(x\), denoted by
\(\calo(x)\), is the subset of \(X\)
\begin{equation*}
\calo(x)=\{gx:g\in G\}\subseteq X
\end{equation*}
the \textbf{stabilizer} of \(x\), denoted by \(G_x\), is the subgroup
\begin{equation*}
G_x=\{g\in G:gx=x\}\le G
\end{equation*}
\end{definition}

\begin{examplle}[]
\begin{enumerate}
\item Caylay's theorem says that \(G\) acts on itself by translation:
\(\tau_g:a\mapsto ga\).  We say \(G\) acts \textbf{transitively} on \(X\) if there is
only one orbit.
\item When \(G\) acts on \(G/H\) by translation \(\tau_g:aH\mapsto gaH\), then the
orbit \(\calo(aH)=G/H\)
\item When a group \(G\) acts on itself by conjugation, then the orbit  \(\calo(x)\)
is 
\begin{equation*}
\{y\in G:y=axa^{-1}\text{ for some }a\in G\}
\end{equation*}
in this case, \(\calo(x)\) is called the \textbf{conjugacy class} of \(x\), and it is
commonly denoted by \(x^G\).

\textbf{centralizer} \(C_G(x)=\{g\in G:gxg^{-1}=x\}\)
\item Let \(X=\{1,2,\dots,n\}\), let \(\alpha\in S_n\) and regard the cyclic group
\(G=\la\alpha\ra\) as acting on \(X\). If \(i\in X\), then
\begin{equation*}
\calo(i)=\{\alpha^k(i):k\in\Z\}
\end{equation*}
Let the complete factorization of \(\alpha\) be \(\alpha=\beta_1\dots\beta_{t(\alpha)}\),
and let \(i=i_1\) be moved by \(\alpha\). If the cycle involving \(i_1\) is
\(\beta_j=(i_1 i_2 \dots i_r)\),
\begin{equation*}
\calo(i)=\{i_1,\dots,i_r\}
\end{equation*}
where \(i=i_1\). It follows that \(\abs{\calo(i)}=r\). The stabilizer \(G_l\) of
a number \(l\) is \(G\) if \(\alpha\) fixes \(l\)
\end{enumerate}
\end{examplle}


\textbf{Normalizer}
\begin{equation*}
N_G(H)=\{g\in G:gHg^{-1}=H\}
\end{equation*}


\begin{proposition}[]
If \(G\) acts on a set \(X\), then \(X\) is the disjoint union of the orbits. If
\(X\) is finite, then
\begin{equation*}
\abs{X}=\displaystyle\sum_i\abs{\calo(x_i)}
\end{equation*}
where \(x_i\) is chosen from each orbit
\end{proposition}

\begin{proof}
\(x\equiv y\Leftrightarrow\) there exists \(g\in G\) with \(y=gx\) is an
equivalence relation
\end{proof}


\begin{theorem}[]
\label{thm2.98}
If \(G\) acts on a set \(X\) and \(x\in X\) then
\begin{equation*}
\abs{\calo(x)}=[G:G_x]
\end{equation*}
\end{theorem}

\begin{proof}
Let \(G/G_x\) denote the family of cosets. Construct a bijection
\(\varphi:G/G_x\to \calo(x)\)
\end{proof}

\begin{corollary}[]
\label{cor2.99}
   If a finite group \(G\) acts on a set X, then the number of elements in any
   orbit is a divisor of \(\abs{G}\). 
\end{corollary}

\begin{corollary}[]
\label{cor2.100}
If \(x\) lies in a finite group \(G\), then the number of conjugates of \(x\) is
the index of its centralizer
\begin{equation*}
\abs{x^G}=[G:C_G(x)]
\end{equation*}
and hence it's a divisor of \(G\)
\end{corollary}

\begin{proof}
\(x^G\) is the orbit, \(C_G(x)\) is the stabilizer
\end{proof}

\begin{proposition}[]
If \(H\) is a subgroup of a finite group \(G\), then the number of conjugates of
\(H\) in \(G\) is \([G:N_G(H)]\)
\end{proposition}

\begin{proof}
Similar to theorem \ref{thm2.98}
\end{proof}

\begin{theorem}[Cauchy]
\label{thmCauchy}
If \(G\) is a finite group whose order is divisible by a prime \(p\), then \(G\)
contains an element of order \(p\)
\end{theorem}

\begin{proof}
Prove by induction on \(m\ge 1\), where \(\abs{G}=mp\). If \(m=1\), it's obvious.

If \(x\in G\) , then \(\abs{x^G}=[G:C_G(x)]\). If \(x\not\in Z(G)\), then \(x^G\)
has more than one element, so \(\abs{C_G(x)}<\abs{G}\). If \(p\mid \abs{C_G(x)}\), by
inductive hypothesis, we are done. Else if \(p\nmid \abs{C_G(x)}\) for all
noncentral \(x\) and \(\abs{G}=[G:C_G(x)]\abs{C_G(x)}\), we have
\begin{equation*}
p\mid[G:C_G(x)]
\end{equation*}
\(Z(G)\) consists of all those elements with \(\abs{X^G}=1\), we have
\begin{equation*}
\abs{G}=\abs{Z(G)}+\displaystyle\sum_i[G:C_G(x_i)]
\end{equation*}
Hence \(p\mid\abs{Z(G)}\) and by proposition \ref{prop2.78}
\end{proof}

\begin{definition}[]
The \textbf{class equation} of a finite group \(G\) is
\begin{equation*}
\abs{G}=\abs{Z(G)}+\displaystyle\sum_i[G:C_G(x_i)]
\end{equation*}
where each \(x_i\) is selected from each conjugacy class having more than one element
\end{definition}

\begin{definition}[]
If \(p\) is a prime, then a finite group \(G\) is called a \textbf{p-group} if
\(\abs{G}=p^n\) for some \(n\ge 0\)
\end{definition}

\begin{theorem}[]
\label{thm2.103}
If \(p\) is a prime and \(G\) is a p-group, then \(Z(G)\neq\{1\}\)
\end{theorem}

\begin{proof}
Consider 
\begin{equation*}
\abs{G}=\abs{Z(G)}+\displaystyle\sum_i[G:C_G(x_i)]
\end{equation*}
\end{proof}

\begin{corollary}[]
If \(p\) is a prime, then every group \(G\) of order \(p^2\) is abelian
\end{corollary}
\begin{proof}
If \(G\) is not abelian, then \(Z(G)\) has order \(p\). The center is always normal, and
so \(G/Z(G)\) is defined; it has order \(p\) and is cyclic by Lagrange's theorem. 
This contradicts Exercise \ref{ex2.69}
\end{proof}

\begin{examplle}[]
Cauchy's theorem and Fermat's theorem are special cases of some common theorem.

If \(G\) is a finite group and \(p\) is a prime, define
\begin{equation*}
X=\{(a_0,a_1,\dots,a_{p-1})\in G^p:a_0a_1\dots a_{p-1}=1\}
\end{equation*}
Note that \(\abs{X}=\abs{G}^{p-1}\), for having chosen the last \(p-1\) entries
arbitrarily, the 0th entry must equal \((a_1a_2\dots a_{p-1})^{-1}\). Introduce an
action of \(\I_p\) on \(X\) by defining, for \(0\le i\le p-1\),
\begin{equation*}
[i](a_0,\dots,a_{p-1})=(a_{i+1},\dots,a_{p-1},a_0,\dots,a_i)
\end{equation*}
The product of the new \(p\)-tuple is a conjugate of \(a_0a_1\dots a_{p-1}\)
\begin{equation*}
a_{i+1}\dots a_{p-1}a_0\dots a_{i}=(a_0\dots a_i)^{-1}(a_0\dots a_{p-1})
(a_0\dots a_i)
\end{equation*}
This conjugate is \(1\) for \(g^{-1}1g=1\), and so \([i](a_0,\dots,a_{p-1})\in X\). By
Corollary \ref{cor2.99}, the size of every orbit of \(X\) is a divisor of
\(\abs{\I_p}=p\). Now orbits with just one element consists of a \(p\)-tuple all
of whose entries \(a_i\) are equal, for all cyclic permutations of the \(p\)-tuple
are the same. In other words, such an orbit corresponds to an element \(a\in G\)
with \(a^p=1\). Clearly \((1,1,\dots,1)\) is such an orbit; if it were the only such
, then we would have
\begin{equation*}
\abs{G}^{p-1}=\abs{X}=1+kp
\end{equation*}
That is, \(\abs{G}^{p-1}\equiv 1\mod p\). If \(p\) is a divisor of \(\abs{G}\), then
we have a contradiction and thus proved Cauchy's theorem.
\end{examplle}

\begin{proposition}[]
If \(G\) is a group of order \(\abs{G}=p^e\) then \(G\) has a normal subgroup of order
\(p^k\) for every \(k\le e\)
\end{proposition}

\begin{proof}
We prove the result by induction on \(e\ge 0\).

By Theorem \ref{thm2.103}, \(Z(G)\neq\{1\}\). Let \(Z\le Z(G)\) be a subgroup of order
\(p\) and \(Z\) is normal. If \(k\le e\), then \(p^{k-1}\le p^{e-1}=\abs{G/Z}\). By
induction, \(G/Z\) has a normal subgroup \(H^*\) of order \(p^{k-1}\). The
correspondence theorem says there is a subgroup \(H\) of \(G\) containing \(Z\) with
\(H^*=H/Z\); moreover \(H^*\triangleleft G/Z\)  implies \(H\triangleleft G\). But
\(\abs{H/Z}=p^{k-1}\) implies \(\abs{H}=p^k\) as desired.
\end{proof}

\begin{definition}[]
A group \(G\neq\{1\}\) is called \textbf{simple} if \(G\) has no normal subgroups other than
\(\{1\}\) and  \(G\) itself.
\end{definition}

\begin{proposition}[]
An abelian group \(G\) is simple if and only if it is finite and of prime order
\end{proposition}
\begin{proof}
Assume \(G\) is simple. Since \(G\) is abelian, every subgroup is normal, and so \(G\)
has no subgroups otherthan \(\{1\}\) and \(G\). Choose \(x\in G\) with \(x\neq 1\).
Since \(\la x\ra\le G\), we have \(\la x\ra =G\). If \(x\) has infinite order, then
all the powers of \(x\) are distinct, and so \(\la x^2\ra<\la x\ra\) is a forbidden
subgroup of \(\la x\ra\), a contradiction. Therefore every \(x\in G\) has finite
order. If \(x\) has order \(m\) and if \(m\) is composite, say \(m=kl\), then 
\(\la x^k\ra\) is a proper subgroup of \(\la x\ra\), a contradiction. Therefore
\(G=\la x\ra\) has prime order.
\end{proof}

Suppose that an element \(x\in G\) has \(k\) conjugates, that is 
\begin{equation*}
\abs{x^G}=\abs{\{gxg^{-1}:g\in G\}}=k
\end{equation*}
If there is a subgroup \(H\le G\) with \(x\in H\le G\), how many conjugates does \$x
\$ have in \(H\)?

Since
\begin{equation*}
x^H=\{hxh^{-1}:h\in H\}\subseteq x^G
\end{equation*}
we have \(\abs{x^H}\le\abs{x^G}\). It is possible that there is a strict
inequality \(\abs{x^H}<\abs{x^G}\). For example, take \(G=S_3,x=(1\; 2)\), and
\(H=\la x\ra\). Now let us consider this question, in particular, for
\(G=S_5,x=(1\;2\;3), H=A_5\)

\begin{lemma}[]
All 3-cycles are conjugate in \(A_5\)
\end{lemma}

\begin{proof}
Let \(G=S_5,\alpha=(1\; 2\;3), H=A_5\). We know that \(\abs{\alpha^{S_5}}=20\), for there
are 20 3-cycles in \(S_5\). Therefore, \(20=\abs{S_5}/\abs{C_{S_5}(\alpha)}\) by
Corollary \ref{cor2.100} , so that \(\abs{C_{S_5}(\alpha)}=6\). Here they are
\begin{equation*}
(1),\;(1\;2\;3),\;(1\;3\;2),\;(4\;5),\;(4\;5)(1\;2\;3),\;(4\;5\;)(1\;3\;2)
\end{equation*}
The last there of these are odd permutations, so that \(\abs{C_{A_5}(\alpha)}=3\). We
conclude that
\begin{equation*}
\abs{\alpha^{A_5}}=\abs{A_5}/\abs{C_{A_5}(\alpha)}=20
\end{equation*}
that is all 3-cycles are conjugate to \(\alpha\) in \(A_5\)
\end{proof}

\begin{lemma}[]
\label{lemma2.109}
If \(n\ge 3\), every element in \(A_n\) is a 3-cycle or a product of 3-cycles
\end{lemma}
\begin{proof}
Since each \(\beta\) equals \(\tau_1\dots\tau_{2q}\)
\end{proof}

\begin{theorem}[]
\(A_5\) is a simple group
\end{theorem}
\begin{proof}
If \(H\triangleleft A_5\) and \(H\neq\{(1)\}\). Now if \(H\) contains a 3-cycle, then
normality forces \(H\) to contain all its conjugates. Therefore it suffices to
prove that \(H\) contains 3-cycle.

Since \(\sigma\in H\), we may assume, after a harmless relabeling, that either
\(\sigma=(1\;\;2\;3),\sigma=(1\;2)(3\;4)\) or \(\sigma=(1\;2\;3\;4\;5x)\)

If \(\sigma=(1\;2)(3\;4)\), define \(\tau=(1\;2)(3\;5)\). Now
\((3\;5\;4)=(\tau\sigma\tau^{-1})\sigma^{-1}\in H\). If \(\sigma=(1\;2\;3\;4\;5)\),
define \(\rho=(1\;3\;2)\) and \((1\;3\;4)=\rho\sigma\rho^{-1}\sigma^{-1}\in H\)
\end{proof}

\(A_4\) is not simple for \(\bV\triangleleft A_4\).

\begin{lemma}[]
\(A_6\) is a simple group
\end{lemma}

\begin{proof}
Let \(\{1\}\neq H\triangleleft A_6\); we must show that \(H=A_6\). Assume that there
is some \(\alpha\in H\) with \(\alpha\neq (1)\) that fixes some \(i\), where \(1\le
i\le 6\). Define
\begin{equation*}
F=\{\sigma\in A_6:\sigma(i)=i\}
\end{equation*}
Note that \(\alpha\in H\cap F\), so that \(H\cap F\neq\{(1)\}\). The second
isomorphism theorem gives \(H\cap F\triangleleft F\). But \(F\) is simple for
\(F\cong A_5\), we have \(H\cap F=F\): that is \(F\le H\). It follows that \(H\)
contains a 3-cycle, and so \(H=A_6\) by Exercise \ref{ex2.91}.

If there is no \(\alpha\in H\) with \(\alpha\neq\{1\}\) that fixes some \(i\) with
\(1\le i\le 6\). If we consider the cycle structures of permutations in \(A_6\),
however, any such \(\alpha\) must have cycle structure \((1\;2)(3\;4\;5\;6)\) or
\((1\;2\;3)(4\;5\;6)\). In the first case \(\alpha^2\in H\), \(\alpha^2\in H\)
fixes 1. In the second case \(\alpha(\beta\alpha^{-1}\beta^{-1})\) where
\(\beta=(2\;3\;4)\) fixes 1.
\end{proof}

\begin{theorem}[]
\(A_n\) is a simple group for all \(n\ge 5\)
\end{theorem}

\begin{proof}
If \(H\) is a nontrivial normal subgroup of \(A_n\), then we must show that \(H=A_n\).
By Exercise \ref{ex2.91} it suffices to prove that \(H\) contains a 3-cycle. If
\(\beta\in H\) is nontrivial, then there exists some \(i\) that \(\beta\) moves: say,
\(\beta(i)=j\neq i\). Choose a 3-cycle \(\alpha\) that fixes \(i\) and moves \(j\). The
permutations \(\alpha\) and \(\beta\) do not commute. It follows that
\(\gamma=(\alpha\beta\alpha^{-1})\beta^{-1}\) is a nontrivial element of \(H\). But
\(\beta\alpha^{-1}\beta^{-1}\) is a 3-cycle, and so
\(\gamma=\alpha(\beta\alpha^{-1}\beta^{-1})\) is a product of two 3-cycles. Hence
\(\gamma\) moves at most 6 symbols, say \(i_1,\dots,i_6\). Define
\begin{equation*}
F=\{\sigma\in A_n:\sigma\text{ fixes all }i\neq i_1,\dots,i_6\}
\end{equation*}
Now \(F\cong A_6\) and \(\gamma\in H\cap F\). Hence \(H\cap F\triangleleft F\). But
\(F\) is simple, and so \(H\cap F=F\); that is \(F\le H\). Therefore \(H\) contains a
3-cycle 
\end{proof}

\begin{theorem}[Burnside's Lemma]
Let \(G\) act on a finite set \(X\). If \(N\) is the number of orbits, then
\begin{equation*}
N=\frac{1}{\abs{G}}\displaystyle\sum_{\tau\in G}Fix(\tau)
\end{equation*}
where \(Fix(\tau)\) is the number of \(x\in X\) fixed by \(\tau\)
\end{theorem}
\begin{proof}
List the elements of \(X\) as follows: Choose \(x_1\in X\) and then list all the
elements \(x_1,\dots,x_r\) in the orbit \(\calo(x_1)\); then choose
\(x_{r+1}\not\in\calo(x_1)\), and so on until all the elements of \(X\) are listed.
Now list the elements \(\tau_1,\dots,\tau_n\) of \(G\) and form the following array,
where
\begin{equation*}
f_{i,j}=
\begin{cases}
1&\text{ if }\tau_i\text{ fixes }x_j\\
0&\text{ if }\tau_i\text{ moves }x_j
\end{cases}
\end{equation*}
\[
\begin{tabular}{c|cccccc}
  & $x_1$ & $x_2$ & $\dots$ & $x_{r+1}$ & $x_{r+2}$ & $\dots$ \\
\hline $\tau_1$ & $f_{1,1}$ & $f_{1,2}$ & $\dots$ & $f_{1,r+1}$ & $f_{1,r+2}$ & $\dots$ \\
 $\vdots$ &  &  &  &  &  &  \\
 $\tau_n$ & $f_{n,1}$ & $f_{n,2}$ & $\dots$ & $f_{n,r+1}$ & $f_{n,r+2}$ & $\dots$ \\
\end{tabular}
\]
Now \(Fix(\tau_i)\) is the number of 1's in the \(i\)th row. therefore 
\(\sum_{\tau\in G}Fix(\tau)\) is the total number of 1's in the array. The number
of 1's in column 1 is \(\abs{G_{x_1}}\). By Exercise \ref{ex2.99}
\(\abs{G_{x_1}}=\abs{G_{x_2}}\). By Theorem \ref{thm2.98} the number of 1's in the
\(r\) columns labels by the \(x_i\in\calo(x_i)\) is thus
\begin{equation*}
r\abs{G_{x_1}}=\abs{\calo(x_1)}\cdot\abs{G_{x_1}}=(\abs{G}/
\abs{G_{x_1}})\abs{G_{x_1}}=\abs{G}
\end{equation*}
Therefore
\begin{equation*}
\displaystyle\sum_{\tau\in G}Fix(\tau)=N\abs{G}
\end{equation*}
\end{proof}

We are going to use Burnside's lemma to solve problems of the following sort.
How many striped flags are there having six stripes each of which can be colored
red, white or blue?
\begin{center}
\begin{tabular}{|c|c|c|c|c|c|}
\hline
r & w & b & r & w & b\\
\hline
\end{tabular}
\end{center}

\begin{center}
\begin{tabular}{|c|c|c|c|c|c|}
\hline
b & w & r & b & w & r\\
\hline
\end{tabular}
\end{center}

Let \(X\) be the set of all 6-tuples of colors: if \(x\in X\), then
\begin{equation*}
x=(c_1,c_2,c_3,c_4,c_5,c_6)
\end{equation*}

Let \(\tau\) be the permutation that reserves all the indices:
\[
\tau=\begin{pmatrix}
 1 & 2 & 3 & 4 & 5 & 6 \\
 6 & 5 & 4 & 3 & 2 & 1 \\
\end{pmatrix}=(1\;6)(2\;5)(3\;4)
\]

(thus \(\tau\) turns over each 6-tuple \(x\) of colored stripes). The cyclic group
\(G=\la\tau\ra\) acts on \(X\); since \(\abs{G}=2\), the orbit of any 6-tuple \(x\)
consists of either 1 or 2 elements. Since a flag is unchanged by turning it
over, it is reasonable to identify a flag with an orbit of 6-tuple. For example,
the orbit consisting of the 6-tuples
\begin{equation*}
(r,w,b,r,w,b)\text{ and }(b,w,r,b,w,r)
\end{equation*}
above. The number of flags is thus the number \(N\) of orbits; by Burnside's
lemma, \(N=\frac{1}{2}[Fix((1))+Fix(\tau)]\). The identity permutation \((1)\) fixes
every \(x\in X\), and so \(Fix((1))=3^6\). Now \(\tau\) fixes a 6-tuple \(x\) if it's a
"palindrome". It follows that \(Fix(x)=3^3\). The number of flags is thus
\begin{equation*}
N=\frac{1}{2}(3^6+3^3)=378
\end{equation*}

\begin{definition}[]
If a group \(G\) acts on \(X=\{1,\dots,n\}\) and if \(\calc\) is a set of \(q\) colors,
then \(G\) acts on the set \(\calc^n\) of all \(n\)-tuples of colors by
\begin{equation*}
\tau(c_1,\dots,c_n)=(c_{\tau1},\dots,c_{\tau n})\text{ for all }\tau\in G
\end{equation*}
An orbit of \((c_1,\dots,c_n)\in\calc^n\) is called a \textbf{\((q,G)\)-coloring} of \(X\).
\end{definition}

\begin{examplle}[]
Color each square in a \(4\times 4\) grid red or black.

If \(X\) consists of the 16 squares in the grid and if \(\calc\) consists of the two
colors red and black, then the cyclic group \(G=\la R\ra\) or order 4 acts on \(X\),
where \(R\) is a clockwise rotation by \(\ang{90}\); 

\begin{figure}[h]
\centering
\begin{subfigure}[b]{0.4\textwidth}
\begin{tabular}{|c|c|c|c|}
\hline
1 & 2 & 3 & 4\\
\hline
5 & 6 & 7 & 8\\
\hline
9 & 10 & 11 & 12\\
\hline
13 & 14 & 15 & 16\\
\hline
\end{tabular}
\end{subfigure}
\begin{subfigure}[b]{0.4\textwidth}
\begin{tabular}{|c|c|c|c|}
\hline
13 & 9 & 5 & 1\\
\hline
14 & 10 & 6 & 2\\
\hline
15 & 11 & 7 & 3\\
\hline
16 & 12 & 8 & 4\\
\hline
\end{tabular}
\end{subfigure}
\label{fig2.10}
\end{figure}

Figure shows how \(R\) acts: the right square is \(R)\)'s action on the left
square. In cycle notation
\begin{align*}
&R=(1,\;4,\;16,\;13)(2,\;8,\;15,\;9)(3,\;12,\;14,\;5)(6,\;7,\;11,\;10)\\
&R^2=(1,\;16)(4,\;13)(2,\;15)(8,\;9)(3,\;14)(12,\;5)(6,\;11)(7,\;10)\\
&R^3=(1,\;13,\;16,\;4)(2,\;9,\;15,\;8)(3,\;5,\;14,\;12)(6,\;10,\;11,\;7)
\end{align*}

By Burnside's lemma, the number of chessboards is
\begin{equation*}
\frac{1}{4}[Fix((1))+Fix(R)+Fix(R^2)+Fix(R^3)]
\end{equation*}
\end{examplle}



\begin{exercise}
Prove that if \(p\) is a prime and \(G\) is a finite group in which every element
has order a power of \(p\), then \(G\) is a \(p\)-group. (A possibly infinite group
\(G\)) is called a \textbf{\(p\)-group} if every element in \(G\) has order a power of \(p\)
\end{exercise}
\begin{proof}
By Cauchy's theorem \ref{thmCauchy}
\end{proof}

\begin{exercise}
\label{ex2.91}
\begin{enumerate}
\item For all \(n\ge 5\), prove that all 3-cycles are conjugate in \(A_n\)
\item Prove that if a normal subgroup \(H\triangleleft A_n\) contains a 3-cycle,
where \(n\ge 5\), then \(H=A_n\)
\end{enumerate}
\end{exercise}
\begin{proof}
\begin{enumerate}
\item If \((1\;2\;3)\) and \((i\; j\; k)\) are not disjoint. As Example \ref{example2.8}
illustrated, \(\alpha\in S_5\) 

If they are disjoint, simple
\item By lemma \ref{lemma2.109}
\end{enumerate}
\end{proof}

\begin{exercise}
\label{ex2.99}
\begin{enumerate}
\item Let a group \(G\) act on a set \(X\), and suppose that \(x,y\in X\) lie in the same
orbit: y=gx for some \(g\in G\). Prove that \(G_y=gG_xg^{-1}\)
\item Let \(G\) be a finite group acting on a set \(X\); prove that if \$\$x,y\(\in\) X lie
in the same orbit, then \(\abs{G_x}=\abs{G_y}\)
\end{enumerate}
\end{exercise}

\begin{proof}
\begin{enumerate}
\item If \(f\in G_x\), then \(gfg^{-1}(y)=gfg^{-1}gx=gx=y\)
\item There is a bijection.
\end{enumerate}
\end{proof}
\section{Commuctative Rings \rom{1}}
\label{sec:org32d5992}
\subsection{First Properties}
\label{sec:orgdb7af21}
\index{commutative ring}
\begin{definition}[]
A \textbf{commutative ring} \(R\) is a set with two binary operations, addition and
multiplication s.t.
\begin{enumerate}
\item \(R\) is an abelian group under addition
\item (\textbf{commutativity}) \(ab=ba\) for all \(a,b\in R\)
\item (\textbf{associativity}) \(a(bc)=(ab)c\) for every \(a,b,c\in R\)
\item there is an element \(1\in R\) with \(1a=a\) for every \(a\in R\)
\item (\textbf{distributivity}) \(a(b+c)=ab+ac\) for every \(a,b,c\in R\)
\end{enumerate}
\end{definition}

The element 1 in a ring \(R\) has several names: it is called \textbf{one}, the \textbf{unit} of
\(R\), or the \textbf{identity} in \(R\)

\begin{examplle}[]
\begin{enumerate}
\item \(\Z,\Q,\R\) and \(\C\) are commutative rings with the usual addition and
multiplication
\item Consider the set \(R\) of all real numbers \(x\) of the form
\begin{equation*}
x=a+b\omega
\end{equation*}
where \(a,b\in\Q\) and \(\omega=\sqrt[3]{2}\). \(R\) is closed under ordinary
addition. However, if \(R\) is closed under multiplication, then
\(\omega^2\in R\) and there are rationals \(a\) and \(b\) with
\begin{align*}
&\omega^2=a+b\omega\\
&2=a\omega+b\omega^2\\
&b\omega^2=ab+b^2\omega
\end{align*}
Hence \(2-a\omega=ab+b^2\omega\) and so
\begin{equation*}
2-ab=(b^2+a)\omega
\end{equation*}
A contradiction.
\end{enumerate}
\end{examplle}


\begin{proposition}[]
Let \(R\) be a commutative ring.
\begin{enumerate}
\item \(0\cdot a=0\) for every \(a\in R\)
\item If \(1=0\) then \(R\) consists of the single element 0. In this case \(R\)
is called the \textbf{zero ring}
\item If \(-a\) is the additive inverse of \(a\), then \((-1)(-a)=a\)
\item \((-1)a=-a\) for every \(a\in R\)
\item If \(n\in\N\) and \(n1=0\), then \(na=0\) for all \(a\in R\)
\item The binomial theorem holds: if \(a,b\in R\), then 
\begin{equation*}
(a+b)^n=\displaystyle\sum_{r=0}^n\binom{n}{r}a^rb^{n-r}
\end{equation*}
\end{enumerate}
\end{proposition}

\begin{proof}
\begin{enumerate}
\setcounter{enumi}{5}
\item \(\binom{n+1}{r}=\binom{n}{r-1}+\binom{n}{r}\)
\end{enumerate}
\end{proof}

\begin{definition}[]
A subset \(S\) of a commutative ring \(R\) is a \textbf{subring} of \(R\) if
\begin{enumerate}
\item \(1\in S\)
\item if \(a,b\in S\) then \(a-b\in S\)
\item if \(a,b\in S\), then \(ab\in S\)
\end{enumerate}
\end{definition}

\textbf{Notation}. The tradition in ring theory is to write \(S\subseteq R\) for a
subring

\begin{proposition}[]
A subring \(S\) of a commutative ring \(R\) is itself a commutative ring.
\end{proposition}

\begin{definition}[]
A \textbf{domain} (often called an \textbf{integral domain}) is a commutative ring \(R\) that
satisfies two extra axioms: first
\begin{equation*}
1\neq 0
\end{equation*}
second, the \textbf{cancellation law} for multiplication: for all \(a,b,c\in R\)
\begin{equation*}
\text{ if } ca=cb\text{ and }c\neq 0,\text{ then }a=b
\end{equation*}
\end{definition}

\begin{proposition}[]
A nonzero commutative ring \(R\) is a domain if and only if the product of
any two nonzero elements of \(R\) is nonzero
\end{proposition}
\begin{proof}
\(ab=ac\) if and only if \(a(b-c)=0\)
\end{proof}

\begin{proposition}[]
The commutative ring \(\I_m\) is a domain if and only if \(m\) is a prime
\end{proposition}
\begin{proof}
If \(m=ab\), where \(1<a,b<m\), then \([a],[b]\neq[0]\) yet
\([a][b]=[m]=[0]\)

Conversely, if \(m\) is a prime and \([a][b]=[ab]=[0]\), then \(m\mid ab\)
\end{proof}

\begin{examplle}[]
\begin{enumerate}
\item Let \(\calf(\R)\) be the set of all the function \(\R\to \R\) equipped
with the operations of \textbf{point-wise addition} and \textbf{point-wise multiplication}:
Given \(f,g\in\calf(\R)\), define functions \(f+g\) and \(fg\) by
\begin{equation*}
f+g:a\mapsto f(a)+f(b)\quad\text{ and }\quad fg:a\mapsto f(a)g(a)
\end{equation*}
We claim that \(\calf(\R)\) with these operations is a commutative ring.
The zero element is the constant function \(z\) with value 0.
\(\calf(\R)\) is not a domain by
\begin{equation*}
f(a)=
\begin{cases}
a&\text{ if }a\le 0\\
0\\
\end{cases}\e g(a)=
\begin{cases}
0&\text{ if }a\le 0\\
a
\end{cases}
\end{equation*}
\end{enumerate}
\end{examplle}

\index{division}
\begin{definition}[]
Let \(a\) and \(b\) be elements of a commutative ring \(R\). Then \(a\) \textbf{divides}
\(b\) \textbf{in} \(R\) (or \(a\) is a \textbf{divisor} of \(b\) or \(b\) is a \textbf{multiple} of \(a\)),
denoted by \(a\mid b\), if there exists an element \(c\in R\) with \(b=ca\)
\end{definition}

\begin{definition}[]
An element \(u\) in a commutative ring \(R\) is called a \textbf{unit} if \(u\mid 1\) in \(R\).
\end{definition}

\begin{proposition}[]
Let \(R\) be a domain, and let \(a,b\in R\) be nonzero. Then \(a\mid b\) and
\(b\mid a\) if and only if \(b=ua\) for some unit \(u\in R\)
\end{proposition}

\begin{proposition}[]
If \(a\) is an integer, then \([a]\) is a unit in \(\I_m\) if and only if
\(a\) and \(m\) are relatively prime. 
\end{proposition}

\begin{corollary}[]
If \(p\) is a prime, then every nonzero \([a]\) in \(\I_p\) is a unit.
\end{corollary}

\begin{definition}[]
If \(R\) is a commutative ring, then the \textbf{group of units} of \(R\) is
\begin{equation*}
U(R)=\{\text{all units in }R\}
\end{equation*}
\end{definition}

\begin{definition}[]
A \textbf{field} \(F\) is a commutative ring in which \(1\neq0\) and every nonzero element
\(a\) is a unit; that is, there is \(a^{-1}\in F\) with \(a^{-1}a=1\)
\end{definition}
A commutative ring \(R\) is a field if and only if \(U(R)=R^{\times}\), the
nonzero elements of \(R\).

\begin{proposition}[]
Every field \(F\) is a domain
\end{proposition}
\begin{proof}
\(ab=ac,b=a^{-1}ab=a^{-1}(ac)=c\)
\end{proof}

\begin{proposition}[]
The commutative ring \(\I_m\) is a field if and only if \(m\) is prime
\end{proposition}

\begin{theorem}[]
\label{thm3.13}
If \(R\) is a domain then there is a field \(F\) containing \(R\) as a subring.
Moreover, \(F\) can be chosen so that for each \(f\in  F\), there are \(a,b\in
   R\) with 
\(b\neq 0\) and \(f=ab^{-1}\) 
\end{theorem}

\begin{proof}
Let \(X=\{(a,b)\in R\times R:b\neq 0\}\) and define a relation \(\equiv\) on
\(X\) by \((a,b)\equiv(c,d)\) if \(ad=bc\). We claim that \(\equiv\) is an
equivalence relation. If \((a,b)\equiv(c,d)\) and \((c,d)\equiv(e,f)\), then
\(ad=bc,cf=de\) and \(adf=b(cf)=bde\), gives \(af=be\)

Denote the equivalence class of (a,b) by \([a,b]\), define \(F\) as the set of
all equivalence classes \([a,b]\) and equip \(F\) with the following addition and
multiplication 
\begin{align*}
&[a,b]+[c,d]=[ad+bc,bd]\\
&[a,b][c,d]=[ac,bd]
\end{align*}
Show addition and multiplication are well-defined.
\end{proof}

\begin{definition}[]
The field \(F\) constructed from \(R\) in Theorem \ref{thm3.13} is called the
\textbf{fraction field} of \(R\), denoted by \(\Frac(R)\), and we denote
\([a,b]\in\Frac(R)\) by \(a/b\)
\end{definition}

Note that \(\Frac(\Z)=\Q\)
\subsection{Polynomials}
\label{sec:org62430de}
\begin{definition}[]
If \(R\) is a commutative ring, then a \textbf{sequence} \(\sigma\) in \(R\) is
\begin{equation*}
\sigma=(s_0,s_1,\dots,s_i,\dots)
\end{equation*}
the entries \(s_i\in R\) for all \(i\ge 0\) are called the \textbf{coefficients} of \(\sigma\)
\end{definition}

\begin{definition}[]
A sequence \(\sigma=(s_0,\dots,s_i,\dots)\) in a commutative ring \(R\) is
called a \textbf{polynomial} if there is some integer \(m\ge 0\) with \(s_i=0\) for all
\(i>m\); that is 
\begin{equation*}
\sigma=(s_0,\dots,s_m,0,\dots)
\end{equation*}
A polynomial has only finitely many nonzero coefficients. The \textbf{zero
polynomial}, denoted by \(\sigma=0\)
\end{definition}

\begin{definition}[]
If \(\sigma(s_0,\dots,s_n,0,\dots)\neq0\) is a polynomial, we call \(s_n\) the
\textbf{leading coefficient} of \(\sigma\), we call \(n\) the \textbf{degree} of \(\sigma\), an we
denote \(n\) by \(\deg(\sigma)\) 
\end{definition}

\textbf{Notation}. If \(R\) is a commutative ring, then the set of all polynomials with
coefficients in \(R\) is denoted by \(R[x]\)


\begin{proposition}[]
\label{prop3.14}
If \(R\) is a commutative ring, then \(R[x]\) is a commutative ring that contains
\(R\) as a subring
\end{proposition}

\begin{proof}
\(\sigma=(s_0,s_1,\dots),\tau=(t_0,t_1,\dots)\)
\begin{align*}
&\sigma+\tau=(s_0+t_0,s_1+t_1,\dots)\\
&\sigma\tau=(c_0,c_1,\dots)
\end{align*}
where \(c_k=\sum_{i+j=k}s_it_j=\sum_{i=0}^ks_it_{k-i}\).
\end{proof}

\begin{lemma}[]
\label{lemma3.15}
Let \(R\) be a commutative ring and let \(\sigma,\tau\in R[x]\) be nonzero
polynomials.
\begin{enumerate}
\item Either \(\sigma\tau=0\) or \(\deg(\sigma\tau)\le\deg(\sigma)+\deg(\tau)\)
\item If \(R\) is a domain, then \(\sigma\tau\neq0\) and 
\begin{equation*}
\deg(\sigma\tau)=\deg(\sigma)+\deg(\tau)
\end{equation*}
\item If \(R\) is a domain, then \(R[x]\) is a domain
\end{enumerate}
\end{lemma}

\begin{proof}
\(\sigma=(s_0,s_1,\dots),\tau=(t_0,t_1,\dots)\) have degrees \(m\) and \(n\) respectively.
\begin{enumerate}
\item if \(k>m+n\), then each term in \(\sum_is_it_{k-i}\) is 0
\item Each term in \(\sum_is_it_{m+n-i}\) is 0 with the possible exception of
\(s_mt_n\). Since \(R\) is a domain, \(s_m\neq 0\) and \(t_n\neq0\) imply \(s_mt_n\neq0\).
\end{enumerate}
\end{proof}

\begin{definition}[]
If \(R\) is a commutative ring, then \(R[x]\) is called the \textbf{ring of polynomials
over \(R\)}
\end{definition}

\begin{definition}[]
Define the element \(x\in R[x]\) by
\begin{equation*}
x=(0,1,0,0,\dots)
\end{equation*}
\end{definition}

\begin{lemma}[]
\begin{enumerate}
\item IF \(\sigma=(s_0,\dots)\), then
\begin{equation*}
x\sigma=(0,s_0,s_1,\dots)
\end{equation*}
\item If \(n\ge 1\), then \(x^n\) is the polynomial having 0 everywhere except for 1
in the \(n\)th coordinate
\item If \(r\in R\), then
\begin{equation*}
(r,0,\dots)(s_0,s_1,\dots,s_j,\dots)=(rs_0,rs_1,\dots,rs_j,\dots)
\end{equation*}
\end{enumerate}
\end{lemma}

\begin{proposition}[]
If \(\sigma=(s_0,\dots,s_n,0,\dots)\), then
\begin{equation*}
\sigma=s_0+s_1x+s_2x^2+\dots+s_nx^n
\end{equation*}
where each element \(s\in R\) is identified with the polynomial \((s,0,\dots)\)
\end{proposition}

As a customary, we shall write 
\begin{equation*}
f(x)=s_0+s_1x+\dots+s_nx^n
\end{equation*}
instead of \(\sigma\). \(s_0\) is called its \textbf{constant term}. If \(s_n=1\) , then
\(f(x)\) is called \textbf{monic}.

\begin{corollary}[]
Polynomials \(f(x)=s_0+\dots+s_nx^n\) and \(g(x)=t_0+\dots+t_mx^m\) are equal if
and only if \(n=m\) and \(s_i=t_i\) for all \(i\).
\end{corollary}

If \(R\) is a commutative ring, each polynomial \(f(x)=s_0+\dots+s_nx^n\)
defines a \textbf{polynomial function} \(f:R\to R\) by evaluation: If \(a\in R\), define
\(f(a)=s_0+\dots+s_na^n\in R\).


\begin{definition}[]
Let \(k\) be a field. The fraction field of \(k[x]\), denoted by \(k(x)\), is
called the \textbf{field of rational function} over \(k\)
\end{definition}

\begin{proposition}[]
If \(k\) is a field, then the elements of \(k(x)\) have the form \(f(x)/g(x)\)
where \(f(x),g(x)\in k[x]\) and \(g(x)\neq 0\)
\end{proposition}

\begin{proposition}[]
If \(p\) is a prime, then the field of rational functions \(\I_p(x)\) is a n
infinite field containing \(\I_p\) as a subfield.
\end{proposition}

\begin{proof}
By Lemma \ref{lemma3.15} (3), \(\I_p[x]\) is an infinite domain for the powers
\(x^n\) for \(n\in\N\) are distinct. Thus its fraction filed \(\I_p(x)\) is an
infinite field containing \(\I_p[x]\) as a subring. But \(\I_p[x]\) contains
\(\I_p\) as a subring, by Proposition \ref{prop3.14}.
\end{proof}


\(R[x]\) is often called the ring of all \textbf{polynomials over \(R\) in one variable}.
If we write \(A=R[x]\), then \(A[y]\) is called the ring of all 
\textbf{polynomials over \(R\) in two variables \(x\) and \(y\)}, and it is denoted by \(R[x,y]\).

\begin{exercise}
Show that if \(R\) is a commutative ring, then \(R[x]\) is never a field
\end{exercise}

\begin{proof}
If \(R[x]\) is a field, then \(x^{-1}\in R[x]\) and \(x^{-1}=\sum_ic_ix^i\).
However
\begin{equation*}
\deg(xx^{-1})=\deg(1)=1=\deg(x)+\deg(x^{-1})
\end{equation*}
A contradiction.
\end{proof}

\begin{exercise}
\label{ex3.22}
Show that the polynomial function defined by \(f(x)=x^p-x\in\I_p[x]\) is
identically zero.
\end{exercise}

\begin{proof}
By Fermat's theorem \ref{Fermat}, \(a^p\equiv a\mod p\)
\end{proof}
\subsection{Greatest Common Divisors}
\label{sec:org2100a20}
\begin{theorem}[Division Algorithm]
Assume that \(k\) is a field and that \(f(x),g(x)\in k[x]\) with \(f(x)\neq 0\).
Then there are unique polynomials \(q(x),r(x)\in k[x]\) with
\begin{equation*}
g(x)=q(x)f(x)+r(x)
\end{equation*}
and either \(r(x)=0\) or \(\deg(r)<\deg(f)\)
\end{theorem}

\begin{proof}
We first prove the existence of such \(q\) and \(r\). If \(f\mid g\), then
\(g=qf\) for some \(q\); define the remainder \(r=0\). If \(f\nmid g\), then
consider all polynomials of the form \(g-qf\) as \(q\) varies over \(k[x]\). The
least integer axiom provides a polynomial \(r=g-qf\) having least degree
among all such polynomials. Since \(g=qf+r\), it suffices to show that
\(\deg(r)<\deg(f)\). Write \(f(x)=s_nx^n+\dots+s_1x+s_0\) and
\(r(x)=t^mx^m+\dots t_0\). Now \(s_n\neq 0\) implies that \(s_n\) is a unit
because \(k\) is a field and so \(s_n^{-1}\in k\). If \(\deg(r)\ge\deg(f)\), define
\begin{equation*}
h(x)=r(x)-t_ms_n^{-1}x^{m-n}f(x)
\end{equation*}
that is, if \(\LT(f)=s_nx^n\), where LT abbreviates \textbf{leading term}, then
\begin{equation*}
h=r-\frac{\LT(r)}{\LT(f)}f
\end{equation*}
note that \(h=0\) or \(\deg(h)<\deg(r)\). If \(h=0\), then \(r=[\LT(r)/\LT(f)]f\)
and
\begin{align*}
g&=qf+r=qf+\frac{\LT(r)}{\LT(f)}f\\
&=\left[q+\frac{\LT(r)}{\LT(f)}\right]f
\end{align*}
contradicting \(f\nmid g\). If \(h\neq0\), then \(\deg(h)<\deg(r)\) and
\begin{equation*}
g-qf=r=h+\frac{\LT(r)}{\LT(f)}f
\end{equation*}
Thus \(g-[q+\LT(r)/\LT(f)]f=h\), contradicting \(r\) being a polynomial of least
degree having this form. Therefore \(\deg(r)<\deg(f)\)

To prove uniqueness of \(q(x)\) and \(r(x)\) assume that \(g=q'f+r'\), where
\(\deg(r')<\deg(f)\). Then
\begin{equation*}
(q-q')f=r'-r
\end{equation*}
If \(r'\neq r\), then each side has a degree. But
\(\deg((q-q')f)=\deg(q-q')+\deg(f)\ge\deg(f)\), while
\(\deg(r'-r)\le\max\{\deg(r'),\deg(r)\}<deg(f)\), a contradiction. Hence
\(r'=r\) and \((q-q')f=0\). As \(k[x]\) is a domain and \(f\neq 0\), it follows that
\(q-q'=0\) and \(q=q'\)
\end{proof}

\begin{definition}[]
If \(f(x)\) and \(g(x)\) are polynomials in \(k[x]\), where \(k\) is a field, then
the polynomials \(q(x)\) and \(r(x)\) occurring in the division algorithm are
called the \textbf{quotient} and the \textbf{remainder} after dividing \(g(x)\) by \(f(x)\)
\end{definition}

The hypothesis that \(k\) is a filed is much too strong: long division can be
carried out in \(R[x]\) for every commutative ring \(R\) as long as the leading
coefficient of \(f(x)\) is a unit in \(R\); in particular, long division is
always possible when \(f(x)\) is monic.

\begin{corollary}[]
Let \(R\) be a commutative ring and let \(f(x)\in R[x]\) be a monic polynomial.
If \(g(x)\in R[x]\), then there exists \(q(x),r(x)\in R[x]\) with
\begin{equation*}
g(x)=q(x)f(x)+r(x)
\end{equation*}
where either \(r(x)=0\) or \(\deg(r)<\deg(f)\)
\end{corollary}

\begin{proof}
Note that \(\LT(r)/\LT(f)\in R\) because \(f(x)\) is monic
\end{proof}

\begin{definition}[]
If \(f(x)\in k[x]\), where \(k\) is a field, then a \textbf{root} of \(f(x)\) \textbf{in \(k\)} is an
element \(a\in k\) with \(f(a)=0\)
\end{definition}

\begin{lemma}[]
Let \(f(x)\in k[x]\), where \(k\) is a field, and let \(u\in k\). Then there is
\(q(x)\in k[x]\) with
\begin{equation*}
f(x)=q(x)(x-u)+f(u)
\end{equation*}
\end{lemma}
\begin{proof}
The division algorithm gives
\begin{equation*}
f(x)=q(x)(x-u)+r
\end{equation*}
Now evaluate
\begin{equation*}
f(u)=q(u)(u-u)+r
\end{equation*}
and so \(r=f(u)\)
\end{proof}

\begin{proposition}[]
\label{prop3.24}
If \(f(x)\in k[x]\), where \(k\) is a field, then \(a\) is a root of \(f(x)\) in \(k\)
if and only if \(x-a\) divides \(f(x)\) in \(k[x]\)
\end{proposition}

\begin{proof}
If \(a\) is a root of \(f(x)\) in \(k\), then \(f(a)=0\) and the lemma gives
\(f(x)=q(x)(x-a)\). 
\end{proof}

\begin{theorem}[]
\label{thm3.25}
Let \(k\) be a field and let \(f(x)\in k[x]\). If \(f(x)\) has degree \(n\), then
\(f(x)\) has at most \(n\) roots in \(k\)
\end{theorem}

\begin{proof}
We prove the statement by induction on \(n\ge 0\). If \(n=0\), then \(f(x)\) is a
nonzero constant, and so the number of its roots in \(k\) is zero. Now let
\(n>0\). If \(f(x)\) has no roots in \(k\), then we are done. Otherwise we may
assume that there is \(a\in k\) with \(a\) a root of \(f(x)\); hence by Proposition
\ref{prop3.24}
\begin{equation*}
f(x)=q(x)(x-a)
\end{equation*}
moreover, \(q(x)\in k[x]\) has degree \(n-1\). 
\end{proof}

\begin{examplle}[]
Theorem \ref{thm3.25} is not true for polynomials with coefficients in an
arbitrary commutative ring \(R\). For example, if \(R=\I_8\), then the quadratic
polynomial \(x^2-1\) has 4 roots: \([1],[3],[5],[7]\)
\end{examplle}


\begin{corollary}[]
Every \(n\)th root of unity in \(\C\)  is equal to
\begin{equation*}
e^{2\pi ik/n}=\cos\left(\frac{2\pi k}{n}\right)+i\sin\left(\frac{2\pi k}{n}\right)
\end{equation*}
where \(k=0,1,\dots,n-1\)
\end{corollary}

\begin{corollary}[]
Let \(k\) be an infinite field and let \(f(x)\) and \(g(x)\) be polynomials in
\(k[x]\). If \(f(x)\) and \(g(x)\) determine the same polynomial function, then
\(f(x)=g(x)\) 
\end{corollary}

\begin{proof}
If \(f(x)\neq g(x)\), then the polynomial \(h(x)=f(x)-g(x)\) is nonzero, so that
it has some degree, say \(n\). Now every element of \(k\) is a root of \(h(x)\);
since \(k\) is infinite, \(h(x)\) has more than \(n\) roots, a contradiction.
\end{proof}

\begin{theorem}[]
If \(k\) is a field and \(G\) is a finite subgroup of the multiplicative group
\(k^\times\)., then \(G\) is cyclic. In particular, if \(k\) itself is finite, then
\(k^\times\) is cyclic.
\end{theorem}

\begin{proof}
Let \(d\) be a divisor of \(\abs{G}\). If there are two subgroups of \(G\) of order
\(d\), say \(S\) and \(T\), then \(\abs{S\cup T}>d\). But each \(a\in S\cup T\)
satisfies \(a^d=1\) and hence it's a root of \(x^d-1\), a contradiction. Thus \(G\)
is cyclic, by Theorem \ref{thm2.86}.
\end{proof}

\begin{definition}[]
If \(k\) is a finite field, a generator of the cyclic group \(k^\times\) is
called a \textbf{primitive element} of \(k\)
\end{definition}

\begin{definition}[]
If \(f(x)\) and \(g(x)\) are polynomials in \(k[x]\), where \(k\) is a field, then a
\textbf{common divisor} is a polynomial \(c(x)\in k[x]\) with \(c(x)\mid f(x)\) and
\(c(x)\mid g(x)\). If \(f(x)\) and \(g(x)\) in \(k[x]\) are not both 0, define their
\textbf{greatest common divisor}, abbreviated gcd, to be the monic common divisor
having largest degree. If \(f(x)=0=g(x)\), define their \(\gcd=0\). The gcd of
\(f(x)\) and \(g(x)\) is often denoted by \((f,g)\)
\end{definition}

\begin{theorem}[]
If \(k\) is a field and \(f(x),g(x)\in k[x]\), then their \(\gcd d(x)\) is a
nonlinear combination of \(f(x)\) and \(g(x)\); that is there are \(s(x),t(x)\in
   k[x]\) with
\begin{equation*}
d(x)=s(x)f(x)+t(x)g(x)
\end{equation*}
\end{theorem}

\begin{corollary}[]
Let \(k\) be a field and let \(f(x),g(x)\in k[x]\). A monic common divisor
\(d(x)\) is the gcd if and only if \(d(x)\) is divisible by every common divisor
\end{corollary}

\index{irreducible}
\begin{definition}[]
An element \(p\) in a domain \(R\) is \textbf{irreducible} if \(p\) is neither 0 nor a unit
and in any factorization \(p=uv\) in \(R\), either \(u\) or \(v\) is a unit. Elements
\(a,b\in R\) are \textbf{associates} if there is a unit \(u\in R\) with \(b=ua\)
\end{definition}

For example, a prime \(p\) is irreducible in \(\Z\)

\begin{proposition}[]
If \(k\) is a field, then a polynomial \(p(x)\in k[x]\) is irreducible if and
only if \(\deg(p)=n\ge 1\) and there is no factorization in \(k[x]\) of the
form \(p(x)=g(x)h(x)\) in which both factors have degree smaller than \(n\)
\end{proposition}

\begin{proof}
We show fist that \(h(x)\in k[x]\) is a unit if and only if \(\deg(h)=0\). If
\(h(x)u(x)=1\), then \(\deg(h)+\deg(u)=\deg(1)=0\), we have \(\deg(h)=0\).
Conversely if \(\deg(h)=0\), then \(h(x)\) is a nonzero constant; that is, 
\(h\in k\); since \(k\) is a field, \(h\) has an inverse

If \(p(x)\) is irreducible, then its only factorization are of the form
\(p(x)=g(x)h(x)\) where \(g(x)\) or \(h(x)\) is a unit; that is, either \(\deg(g)=0\)
or \(\deg(h)=0\). 

Conversely, if \(p(x)\) is reducible, then it has factorization \(p(x)=g(x)h(x)\)
where neither \(g(x)\) nor \(h(x)\) is a unit;
\end{proof}

\begin{corollary}[]
\label{cor3.34}
Let \(k\) be a field and let \(f(x)\in k[x]\) be a quadratic or cubic polynomial.
Then \(f(x)\) is irreducible in \(k[x]\) if and only if \(f(x)\) does not have a root
in \(k\)
\end{corollary}

\begin{proof}
If \(f(x)=g(x)h(x)\), then \(\deg(f)=\deg(g)+\deg(h)\)
\end{proof}

\begin{examplle}[]
\begin{enumerate}
\item We determine the irreducible polynomials in \(\I_2[x]\) of small degree.

As always, the linear polynomials \(x\) and \(x+1\) are irreducible

There are four quadratics: \(x^2,x^2+x,x^2+1,x^2+x+1\)
\end{enumerate}
\end{examplle}

\begin{lemma}[]
Let \(k\) be a field, let \(p(x),f(x)\in k[x]\), and let \(d(x)=(p,f)\). If \(p(x)\)
is a monic irreducible polynomial, then
\begin{equation*}
d(x)=
\begin{cases}
1&\text{if }p(x)\nmid f(x)\\
p(x)&\text{if }p(x)\mid f(x)
\end{cases}
\end{equation*}
\end{lemma}

\begin{theorem}[Euclid's Lemma]
Let \(k\) be a field and let \(f(x),g(x)\in k[x]\). If \(p(x)\) is an irreducible
polynomial in \(k[x]\), and \(p(x)\mid f(x)g(x)\), then either
\begin{equation*}
p(x)\mid f(x)\hspace{0.5cm}\text{or}\hspace{0.5cm}p(x)\mid g(x)
\end{equation*}
More generally, if \(p(x)\mid f_1(x)\dots f_n(x))\), then \(p(x)\mid f_i(x)\)
for some \(i\)
\end{theorem}

\begin{proof}
Assume \(p\mid fg\) but that \(p\nmid f\). Since \(p\) is irreducible, \((p,f)=1\),
and so \(1=sp+tf\) for some polynomials \(s\) and \(t\). Therefore
\begin{equation*}
g=spg+tfg
\end{equation*}
and so \(p\mid g\)
\end{proof}

\begin{definition}[]
Two polynomials \(f(x),g(x)\in k[x]\) where \(k\) is a field, are called
\textbf{relatively prime} if their gcd is 1
\end{definition}

\begin{corollary}[]
Let \(f(x),g(x),h(x)\in k[x]\), where \(k\) is a field and let \(h(x)\) and
\(f(x)\) be relatively prime. If \(h(x)\mid f(x)g(x)\), then \(h(x)\mid g(x)\)
\end{corollary}

\begin{definition}[]
If \(k\) is a field, then a rational function \(f(x)/g(x)\in k(x)\) is in
\textbf{lowest terms} if \(f(x)\) and \(g(x)\) are relatively prime
\end{definition}


\begin{proposition}[]
If \(k\) is a field, every nonzero \(f(x)/g(x)\in k(x)\) can be put in lowest terms
\end{proposition}


\begin{theorem}[Euclidean Algorithm]
If \(k\) is a field and \(f(x),g(x)\in k[x]\), then there are algorithms for
computing \(\gcd(f,g)\) as well as for finding a pair of polynomials \(s(x)\)
and \(t(x)\) with 
\begin{equation*}
(f,g)=s(x)f(x)+t(x)g(x)
\end{equation*}
\end{theorem}

\begin{proof}
\begin{gather*}
g=q_1f+r_1\\
f=q_2r_1+r_2\\
r_1=q_3r_2+r_3\\
\vdots\\
r_{n-4}=q_{n-2}r_{n-3}+r_{n-2}\\
r_{n-3}=q_{n-1}r_{n-2}+r_{n-1}\\
r_{n-2}=q_nr_{n-1}+r_n\\
r_{n-1}=q_{n+1}r_n
\end{gather*}
Since the degrees of the remainders are strictly decreasing, this procedure
must stop after a finite number of steps. The claim is that \(d=r_n\) is the
gcd. If \(c\) is any common divisor of \(f\) and \(g\), then \(c\mid r_i\)  for every
\(i\). Also
\begin{align*}
r_n&=r_{n-2}-q_nr_{n-1}\\
&=r_{n-2}-q_n(r_{n-3}-q_{n-1}r_{n-2})\\
&=(1+q_{n-1})r_{n-2}-q_nr_{n-3}\\
&=(1+q_{n-1})(r_{n-4}-q_{n-2}r_{n-3})-q_nr_{n-3}\\
&=(1+q_{n-1})r_{n-4}-[(1+q_{n-1})q_{n-2}+q_n]r_{n-3}\\
&\vdots\\
&=sf+tg
\end{align*}
\end{proof}

\begin{corollary}[]
Let \(k\) be a subfield of a field \(K\), so that \(k[x]\) is a subring of \(K[x]\).
If \(f(x),g(x)\in k[x]\), then their gcd in \(k[x]\) is equal to their gcd in \(K[x]\)
\end{corollary}

\begin{proof}
The division algorithm in \(K[x]\) gives
\begin{equation*}
g(x)=Q(x)f(x)+R(x)
\end{equation*}
\(k[x]\) gives
\begin{equation*}
g(x)=q(x)f(x)+r(x)
\end{equation*}
and this also holds in \(K[x]\). So that uniqueness of quotient and remainder
gives \(Q(x)=q(x),R(x)=r(x)\).
\end{proof}

\begin{theorem}[Unique Factorization]
If \(k\) is a field, then every polynomial \(f(x)\in k[x]\) of degree \(\ge1\) is
a product of a nonzero constant and monic irreducibles. Moreover, if \(f(x)\)
has two such factorizations
\begin{equation*}
f(x)=ap_1(x)\dots p_m(x)\hspace{0.5cm}\text{and}\hspace{0.5cm}
f(x)=bq_1(x)\dots q_n(x)
\end{equation*}
then \(a=b,m=n\) and the \(q\)'s may be reindexed so that \(q_i=p_i\) for all \(i\)
\end{theorem}

\begin{proof}
We prove the existence of a factorization for a polynomial \(f(x)\) by
induction on \(\deg(f)\ge1\). If \(\deg(f)=1=\), then \(f(x)=ax+c=a(x+a^{-1}c)\).
As every linear polynomial, \(x+a^{-1}c\) is irreducible.

Assume now that \(\deg(f)\ge1\). If \(f(x)\) is irreducible and its leading
coefficient is \(a\), write \(f(x)=a(a^{-1}f(x))\); we are done. If \(f(x)\) is not
irreducible, then \(f(x)=g(x)h(x)\), where \(\deg(g)<\deg(f)\) and
\(\deg(h)<\deg(f)\). By the inductive hypothesis, 
\(g(x)=bp_1(x)\dots p_m(x)\) and \(h(x)=cq_1(x)\dots q_n(x)\). It follows
that 
\begin{equation*}
f(x)=(bc)p_1(x)\dots p_m(x)q_x(x)\dots q_n(x)
\end{equation*}

We now prove by induction on \(M=\max\{m,n\}\ge1\) if there is an
equation
\begin{equation*}
ap_1(x)\dots p_m(x)=bq_1(x)\dots q_n(x)
\end{equation*}
where \(a\) and \(b\) are nonzero constants and the \(p\)'s and \(q\)'s are monic
irreducibles. For the inductive step, \(p_m(x)\mid q_1(x)\dots q_n(x)\). By
Euclid's lemma, there is \(i\) with \(p_m(x)\mid q_i(x)\). But \(q_i(x)\) are
monic irreducible, so that \(q_i(x)=p_m(x)\). Canceling this factor we will use
inductive hypothesis
\end{proof}

Let \(k\) be a field and assume that there are \(a,r_1,\dots,r_n\in k\) with
\begin{equation*}
f(x)=a \displaystyle\prod_{i=1}^n(x-r_i)
\end{equation*}

If \(r_1,\dots,r_s\) where \(s\le n\) are the distinct roots of \(f(x)\), then
collecting terms gives
\begin{equation*}
f(x)=a(x-r_1)^{e_1}\dots (x-r_s)^{e_s}
\end{equation*}
where \(r_j\) are distinct and \(e_j\ge1\). We call \(e_j\) the \textbf{multiplicity} of the
root \(r_j\). 

\begin{theorem}[]
\label{thm3.43}
Let \(f(x)=a_0+a_1x+\dots+a_nx^n\in\Z[x]\subseteq\Q[x]\). Every rational root
\(r\) of \(f(x)\) has the form \(b/c\), where \(b\mid a_0\) and \(c\mid a_n\)
\end{theorem}

\begin{proof}
We may assume that \(r=b/c\) is in lowest form. 
\begin{gather*}
0=f(b/c)=a_0+a_1(b/c)+\dots+a_n(b/c)^n\\
0=a_0c^n+a_1bc^{n-1}+\dots+a_nb^n
\end{gather*}
Hence \(a_0c^n=b(-a_1c^{n-1}-\dots-a_nb^{n-1})\), that is \(b\mid a_0c^n\).
\end{proof}

\begin{definition}[]
A complex number \(\alpha\) is called an \textbf{algebraic integer} if \(\alpha\) is a root of a monic
\(f(x)\in\Z[x]\)
\end{definition}

\begin{corollary}[]
\label{cor3.44}
A rational number \(z\) that is an algebraic integer must lie in \(\Z\). More
precisely, if \(f(x)\in\Z[x]\subseteq\Q[x]\) is a monic polynomial, then
every rational root of \(f(x)\) is an integer that divides the constant term
\end{corollary}

\begin{proof}
\(a_n=1\) in Theorem \ref{thm3.43}
\end{proof}

For example, consider \(f(x)=x^3+4x^2-2x-1\in\Q[x]\). By Corollary
\ref{cor3.34}, this cubic is irreducible if and only if it has no rational
root. As \(f(x)\) is monic, the candidates for rational roots are \(\pm 1\),
for these are the only divisor of -1 in \(\Z\). Thus \(f(x)\) has no roots in
\(\Q\) and hence \(f(x)\) is irreducible in \(\Q[x]\)
\subsection{Homomorphisms}
\label{sec:org423167d}
\begin{definition}[]
If \(A\) and \(R\) are (commutative) rings, a \textbf{(ring) homomorphism} is a function
\(f:A\to R\) s.t.
\begin{enumerate}
\item \(f(1)=1\)
\item \(f(a+a')=f(a)+f(a')\)
\item \(f(aa')=f(a)f(a')\)
\end{enumerate}
\end{definition}


\begin{examplle}[]
\begin{enumerate}
\item Let \(R\) be a domain and let \(F=\Frac(R)\). 
\(R'=\{[a,1]:a\in R\}\subseteq F\), then the function \(f:R\to R'\) given
by \(f(a)=[a,1]\), is an isomorphism
\item Complex conjugation \(z=a+ib\mapsto\overline{z}=a-ib\) is an isomorphism
\(\C\to \C\).
\item Let \(R\) be a commutative ring, and let \(a\in R\). Define the \textbf{evaluation
homomorphism} \(e_a:R[x]\to R\) by \(e_a(f(x))=f(a)\).
\end{enumerate}
\end{examplle}

\begin{lemma}[]
If \(f:A\to R\) is a ring homomorphism, then for all \(a\in A\)
\begin{enumerate}
\item \(f(a^n)=f(a)^n\)
\item if \(a\) is a unit, then \(f(a)\) is a unit and \(f(a^{-1})=f(a)^{-1}\)
\item if \(f:A\to R\) is a ring homomorphism, then
\begin{equation*}
f(U(A))\le U(R)
\end{equation*}
where \(U(A)\) is the group of units of \(A\); if \(f\) is an isomorphism,
then
\begin{equation*}
U(A)\cong U(R)
\end{equation*}
\end{enumerate}
\end{lemma}

\begin{proposition}[]
If \(R\) and \(S\) are commutative rings and \(\varphi:R\to S\) is a ring
homomorphism, then there is a ring homomorphism \(\varphi^*:R[x]\to S[x]\)
given by
\begin{equation*}
\varphi^*:r_0+r_1x+r_2x^2+\dots\mapsto\varphi(r_0)+\varphi(r_1)x+
\varphi(r_2)x^2+\dots
\end{equation*}
\end{proposition}

\begin{definition}[]
If \(f:A\to R\) is a ring homomorphism, then its \textbf{kernel} is
\begin{equation*}
\ker f=\{a\in A:f(a)=0\}
\end{equation*}
and its \textbf{image} is 
\begin{equation*}
\im f=\{r\in R:\exists a\in R\e r=f(a)\}
\end{equation*}
\end{definition}

The kernel of a group homomorphism is not merely a subgroup; it is a \textbf{normal}
subgroup. Similarly, the kernel of a ring homomorphism is almost a subring
(\(1\not\in\ker f\))
and is closed under multiplication.

\index{ideal}
\begin{definition}[]
An \textbf{ideal} in a commutative ring \(R\) is a subset \(I\) of \(R\) s.t. 
\begin{enumerate}
\item \(0\in I\)
\item if \(a,b\in I\), then \(a+b\in I\)
\item if \(a\in I\) and \(r\in R\), then \(ra\in I\)
\end{enumerate}
\end{definition}

An ideal \(I\neq R\) is called a \textbf{proper ideal}

\begin{examplle}[]
If \(b_1,\dots,b_n\in R\), then the set of all linear combinations
\begin{equation*}
I=\{r_1b_1+\dots+r_nb_n:r_i\in R\}
\end{equation*}
is an ideal in \(R\). We write \(I=(b_1,\dots,b_n)\) in this case and we call
\(I\) the \textbf{ideal generated by} \(b_1,\dots,b_n\). In particular, if \(n=1\), then
\begin{equation*}
I=(b)=\{rb:r\in R\}
\end{equation*}
is an ideal in \(R\); \((b)\) consists of all the multiplies of \(b\) and it is
called the \textbf{principal ideal} generated by \(b\). Notice that \(R\) and \(\{0\}\) are
always principal ideals: \(R=(1),\{0\}=(0)\)
\end{examplle}


\begin{proposition}[]
If \(f:A\to R\) is a ring homomorphism, then \(\ker f\) is an ideal in \(A\)
and \(\im f\) is a subring of \(R\). Moreover, if \(A\) and \(R\) are not zero rings,
then \(\ker f\) is a proper ideal.
\end{proposition}

\begin{examplle}[]
\begin{enumerate}
\item If an ideal \(I\) in a commutative ring \(R\) contains 1, then \(I=R\)
\item it follows from 1 that if \(R\) is a field, then the only ideals are \(\{0\}\)
and \(R\)
\end{enumerate}
\end{examplle}

\begin{proposition}[]
A ring homomorphism \(f:A \to R\) is an injection if and only if \(\ker f=\{0\}\)
\end{proposition}

\begin{corollary}[]
If \(f:k\to R\) is a ring homomorphism, where \(k\) is a field and \(R\) is not the
zero ring, then \(f\) is an injection
\end{corollary}

\begin{proof}
the only proper ideal in \(k\) is \(\{0\}\)
\end{proof}

\begin{theorem}[]
If \(k\) is a field, then every ideal \(I\) in \(k[x]\) is a principal ideal.
Moreover, if \(I\neq\{0\}\), there is a monic polynomial that generates \(I\)
\end{theorem}

\begin{proof}
If \(k\) is a field, then \(k[x]\) is an example of a \textbf{euclidean ring}. Follows
Theorem \ref{thm3.60}
\end{proof}

\index{principal ideal domain}

\begin{definition}[]
A domain \(R\) is a \textbf{principal ideal domain} (PID) if every ideal in \(R\) is a principal
ideal. 
\end{definition}

\begin{examplle}[]
\begin{enumerate}
\item The ring of integers is a PID
\item Every field is a PID
\item If \(k\) is a field, then the polynomial ring \(k[x]\) is a PID
\item There are rings other than \(\Z\) and \(k[x]\) where \(k\) is a field that
have a division algorithm; they are called \textbf{euclidean rings}.
\end{enumerate}
\end{examplle}

\begin{examplle}[]
Let \(R=\Z[x]\). The set of all polynomials with even constant term is an
ideal in \(\Z[x]\). We show that \(I\) is not a principal ideal.

Suppose there is \(d(x)\in\Z[x]\) with \(I=(d(x))\). The constant \(2\in I\),
so that there is \(f(x)\in\Z[x]\) with \(2=d(x)f(x)\). We have
\(0=\deg(2)=\deg(d)+\deg(f)\). The candidates for \(d(x)\) are \(\pm1\) and
\(\pm2\). Suppose \(d(x)=\pm2\); since \(x\in I\), there is \(g(x)\in\Z[x]\) with
\(x=d(x)g(x)=\pm2g(x)\). But every coefficients on the right side is even.
This contradiction gives \(d(x)=\pm1\). Hence \(I=\Z[x]\), another
contradiction. Therefore \(I\) is not a principal ideal.
\end{examplle}

\begin{definition}[]
An element \(\delta\) in a commutative ring \(R\) is a \textbf{greatest commmon divisor}, gcd, of
elements \(\alpha,\beta \in R\) if
\begin{enumerate}
\item \(\delta\) is a common divisor of \(\alpha\) and \(\beta\)
\item if \(\gamma\) is any common divisor of \(\alpha\) and \(\beta\), then \(\gamma\mid\delta\)
\end{enumerate}
\end{definition}

\begin{remark}
Let \(R\) be a PID and let \(\pi,\alpha\in R\) with \(\pi\) irreducible. A gcd \(\delta\) of
\(\pi\) and \(\alpha\) is a divisor of \(\pi\). Hence \(\pi=\delta\epsilon\). And
irreducibility of \(\pi\) forces either \(\delta\) or \(\epsilon\) to be a unit. Now
\(\alpha=\delta\beta\). If \(\delta\) is not a unit, then \(\epsilon\) is a unit and so 
\begin{equation*}
\alpha=\delta\beta=\pi\epsilon^{-1}\beta
\end{equation*}
that is \(\pi\mid\alpha\). We conclude that if \(\pi\nmid\alpha\) then \(\delta\) is a
unit; that is 1 is a gcd of \(\pi\) and \(\alpha\)
\end{remark}

\begin{theorem}[]
\label{thm3.57}
Let \(R\) be a PID
\begin{enumerate}
\item Every \(\alpha,\beta\in R\) has a gcd, \(\delta\), which is a linear combination of
\(\alpha\) and \(\beta\)
\begin{equation*}
\delta=\sigma\alpha+\tau\beta
\end{equation*}
\item If an irreducible element \(\pi\in R\) divides a product \(\alpha\beta\), then
either \(\pi\mid\alpha\) or \(\pi\mid\beta\)
\end{enumerate}
\end{theorem}
\begin{proof}
\begin{enumerate}
\item We may assume that at least one of \(\alpha\) and \(\beta\) is not zero. Consider the set \(I\)
of all the linear combinations
\begin{equation*}
I=\{\sigma\alpha+\tau\beta:\sigma,\tau\in R\}
\end{equation*}
\(I\) is an ideal and so there is \(\delta\in I\) with \(I=(\delta)\); we claim
that \(\delta\) is gcd of \(\alpha\) and \(\beta\)
\item If \(\pi\nmid\alpha\), then the remark says that 1 is a gcd of \(\pi\) and \(\alpha\).
Thus \(1=\sigma\pi+\tau\alpha\) and so
\begin{equation*}
\beta=\sigma\pi\beta+\tau\alpha\beta
\end{equation*}
Since \(\pi\mid\alpha\beta\), it follows that \(\pi\mid\beta\)
\end{enumerate}
\end{proof}

\index{least common multiple}
\begin{definition}[]
If \(f\) and \(g\) are elements in a commutative ring \(R\), then a \textbf{common multiple}
is an element \(m\in R\) with \(f\mid m\) and \(g\mid m\). If \(f\) and \(g\) in
\(R\) are not both 0, define their \textbf{least common multiple}, abbreviated lcm.
\end{definition}

\begin{exercise}
\label{ex3.47}
\begin{enumerate}
\item If \(A\) and \(R\) are domains and \(\varphi:A\to R\) is a ring homomorphism,
prove that 
\begin{equation*}
[a,b]\to [\varphi(a),\varphi(b)]
\end{equation*}
is a ring homomorphism \(\Frac(A)\to\Frac(B)\)
\item Prove that if a field \(k\) contains an isomorphic copy of \(\Z\) as a
subring, then \(k\) must contain an isomorphic copy of \(\Q\)
\item Let \(R\) be a domain and let \(\varphi:R\to k\) be an injective ring
homomorphism, where \(k\) is a field. Prove that there exists a unique ring
homomorphism \(\Phi:\Frac(R)\to k\) extending \(\varphi\) ; that is, \(\Phi|R=\varphi\)
\end{enumerate}
\end{exercise}

\begin{proof}
\begin{enumerate}
\item \begin{align*}
f([1,1])&=[1,1]\\
f([a,b]+[c,d])&=f([ad+bc,bd])=[\varphi(ad+bc),\varphi(bd)]\\
&=[\varphi(a)\varphi(d)+\varphi(b)\varphi(c),\varphi(b)\varphi(d)]\\
&=[\varphi(a),\varphi(b)]+[\varphi(c),\varphi(d)]\\
&=f([a,b])+f([c,d])\\
f([a,b][c,d])&=f([ac,bd])=[\varphi(ac),\varphi(bd)]=[\varphi(a)\varphi(c),
\varphi(b)\varphi(d)]\\&=f([a,b])f([c,d])
\end{align*}
\item Suppose \(k'\le k\) and \(k'\cong\Z\), then \(\Frac(k')\cong\Frac(\Z)\).
Obviously.
\item \(k\) is a field and has inverse.
\end{enumerate}
\end{proof}

\subsection{Euclidean Rings}
\label{sec:org5a72b46}
\index{degree function} \index{euclidean ring}
\begin{definition}[]
A \textbf{euclidean ring} is a domain that is equipped with a function
\begin{equation*}
\partial:R-\{0\}\to\N
\end{equation*}
called a \textbf{degree function}, s.t.
\begin{enumerate}
\item \(\partial(f)\le\partial(fg)\) for all \(f,g\in R\) with \(f,g\neq0\)
\item for all \(f,g\in R\) with \(f\neq0\), there exists \(q,r\in R\) with
\begin{equation*}
g=qf+r
\end{equation*}
where either \(r=0\) or \(\partial(r)<\partial(f)\)
\end{enumerate}
\end{definition}

\begin{examplle}[]
\begin{enumerate}
\item The integers \(\Z\) is a euclidean ring with the degree function
\(\partial(m)=\abs{m}\). In \(\Z\) we have
\begin{equation*}
\partial(mn)=\abs{mn}=\abs{m}\abs{n}=\partial(m)\partial(n)
\end{equation*}
\item when \(k\) is a field, the domain \(k[x]\) is a euclidean ring with degree
function the usual degree of a nonzero polynomial. In \(k[x]\), we have
\begin{align*}
\partial(fg)=\deg(fg)=\deg(f)+\deg(g)=\partial(f)+\partial(g)
\end{align*}
If a degree function is multiplicative, then \(\partial\) is called a \textbf{norm}
\item The Gaussian integers \(\Z[i]\) form a euclidean ring whose degree
function 
\begin{equation*}
\partial(a+bi)=a^2+b^2
\end{equation*}
is a norm. One reason to show that \(\Z[i]\) is a euclidean ring is that
it is a PID, and hence it has unique factorization of its elements of into
products of irreducibles.

\(\partial\) is a multiplicative degree function for 
\begin{equation*}
\partial(\alpha\beta)=\alpha\beta\overline{\alpha\beta}=\alpha\beta\overline{\alpha}
\overline{\beta}=\alpha\overline{\alpha}\beta\overline{\beta}= \partial(\alpha)\partial(\beta)
\end{equation*}

Let us show that \(\partial\) satisfies the second desired property. Given
\(\alpha,\beta\in\Z[i]\) with \(\beta\neq0\), regard \(\alpha/\beta\) as an
element of \(\C\). Rationalizing the denominator gives
\(\alpha/\beta=\alpha\overline{\beta}/\beta\overline{\beta}=\alpha
      \overline{\beta}/\partial{\beta}\), so that
\begin{equation*}
a/\beta=x+yi
\end{equation*}
where \(x,y\in\Q\). Write \(x=a+u\) and \(y=b+v\), where \(a,b\in\Z\)are
integers closest to \(x\) and \(y\), respectively; thus
\(\abs{u},\abs{v}\le1/2\). It follows that
\begin{equation*}
\alpha=\beta(a+bi)+\beta(u+vi)
\end{equation*}
Notice that \(\beta(u+vi)\in\Z[i]\). Finally we have
\begin{equation*}
\partial(\beta(u+vi))=\partial(\beta)\partial(u+vi)<\partial(\beta)
\end{equation*}
And so \(\Z[i]\) is a euclidean ring whose degree function is a norm

Note that quotients and remainders are not unique because of the choice
\end{enumerate}
\end{examplle}

\begin{theorem}[]
\label{thm3.60}
Every euclidean ring \(R\) is a PID
\end{theorem}

\begin{proof}
Let \(I\) be an ideal in \(R\). If \(I\neq\{0\}\), by the least integer axiom, the
set of all degrees of nonzero elements in \(I\) has a smallest element, say
\(n\); choose \(d\in I\) with \(\partial(d)=n\). Clearly \((d)\subseteq I\). For any
\(a\in I\), then there are \(q,r\in R\) with \(a=qd+r\), where either \(r=0\)
or \(\partial(r)<\partial(a)\). But \(r=a-qd\in I\) and so \(d\) having the least degree
implies that \(r=0\). Hence \(a=qd\in(d)\).
\end{proof}

\begin{corollary}[]
The ring of Gaussian integers \(\Z[i]\) is a PID
\end{corollary}

\begin{definition}[]
An element \(u\) in a domain \(R\) is a \textbf{universal side divisor} if \(u\) is not a unit
and for every \(x\in R\), either \(u\mid x\) or there is a unit \(z\in R\) with \(u\mid(x+z)\)
\end{definition}


\begin{proposition}[]
If \(R\) is a euclidean ring but not a field, then \(R\) has a universal side divisor
\end{proposition}

\begin{proof}
Define 
\begin{equation*}
S=\{\partial(v):v\neq 0\text{ and }v\text{ is not a unit}\}
\end{equation*}
where \(\partial\) is the degree function on \(R\). Since \(R\) is not a field, \(S\) is a
nonempty subset of the natural number. By the least integer axiom, \(S\) has a
smallest element, say, \(\partial(u)\). We claim that \(u\) is a universal side
divisor. If \(x\in R\), then there are \(q,r\) with \(x=qu+r\).
\end{proof}

\begin{proposition}[]
\label{prop3.64}
\begin{enumerate}
\item Let \(R\) be a euclidean ring \(R\) that is not a field. If the degree function
\(\partial\) is a norm, then \(\alpha\) is a unit if and only if \(\partial(\alpha)=1\)
\item Let \(R\) be a euclidean ring \(R\) that is not a field. If the degree function
\(\partial\) is a norm and if \(\partial(a)=p\), where \(p\) is a prime, then \(\alpha\) is not
irreducible
\item The only units in the ring \(\Z[i]\) of Gaussian integers are \(\pm1\) and
\(\pm i\)
\end{enumerate}
\end{proposition}

\begin{proof}
\begin{enumerate}
\item Since \(1^2=1\), we have \(\partial(1)^2=\partial(1)\), so that \(\partial(1)=0\) or
\(\partial(1)=1\). If \(\partial(1)=0\), then \(\partial(a)=\partial(1a)=0\). But \(R\) is not a
field, and so \(\partial\) is not identically zero. We conclude that \(\partial(1)=1\)

If \(a\in R\) is a unit, then there is \(\beta\in R\) with
\(\alpha\beta=1\). Therefore \(\partial(\alpha)\partial(\beta)=1\) and hence \(\partial(\alpha)=1\)

For the converse, we begin by showing that there is no element \(\beta\in
      R\) with 
\(\partial(\beta)=0\). If such an element exists, the division algorithms gives 
\(1=q\beta+r\) and so \(\partial(r)=0\). That is \(\beta\) is a unit, then \(\partial(\beta)=1\), a
contradiction

Assume now that \(\partial(\alpha)=1\). The division algorithm gives
\begin{equation*}
\alpha=q\alpha^2+r
\end{equation*}
As \(\partial(\alpha^2)=\partial(\alpha)^2=1\), \(r=0\) or \(\partial(r)=0\), which would not occur.
Hence \(r=0\) and \(\alpha=q\alpha^2\). It follows that \(1=q\alpha\), and
so \(\alpha\) is a unit
\item If on the contrary, \(\alpha=\beta\gamma\), where neither \(\beta\) or
\(\gamma\) is a unit, then \(p=\partial(\alpha)=\partial(\beta)\partial(\gamma)\).
\item If \(\alpha=a+bi\in\Z[i]\) is a unit, then \(1=\partial(\alpha)=a^2+b^2\).
\end{enumerate}
\end{proof}

\begin{lemma}[]
If \(p\) is a prime and \(p\equiv 1\mod 4\), then there is an integer \(m\) with
\begin{equation*}
 m^2\equiv -1\mod p
\end{equation*}
\end{lemma}
\begin{proof}
If \(G=(\I_p)^\times\) is the multiplicative group of nonzero elements in
\(\I_p\), then \(\abs{G}=p-1\equiv 0\mod 4\). By Proposition \ref{prop2.78},
\(G\) contains a subgroup \(S\) of order \(4\). By Exercise \ref{ex2.36} either \(S\)
is cyclic or \(a^2=1\) for all \(a\in S\). Since \(\I_p\) is a field,
however, it cannot contain four roots of the quadratic \(x^2-1\). Therefore,
\(S\) is cyclic, say \(S=\la[m]\ra\) where \([m]\) is the congruence class of
\(m\) mod \(p\). Since \([m]\) has order 4, we have \([m^4]=[1],[m^2]\neq1\), and
so \([m^2]=[-1]\) for \([-1]\)  is the unique element in \(S\) of order 2.
Therefore, \(m^2\equiv -1\mod p\)
\end{proof}

\begin{theorem}[Fermat's Two-Squares Theorem]
\label{thm3.66}
An odd prime \(p\) is a sum of two squares,
\begin{equation*}
p=a^2+b^2
\end{equation*}
where \(a\) and \(b\) are integers if and only if \(p\equiv 1\mod 4\)
\end{theorem}

\begin{proof}
Assume that \(p=a^2+b^2\). Since \(p\) is odd, \(a\) and \(b\) have different
parity; say, \(a\) is even and \(b\) is odd. Hence \(a=2m\) and \(b=2n+1\) and
\begin{equation*}
p=a^2+b^2=4m^2+4n^2+4n+1\equiv1\mod 4
\end{equation*}
Conversely, assume that \(p\equiv1\mod4\). By the lemma, there is an integer
\(m\) s.t.
\begin{equation*}
p\mid(m^2+1)
\end{equation*}
In \(\Z[i]\), there is a factorization \(m^2+1=(m+i)(m-i)\) and so 
\begin{equation*}
p\mid(m+i)(m-i) \text{ in }\Z[i]
\end{equation*}
If \(p\mid(m\pm i)\) in \(\Z[i]\), then there are integers \(u\) and \(v\) with
\(m\pm i=p(u+iv)\). Comparing the imaginary parts gives \(pv=1\), a
contradiction. We conclude that \(p\) does not satisfy the analog of Euclid's
lemma in Theorem \ref{thm3.57}; it follows from Exercise \ref{ex3.62} that  \(p\)
is not irreducible. Hence there is a factorization
\begin{equation*}
p=\alpha\beta\in\Z[i]
\end{equation*}
Therefore, taking norms gives an equation in \(\Z\)
\begin{align*}
p^2&=\partial(p)=\partial(\alpha\beta)\\
&=\partial(\alpha)\partial(\beta)=(a^2+b^2)(c^2+d^2)
\end{align*}
By Proposition \ref{prop3.64}, the only units in \(\Z[i]\) are \(\pm1\) and
\(\pm i\), so that any nonzero Gaussian integers that is not a unit has a norm
\(>1\); therefore \(a^2+b^2\neq1\) and \(c^2+d^2\neq1\). Euclid's lemma now
gives \(p\mid a^2+b^2\)  or \(p\mid c^2+d^2\); then fundamental theorem of arithmetic
gives \(p=a^2+b^2\).
\end{proof}

\begin{lemma}[]
\label{lemma3.67}
If \(\alpha\in\Z[i]\) is irreducible, then there is a unique prime number \(p\)
with \(a\mid p\) in \(\Z[i]\)
\end{lemma}

\begin{proof}
Since \(\partial(\alpha)=\alpha\overline{\alpha}\), we have \(\alpha\mid\partial(\alpha)\). Now
\(\partial(\alpha)=p_1\dots p_n\). If \(\alpha\mid q\) for some prime \(q\neq p_i\), then 
\(\alpha\mid(q,p_i)=1\), forcing \(\alpha\) to be unit. A contradiction
\end{proof}

\begin{proposition}[]
Let \(\alpha=a+bi\in\Z[i]\) be neither 0 nor a unit. Then \(\alpha\) is irreducible if
and only if
\begin{enumerate}
\item \(\alpha\) is an associate of a prime \(p\) in \(\Z\) of the form \(p=4m+3\); or
\item \(\alpha\) is an associate of \(1+i\) or its conjugate; or
\item \(\partial(\alpha)=a^2+b^2\) is a prime in \(\Z\) of the form \(4m+1\)
\end{enumerate}
\end{proposition}

\begin{proof}
By Lemma \ref{lemma3.67} there is a unique prime number \(p\) divides by \(\alpha\) in
\(\Z[i]\). Since \(\alpha\mid p\), we have \(\partial(\alpha)\mid\partial(p)=p^2\) in
\(\Z\), so that \(\partial(\alpha)=p\) or \(\partial(\alpha)=p^2\).
\begin{enumerate}
\item \(p\equiv 3\mod4\)

By Theorem \ref{thm3.66} \(p^2=a^2+b^2\). We have \(\alpha\beta=p\) and
\(\partial(\alpha)\partial(\beta)=\partial(p)\). Therefore, \(p^2\partial(\beta)=p^2\) and
\(\partial(\beta)=1\). Thus \(\beta\) is a unit by Proposition \ref{prop3.64} and \(p\) is
irreducible.
\item \(p\equiv 2\mod 4\)

\(a^2+b^2=2\)
\item \(p\equiv1\mod4\)
If \(\partial(\alpha)=p^2\), \(\beta\) is a unit as case 1. Now 
\(\alpha\overline{\alpha}=p^2=(\alpha\beta)^2\), so that
\(\overline{\alpha}=\alpha\beta^2\) but \(\beta^2=\pm1\) by Proposition \ref{prop3.64}
\end{enumerate}
\end{proof}

\begin{exercise}
\label{ex3.62}
If \(R\) is a euclidean ring and \(\pi\in R\) is irreducible, prove that 
\(\pi\mid\alpha\beta\) implies \(\pi\mid\alpha\) or \(\pi\mid\beta\)
\end{exercise}

\begin{proof}
\(R\) is PID and follow Theorem \ref{thm3.57}.
\end{proof}
\subsection{Linear Algebra}
\label{sec:org0181a12}
\subsubsection{Vector Spaces}
\label{sec:org2198e9f}
\index{vector space} \index{vector}

\begin{definition}[]
If \(k\) is a field, then a \textbf{vector space over} \(k\) is an (additive) abelian
group \(V\) equipped with a \textbf{scalar multiplication}; there is a function 
\(k\times V\to V\), denoted by \((a,v)\mapsto av\) s.t. for all \(a,b,1\in
    k\) and all 
\(u,v\in V\)
\begin{enumerate}
\item \(a(u+v)=au+av\)
\item \((a+b)v=av+bv\)
\item \((abv)=a(bv)\)
\item \(1v=v\)
\end{enumerate}


The elements of \(V\) are called \textbf{vectors} and the elements of \(k\) are called \textbf{scalars}
\end{definition}

\begin{examplle}[]
\begin{enumerate}
\item Euclidean space \(V=\R^n\) is a vector space over \(\R\)
\item If \(R\) is a commutative ring and \(k\) is a subring that is a field, then
\(R\) is a vector space over \(k\)

For example, if \(k\) is a field, then the polynomial ring \(R=k[x]\) is a
vector space over \(k\).
\end{enumerate}
\end{examplle}

\index{subspace}
\begin{definition}[]
If \(V\) is a vector space over a field \(k\), then a \textbf{subspace} of \(V\) is a
subset \(U\) of \(V\) s.t.
\begin{enumerate}
\item \(0\in U\)
\item \(u,u'\in U\) imply \(u+u'\in U\)
\item \(u\in U\) and \(a\in k\) imply \(au\in U\)
\end{enumerate}
\end{definition}


\index{k-linear combination} \index{span}
\begin{definition}[]
Let \(V\) be a vector space over a field \(k\). A \textbf{\(k\)-linear combination} of a
list \(v_1,\dots,v_n\) in \(V\) is a vector of \(v\) of the form
\begin{equation*}
v=a_1v_1+\dots+a_nv_n
\end{equation*}
where \(a_i\in k\) for all \(i\)
\end{definition}

\begin{definition}[]
If \(X=v_1,\dots,v_m\) is a list in a vector space \(V\), then
\begin{equation*}
\la v_1,\dots,v_m\ra
\end{equation*}
the set of all the \(k\)-linear combinations of \(v_1,\dots,v_m\) is called
the \textbf{subspace spanned by \(X\)}. We also say that \(v_1,\dots,v_m\) \textbf{spans}
\(\la v_1,\dots,v_m\ra\)
\end{definition}

\begin{lemma}[]
Let \(V\) be a vector space over a field \(k\)
\begin{enumerate}
\item Every intersection of subspaces of \(V\) is itself a subspace
\item If \(X=v_1,\dots,v_m\) is a list in \(V\), then the intersection of all the
subspaces of \(V\) containing \(X\) is \(\la v_1,\dots,v_m\ra\), and so 
\(\la v_1,\dots,v_m\ra\) is the \textbf{smallest subspace}
\end{enumerate}
\end{lemma}

\begin{examplle}[]
Let \(V=\R^2\), let \(e_1=(1,0)\) and let \(e_2=(0,1)\). then \(V=\la
       e_1,e_2\ra\) 
\end{examplle}

\begin{definition}[]
A vector space \(V\) is called \textbf{finite-dimensional} if it is spanned by a finite
list; otherwise \(V\) is called \textbf{infinite-dimensional}
\end{definition}

\textbf{Notation}. If \(v_1,\dots,v_m\) is a list, then
\(v_1,\dots,\what{v_i},\dots,v_m\) is the shorter list with \(v_i\) deleted

\begin{proposition}[]
\label{prop3.73}
If \(V\) is a vector space, then the following conditions on a list
\(X=v_1,\dots,v_m\) spanning \(V\) are equivalent
\begin{enumerate}
\item \(X\) is not a shortest spanning list
\item some \(v_i\) is in the subspace spanned by the others; that is
\begin{equation*}
v_i\in\la v_1,\dots,\what{v_i},\dots,v_m\ra
\end{equation*}
\item there are scalars \(a_1,\dots,a_m\) not all zero with
\begin{equation*}
\displaystyle\sum_{l=1}^ma_lv_l=0
\end{equation*}
\end{enumerate}
\end{proposition}


\index{linearly dependent}
\begin{definition}[]
A list \(X=v_1,\dots,v_m\) in a vector space \(V\) is \textbf{linearly dependent} if
there are scalars \(a_1,\dots,a_m\) not all zero, with
\(\sum_{l=1}^ma_lv_l=0\); otherwise \(X\) is called \textbf{linearly independent}
\end{definition}


\begin{corollary}[]
If \(X=v_1,\dots,v_m\) is a list spanning a vector space \(V\), then \(X\) is a
shortest spanning list if and only if \(X\) is linearly independent
\end{corollary}


\index{basis}
\begin{definition}[]
A \textbf{basis} of a vector space \(V\) is a linearly independent list that spans \(V\)
\end{definition}

\begin{proposition}[]
Let \(X=v_1,\dots,v_n\) be a list in a vector space \(V\) over a field \(k\).
Then \(X\) is a basis if and only if each vector in \(V\) has a unique
expression as a \(k\)-linear combination of vectors in \(X\)
\end{proposition}

\begin{proof}
If a vector \(v=\sum a_iv_i=\sum b_iv_i\), then \(\sum(a_i-b_i)=0\)
\end{proof}

\index{coordinate set}
\begin{definition}[]
If \(X=v_1,\dots,v_n\) is a basis of a vector space \(V\) and if \(v\in V\),
then there are unique scalars \(a_1,\dots,a_n\) with
\(v=\sum_{i=1}^na_iv_i\). The \(n\)-tuple \((a_1,\dots,a_n)\) is called the
\textbf{coordinate set} of a vector \(v\in V\) relative to the basis \(X\)
\end{definition}

\begin{theorem}[]
Every finite-dimensional vector space \(V\) has a basis
\end{theorem}

\begin{proof}
A finite spanning list \(X\) exists, since \(V\) is finite-dimensional. If it
is  linearly independent, it is a basis; if not, \(X\) can be shortened to a
spanning list \(X'\) by Proposition \ref{prop3.73}
\end{proof}

\begin{lemma}[]
Let \(u_1,\dots,u_n\) be elements in a vector space \(V\), and let
\(v_1,\dots,v_m\in\la u_1,\dots,u_n\ra\). If \(m>n\), then \(v_1,\dots,v_m\)
is a linearly dependent list
\end{lemma}

\begin{proof}
Induction on \(n\ge 1\)

\emph{Base step}. If \(n=1\)

\emph{Inductive step}. For \(i=1,\dots,m\)
\begin{equation*}
v_i=a_{i1}u_1+\dots+a_{in}u_n
\end{equation*}
We may assume that some \(a_{i1}\neq0\) otherwise 
\(v_1,\dots,v_m\in\la u_2,\dots,u_n\ra\) , and the inductive hypothesis
applies. Changing notation 
if necessary we may assume \(a_{11}\neq 0\). For each \(i\ge 2\), define
\begin{equation*}
v_i'=v_i-a_{i1}a_{11}^{-1}v_1\in\la u_2,\dots,u_n\ra
\end{equation*}

Since \(m-1>n-1\)
\end{proof}

\begin{corollary}[]
A homogeneous system of linear equations, over a field \(k\), with more
unknowns than equations has a nontrivial solution.
\end{corollary}

\begin{proof}
An \(n\)-tuple \((\beta_1,\dots,\beta_n)\) is a solution of a system
\begin{gather*}
\alpha_{11}x_1+\dots+\alpha_{1n}x_n=0\\
\vdots\quad\vdots\quad\vdots\\
\alpha_{m1}x_1+\dots+\alpha_{mn}x_n=0
\end{gather*}
if \(\alpha_{i1}\beta_1+\dots+\alpha_{in}\beta_n=0\) for all \(i\). In other
words, if \(c_1,\dots,c_n\) are the columns of the \(m\times n\) coefficient
matrix \(A=[\alpha_{ij}]\), then 
\begin{equation*}
\beta_1c_1+\dots+\beta_nc_n=0
\end{equation*}
Note that \(c_i\in k^m\). Now \(k^m\) can be spanned by \(m\) vectors. Since
\(n>m\), \(c_1,\dots,c_n\) is linearly dependent
\end{proof}

\begin{theorem}[Invariance of Dimension]
If \(X=x_1,\dots,x_n\) and \(Y=y_1,\dots,y_m\) are bases of a vector space
\(V\), then \(m=n\)
\end{theorem}

\begin{proof}
Otherwise \(n<m\) or \(m<n\)
\end{proof}

\index{dimension}
\begin{definition}[]
If \(V\) is a finite-dimensional vector space over a field \(k\), then its
\textbf{dimension} denoted by \(\dim_k(V)\) or \(\dim(V)\), is the number of elements
in a basis of \(V\)
\end{definition}

\begin{examplle}[]
Let \(X=\{x_1,\dots,x_n\}\) be a finite set. Define
\begin{equation*}
k^X=\{\text{functions }f:X\to k\}
\end{equation*}
Now \(k^X\) is a vector space if we define addition
\begin{equation*}
f+f':x\mapsto f(x)+f'(x)
\end{equation*}
and scalar multiplication for \(a\in k\)
\begin{equation*}
af:x\mapsto af(x)
\end{equation*}

It's easy to check that the set of \(n\) functions of the form \(f_x\),
where \(x\in X\) defined by
\begin{equation*}
f_x(y)=
\begin{cases}
1&\text{if }y=x\\
0
\end{cases}
\end{equation*}
form a basis.

An \(n\)-tuple \((a_1,\dots,a_n)\) is really a function \(f:\{1,\dots,n\}\to
    k\) with \(f(i)=a_i\)
\end{examplle}

\begin{lemma}[]
\label{lemma3.73}
If \(X=v_1,\dots,v_n\) is a linearly dependent list of vectors in a vector
space \(V\), then there exists \(v_r\) with \(r\ge1\) with \(v_r\in\la v_1,\dots,v_{r-1}\ra\)
\end{lemma}

\begin{lemma}[Exchange Lemma]
\label{lemma3.74}
If \(X=x_1,\dots,x_m\) is a basis of a vector space \(V\) and
\(y_1,\dots,y_n\) is a linearly independent subset of \(V\), then \(n\le m\)
\end{lemma}

\begin{proof}
We begin by showing that one of the \(x\)'s in \(X\) can be replaced by
\(y_n\) so that the new list still spans \(V\). Now \(y_n\in\la X\ra\), so
that the list
\begin{equation*}
y_n,x_1,\dots,x_m
\end{equation*}
is linearly dependent. By Lemma \ref{lemma3.73} there is some \(i\) with 
\(x_i=ay_n+\sum_{j<i}a_jx_j\). Throwing out \(x_i\) and replacing it by
\(y_n\) gives a spanning list 
\begin{equation*}
X'=y_n,x_1,\dots,\what{x_i},\dots,x_m
\end{equation*}

Now repeat this argument for the spanning list 
\(y_{n-1},y_n,x_1,\dots,\what{x_i},\dots,x_m\). It follows that the
disposable vector must be one of the remaining \(x\)'s, say \(x_l\). After
throwing out \(x_l\), we have a new spanning list \(X''\). If \(n>m\), then
this procedure ends with a spanning list consisting of \(m\) \(y\)'s and no
\(x\)'. Thus a proper sublist of \(Y=y_1,\dots,y_n\) spans \(V\), a contradiction
\end{proof}

\begin{theorem}[Invariance of Dimension]
If \(X=x_1,\dots,x_n\) and \(Y=y_1,\dots,y_m\) are bases of a vector space
\(V\), then \(m=n\)
\end{theorem}

\begin{proof}
By Lemma \ref{lemma3.74}, \(n\le m\) and \(m\le n\)
\end{proof}

\begin{definition}[]
A \textbf{longest} (or a \textbf{maximal}) linearly independent list \(u_1,\dots,u_m\) is a
linearly independent list for which there is no vector \(v\in V\) s.t. 
\(u_1,\dots,u_m,v\) is linearly independent
\end{definition}

\begin{lemma}[]
If \(V\) is a finite-dimensional vector space, then a longest linearly
independent list \(v_1,\dots,v_n\) is a basis of \(V\)
\end{lemma}

\begin{proposition}[]
Let \(Z=u_1,\dots,u_m\) be a linearly independent list in an
\(n\)-dimensional vector space \(V\). Then \(Z\) can be extended to a basis
\end{proposition}

\begin{corollary}[]
If \(\dim(V)=n\), then any list of \(n+1\) or more vectors is linearly dependent
\end{corollary}

\begin{corollary}[]
Let \(V\) be a vector space with \(\dim(V)=n\)
\begin{enumerate}
\item A list of \(n\) vectors that spans \(V\) must be linearly independent
\item Any linearly independent list of \(n\) vectors must span \(V\)
\end{enumerate}
\end{corollary}

\begin{corollary}[]
Let \(U\) be a subspace of a vector space \(V\) of dimension \(n\)
\begin{enumerate}
\item \(U\) is finite-dimensional and \(\dim(U)\le\dim(V)\)
\item If \(\dim(U)=\dim(V)\), then \(U=V\)
\end{enumerate}
\end{corollary}
\subsubsection{Linear Tranformations}
\label{sec:org1d5ea4d}
\index{linear transformation}
\begin{definition}[]
If \(V\) and \(W\) are vector spaces over a field \(k\), then a function
\(T:V\to W\) is a \textbf{linear transformation} if for all vectors \(u,v\in V\), and
all scalars \(a\in k\)
\begin{enumerate}
\item \(T(u+v)=T(u)+T(v)\)
\item \(T(av)=aT(v)\)
\end{enumerate}
\end{definition}

\index{singular}
We say that a linear transformation \(T\) is \textbf{nonsingular} (or is an
\textbf{isomorphism}) if \(T\) is a bijection.

\begin{examplle}[]
\begin{enumerate}
\item If \(\theta\) is an angle, then the rotation about the origin by \(\theta\) is a linear
transformation \(R_\theta:\R^2\to \R^2\)
\item If \(V\) and \(W\) are vector spaces over a field \(k\), write
\(\Hom_k(V,W)\) for the set of all linear transformations \(V\to W\).
It's a vector space
\end{enumerate}
\end{examplle}

\index{general linear group}
\begin{definition}[]
If \(V\) is a vector space over a field \(k\), then the \textbf{general linear group},
denoted by \(\gl(V)\), is the set of all nonsingular linear transformations 
\(V\to V\)
\end{definition}

A composite \(ST\) of linear transformation \(S\) and \(T\) is again a linear
transformation 

\begin{theorem}[]
\label{thm3.92}
Let \(v_1,\dots,v_n\) be a basis of a vector space \(V\) over a field \(k\). If
\(W\) is a vector space over \(k\) and \(u_1,\dots,u_n\) is a list in \(W\), then
there exists a unique linear transformation \(T:V\to W\) with \(T(v_i)=u_i\)
for all \(i\)
\end{theorem}

\begin{proof}
Each \(v\in V\) has a unique expression of the form \(v=\sum_ia_iv_i\) and
so \(T:V\to W\) given by \(T(v)=\sum a_iu_i\) is a well-defined function

To prove the uniqueness of \(T\), assume that \(S:V\to W\) is a linear
transformation with 
\begin{equation*}
S(v_i)=u_i=T(v_i)
\end{equation*}
Then 
\begin{align*}
S(v)&=S(\sum a_iv_i)=\sum S(a_iv_i)\\
&=\sum a_iS(v_i)=\sum a_iT(v_i)=T(v)
\end{align*}
\end{proof}

\begin{corollary}[]
If two linear transformations \(S,T:V\to W\) agree on a basis, then \(S=T\)
\end{corollary}

\begin{proposition}[]
If \(T:k^n\to k^m\) is a linear transformation, then there exists an
\(m\times n\) matrix \(A\) s.t.
\begin{equation*}
T(y)=Ay
\end{equation*}
for all \(y\in k^n\) (here \(y\) is an \(n\times 1\) column matrix) 
\end{proposition}

\begin{proof}
If \(e_1,\dots,e_n\) is the standard basis of \(k^n\) and
\(e_1',\dots,e_m'\) is the standard basis of \(k^m\), define \(A=[a_{ij}]\)
to be the matrix whose \(j\)th column is the coordinate set of \(T(e_j)\).
If \(S:k^n\to k^m\) is defined by \(S(y)=Ay\), then \(S=T\) since they agree
on a basis: \(T(e_j)=\sum_ia_{ij}e_i'=Ae_j\)
\end{proof}

\index{matrix}
\begin{definition}[]
Let \(X=v_1,\dots,v_n\) be a basis of \(V\) and let \(Y=w_1,\dots,w_m\) be a
basis of \(W\). If \(T:V\to W\) is a linear transformation, then the \textbf{matrix
of \(T\)} is the \(m\times n\) matrix \(A=[a_{ij}]\), whose \(j\)th column
\(a_{1j},a_{2j},\dots,a_{mj}\) is the coordinate set of \(T(v_j)\)
determined by \(w\)'s: \(T(v_j)=\sum_{i=1}^ma_{ij}w_j\). The matrix \(A\) does
depend on the choice of bases \(X\) and \(Y\): we will write
\begin{equation*}
A={}_Y[T]_X
\end{equation*}

In case \(V=W\), we often let the basis \(X=v_1,\dots,v_n\) and
\(w_1,\dots,w_m\) coincide. If \(1_V:V\to V\), given by \(v\mapsto v\) is
the identity linear transformation, then \({}_X[1_V]_X\) is the
\(n\times n\)  \textbf{identity matrix \(I_n\)}, defined by
\begin{equation*}
I=[\delta_{ij}]
\end{equation*}
where \(\delta_{ij}\) is the Kronecker delta. A matrix is \textbf{nonsingular} if it
has inverse.
\end{definition}

\begin{examplle}[]
Let \(T:V\to W\) be a linear transformation, and let \(X=v_1,\dots,v_n\) and
\(Y=w_1,\dots,w_n\) be bases of \(V\) and \(W\) ,respectively. The matrix for
\(T\) is set up from the equation
\begin{equation*}
T(v_j)=a_{1j}w_1+\dots+a_{mj}w_m
\end{equation*}
\end{examplle}

\begin{examplle}[]
\begin{enumerate}
\item Let \(T:\R^2\to \R^2\) be rotation by \(\ang{90}\). The matrix of \(T\)
related to the standard basis \(X=(1,0),(0,1)\) is 
\[
{}_X[T]_X=\begin{bmatrix}
 0 & -1 \\
 1 & 0 \\
\end{bmatrix}
\]

However if \(Y=(0,1)(1,0)\), then
\[
{}_Y[T]_Y=\begin{bmatrix}
 0 & 1 \\
 -1 & 0 \\
\end{bmatrix}
\]
\item Let \(k\) be a field, let \(T:V\to V\) be a linear transformation on a
two-dimensional vector space, and assume that there is some vector
\(v\in V\) with \(T(v)\) not a scalar multiple of \(v\). The assumption on
\(v\) says that the list \(X=v,T(v)\) is linearly independent, and hence
it's a basis of \(V\). Write \(v_1=v,v_2=Tv\).

We compute \({}_X[T]_X\)
\begin{equation*}
T(v_1)=v_2\quad\text{ and }\quad T(v_2)=av_1+bv_2
\end{equation*}
for some \(a,b\in k\). We conclude that
\[
{}_X[T]_X=\begin{bmatrix}
 0 & a \\
 1 & b \\
\end{bmatrix}
\]
\end{enumerate}
\end{examplle}

\begin{proposition}[]
\label{prop3.97}
Let \(V\) and \(W\) be a vector spaces over a field \(k\), and let
\(X=v_1,\dots,v_n\) and \(Y=w_1,\dots,w_m\) be bases of \(V\) and \(W\),
respectively. If \(\Hom_k(V,W)\) denotes the set of all linear
transformations \(T:V\to W\) and \(\Mat_{m\times n}k\) denotes the set of
all \(m\times n\) matrices with entries in \(k\), then the function 
\(T\mapsto{}_Y[T]_X\) is a bijection \(\Hom_k(V,W)\to\Mat_{m\times n}(k)\) 
\end{proposition}

\begin{proof}
Given a matrix \(A\), its columns define vectors in \(W\); in more detail,
if the \(j\)th column of \(A\) is \(a_{1j},\dots,a_{mj}\), define
\(z_j=\sum_{i=1}^ma_{ij}w_i\). By Theorem \ref{thm3.92}, there exists a linear
transformation \(T:V\to W\) with \(T(v_j)=z_j\) and \({}_Y[T]_X=A\).
\end{proof}

\begin{proposition}[]
\label{prop3.98}
Let \(T:V\to W\) and \(S:W\to U\) be linear transformations. Choose bases 
\(X=x_1,\dots,x_n\) of \(V\), \(Y=y_1,\dots,y_m\) of \(W\), and
\(Z=z_1,\dots,z_l\) of \(U\), then
\begin{equation*}
{}_Z[S\circ T]_X=({}_Z[S]_Y)({}_Y[T]_X)
\end{equation*}
\end{proposition}

\begin{proof}
Let \({}_Y[T]_X=[a_{ij}]\), so that \(T(x_j)=\sum_pa_{pj}y_p\), and let
\({}_Z[S]_Y=[b_{qp}]\), so that \(S(y_p)=\sum_qb_{qp}z_q\). Then
\begin{align*}
ST(x_j)=S(T(x_j))&=S(\displaystyle\sum_{p}a_{pj}y_p)\\
&=\displaystyle\sum_{p}a_{pj}S(y_p)=\displaystyle\sum_{p}
\displaystyle\sum_qa_{pj}b_{qp}z_q=\displaystyle\sum_{q}c_{qj}z_q
\end{align*}
where \(c_{qj}=\sum_{p}b_{qp}a_{pj}\). Therefore
\begin{equation*}
{}_Z[ST]_X=[c_{qj}]={}_Z[S]_Y{}_Y[T]_X
\end{equation*}
\end{proof}

\begin{corollary}[]
Matrix multiplication is associative
\end{corollary}

\begin{proof}
Let \(A\) be an \(m\times n\) matrix, let \(B\) be an \(n\times p\) matrix,
and let \(C\) be a \(p\times q\) matrix. By Theorem \ref{thm3.92}, there are
linear transformations
\begin{equation*}
k^q\xrightarrow{T}k^p\xrightarrow{S}k^n\xrightarrow{R}k^m
\end{equation*}
with \(C=[T],B=[S],A=[R]\)

Then 
\begin{equation*}
[R\circ(S\circ T)]=[R][S\circ T]=[R]([S][T])=A(BC)
\end{equation*}
On the other hand
\begin{equation*}
[(R\circ S)\circ T]=[R\circ S][T]=([R][S])[T]=(AB)C
\end{equation*}
\end{proof}

\begin{corollary}[]
\label{cor3.100}
Let \(T:V\to W\) be a linear transformation of vector space \(V\) over a
field \(k\), and let \(X\) and \(Y\) be bases of \(V\) and \(W\),
respectively. If \(T\) is nonsingular, then the matrix of \(T^{-1}\) is the
inverse of the matrix of \(T\)
\begin{equation*}
{}_X[T^{-1}]_Y=({}_Y[T]_X)^{-1}
\end{equation*}
\end{corollary}

\begin{proof}
\(I={}_Y[1_W]_Y={}_Y[T]_{XX}[T^{-1}]_Y\) and \(I={}_X[1_V]_X={}_X[T^{-1}]_{YY}[T]_X\)
\end{proof}

\begin{corollary}[]
\label{cor3.101}
Let \(T:V\to V\) be a linear transformation on a vector space \(V\) over a
field \(k\). If \(X\) and \(Y\) are bases of \(V\), then there is a nonsingular
matrix \(P\) with entries in \(k\) so that
\begin{equation*}
{}_Y[T]_Y=P(_X[T]_X)P^{-1}
\end{equation*}
Conversely, if \(B=PAP^{-1}\), where \(B,A,P\) are \(n\times n\) matrices
with entries in \(k\) and \(P\) is nonsingular, then there is a linear
transformation \(T:k^n\to k^n\) and bases \(X\) and \(Y\) of \(k^n\) s.t.
\(B=_Y[T]_Y,A=_X[T]_X\)
\end{corollary}

\begin{proof}
The first statement follows from Proposition \ref{prop3.98} and associativity
\begin{equation*}
{}_Y[T]_Y=_Y[1_VT1_V]_Y=(_Y[1_V]_X)(_X[T]_X)(_X[1_V]_Y)
\end{equation*}
Set \(P=_Y[1_V]_X\)

For the converse, let \(E=e_1,\dots,e_n\) be the standard basis of \(k^n\),
and define \(T:k^n\to k^n\) be \(T(e_j)=Ae_j\). If follows that
\(A=_E[T]_E\). Now define a basis \(Y=y_1,\dots,y_n\) by \(y_j=P^{-1}e_j\).
\(Y\) is a basis because \(P^{-1}\) is nonsingular. It suffices to prove that 
\(B=_Y[T]_Y\); that is \(T(y_j)=\sum_ib_{ij}y_i\), where \(B=[b_{ij}]\)
\begin{align*}
T(y_j)&=Ay_y=AP^{-1}e_j=P^{-1}Be_j\\
&=P^{-1}\displaystyle\sum_{i}b_{ij}e_i
=\displaystyle\sum_{i}b_{ij}P^{-1}e_i\\
&=\displaystyle\sum_{i}b_{ij}y_i
\end{align*}
\end{proof}

\index{similar}
\begin{definition}[]
Two \(n\times n\) matrices \(B\)  and \(A\) with entries in field \(k\) are
\textbf{similar} if there is a nonsingular matrix \(P\) with entries in \(k\) with
\(B=PAP^{-1}\) 
\end{definition}



Corollary \ref{cor3.101} says the two matrices arise from the same linear
transformation on a vector space \(V\) if and only if they are similar

\begin{definition}[]
If \(T:V\to W\) is a linear transformation, then the \textbf{kernel} (or the \textbf{null
space}) of \(T\) is
\begin{equation*}
\ker T=\{v\in V:T(v)=0\}
\end{equation*}
and the \textbf{image} of \(T\) is
\begin{equation*}
\im T=\{w\in W:w=T(v)\text{ for some }v\in V\}
\end{equation*}
\end{definition}

\begin{proposition}[]
Let \(T:V\to W\) be a linear transformation
\begin{enumerate}
\item \(\ker T\) is a subspace of \(V\) and \(\im T\) is a subspace of \(W\)
\item \(T\) is injective if and only if \(\ker T=\{0\}\)
\end{enumerate}
\end{proposition}

\begin{lemma}[]
\label{lemma3.103}
Let \(T:V\to W\) be a linear transformation
\begin{enumerate}
\item If \(T\) is nonsingular, then for every basis \(X=v_1,\dots,v_n\) of \(V\),
we have \(T(X)=T(v_1),\dots,T(v_n)\) a basis of \(W\)
\item Conversely, if there exists some basis \(X=v_1,\dots,v_n\) of \(V\) for
which \(T(X)\) is a basis of \(W\), then \(T\) is nonsingular
\end{enumerate}
\end{lemma}
\begin{proof}
\begin{enumerate}
\item If \(\sum c_iT(v_i)=0\), then \(T(\sum c_iv_i)=0\) and so 
\(\sum c_iv_i\in\ker T=\{0\}\). Hence each \(c_i=0\) because \(X\) is linearly
independent. If \(w\in W\), then the surjectivity of \(T\) provides \(v\in V\)
with \(w=T(v)\). But \(v=\sum a_iv_i\), and so 
\(w=T(v)=T(\sum a_iv_i)=\sum a_iT(v_i)\). Therefore \(T(X)\) is a basis of
\(W\)
\item Let \(w\in W\). Since \(T(X)\) is a basis of \(W\), we have 
\(w=\sum c_iT(v_i)=T(\sum c_iv_i)\). Add so \(T\) is surjective. If 
\(\sum c_iv_i\in\ker T\), then \(\sum c_iT(v_i)=0\) and so linear
independence gives all \(c_i=0\); hence \(\ker T=\{0\}\). Therefore \(T\)
is nonsingular
\end{enumerate}
\end{proof}

\begin{theorem}[]
If \(V\) is an \(n\)-dimensional vector space over a field \(k\), then \(V\) is
isomorphic to \(k^n\)
\end{theorem}

\begin{proof}
Choose a basis \(v_1,\dots,v_n\) of \(V\). If \(e_1,\dots,e_n\) is the
standard basis of \(k^n\), then Theorem \ref{thm3.92} says that there is a
linear transformation \(T:V\to k^n\) with \(T(v_i)=e_i\); by Lemma
\ref{lemma3.103} \(T\) is nonsingular
\end{proof}

\begin{corollary}[]
Two finite-dimensional vector space \(V\) and \(W\) over a field \(k\) are
isomorphic if and only if \(\dim(V)=\dim(W)\)
\end{corollary}

\begin{proposition}[]
\label{prop3.106}
Let \(V\) be a finite-dimensional vector space with \(\dim(V)=n\), and let 
\(T:V\to V\) be a linear transformation. The following statements are
equivalent
\begin{enumerate}
\item \(T\) is an isomorphism
\item \(T\) is surjective
\item \(T\) is injective
\end{enumerate}
\end{proposition}

\begin{proof}
\(2\to 3\). Let \(v_1,\dots,v_n\) be the basis of \(V\). Since \(T\) is
surjective, there are vectors \(u_1,\dots,u_n\) with \(Tu_i=v_i\). We claim
that \(u_1,\dots,u_n\) are linearly independent. To show that \(T\) is
injective, it suffices to show that \(\ker T=\{0\}\)

\(3\to 1\). Let \(v_1,\dots,v_n\) be a basis of \(V\). If \(c_1,\dots,c_n\)
are scalars not all 0, then \(\sum c_iv_i\neq0\). Since \(T\) is injective, it
follows that \(\sum c_iT(v_i)\neq0\) and so \(Tv_1,\dots,Tv_n\) are linearly
independent. Therefore Lemma \ref{lemma3.103} shows that \(T\) is an isomorphism
\end{proof}

\begin{corollary}[]
If \(A\) and \(B\) are \(n\times n\) matrices with \(AB=I\), then \(BA=I\).
Therefore \(A\) is nonsingular with inverse \(B\)
\end{corollary}
\begin{proof}
There are linear transformations \(T,S:k^n\to k^n\) with \([T]=A,[S]=B\),
and \(AB=I\) gives
\begin{equation*}
[TS]=[T][S]=[1_{k^n}]
\end{equation*}
Since \(T\mapsto[T]\) is a bijection, by Proposition \ref{prop3.97}, it
follows that \(TS=1_{k^n}\). Hence \(T\) is a surjection and \(S\) is an
injection by Proposition \ref{prop1}.. But Proposition \ref{prop3.106} says taht \(T,S\) are both
isomorphism, so that \(S=T^{-1}\) and \(TS=1_{k^n}=ST\)
\end{proof}

\begin{definition}[]
The set of all nonsingular \(n\times n\) matrices with entries in \(k\) is
denoted by \(\gl(n,k)\)
\end{definition}

It's easy to prove that \(\gl(n,k)\) is a group

\begin{proposition}[]
Let \(V\) be an \(n\)-dimensional vector space over a field \(k\), and let
\(X=v_1,\dots,v_n\) be a basis of \(V\). Then \(\mu:\gl(V)\to\gl(n,k)\) defined
by \(T\mapsto[T]={}_X[T]_X\) is an isomorphism
\end{proposition}

\begin{proof}
By Proposition \ref{prop3.97} the function \(\mu':T\mapsto[T]\) is a bijection
\begin{equation*}
\Hom_k(V,V)\to\Mat_n(k)
\end{equation*}

If \(T\in\gl(V)\), then \([T]\) is a nonsingular matrix by Corollary
\ref{cor3.100}; that is, if \(\mu\) is the restriction of \(\mu'\), then 
\(\mu:\gl(V)\to\gl(n,k)\) is an injective homomorphism.

If \(A\in\gl(n,k)\), then \(A=[T]\) for some \(T:V\to V\). It suffices to
show that \(T\) is an isomorphism; that is, \(T\in\gl(V)\). Since \([T]\) is
a nonsingular matrix, there is a matrix \(B\) with \([T]B=I\). Now \(B=[S]\)
for some \(S:V\to V\) and 
\begin{equation*}
[TS]=[T][S]=I=[1_V]
\end{equation*}
\end{proof}

\begin{definition}[]
A linear transformation \(T:V\to V\) is a \textbf{scalar transformation} if there is 
\(c\in k\) with \(T(v)=cv\) for all \(v\in V\); that is \(T=c1_V\). A
\textbf{scalar matrix} is a matrix of the form \(cI\)
\end{definition}

\begin{corollary}[]
\begin{enumerate}
\item The center of the group \(\gl(V)\) consists of all the nonsingular scalar
transformations
\item The center of the group \(\gl(n,k)\) consists of all the nonsingular
scalar matrices
\end{enumerate}
\end{corollary}
\subsection{Quotient Rings and Finite Fields}
\label{sec:orgf042941}
\begin{theorem}[]
\label{thm3.110}
If \(I\) is an ideal in a commutative ring \(R\), then the additive abelian group
\(R/I\) can be made into a commutative ring in such a way that the natural
map \(\pi:R\to R/I\) is a surjective ring homomorphism
\end{theorem}

\begin{proof}
Define multiplication on the additive abelian group \(R/I\) by
\begin{equation*}
(a+I)(b+I)=ab+I
\end{equation*}
\end{proof}

\begin{definition}[]
The commutative ring \(R/I\) constructed in Theorem \ref{thm3.110} is called
the \textbf{quotient ring} of \(R\) modulo \(I\)
\end{definition}


\begin{corollary}[]
If \(I\) is an ideal in a commutative ring \(R\), then there are a commutative
ring \(A\) and a ring homomorphism \(\pi:R\to A\) with \(I=\ker\pi\)
\end{corollary}

\begin{proof}
Natural map \(\pi:R\to R/I\)
\end{proof}

\begin{theorem}[First Isomorphism Theorem]
If \(f:R\to A\) is a homomorphism of rings, then \(\ker f\) is an ideal in
\(R\), \(\im f\) is a subring of \(A\), and
\begin{equation*}
R/\ker f\cong\im f
\end{equation*}
\end{theorem}

\begin{definition}[]
If \(k\) is a field, the intersection of all the subfields of \(k\) is called the
\textbf{prime field} of \(k\)
\end{definition}

Every subfield of \(\C\) contains \(\Q\) and so the prime field of \(\C\) and
of \(\R\) is \(\Q\).

\textbf{Notation}. From now on, we will denote \(\I_p\) by \(\F_p\) when we are
regarding it as a field.

\begin{proposition}[]
If \(k\) is a field, then its prime field is isomorphic to \(\Q\) or to \(\F_p\)
for some prime \(p\)
\end{proposition}

\begin{proof}
Consider the ring homomorphism \(\chi:\Z\to k\) defined by \(\chi(n)=n\epsilon\),
where we denote the \textbf{one} in \(k\) by \(\epsilon\). Since every ideal in \(\Z\) is
principal, there is an integer \(m\) with \(\ker\chi=(m)\). If \(m=0\), then
\(\chi\) is an injection, and so there is an isomorphism copy of \(\Z\) taht is a
subring of \(k\). By Exercise \ref{ex3.47} 
\end{proof}

\(aef\). 
\section{Index}
\label{sec:org781d647}
\printindex
\end{document}
\message{ !name(AdvancedModernAlgebra.tex) !offset(-3984) }
