% Created 2019-09-12 四 13:45
% Intended LaTeX compiler: pdflatex
\documentclass[11pt]{article}
\usepackage[utf8]{inputenc}
\usepackage[T1]{fontenc}
\usepackage{graphicx}
\usepackage{grffile}
\usepackage{longtable}
\usepackage{wrapfig}
\usepackage{rotating}
\usepackage[normalem]{ulem}
\usepackage{amsmath}
\usepackage{textcomp}
\usepackage{amssymb}
\usepackage{capt-of}
\usepackage{hyperref}
\usepackage{minted}
% TIPS
% \substack{a\\b} for multiple lines text





% pdfplots will load xolor automatically without option
\usepackage[dvipsnames]{xcolor}

\usepackage{forest}
% two-line text in node by [two \\ lines]
% \begin{forest} qtree, [..] \end{forest}
\forestset{
  qtree/.style={
    baseline,
    for tree={
      parent anchor=south,
      child anchor=north,
      align=center,
      inner sep=1pt,
    }}}
%\usepackage{flexisym}
% load order of mathtools and mathabx, otherwise conflict overbrace

\usepackage{mathtools}
%\usepackage{fourier}
\usepackage{pgfplots}
\usepackage{amsthm}
\usepackage{amsmath}
%\usepackage{unicode-math}
%
\usepackage{commath}
%\usepackage{,  , }
\usepackage{amsfonts}
\usepackage{amssymb}
% importing symbols https://tex.stackexchange.com/questions/14386/importing-a-single-symbol-from-a-different-font
%mathabx change every symbol
% use instead stmaryrd
%\usepackage{mathabx}
\usepackage{stmaryrd}
\usepackage{empheq}
\usepackage{tikz}
\usepackage{tikz-cd}
%\usepackage[notextcomp]{stix}
\usetikzlibrary{arrows.meta}
\usepackage[most]{tcolorbox}
%\utilde
%\usepackage{../../latexpackage/undertilde/undertilde}
% left and right superscript and subscript
\usepackage{actuarialsymbol}
\usepackage{threeparttable}
\usepackage{scalerel,stackengine}
\usepackage{stackrel}
% \stackrel[a]{b}{c}
\usepackage{dsfont}
% text font
\usepackage{newpxtext}
%\usepackage{newpxmath}

%\newcounter{dummy} \numberwithin{dummy}{section}
\newtheorem{dummy}{dummy}[section]
\theoremstyle{definition}
\newtheorem{definition}[dummy]{Definition}
\newtheorem{corollary}[dummy]{Corollary}
\newtheorem{lemma}[dummy]{Lemma}
\newtheorem{proposition}[dummy]{Proposition}
\newtheorem{theorem}[dummy]{Theorem}
\theoremstyle{definition}
\newtheorem{example}[dummy]{Example}
\theoremstyle{remark}
\newtheorem*{remark}{Remark}


\newcommand\what[1]{\ThisStyle{%
    \setbox0=\hbox{$\SavedStyle#1$}%
    \stackengine{-1.0\ht0+.5pt}{$\SavedStyle#1$}{%
      \stretchto{\scaleto{\SavedStyle\mkern.15mu\char'136}{2.6\wd0}}{1.4\ht0}%
    }{O}{c}{F}{T}{S}%
  }
}

\newcommand\wtilde[1]{\ThisStyle{%
    \setbox0=\hbox{$\SavedStyle#1$}%
    \stackengine{-.1\LMpt}{$\SavedStyle#1$}{%
      \stretchto{\scaleto{\SavedStyle\mkern.2mu\AC}{.5150\wd0}}{.6\ht0}%
    }{O}{c}{F}{T}{S}%
  }
}

\newcommand\wbar[1]{\ThisStyle{%
    \setbox0=\hbox{$\SavedStyle#1$}%
    \stackengine{.5pt+\LMpt}{$\SavedStyle#1$}{%
      \rule{\wd0}{\dimexpr.3\LMpt+.3pt}%
    }{O}{c}{F}{T}{S}%
  }
}

\newcommand{\bl}[1] {\boldsymbol{#1}}
\newcommand{\Wt}[1] {\stackrel{\sim}{\smash{#1}\rule{0pt}{1.1ex}}}
\newcommand{\wt}[1] {\widetilde{#1}}
\newcommand{\tf}[1] {\textbf{#1}}


%For boxed texts in align, use Aboxed{}
%otherwise use boxed{}

\DeclareMathSymbol{\widehatsym}{\mathord}{largesymbols}{"62}
\newcommand\lowerwidehatsym{%
  \text{\smash{\raisebox{-1.3ex}{%
    $\widehatsym$}}}}
\newcommand\fixwidehat[1]{%
  \mathchoice
    {\accentset{\displaystyle\lowerwidehatsym}{#1}}
    {\accentset{\textstyle\lowerwidehatsym}{#1}}
    {\accentset{\scriptstyle\lowerwidehatsym}{#1}}
    {\accentset{\scriptscriptstyle\lowerwidehatsym}{#1}}
}

\usepackage{graphicx}
    
% text on arrow for xRightarrow
\makeatletter
%\newcommand{\xRightarrow}[2][]{\ext@arrow 0359\Rightarrowfill@{#1}{#2}}
\makeatother


\newcommand{\dom}[1]{%
\mathrm{dom}{(#1)}
}

% Roman numerals
\makeatletter
\newcommand*{\rom}[1]{\expandafter\@slowromancap\romannumeral #1@}
\makeatother

\def \fR {\mathfrak{R}}
\def \bx {\boldsymbol{x}}
\def \bz {\boldsymbol{z}}
\def \ba {\boldsymbol{a}}
\def \bh {\boldsymbol{h}}
\def \bo {\boldsymbol{o}}
\def \bU {\boldsymbol{U}}
\def \bc {\boldsymbol{c}}
\def \bV {\boldsymbol{V}}
\def \bI {\boldsymbol{I}}
\def \bK {\boldsymbol{K}}
\def \bt {\boldsymbol{t}}
\def \bb {\boldsymbol{b}}
\def \bA {\boldsymbol{A}}
\def \bX {\boldsymbol{X}}
\def \bu {\boldsymbol{u}}
\def \bS {\boldsymbol{S}}
\def \bZ {\boldsymbol{Z}}
\def \bz {\boldsymbol{z}}
\def \by {\boldsymbol{y}}
\def \bw {\boldsymbol{w}}
\def \bT {\boldsymbol{T}}
\def \bF {\boldsymbol{F}}
\def \bS {\boldsymbol{S}}
\def \bm {\boldsymbol{m}}
\def \bW {\boldsymbol{W}}
\def \bR {\boldsymbol{R}}
\def \bQ {\boldsymbol{Q}}
\def \bS {\boldsymbol{S}}
\def \bP {\boldsymbol{P}}
\def \bT {\boldsymbol{T}}
\def \bY {\boldsymbol{Y}}
\def \bH {\boldsymbol{H}}
\def \bB {\boldsymbol{B}}
\def \blambda {\boldsymbol{\lambda}}
\def \bPhi {\boldsymbol{\Phi}}
\def \btheta {\boldsymbol{\theta}}
\def \bTheta {\boldsymbol{\Theta}}
\def \bmu {\boldsymbol{\mu}}
\def \bphi {\boldsymbol{\phi}}
\def \bSigma {\boldsymbol{\Sigma}}
\def \lb {\left\{}
\def \rb {\right\}}
\def \la {\langle}
\def \ra {\rangle}
\def \caln {\mathcal{N}}
\def \dissum {\displaystyle\Sigma}
\def \dispro {\displaystyle\prod}
\def \E {\mathbb{E}}
\def \Q {\mathbb{Q}}
\def \N {\mathbb{N}}
\def \V {\mathbb{V}}
\def \R {\mathbb{R}}
\def \P {\mathbb{P}}
\def \A {\mathbb{A}}
\def \Z {\mathbb{Z}}
\def \I {\mathbb{I}}
\def \C {\mathbb{C}}
\def \cala {\mathcal{A}}
\def \calb {\mathcal{B}}
\def \calq {\mathcal{Q}}
\def \calp {\mathcal{P}}
\def \cals {\mathcal{S}}
\def \calg {\mathcal{G}}
\def \caln {\mathcal{N}}
\def \calr {\mathcal{R}}
\def \calm {\mathcal{M}}
\def \calc {\mathcal{C}}
\def \calf {\mathcal{F}}
\def \calk {\mathcal{K}}
\def \call {\mathcal{L}}
\def \calu {\mathcal{U}}
\def \bcup {\bigcup}


\def \uin {\underline{\in}}
\def \oin {\overline{\in}}
\def \uR {\underline{R}}
\def \oR {\overline{R}}
\def \uP {\underline{P}}
\def \oP {\overline{P}}

\def \Ra {\Rightarrow}

\def \e {\enspace}

\def \sgn {\operatorname{sgn}}
\def \gen {\operatorname{gen}}
\def \ker {\operatorname{ker}}
\def \im {\operatorname{im}}

\def \tril {\triangleleft}

% \varprod
\DeclareSymbolFont{largesymbolsA}{U}{txexa}{m}{n}
\DeclareMathSymbol{\varprod}{\mathop}{largesymbolsA}{16}

% \bigtimes
\DeclareFontFamily{U}{mathx}{\hyphenchar\font45}
\DeclareFontShape{U}{mathx}{m}{n}{
      <5> <6> <7> <8> <9> <10>
      <10.95> <12> <14.4> <17.28> <20.74> <24.88>
      mathx10
      }{}
\DeclareSymbolFont{mathx}{U}{mathx}{m}{n}
\DeclareMathSymbol{\bigtimes}{1}{mathx}{"91}
% \odiv
\DeclareFontFamily{U}{matha}{\hyphenchar\font45}
\DeclareFontShape{U}{matha}{m}{n}{
      <5> <6> <7> <8> <9> <10> gen * matha
      <10.95> matha10 <12> <14.4> <17.28> <20.74> <24.88> matha12
      }{}
\DeclareSymbolFont{matha}{U}{matha}{m}{n}
\DeclareMathSymbol{\odiv}         {2}{matha}{"63}


\newcommand\subsetsim{\mathrel{%
  \ooalign{\raise0.2ex\hbox{\scalebox{0.9}{$\subset$}}\cr\hidewidth\raise-0.85ex\hbox{\scalebox{0.9}{$\sim$}}\hidewidth\cr}}}
\newcommand\simsubset{\mathrel{%
  \ooalign{\raise-0.2ex\hbox{\scalebox{0.9}{$\subset$}}\cr\hidewidth\raise0.75ex\hbox{\scalebox{0.9}{$\sim$}}\hidewidth\cr}}}

\newcommand\simsubsetsim{\mathrel{%
  \ooalign{\raise0ex\hbox{\scalebox{0.8}{$\subset$}}\cr\hidewidth\raise1ex\hbox{\scalebox{0.75}{$\sim$}}\hidewidth\cr\raise-0.95ex\hbox{\scalebox{0.8}{$\sim$}}\cr\hidewidth}}}
\newcommand{\stcomp}[1]{{#1}^{\mathsf{c}}}


\author{Joseph J. Rotman}
\date{\today}
\title{Advanced Modern Algebra}
\hypersetup{
 pdfauthor={Joseph J. Rotman},
 pdftitle={Advanced Modern Algebra},
 pdfkeywords={},
 pdfsubject={},
 pdfcreator={Emacs 26.2 (Org mode 9.2.5)}, 
 pdflang={English}}
\begin{document}

\maketitle
\tableofcontents \clearpage
\section{Group \rom{1}}
\label{sec:org5b13d84}
\subsection{Permutations}
\label{sec:org6f2e465}
\begin{definition}[]
A \textbf{permutation} of a set \(X\) is a bijection from \(X\) to itself.
\end{definition}


\begin{definition}[]
The family of all the permutations of a set \(X\), denoted by \(S_X\) is called
the \textbf{symmetric group} on \(X\). When \(X=\lb 1,2,\dots,n\rb\), \(S_X\) is
usually denoted by \(X_n\) and is called the \textbf{symmetric group on } \(n\)
\textbf{letters} 
\end{definition}

\begin{definition}[]
Let \(i_1,i_2,\dots,i_r\) be distinct integers in \(\lb 1,2,\dots,n\rb\). If
\(\alpha\in S_n\) fixes the other integers and if
\begin{equation*}
\alpha(i_1)=i_2,\alpha(i_2)=i_3,\dots,\alpha(i_{r-1})=i_r,\alpha(i_r)=i_1
\end{equation*}
then \(\alpha\) is called an textbf\{r-cycle\}. \(\alpha\) is a cycle of
\textbf{length} \(r\) and denoted by
\begin{equation*}
\alpha=(i_1\; i_2\;\dots\; i_r)
\end{equation*}
\end{definition}

2-cycles are also called the \textbf{transpositions}.

\begin{definition}[]
Two permutations \(\alpha,\beta\in S_n\) are \textbf{disjoint} if every \(i\)
moved by one is fixed by the other.
\end{definition}

\begin{lemma}[]
Disjoint permutations \(\alpha,\beta\in S_n\) commute
\end{lemma}

\begin{proposition}[]
Every permutation \(\alpha\in S_n\) is either a cycle or a product of disjoint cycles.
\end{proposition}

\begin{proof}
Induction on the number \(k\) of points moved by \(\alpha\)
\end{proof}

\begin{definition}[]
A \textbf{complete factorization} of a permutation \(\alpha\) is a
factorization of \(\alpha\) into disjoint cycles that contains exactly one
1-cycle \((i)\) for every \(i\) fixed by \(\alpha\)
\end{definition}

\begin{theorem}[]
Let \(\alpha\in S_n\) and let \(\alpha=\beta_1\dots\beta_t\) be a complete
factorization into disjoint cycles. This factorization is unique except for
the order in which the cycles occur
\end{theorem}

\begin{proof}
for all \(i\), if \(\beta_t(i)\neq i\), then \(\beta_t^k(i)\neq\beta_t^{k-1}(i)\)
for any \(k\ge 1\)
\end{proof}

\begin{lemma}[]
If \(\gamma,\alpha\in S_n\), then \(\alpha\gamma\alpha^{-1}\) has the same cycle
structure as \(\gamma\). In more detail, if the complete factorization of
\(\gamma\) is
\begin{equation*}
\gamma=\beta_1\beta_2\dots(i_1\; i_2\;\dots)\dots\beta_t
\end{equation*}
then \(\alpha\gamma\alpha^{-1}\) is permutation that is obtained from \(\gamma\)
by applying \(\alpha\) to the symbols in the cycles of \(\gamma\)
\end{lemma}

Example. Suppose
\begin{gather*}
\beta=(1\;2\;3)(4)(5)\\
\gamma=(5\;2\;4)(1)(3)
\end{gather*}
then we can easily find the \(\alpha\)
\begin{equation*}
\alpha=
\begin{pmatrix}
1&2&3&4&5\\
5&2&4&1&3
\end{pmatrix}
\end{equation*}
\begin{theorem}[]
Permutations \(\gamma\) and \(\sigma\) in \(S_n\) has the same cycle structure if
and only if there exists \(\alpha\in S_n\) with \(\sigma=\alpha\gamma\alpha^{-1}\)
\end{theorem}


\begin{proposition}[]
If \(n\ge 2\) then every \(\alpha\in S_n\) is a product of tranpositions
\end{proposition}
\begin{proof}
\((1\;2\;\dots\; r)=(1\; r)(1\; r-1)\dots(1\; 2)\)
\end{proof}


\begin{definition}[]
A permutation \(\alpha\in S_n\) is \textbf{even} if it can be factored into a
product of an even number of transpositions. Otherwise \textbf{odd}
\end{definition}

\begin{definition}[]
If \(\alpha\in S_n\) and \(\alpha=\beta_1\dots\beta_t\) is a complete
factorization, then \textbf{signum} \(\alpha\) is defined by
\begin{equation*}
\sgn(\alpha)=(-1)^{n-t}
\end{equation*}
\end{definition}

\begin{theorem}[]
For all \(\alpha,\beta\in S_n\)
\begin{equation*}
\sgn(\alpha\beta)=\sgn(\alpha)\sgn(\beta)
\end{equation*}
\end{theorem}

\begin{theorem}[]
\begin{enumerate}
\item Let \(\alpha\in S_n\); if \(\sgn(\alpha)=1\) then \(\alpha\) is even. otherwise
odd
\item A permutation \(\alpha\) is odd if and only if it's a product of an odd
number of transpositions
\end{enumerate}
\end{theorem}

\begin{corollary}[]
Let \(\alpha,\beta\in S_n\). If \(\alpha\) and \(\beta\) have the same parity, then
\(\alpha\beta\) is even while if \(\alpha\) and \(\beta\) have distinct parity,
\(\alpha\beta\) is odd
\end{corollary}
\subsection{Groups}
\label{sec:orgb493827}
\begin{definition}[]
A \textbf{binary operation} on a set \(G\) is a function
\begin{equation*}
*:G\times G\to G
\end{equation*}
\end{definition}

\begin{definition}[]
A \textbf{group} is a set \(G\) equipped with a binary operation * s.t.
\begin{enumerate}
\item the \textbf{associative law} holds
\item \textbf{identity}
\item every \(x\in G\) has an \textbf{inverse}, there is a \(x'\in G\)  with 
\(x*x'=e=x'*x\)
\end{enumerate}
\end{definition}

\begin{definition}[]
A group \(G\) is called \textbf{abelian} if it satisfies the
\textbf{commutative law}
\end{definition}

\begin{lemma}[]
Let \(G\) be a group
\begin{enumerate}
\item The \textbf{cancellation laws} holds: if either \(x*a=x*b\) or \(a*x=b*x\), then
\(a=b\)
\item \(e\) is unique
\item Each \(x\in G\) has a unique inverse
\item \((x^{-1})^{-1}=x\)
\end{enumerate}
\end{lemma}

\begin{definition}[]
An expression \(a_1a_2\dots a_n\) \textbf{needs no parentheses} if all the ultimate
products it yields are equal
\end{definition}

\begin{theorem}[Generalized Associativity]
If \(G\) is a group and \(a_1,a_2,\dots,a_n\in G\) then the expression
\(a_1a_2\dots a_n\) needs no parentheses
\end{theorem}

\begin{definition}[]
Let \(G\) be a group and let \(a\in G\). If \(a^k=1\) for some \(k>1\) then the
smallest such exponent \(k\ge 1\) is called the \tf{order} or \(a\); if no such
power exists, then one says that \(a\) has \tf{infinite order}
\end{definition}

\begin{proposition}[]
If \(G\) is a finite group, then every \(x\in G\) has finite order
\end{proposition}

\begin{definition}[]
A \tf{motion} is a distance preserving bijection \(\varphi:\R^2\to\R^2\). If
\(\pi\) is a polygon in the plane, then its \tf{symmetry group} \(\Sigma(\pi)\)
consists of all the motions \(\varphi\) for which \(\varphi(\pi)=\pi\). The
elements of \(\Sigma(\pi)\) are called the \tf{symmetries} of \(\pi\)
\end{definition}

Let \(\pi_4\) be a square. Then the group \(\Sigma(\pi_4)\) is called the
\tf{dihedral group} with 8 elements, denoted by \(D_8\)

\begin{definition}[]
If \(\pi_n\) is a regular polygon with \(n\) vertices \(v_1,\dots,v_n\) and center
\(O\), then the symmetry group \(\Sigma(\pi_n)\) is called the \tf\{dihedral
group\} with \(2n\) elements, and it's denoted by \(D_{2n}\)
\end{definition}
\subsection{Lagrange's theorem}
\label{sec:org16e8247}
\begin{definition}[]
A subset \(H\) of a group \(G\) is a \tf{subgroup} if
\begin{enumerate}
\item \(1\in H\)
\item if \(x,y\in H\), then \(xy\in H\)
\item if \(x\in H\), then \(x^{-1}\in H\)
\end{enumerate}
\end{definition}

If \(H\) is a subgroup of \(G\), we write \(H\le G\). If \(H\) is a proper subgroup,
then we write \$H<\$G

\begin{proposition}[]
A subset \(H\) of a group \(G\) is a subgroup if and only if \(H\) is nonempty and
whenever \(x,y\in H\), \(xy^{-1}\in H\)
\end{proposition}

\begin{proposition}[]
A nonempty subset \(H\) of a finite group \(G\) is a subgroup if and only if \(H\)
is closed; that is, if \(a,b\in H\), then \(ab\in H\)
\end{proposition}

\begin{definition}[]
If \(G\) is a group and \(a\in G\)
\begin{equation*}
\langle a\rangle=\{a^n:n\in\Z\}=\{\text{all powers of } a\}
\end{equation*}
\(\la a\ra\) is called the \tf{cyclic subgroup} of \(G\) \tf{generated} by \(a\). A
group \(G\) is called \tf{cyclic} if there exists \(a\in G\) s.t. \(G=\la a\ra\),
in which case \(a\) is called the \tf{generator}
\end{definition}

\begin{definition}[]
The \tf{integers mod $m$}, denoted by \(\I_m\) is the family of all congruence
classes mod \(m\)
\end{definition}


\begin{proposition}[]
Let \(m\ge 2\) be a fixed integer
\begin{enumerate}
\item If \(a\in \Z\), then \([a]=[r]\) for some \(r\) with \(0\le r<m\)
\item If \(0\le r'<r<m\), then \([r']\neq[r]\)
\item \(\I_m\) has exactly \(m\) elements
\end{enumerate}
\end{proposition}

\begin{theorem}[]
\begin{enumerate}
\item If \(G=\la a\ra\) is a cyclic group of order \(n\), then \(a^k\) is a generator
of \(G\) if and only if \((k,n)=1\)
\item If \(G\) is a cyclic group of order \(n\) and \(\gen(G)=\{\text{all generators
      of } G\}\), then
\begin{equation*}
\abs{\gen(G)}=\phi(n)
\end{equation*}
where \(\phi\) is the Euler \(\phi\)-function
\end{enumerate}
\end{theorem}
\begin{proof}
\begin{enumerate}
\item there is \(t\in\N\) s.t. \(a^{kt}=a\) hence \(a^{kt-1}=1\) and \(n\mid kt-1\)
\end{enumerate}
\end{proof}

\begin{proposition}[]
Let \(G\) be a finite group and let \(a\in G\). Then the order of \(a\) is
\(\abs{\la a\ra}\).
\end{proposition}

\begin{definition}[]
If \(G\) is a finite group, then the number of elements in \(G\), denoted by
\(\abs{G}\) is called the \tf{order} of \(G\)
\end{definition}


\begin{proposition}[]
The intersection \(\bigcap_{i\in I}H_i\) of any family of subgroups of a group
\(G\) is again a subgroup of \(G\)
\end{proposition}


\begin{corollary}[]
If \(X\) is a subset of a group \(G\), then there is a subgroup \(\la X\ra\) of \(G\)
containing \(X\) tHhat is \tf{smallest} in the sense that \(\la X\ra\le H\) for
of \(G\) that contains \(X\)
every subgroup \$\$
\end{corollary}


\begin{definition}[]
If \(X\) is a subset of a group \(G\), then \(\la X\ra\) is called the \tf\{subgroup
generated by\} \(X\)
\end{definition}

A \tf{word} on \(X\) is an element \(g\in G\) of the form \(g=x_1^{e_1}\dots
   x_n^{e_n}\) where \(x_i\in X\) and \(e_i=\pm\) for all \(i\)

\begin{proposition}[]
If \(X\) is a nonempty subset of a group \(G\), then \(\la X\ra\) is the set of all
words on \(X\)
\end{proposition}


\begin{definition}[]
If \(H\le G\) and \(a\in G\), then the \tf{coset} \(aH\) is the subset \(aH\) of \(G\),
where
\begin{equation*}
aH=\{ah:h\in H\}
\end{equation*}
\end{definition}
\(aH\) \tf{left coset}, \(Ha\) \tf{right coset}

\begin{lemma}[]
\(H\le G,a,b\in G\)
\begin{enumerate}
\item \(aH=bH\) if and only if \(b^{-1}a\in H\)
\item if \(aH\cap bH\neq\emptyset\), then \(aH=bH\)
\item \(\abs{aH}=\abs{H}\) for all \(a\in G\)
\end{enumerate}
\end{lemma}
\begin{proof}
define a relation \(a\equiv b\) if \(b^{-1}a\in H\)
\end{proof}


\begin{theorem}[Lagrange's Theorem]
If \(H\) is a subgroup of a finite group \(G\), then \(\abs{H}\) is a divisor of \(\abs{G}\)
\end{theorem}

\begin{proof}
Let \(\{a_1H,a_2H,\dots,a_tH\}\) be the family of all the distinct cosets of
\(H\) in \(G\). Then
\begin{equation*}
G=a_1H\cup a_2H\cup\dots\cup a_tH
\end{equation*}
hence
\begin{equation*}
\abs{G}=\abs{a_1H}+\dots+\abs{a_tH}
\end{equation*}
But \(\abs{a_iH}=\abs{H}\) for all \(i\). Hence \(\abs{G}=t\abs{H}\)
\end{proof}

\begin{definition}[]
The \tf{index} of a subgroup \(H\) in \(G\) denoted by \([G:H]\), is the number of
left cosets of \(H\) in \(G\)
\end{definition}

Note that \(\abs{G}=[G:H]\abs{H}\)

\begin{corollary}[]
If \(G\) is a finite group and \(a\in G\), then the order of \(a\) is a divisor of
\(\abs{G}\) 
\end{corollary}

\begin{corollary}[]
If \(G\) is a finite group, then \(a^{\abs{G}}=1\) for all \(a\in G\)
\end{corollary}

\begin{corollary}[]
If \(p\) is a prime, then every group \(G\) of order \(p\) is cyclic
\end{corollary}

\begin{proposition}[]
The set \(U(\I_m)\), defined by
\begin{equation*}
 U(\I_m)=\{[r]\in\I_m:(r,m)=1\}

\end{equation*}
is a multiplicative group of order \(\varphi(m)\). If \(p\) is a prime, then
\(U(\I_m)=\I_m^{\times}\).
\end{proposition}

\begin{corollary}[Fermat]
If \(p\) is a prime and \(a\in\Z\), then
\begin{equation*}
a^p\equiv a\mod p
\end{equation*}
\end{corollary}

\begin{proof}
suffices to show \([a^p]=[a]\) in \(\I_p\). If \([a]=[0]\), then \([a^p]=[a]^p=[0]\).
Else, since \(\abs{\I_p^\times}=p\), \([a]^{p-1}=[1]\)
\end{proof}


\begin{theorem}[Euler]
If \((r,m)=1\), then
\begin{equation*}
r^{\phi(m)}\equiv 1\mod m
\end{equation*}
\end{theorem}

\begin{theorem}[Wilson's Theorem]
An integer \(p\) is a prime if and only if
\begin{equation*}
(p-1)!\equiv -1\mod p
\end{equation*}
\end{theorem}
\end{document}