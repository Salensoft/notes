% Created 2020-01-09 四 22:15
% Intended LaTeX compiler: pdflatex
\documentclass[11pt]{article}
\usepackage[utf8]{inputenc}
\usepackage[T1]{fontenc}
\usepackage{graphicx}
\usepackage{grffile}
\usepackage{longtable}
\usepackage{wrapfig}
\usepackage{rotating}
\usepackage[normalem]{ulem}
\usepackage{amsmath}
\usepackage{textcomp}
\usepackage{amssymb}
\usepackage{capt-of}
\usepackage{hyperref}
\usepackage{minted}
% TIPS
% \substack{a\\b} for multiple lines text





% pdfplots will load xolor automatically without option
\usepackage[dvipsnames]{xcolor}

\usepackage{forest}
% two-line text in node by [two \\ lines]
% \begin{forest} qtree, [..] \end{forest}
\forestset{
  qtree/.style={
    baseline,
    for tree={
      parent anchor=south,
      child anchor=north,
      align=center,
      inner sep=1pt,
    }}}
%\usepackage{flexisym}
% load order of mathtools and mathabx, otherwise conflict overbrace

\usepackage{mathtools}
%\usepackage{fourier}
\usepackage{pgfplots}
\usepackage{amsthm}
\usepackage{amsmath}
%\usepackage{unicode-math}
%
\usepackage{commath}
%\usepackage{,  , }
\usepackage{amsfonts}
\usepackage{amssymb}
% importing symbols https://tex.stackexchange.com/questions/14386/importing-a-single-symbol-from-a-different-font
%mathabx change every symbol
% use instead stmaryrd
%\usepackage{mathabx}
\usepackage{stmaryrd}
\usepackage{empheq}
\usepackage{tikz}
\usepackage{tikz-cd}
%\usepackage[notextcomp]{stix}
\usetikzlibrary{arrows.meta}
\usepackage[most]{tcolorbox}
%\utilde
%\usepackage{../../latexpackage/undertilde/undertilde}
% left and right superscript and subscript
\usepackage{actuarialsymbol}
\usepackage{threeparttable}
\usepackage{scalerel,stackengine}
\usepackage{stackrel}
% \stackrel[a]{b}{c}
\usepackage{dsfont}
% text font
\usepackage{newpxtext}
%\usepackage{newpxmath}

%\newcounter{dummy} \numberwithin{dummy}{section}
\newtheorem{dummy}{dummy}[section]
\theoremstyle{definition}
\newtheorem{definition}[dummy]{Definition}
\newtheorem{corollary}[dummy]{Corollary}
\newtheorem{lemma}[dummy]{Lemma}
\newtheorem{proposition}[dummy]{Proposition}
\newtheorem{theorem}[dummy]{Theorem}
\theoremstyle{definition}
\newtheorem{example}[dummy]{Example}
\theoremstyle{remark}
\newtheorem*{remark}{Remark}


\newcommand\what[1]{\ThisStyle{%
    \setbox0=\hbox{$\SavedStyle#1$}%
    \stackengine{-1.0\ht0+.5pt}{$\SavedStyle#1$}{%
      \stretchto{\scaleto{\SavedStyle\mkern.15mu\char'136}{2.6\wd0}}{1.4\ht0}%
    }{O}{c}{F}{T}{S}%
  }
}

\newcommand\wtilde[1]{\ThisStyle{%
    \setbox0=\hbox{$\SavedStyle#1$}%
    \stackengine{-.1\LMpt}{$\SavedStyle#1$}{%
      \stretchto{\scaleto{\SavedStyle\mkern.2mu\AC}{.5150\wd0}}{.6\ht0}%
    }{O}{c}{F}{T}{S}%
  }
}

\newcommand\wbar[1]{\ThisStyle{%
    \setbox0=\hbox{$\SavedStyle#1$}%
    \stackengine{.5pt+\LMpt}{$\SavedStyle#1$}{%
      \rule{\wd0}{\dimexpr.3\LMpt+.3pt}%
    }{O}{c}{F}{T}{S}%
  }
}

\newcommand{\bl}[1] {\boldsymbol{#1}}
\newcommand{\Wt}[1] {\stackrel{\sim}{\smash{#1}\rule{0pt}{1.1ex}}}
\newcommand{\wt}[1] {\widetilde{#1}}
\newcommand{\tf}[1] {\textbf{#1}}


%For boxed texts in align, use Aboxed{}
%otherwise use boxed{}

\DeclareMathSymbol{\widehatsym}{\mathord}{largesymbols}{"62}
\newcommand\lowerwidehatsym{%
  \text{\smash{\raisebox{-1.3ex}{%
    $\widehatsym$}}}}
\newcommand\fixwidehat[1]{%
  \mathchoice
    {\accentset{\displaystyle\lowerwidehatsym}{#1}}
    {\accentset{\textstyle\lowerwidehatsym}{#1}}
    {\accentset{\scriptstyle\lowerwidehatsym}{#1}}
    {\accentset{\scriptscriptstyle\lowerwidehatsym}{#1}}
}

\usepackage{graphicx}
    
% text on arrow for xRightarrow
\makeatletter
%\newcommand{\xRightarrow}[2][]{\ext@arrow 0359\Rightarrowfill@{#1}{#2}}
\makeatother


\newcommand{\dom}[1]{%
\mathrm{dom}{(#1)}
}

% Roman numerals
\makeatletter
\newcommand*{\rom}[1]{\expandafter\@slowromancap\romannumeral #1@}
\makeatother

\def \fR {\mathfrak{R}}
\def \bx {\boldsymbol{x}}
\def \bz {\boldsymbol{z}}
\def \ba {\boldsymbol{a}}
\def \bh {\boldsymbol{h}}
\def \bo {\boldsymbol{o}}
\def \bU {\boldsymbol{U}}
\def \bc {\boldsymbol{c}}
\def \bV {\boldsymbol{V}}
\def \bI {\boldsymbol{I}}
\def \bK {\boldsymbol{K}}
\def \bt {\boldsymbol{t}}
\def \bb {\boldsymbol{b}}
\def \bA {\boldsymbol{A}}
\def \bX {\boldsymbol{X}}
\def \bu {\boldsymbol{u}}
\def \bS {\boldsymbol{S}}
\def \bZ {\boldsymbol{Z}}
\def \bz {\boldsymbol{z}}
\def \by {\boldsymbol{y}}
\def \bw {\boldsymbol{w}}
\def \bT {\boldsymbol{T}}
\def \bF {\boldsymbol{F}}
\def \bS {\boldsymbol{S}}
\def \bm {\boldsymbol{m}}
\def \bW {\boldsymbol{W}}
\def \bR {\boldsymbol{R}}
\def \bQ {\boldsymbol{Q}}
\def \bS {\boldsymbol{S}}
\def \bP {\boldsymbol{P}}
\def \bT {\boldsymbol{T}}
\def \bY {\boldsymbol{Y}}
\def \bH {\boldsymbol{H}}
\def \bB {\boldsymbol{B}}
\def \blambda {\boldsymbol{\lambda}}
\def \bPhi {\boldsymbol{\Phi}}
\def \btheta {\boldsymbol{\theta}}
\def \bTheta {\boldsymbol{\Theta}}
\def \bmu {\boldsymbol{\mu}}
\def \bphi {\boldsymbol{\phi}}
\def \bSigma {\boldsymbol{\Sigma}}
\def \lb {\left\{}
\def \rb {\right\}}
\def \la {\langle}
\def \ra {\rangle}
\def \caln {\mathcal{N}}
\def \dissum {\displaystyle\Sigma}
\def \dispro {\displaystyle\prod}
\def \E {\mathbb{E}}
\def \Q {\mathbb{Q}}
\def \N {\mathbb{N}}
\def \V {\mathbb{V}}
\def \R {\mathbb{R}}
\def \P {\mathbb{P}}
\def \A {\mathbb{A}}
\def \Z {\mathbb{Z}}
\def \I {\mathbb{I}}
\def \C {\mathbb{C}}
\def \cala {\mathcal{A}}
\def \calb {\mathcal{B}}
\def \calq {\mathcal{Q}}
\def \calp {\mathcal{P}}
\def \cals {\mathcal{S}}
\def \calg {\mathcal{G}}
\def \caln {\mathcal{N}}
\def \calr {\mathcal{R}}
\def \calm {\mathcal{M}}
\def \calc {\mathcal{C}}
\def \calf {\mathcal{F}}
\def \calk {\mathcal{K}}
\def \call {\mathcal{L}}
\def \calu {\mathcal{U}}
\def \bcup {\bigcup}


\def \uin {\underline{\in}}
\def \oin {\overline{\in}}
\def \uR {\underline{R}}
\def \oR {\overline{R}}
\def \uP {\underline{P}}
\def \oP {\overline{P}}

\def \Ra {\Rightarrow}

\def \e {\enspace}

\def \sgn {\operatorname{sgn}}
\def \gen {\operatorname{gen}}
\def \ker {\operatorname{ker}}
\def \im {\operatorname{im}}

\def \tril {\triangleleft}

% \varprod
\DeclareSymbolFont{largesymbolsA}{U}{txexa}{m}{n}
\DeclareMathSymbol{\varprod}{\mathop}{largesymbolsA}{16}

% \bigtimes
\DeclareFontFamily{U}{mathx}{\hyphenchar\font45}
\DeclareFontShape{U}{mathx}{m}{n}{
      <5> <6> <7> <8> <9> <10>
      <10.95> <12> <14.4> <17.28> <20.74> <24.88>
      mathx10
      }{}
\DeclareSymbolFont{mathx}{U}{mathx}{m}{n}
\DeclareMathSymbol{\bigtimes}{1}{mathx}{"91}
% \odiv
\DeclareFontFamily{U}{matha}{\hyphenchar\font45}
\DeclareFontShape{U}{matha}{m}{n}{
      <5> <6> <7> <8> <9> <10> gen * matha
      <10.95> matha10 <12> <14.4> <17.28> <20.74> <24.88> matha12
      }{}
\DeclareSymbolFont{matha}{U}{matha}{m}{n}
\DeclareMathSymbol{\odiv}         {2}{matha}{"63}


\newcommand\subsetsim{\mathrel{%
  \ooalign{\raise0.2ex\hbox{\scalebox{0.9}{$\subset$}}\cr\hidewidth\raise-0.85ex\hbox{\scalebox{0.9}{$\sim$}}\hidewidth\cr}}}
\newcommand\simsubset{\mathrel{%
  \ooalign{\raise-0.2ex\hbox{\scalebox{0.9}{$\subset$}}\cr\hidewidth\raise0.75ex\hbox{\scalebox{0.9}{$\sim$}}\hidewidth\cr}}}

\newcommand\simsubsetsim{\mathrel{%
  \ooalign{\raise0ex\hbox{\scalebox{0.8}{$\subset$}}\cr\hidewidth\raise1ex\hbox{\scalebox{0.75}{$\sim$}}\hidewidth\cr\raise-0.95ex\hbox{\scalebox{0.8}{$\sim$}}\cr\hidewidth}}}
\newcommand{\stcomp}[1]{{#1}^{\mathsf{c}}}


\author{Joseph J. Rotman}
\date{\today}
\title{Advanced Modern Algebra}
\hypersetup{
 pdfauthor={Joseph J. Rotman},
 pdftitle={Advanced Modern Algebra},
 pdfkeywords={},
 pdfsubject={},
 pdfcreator={Emacs 26.3 (Org mode 9.3)}, 
 pdflang={English}}
\begin{document}

\maketitle
\tableofcontents \clearpage\section{Things Past}
\label{sec:org842d07f}
\subsection{Roots of Unity}
\label{sec:orgebaa42a}
\begin{proposition}[Polar Decomposition]
Every complex number \(z\) has a factorization
\begin{equation*}
z=r(\cos\theta+i\sin\theta)
\end{equation*}
where \(r=\abs{z}\ge0\) and \(0\le\theta\le 2\pi\)
\end{proposition}

\begin{proposition}[Addition Theorem]
If \(z=\cos\theta+i\sin\theta\) and \(w=\cos\psi+i\sin\psi\), then
\begin{equation*}
zw=\cos(\theta+\psi)+i\sin(\theta+\psi)
\end{equation*}
\end{proposition}

\begin{theorem}[De Moivre]
\(\forall x\in\R,n\in\N\)
\begin{equation*}
\cos(nx)+i\sin(nx)=(\cos x+i\sin x)^n
\end{equation*}
\end{theorem}

\begin{theorem}[Euler]
\(e^{ix}=\cos x+i\sin x\)
\end{theorem}

\begin{definition}[]
If \(n\in\N\ge 1\) , an \textbf{nth root of unity} is a complex number \(\xi\) with
\(\xi^n=1\)
\end{definition}

\begin{corollary}[]
Every nth root of unity is equal to
\begin{equation*}
e^{2\pi ik/n}=\cos(\frac{2\pi k}{n})+i\sin(\frac{2\pi k}{n})
\end{equation*}
for \(k=0,1,\dots,n-1\)
\end{corollary}

\begin{equation*}
x^n-1=\displaystyle\prod_{\xi^n=1}(x-\xi)
\end{equation*}

If \(\xi\) is an nth root of unity and if \(n\) is the smallest, then \(\xi\) is a
\textbf{primitive} \(n\)\tf{th root of unity}

\begin{definition}[]
If \(d\in\N^+\) , then the \$d\$th \textbf{cyclotomic polynomial} is 
\begin{equation*}
\Phi_d(x)=\displaystyle\prod(x-\xi)
\end{equation*}
where \(\xi\) ranges over all the \emph{primitive dth} roots of unity
\end{definition}

\begin{proposition}[]
For every integer \(n\ge 1\)
\begin{equation*}
x^n-1=\displaystyle\prod_{d|n}\Phi_d(x)
\end{equation*}
\end{proposition}

\begin{definition}[]
Define \textbf{Euler\} \(\phi\)\tf\{-function} as the degree of the nth cyclotomic
polynomial
\begin{equation*}
\phi(n)=\deg(\Phi_n(x))
\end{equation*}
\end{definition}

\begin{proposition}[]
If \(n\ge1\) is an integer, then \(\phi(n)\) is the number of integers \(k\) with
\(1\le k\le n\) and \((k,n)=1\)
\end{proposition}

\begin{proof}
Suffice to prove \(e^{2\pi ik/n}\) is a primitive nth root of unity if and only
if \(k\) and \(n\) are relatively prime
\end{proof}

\begin{corollary}[]
For every integer \(n\ge 1\), we have
\begin{equation*}
n=\displaystyle\sum_{d|n}\phi(d)
\end{equation*}
\end{corollary}
\section{Group \rom{1}}
\label{sec:org63d0543}
\subsection{Permutations}
\label{sec:orgfd6e395}
\begin{definition}[]
A \textbf{permutation} of a set \(X\) is a bijection from \(X\) to itself.
\end{definition}


\begin{definition}[]
The family of all the permutations of a set \(X\), denoted by \(S_X\) is called
the \textbf{symmetric group} on \(X\). When \(X=\lb 1,2,\dots,n\rb\), \(S_X\) is
usually denoted by \(X_n\) and is called the \textbf{symmetric group on } \(n\)
\textbf{letters} 
\end{definition}

\begin{definition}[]
Let \(i_1,i_2,\dots,i_r\) be distinct integers in \(\lb 1,2,\dots,n\rb\). If
\(\alpha\in S_n\) fixes the other integers and if
\begin{equation*}
\alpha(i_1)=i_2,\alpha(i_2)=i_3,\dots,\alpha(i_{r-1})=i_r,\alpha(i_r)=i_1
\end{equation*}
then \(\alpha\) is called an textbf\{r-cycle\}. \(\alpha\) is a cycle of
\textbf{length} \(r\) and denoted by
\begin{equation*}
\alpha=(i_1\; i_2\;\dots\; i_r)
\end{equation*}
\end{definition}

2-cycles are also called the \textbf{transpositions}.

\begin{definition}[]
Two permutations \(\alpha,\beta\in S_n\) are \textbf{disjoint} if every \(i\)
moved by one is fixed by the other.
\end{definition}

\begin{lemma}[]
Disjoint permutations \(\alpha,\beta\in S_n\) commute
\end{lemma}

\begin{proposition}[]
Every permutation \(\alpha\in S_n\) is either a cycle or a product of disjoint cycles.
\end{proposition}

\begin{proof}
Induction on the number \(k\) of points moved by \(\alpha\)
\end{proof}

\begin{definition}[]
A \textbf{complete factorization} of a permutation \(\alpha\) is a
factorization of \(\alpha\) into disjoint cycles that contains exactly one
1-cycle \((i)\) for every \(i\) fixed by \(\alpha\)
\end{definition}

\begin{theorem}[]
Let \(\alpha\in S_n\) and let \(\alpha=\beta_1\dots\beta_t\) be a complete
factorization into disjoint cycles. This factorization is unique except for
the order in which the cycles occur
\end{theorem}

\begin{proof}
for all \(i\), if \(\beta_t(i)\neq i\), then \(\beta_t^k(i)\neq\beta_t^{k-1}(i)\)
for any \(k\ge 1\)
\end{proof}

\begin{lemma}[]
If \(\gamma,\alpha\in S_n\), then \(\alpha\gamma\alpha^{-1}\) has the same cycle
structure as \(\gamma\). In more detail, if the complete factorization of
\(\gamma\) is
\begin{equation*}
\gamma=\beta_1\beta_2\dots(i_1\; i_2\;\dots)\dots\beta_t
\end{equation*}
then \(\alpha\gamma\alpha^{-1}\) is permutation that is obtained from \(\gamma\)
by applying \(\alpha\) to the symbols in the cycles of \(\gamma\)
\end{lemma}

Example. Suppose
\begin{gather*}
\beta=(1\;2\;3)(4)(5)\\
\gamma=(5\;2\;4)(1)(3)
\end{gather*}
then we can easily find the \(\alpha\)
\begin{equation*}
\alpha=
\begin{pmatrix}
1&2&3&4&5\\
5&2&4&1&3
\end{pmatrix}
\end{equation*}
\begin{theorem}[]
Permutations \(\gamma\) and \(\sigma\) in \(S_n\) has the same cycle structure if
and only if there exists \(\alpha\in S_n\) with \(\sigma=\alpha\gamma\alpha^{-1}\)
\end{theorem}


\begin{proposition}[]
If \(n\ge 2\) then every \(\alpha\in S_n\) is a product of tranpositions
\end{proposition}
\begin{proof}
\((1\;2\;\dots\; r)=(1\; r)(1\; r-1)\dots(1\; 2)\)
\end{proof}


\begin{definition}[]
A permutation \(\alpha\in S_n\) is \textbf{even} if it can be factored into a
product of an even number of transpositions. Otherwise \textbf{odd}
\end{definition}

\begin{definition}[]
If \(\alpha\in S_n\) and \(\alpha=\beta_1\dots\beta_t\) is a complete
factorization, then \textbf{signum} \(\alpha\) is defined by
\begin{equation*}
\sgn(\alpha)=(-1)^{n-t}
\end{equation*}
\end{definition}

\begin{theorem}[]
For all \(\alpha,\beta\in S_n\)
\begin{equation*}
\sgn(\alpha\beta)=\sgn(\alpha)\sgn(\beta)
\end{equation*}
\end{theorem}

\begin{theorem}[]
\begin{enumerate}
\item Let \(\alpha\in S_n\); if \(\sgn(\alpha)=1\) then \(\alpha\) is even. otherwise
odd
\item A permutation \(\alpha\) is odd if and only if it's a product of an odd
number of transpositions
\end{enumerate}
\end{theorem}

\begin{corollary}[]
Let \(\alpha,\beta\in S_n\). If \(\alpha\) and \(\beta\) have the same parity, then
\(\alpha\beta\) is even while if \(\alpha\) and \(\beta\) have distinct parity,
\(\alpha\beta\) is odd
\end{corollary}
\subsection{Groups}
\label{sec:orga41ba89}
\begin{definition}[]
A \textbf{binary operation} on a set \(G\) is a function
\begin{equation*}
*:G\times G\to G
\end{equation*}
\end{definition}

\begin{definition}[]
A \textbf{group} is a set \(G\) equipped with a binary operation * s.t.
\begin{enumerate}
\item the \textbf{associative law} holds
\item \textbf{identity}
\item every \(x\in G\) has an \textbf{inverse}, there is a \(x'\in G\)  with 
\(x*x'=e=x'*x\)
\end{enumerate}
\end{definition}

\begin{definition}[]
A group \(G\) is called \textbf{abelian} if it satisfies the
\textbf{commutative law}
\end{definition}

\begin{lemma}[]
Let \(G\) be a group
\begin{enumerate}
\item The \textbf{cancellation laws} holds: if either \(x*a=x*b\) or \(a*x=b*x\), then
\(a=b\)
\item \(e\) is unique
\item Each \(x\in G\) has a unique inverse
\item \((x^{-1})^{-1}=x\)
\end{enumerate}
\end{lemma}

\begin{definition}[]
An expression \(a_1a_2\dots a_n\) \textbf{needs no parentheses} if all the ultimate
products it yields are equal
\end{definition}

\begin{theorem}[Generalized Associativity]
If \(G\) is a group and \(a_1,a_2,\dots,a_n\in G\) then the expression
\(a_1a_2\dots a_n\) needs no parentheses
\end{theorem}

\begin{definition}[]
Let \(G\) be a group and let \(a\in G\). If \(a^k=1\) for some \(k>1\) then the
smallest such exponent \(k\ge 1\) is called the \textbf{order} or \(a\); if no such
power exists, then one says that \(a\) has \textbf{infinite order}
\end{definition}

\begin{proposition}[]
If \(G\) is a finite group, then every \(x\in G\) has finite order
\end{proposition}

\begin{definition}[]
A \textbf{motion} is a distance preserving bijection \(\varphi:\R^2\to\R^2\). If
\(\pi\) is a polygon in the plane, then its \textbf{symmetry group} \(\Sigma(\pi)\)
consists of all the motions \(\varphi\) for which \(\varphi(\pi)=\pi\). The
elements of \(\Sigma(\pi)\) are called the \textbf{symmetries} of \(\pi\)
\end{definition}

Let \(\pi_4\) be a square. Then the group \(\Sigma(\pi_4)\) is called the
\textbf{dihedral group} with 8 elements, denoted by \(D_8\)

\begin{definition}[]
If \(\pi_n\) is a regular polygon with \(n\) vertices \(v_1,\dots,v_n\) and center
\(O\), then the symmetry group \(\Sigma(\pi_n)\) is called the \tf\{dihedral
group\} with \(2n\) elements, and it's denoted by \(D_{2n}\)
\end{definition}
\subsection{Lagrange's theorem}
\label{sec:org82b75b6}
\begin{definition}[]
A subset \(H\) of a group \(G\) is a \textbf{subgroup} if
\begin{enumerate}
\item \(1\in H\)
\item if \(x,y\in H\), then \(xy\in H\)
\item if \(x\in H\), then \(x^{-1}\in H\)
\end{enumerate}
\end{definition}

If \(H\) is a subgroup of \(G\), we write \(H\le G\). If \(H\) is a proper subgroup,
then we write \(H<G\)

The four permutations
\begin{equation*}
\bV=\{(1),(1 2)(3 4),(1 3)(2 4),(1 4)(2 3)\}
\end{equation*}
form a group because \(\bV\le S_4\)

\begin{proposition}[]
A subset \(H\) of a group \(G\) is a subgroup if and only if \(H\) is nonempty and
whenever \(x,y\in H\), \(xy^{-1}\in H\)
\end{proposition}

\begin{proposition}[]
A nonempty subset \(H\) of a finite group \(G\) is a subgroup if and only if \(H\)
is closed; that is, if \(a,b\in H\), then \(ab\in H\)
\end{proposition}

\begin{definition}[]
If \(G\) is a group and \(a\in G\)
\begin{equation*}
\langle a\rangle=\{a^n:n\in\Z\}=\{\text{all powers of } a\}
\end{equation*}
\(\la a\ra\) is called the \textbf{cyclic subgroup} of \(G\) \textbf{generated} by \(a\). A
group \(G\) is called \textbf{cyclic} if there exists \(a\in G\) s.t. \(G=\la a\ra\),
in which case \(a\) is called the \textbf{generator}
\end{definition}

\begin{definition}[]
The \textbf{integers mod \(m\)}, denoted by \(\I_m\) is the family of all congruence
classes mod \(m\)
\end{definition}


\begin{proposition}[]
Let \(m\ge 2\) be a fixed integer
\begin{enumerate}
\item If \(a\in \Z\), then \([a]=[r]\) for some \(r\) with \(0\le r<m\)
\item If \(0\le r'<r<m\), then \([r']\neq[r]\)
\item \(\I_m\) has exactly \(m\) elements
\end{enumerate}
\end{proposition}

\begin{theorem}[]
\begin{enumerate}
\item If \(G=\la a\ra\) is a cyclic group of order \(n\), then \(a^k\) is a generator
of \(G\) if and only if \((k,n)=1\)
\item If \(G\) is a cyclic group of order \(n\) and \(\gen(G)=\{\text{all generators
      of } G\}\), then
\begin{equation*}
\abs{\gen(G)}=\phi(n)
\end{equation*}
where \(\phi\) is the Euler \(\phi\)-function
\end{enumerate}
\end{theorem}
\begin{proof}
\begin{enumerate}
\item there is \(t\in\N\) s.t. \(a^{kt}=a\) hence \(a^{kt-1}=1\) and \(n\mid kt-1\)
\end{enumerate}
\end{proof}

\begin{proposition}[]
Let \(G\) be a finite group and let \(a\in G\). Then the order of \(a\) is
\(\abs{\la a\ra}\).
\end{proposition}

\begin{definition}[]
If \(G\) is a finite group, then the number of elements in \(G\), denoted by
\(\abs{G}\) is called the \textbf{order} of \(G\)
\end{definition}


\begin{proposition}[]
The intersection \(\bigcap_{i\in I}H_i\) of any family of subgroups of a group
\(G\) is again a subgroup of \(G\)
\end{proposition}


\begin{corollary}[]
If \(X\) is a subset of a group \(G\), then there is a subgroup \(\la X\ra\) of \(G\)
containing \(X\) tHhat is \textbf{smallest} in the sense that \(\la X\ra\le H\) for
every subgroup \(H\) 
of \(G\) that contains \(X\)
\end{corollary}


\begin{definition}[]
If \(X\) is a subset of a group \(G\), then \(\la X\ra\) is called the \textbf{subgroup}
\textbf{generated by} \(X\)
\end{definition}

A \textbf{word} on \(X\) is an element \(g\in G\) of the form \(g=x_1^{e_1}\dots
   x_n^{e_n}\) where \(x_i\in X\) and \(e_i=\pm 1\) for all \(i\)

\begin{proposition}[]
If \(X\) is a nonempty subset of a group \(G\), then \(\la X\ra\) is the set of all
words on \(X\)
\end{proposition}


\begin{definition}[]
If \(H\le G\) and \(a\in G\), then the \textbf{coset} \(aH\) is the subset \(aH\) of \(G\),
where
\begin{equation*}
aH=\{ah:h\in H\}
\end{equation*}
\end{definition}
\(aH\) \textbf{left coset}, \(Ha\) \textbf{right coset}

\begin{lemma}[]
\(H\le G,a,b\in G\)
\begin{enumerate}
\item \(aH=bH\) if and only if \(b^{-1}a\in H\)
\item if \(aH\cap bH\neq\emptyset\), then \(aH=bH\)
\item \(\abs{aH}=\abs{H}\) for all \(a\in G\)
\end{enumerate}
\end{lemma}
\begin{proof}
define a relation \(a\equiv b\) if \(b^{-1}a\in H\)
\end{proof}


\begin{theorem}[Lagrange's Theorem]
If \(H\) is a subgroup of a finite group \(G\), then \(\abs{H}\) is a divisor of \(\abs{G}\)
\end{theorem}

\begin{proof}
Let \(\{a_1H,a_2H,\dots,a_tH\}\) be the family of all the distinct cosets of
\(H\) in \(G\). Then
\begin{equation*}
G=a_1H\cup a_2H\cup\dots\cup a_tH
\end{equation*}
hence
\begin{equation*}
\abs{G}=\abs{a_1H}+\dots+\abs{a_tH}
\end{equation*}
But \(\abs{a_iH}=\abs{H}\) for all \(i\). Hence \(\abs{G}=t\abs{H}\)
\end{proof}

\begin{definition}[]
The \textbf{index} of a subgroup \(H\) in \(G\) denoted by \([G:H]\), is the number of
left cosets of \(H\) in \(G\)
\end{definition}

Note that \(\abs{G}=[G:H]\abs{H}\)

\begin{corollary}[]
If \(G\) is a finite group and \(a\in G\), then the order of \(a\) is a divisor of
\(\abs{G}\) 
\end{corollary}

\begin{corollary}[]
If \(G\) is a finite group, then \(a^{\abs{G}}=1\) for all \(a\in G\)
\end{corollary}

\begin{corollary}[]
If \(p\) is a prime, then every group \(G\) of order \(p\) is cyclic
\end{corollary}

\begin{proposition}[]
The set \(U(\I_m)\), defined by
\begin{equation*}
 U(\I_m)=\{[r]\in\I_m:(r,m)=1\}
\end{equation*}
is a multiplicative group of order \(\phi(m)\). If \(p\) is a prime, then
\(U(\I_p)=\I_p^{\times}\), the nonzero elements of \(\I_p\).
\end{proposition}

\begin{proof}
\((r,m)=1=(r',m)\) implies \((rr',m)=1\). Hence \(U(\I_m)\) is closed under
multiplication. If \((x,m)=1\), then \(rs+sm=1\). There fore \((r,m)=1\). Each of
them have inverse.
\end{proof}

\begin{corollary}[Fermat]
If \(p\) is a prime and \(a\in\Z\), then
\begin{equation*}
a^p\equiv a\mod p
\end{equation*}
\end{corollary}

\begin{proof}
suffices to show \([a^p]=[a]\) in \(\I_p\). If \([a]=[0]\), then \([a^p]=[a]^p=[0]\).
Else, since \(\abs{\I_p^\times}=p-1\), \([a]^{p-1}=[1]\)
\end{proof}


\begin{theorem}[Euler]
If \((r,m)=1\), then
\begin{equation*}
r^{\phi(m)}\equiv 1\mod m
\end{equation*}
\end{theorem}
\begin{proof}
Since \(\abs{U(\I_m)}=\phi(m)\). Lagrange's theorem gives
\([r]^{\phi(m)}=[1]\) for all \([r]\in U(\I_m)\).

In fact we construct a group to prove this.
\end{proof}

\begin{theorem}[Wilson's Theorem]
An integer \(p\) is a prime if and only if
\begin{equation*}
(p-1)!\equiv -1\mod p
\end{equation*}
\end{theorem}

\begin{proof}
Assume that \(p\) is a prime. If \(a_1,\dots,a_n\) is a list of all the elements
of finite abelian group, then product \(a_1a_2\dots a_n\) is the same as the
product of all elements \(a\) with \(a^2=1\). Since \(p\) is prime, \(\I_p^\times\)
has only one element of order 2, namely \([-1]\). It follows that the product
of all the elements in \(\I_p^\times\) namely \([(p-1)!]\) is equal to \([-1]\).

Conversly assume that \(m\) is composite: there are integers \(a\) and \(b\) with
\(m=ab\) and \(1<a\le b<m\). If \(a<b\) then \(m=ab\) is a divisor of \((m-1)!\). If
\(a=b\), then \(m=a^2\). if \(a=2\), then \((a^2-1)!\equiv 2\mod 4\). If \(2<a\), then
\(2a<a^2\) and so \(a\) and \(2a\) are factors of \((a^2-1)!\)
\end{proof}
\subsection{Homomorphisms}
\label{sec:orgec15e46}
\begin{definition}[]
If \((G,*)\) and \((H,\circ)\) are groups, then a function \(f:G\to H\) is a
\textbf{homomorphism} if
\begin{equation*}
f(x*y)=f(x)\circ f(y)
\end{equation*}
for all \(x,y\in G\). If \(f\) is also a bijection, then \(f\) is called an
\textbf{isomorphism}. \(G\) and \(H\) are called \textbf{isomorphic}, denoted by \(G\cong H\)
\end{definition}

\begin{lemma}[]
Let \(f:G\to H\) be a homomorphism
\begin{enumerate}
\item \(f(1)=1\)
\item \(f(x^{-1})=f(x)^{-1}\)
\item \(f(x^n)=f(x)^n\) for all \(n\in\Z\)
\end{enumerate}
\end{lemma}



\begin{definition}[]
If \(f:G\to H\) is a homomorphism, define
\begin{equation*}
\ker f=\{x\in G:f(x)=1
\end{equation*}
and
\begin{equation*}
\im f=\{h\in H:h=f(x)\text{ for some } x\in G\
\end{equation*}
\end{definition}

\begin{proposition}[]
Let \(f:G\to H\) be a homomorphism
\begin{enumerate}
\item \(\ker f\) is a subgroup of \(G\) and \(\im f\) is a subgroup of \(H\)
\item if \(x\in\ker f\) and if \(a\in G\), then \(axa^{-1}\in\ker f\)
\item \(f\) is an injection if and only if \(\ker f=\{1\}\)
\end{enumerate}
\end{proposition}

\begin{proof}
\begin{enumerate}
\setcounter{enumi}{2}
\item \(f(a)=f(b)\Leftrightarrow f(ab^{-1})=1\)
\end{enumerate}
\end{proof}

\begin{definition}[]
A subgroup \(K\) of a group \(G\) is called a \textbf{normal subgroup} if \(k\in K\)
and \(g\in G\) imply \(gkg^{-1}\in K\), denoted by \(K\triangleleft G\)
\end{definition}

\begin{definition}[]
If \(G\) is a group and \(a\in G\), then a \textbf{conjugate} of \(a\) is any element
in \(G\) of the form
\begin{equation*}
gag^{-1}
\end{equation*}
where \(g\in G\)
\end{definition}

\begin{definition}[]
If \(G\) is a group and \(g\in G\), define \textbf{conjugation} \(\gamma_g:G\to G\) by
\begin{equation*}
\gamma_g(a)=gag^{-1}
\end{equation*}
for all \(a\in G\)
\end{definition}

\begin{proposition}[]
\begin{enumerate}
\item If \(G\) is a group and \(g\in G\), then conjugation \(\gamma_g:G\to G\) is an
isomorphism
\item Conjugate elements have the same order
\end{enumerate}
\end{proposition}

\begin{proof}
\begin{enumerate}
\item bijection: \(\gamma_g\circ\gamma_{g^{-1}}=1=\gamma_{g^{-1}}\circ\gamma_g\).
\end{enumerate}
\end{proof}

\begin{examplle}[]
Define the \textbf{center} of a group \(G\), denoted by \(Z(G)\), to be
\begin{equation*}
Z(G)=\{z\in G:zg=gz\text{ for all }g\in G\}
\end{equation*}
\end{examplle}

\begin{examplle}[]
If \(G\) is a group, then an \textbf{automorphism} of \(G\) is an isomorphism \(f:G\to G\).
For example, every conjugation \(\gamma_g\) is an automorphism of \(G\) (it is
called an \textbf{inner automorphism}), for its inverse is conjugation by \(g^{-1}\).
The set \(\aut(G)\) of all the automorphism of \(G\) is itself a group.
\begin{equation*}
\inn(G)=\{\gamma_g:g\in G\}
\end{equation*}
is a subgroup of \(\aut(G)\)
\end{examplle}
\begin{proposition}[]
\begin{enumerate}
\item If \(H\) is a subgroup of index 2 in a group \(G\), then \(g^2\in H\) for every
\(g\in G\)
\item If \(H\) is a subgroup of index 2 in a group \(G\), then \(H\) is a normal
subgroup of \(G\)
\end{enumerate}
\end{proposition}


\begin{definition}[]
The group of \textbf{quaternions} is the group \(\bQ\) of order 8 consisting of the
following matrices in \(GL(2, \C)\)
\begin{equation*}
\bQ=\{I,A,A^2,A^3,B,BA,BA^2,BA^3\}
\end{equation*}
where \(I\) is the identity matrix
\begin{equation*}
A=
\begin{pmatrix}
0&1\\
-1&0
\end{pmatrix}, \text{ and }
B=\begin{pmatrix}
0&i\\
i&0
  \end{pmatrix}
\end{equation*}
\end{definition}

\begin{examplle}[]
\(\bQ\) is normal. By Lagrange's theorem the only possible orders of subgroups
are 1,2,4 or 8. The only subgroup of order 2 is \(\la -I\ra\) since \(-I\) is the
only element of order 2
\end{examplle}
\begin{proposition}[]
The alternating group \(A_4\) is a group of order 12 having no subgroup of
order 6
\end{proposition}
\subsection{Quotient group}
\label{sec:org98add67}
\(\cals(G)\) is the set of all nonempty subsets of a group \(G\). If
\(X,Y\in\cals(G)\), define
\begin{equation*}
XY=\{xy:x\in X\text{ and } y\in Y\}
\end{equation*}

\begin{lemma}[]
\(K\le G\) is normal if and only if
\begin{equation*}
gK=Kg
\end{equation*}
\end{lemma}

A natural question is that whether \(HK\) is a subgroup when \(H\) and \(K\) are
subgroups. The answer is no. Let \(G=S_3,H=\la(1\;2)\ra,K=\la(1\;3)\ra\)


\begin{proposition}[]
\begin{enumerate}
\item If \(H\) and \(K\) are subgroups of a group \(G\), and if one of them is normal,
then \(HK\le G\) and \(HK=KH\)
\item If \(H,K\tril G\), then \(HK\tril G\)
\end{enumerate}
\end{proposition}

\begin{theorem}[]
Let \(G/K\) denote the family of all the left cosets of a subgroup \(K\) of \(G\).
If \(K\tril G\), then
\begin{equation*}
aKbK=abK
\end{equation*}
for all \(a,b\in G\) and \(G/K\) is a group under this operation
\end{theorem}

\begin{proof}
\(aKbK=abKK=abK\)
\end{proof}

\(G/K\) is called the \textbf{quotient group} \(G\) mod \(K\)

\begin{corollary}[]
Every \(K\tril G\) is the kernel of some homomorphism
\end{corollary}

\begin{proof}
Define the \textbf{natural map} \(\pi:G\to G/K\), \(a\mapsto aK\)
\end{proof}

\begin{theorem}[First Isomorphism Theorem]
If \(f:G\to H\) is a homomorphism, then
\begin{equation*}
\ker f\tril G\quad\text{ and }\quad G/\ker f\cong\im f
\end{equation*}
If \(\ker f=K\) and \(\varphi:G/K\to\im f\le H,aK\mapsto f(a)\), then \(\varphi\)
is an isomorphism
\end{theorem}

\begin{remark}
\begin{center}
\begin{tikzcd}
G \arrow[rr,"f"] \arrow[dr,"\pi"]& &
H\\ 
&G/K \arrow[ur,"\varphi"]&
\end{tikzcd}
\end{center}
\end{remark}

\begin{examplle}[]
What's the quotient group \(\R/\Z\)? Define \(f:\R\to S^1\) where \(S^1\) is the
circle group by
\begin{equation*}
f:x\mapsto e^{2\pi ix}
\end{equation*}
\(\R/\Z\cong S^1\)
\end{examplle}

\begin{proposition}[Product Formula]
If \(H\) and \(K\) are subgroups of a finite group \(G\), then
\begin{equation*}
\abs{HK}\abs{H\cap K}=\abs{H}\abs{K}
\end{equation*}
\end{proposition}

\begin{proof}
Define a function \(f:H\times K\to HK,(h,k)\mapsto hk\). Show that
\(\abs{f^{-1}(x)}=\abs{H\cap K}\). 

Claim that if \(x=hk\), then
\begin{equation*}
f^{-1}(x)=\{(hd,d^{-1}k):d\in H\cap K\}
\end{equation*}
\end{proof}

\begin{theorem}[Second Isomorphism Theorem]
If \(H\tril G, K\le G\), then \(HK\le G,H\cap K\tril G\) and
\begin{equation*}
K/(H\cap K)\cong HK/H
\end{equation*}
\end{theorem}

\begin{proof}
\(hkH=kk^{-1}hkH=kh'H=kH\)
\end{proof}

\begin{theorem}[Third Isomorphism Theorem]
If \(H,K\tril G\) with \(K\le H\), then \(H/K\tril G/K\) and
\begin{equation*}
(G/K)/(H/K)\cong G/H
\end{equation*}
\end{theorem}

\begin{theorem}[Correspondence Theorem]
If \(K\tril G, \pi:G\to G/K\) is the natural map, then
\begin{equation*}
S\mapsto \pi(S)=S/K
\end{equation*}
is a bijection between \(Sub(G;K)\), the family of all those subgroups \(S\) of
\(G\) that contain \(K\), and \(Sub(G/K)\), the family of all the subgroups of
\(G/K\). If we denote \(S/K\) by \(S^*\), then
\begin{enumerate}
\item \(T\le S\le G\) if and only if \(T^*\le S^*\), in which case \([S:T]=[S^*:T^*]\)
\item \(T\tril S\) if and only if \(T^*\tril S^*\), in which case \(S/T\cong S^*/T^*\)
\end{enumerate}
\end{theorem}

\begin{center}
\begin{tikzcd}
G \arrow[d,dash] \arrow[rd]&\\
S \arrow[d,dash] \arrow[rd] & G/K \arrow[d,dash]\\
T \arrow[d,dash] \arrow[rd] & S/K=S^* \arrow[d,dash]\\
K  \arrow[rd] & T/K=T^* \arrow[d,dash]\\
& \{1\}
\end{tikzcd}
\end{center}

\begin{proof}
Use \(\pi^{-1}\pi=1\) and \(\pi\pi^{-1}=1\) to prove injectivity and surjectivity
respectively. 

For \([S:T]=[S^*:T^*]\), show there is a bijection between the family of all
cosets of the form \(sT\) and the family of all the cosets of the form
\(s^*T^*\).

injective:
\begin{align*}
\pi(m)T^*=\pi(n)T^*&\Leftrightarrow \pi(m)\pi(n)^{-1}\in T^*\\
&\Leftrightarrow mn^{-1}K\in T/K\\
&\Rightarrow mn^{-1}t^{-1}\in K\\
&\Rightarrow mn^{-1}=tk\in T\\
&\Leftrightarrow mT=nT\\
\end{align*}

surjective:


If \(G\) is finite, then
\begin{align*}
[S^*:T^*]&=\abs{S^*}/\abs{T^*}\\
&=\abs{S/K}/\abs{T/K}\\
&=(\abs{S}/\abs{K})/(\abs{T}/\abs{K})\\
&=\abs{S}/\abs{T}\\
&=[S:T]
\end{align*}

If \(T\tril S\), by third isomorphism theorem, \(T/S\cong (T/K)/(S/K)=T^*/S^*\)

If \(T^*\tril S^*\), 
\begin{equation*}
\pi(sts^{-1})\in \pi(s)T^*\pi(s)^{-1}=T^*
\end{equation*}
so that \(sts^{-1}\in \pi^{-1}(T^*)=T\)
\end{proof}


\begin{proposition}[]
\label{111}
If \(G\) is a finite abelian group and \(d\) is a divisor of \(\abs{G}\), then \(G\)
contains a subgroup of order \(d\)
\end{proposition}

\begin{proof}
Abelian group's subgroup is normal and hence we can build quotient groups.
p90 for proof. Use the correspondence theorem
\end{proof}

\begin{definition}[]
If \(H\) and \(K\) are grops, then their \textbf{direct product}, denoted by 
\(H\times K\) 
, is the set of all ordered pairs \((h,k)\) with the operation
\begin{equation*}
(h,k)(h',k')=(hh',kk')
\end{equation*}
\end{definition}

\begin{proposition}[]
Let \(G\) and \(G'\) be groups and \(K\tril G, K'\tril G'\). Then \(K\times K'\tril
   G\times G'\) and
\begin{equation*}
(G\times G')/(K\times K')\cong (G/K)\times(G'/K')
\end{equation*}
\end{proposition}

\begin{proof}
   
\end{proof}

\begin{proposition}[]
If \(G\) is a group containing normal subgroups \(H\) and \(K\) and \(H\cap K=\{1\}\)
and \(HK=G\), then \(G\cong H\times K\)
\end{proposition}

\begin{proof}
Note \(\abs{HK}\abs{H\cap K}=\abs{H}\abs{K}\). Consider \(\varphi:G\to H\times
   K\). Show it's homo and bijective.
\end{proof}

\begin{theorem}[]
\label{thm2.81}
If \(m,n\) are relatively prime, then
\begin{equation*}
\I_{mn}\cong\I_m\times\I_n
\end{equation*}
\end{theorem}

\begin{proof}
\begin{align*}
f:&\Z\to\I_m\times\I_n\\
&a\mapsto([a]_m,[a]_n)
\end{align*}
is a homo.
\(\Z/\la mn\ra\cong\I_m\times\I_n\)
\end{proof}

\begin{proposition}[]
Let \(G\) be a group, and \(a,b\in G\) be commuting elements of orders \(m,n\). If
\((m,n)=1\), then \(ab\) has order \(mn\)
\end{proposition}

\begin{corollary}[]
If \((m,n)=1\), then \(\phi(mn)=\phi(m)\phi(n)\)
\end{corollary}

\begin{proof}
Theorem \ref{thm2.81} shows that \(f:\I_{mn}\cong\I_m\times\I_n\). The result will
follow if we prove that \(f(U(\I_{mn}))=U(\I_m)\times U(\I_n)\), for then
\begin{align*}
\phi(mn)&=\abs{U(\I_{mn})}=\abs{f(U(\I_{mn}))}\\
&=\abs{U(\I_m)\times U(\I_n)}=\abs{U(\I_m)}\cdot\abs{U(\I_n)}
\end{align*}
If \([a]\in U(\I_{mn})\), then \([a][b]=[1]\) for some \([b]\in\I_{mn}\) and
\begin{equation*}
f([ab])&=([ab]_m,[ab]_n)=([a]_m[b]_m,[a]_n[b]_n)\\
&=([a]_m,[a]_n)([b]_m,[b]_n)=([1]_m,[1]_n)
\end{equation*}
Hence \(f([a])=([a]_m,[a]_n)\in U(\I_m)\times U(\I_n)\)

For the reverse inclusion, if \(f([c])=([c]_m,[c]_n)\in U(\I_m)\times
   U(\I_n)\), then we must show that \([c]\in U(\I_{mn})\). There is \([d]_m\in\I_m\)
with \([c]_m[d]_m=[1]_m\), and there is \([e]_n\I_n\) with \([c]_n[e]_n=[1]_n\).
Since \(f\) is surjective, there is \(b\in\Z\) with
\(([b]_m,[b]_n)=([d]_m,[e]_n)\), so that
\begin{equation*}
f([1])=([1]_m,[1]_n)=([c]_m[b]_m,[c]_n[b]_n)=f([c][b])
\end{equation*}
Since \(f\) is an injection, \([1]=[c][b]\) and \([c]\in U(\I_{mn})\)
\end{proof}

\begin{corollary}[]
\begin{enumerate}
\item If \(p\) is a prime, then \(\phi(p^e)=p^e-p^{e-1}=p^e(1-\frac{1}{p})\)
\item If \(n=p_1^{e_1}\dots p_t^{e_t}\), then
\begin{equation*}
\phi(n)=n(1-\frac{1}{p_1})\dots(1-\frac{1}{p_t})
\end{equation*}
\end{enumerate}
\end{corollary}

\begin{lemma}[]
A cyclic group of order \(n\) has a unique subgroup of order \(d\), for each
divisor \(d\) of \(n\), and this subgroup is cyclic.
\end{lemma}

Define an equivalence relation on a group \(G\) by \(x\equiv y\) if \(\la x\ra=\la
   y\ra\). Denote the equivalence class containing \(x\) by \(\gen(C)\), where \(C=\la
   x\ra\). Equivalence classes form a partition and we get
\begin{equation*}
G=\displaystyle\prod_{C}\gen(C)
\end{equation*}
where \(C\) ranges over all cyclic subgroups of \(G\). Note \(\abs{\gen(C)}=\phi(n)\)

\begin{theorem}[]
A group \(G\) of order \(n\) is cyclic if and only if for each divisor \(d\) of
\(n\), there is at most one cyclic subgroup of order \(d\)
\end{theorem}

\begin{theorem}[]
   If \(G\) is an abelian group of order \(n\) having at most one cyclic subgroup o
f
   order \(p\) for each prime divisor \(p\) of \(n\), then \(G\) is cyclic
\end{theorem}

Exercise:
\begin{itemize}
\item 2.71 Suppose \(H\le G, \abs{H}=\abs{K}\). Since \(\abs{H}=[H:K]\abs{K}\),
\([H:K]=1\). Hence \(H=K\)
\item 2.67 1. \(\inn(S_3)\cong S_3/Z(S_3)\cong S_3\) and \(\abs{\aut(S_3)}\le 6\).
Hence \(\aut(S_3)=\inn(S_3)\)
\end{itemize}
\subsection{Group Actions}
\label{sec:orgf88ec28}
\begin{theorem}[Cayley]
Every group \(G\) is isomorphic to a subgroup of the symmetric group \(S_G\). In
particular, if \(\abs{G}=n\), then \(G\) is isomorphic to a subgroup of \(S_n\)
\end{theorem}

\begin{proof}
For each \(a\in G\), define \(\tau_a(x)=ax\) for every \(x\in G\). \(\tau_a\) is a
bijection for its inverse is \(\tau_{a^{-1}}\)
\begin{equation*}
\tau_a\tau_{a^{-1}}=\tau_1=\tau_{a^{-1}}\tau_a
\end{equation*}
\end{proof}

\begin{theorem}[Representation on Cosets]
Let \(G\) be a group and \(H\le G\) having finite index \(n\). Then there exists a
homomorphism \(\varphi:G\to S_n\) with \(\ker\varphi\le H\)
\end{theorem}

When \(H=\{1\}\), this is the Cayley theorem.

\begin{proposition}[]
Every group \(G\) of order 4 is isomorphic to either \(\I_4\) or the four-group
\(\bV\). And \(\I_4\not\cong\bV\)
\end{proposition}

\begin{proof}
By lagrange's theorem, every element in \(G\) other than 1 has order 2 or 4. If
4, then \(G\) is cyclic.

Suppose \(x,y\neq 1\), then \(xy\neq x,y\). Hence \(G=\{1,x,y,xy\}\).
\end{proof}


\begin{proposition}[]
If \(G\) is a group of order 6, then \(G\) is isomorphic to either \(\I_6\) or
\(S_3\). Moreover \(\I_6\not\cong S_3\)
\end{proposition}

\begin{proof}
If \(G\) is not cyclic. Since \(\abs{G}\) is even, it has some elements having
order 2, say \(t\).

If \(G\) is abelian. Suppose it has another different element \(a\) with order 2.
Then \(H=\{1,a,t,at\}\) is a subgroup which contradict. Hence it must contain
an element \(b\) of order 3. Then \(bt\) has order 6 and \(G\) is cyclic.

If \(G\) is not abelian. If \(G\) doesn't have elements of order 3, then it's
abelian. Hence \(G\) has an element \(s\) of order 3.

Now \(\abs{\la s\ra}=3\), so \([G:\la s\ra]=\abs{G}/\abs{\la s\ra}=2\) and \(\la
   s\ra\) is normal. Since \(t=t^{-1}\), \(tst\in\la s\ra\). If \(tst=s^0=1\), \(s=1\).
If \(tst=s\), \(\abs{\la st\ra}=6\). If \(tst=s^2=s^{-1}\).

Let \(H=\la t\ra\), \(\varphi:G\to S_{G/\la t\ra}\) given by
\begin{equation*}
\varphi(g):x\la t\ra\to gx\la t\ra
\end{equation*}
By representation on cosets, \(\ker\varphi\le\la t\ra\). Hence
\(\ker\varphi=\{1\}\) or \(\ker\varphi=\la t\ra\). Since
\begin{equation*}
\varphi(t)=
\begin{pmatrix}
\la t\ra&s\la t\ra&s^2\la t\ra\\
t\la t\ra&ts\la t\ra&ts^2\la t \ra
\end{pmatrix}
\end{equation*}
If \(\varphi(t)\) is the identity permutation, then \(ts\la t\ra=s\la t\ra\), so
that \(s^{-1}ts\in\la t\ra=\{1,t\}\). But now \(s^{-1}ts=t\). Therefore
\(t\not\in\ker\varphi\) and \(\ker\varphi=\{1\}\). Therefore \(\varphi\) is
injective. Because \(\abs{G}=\abs{S_3}\), \(G\cong S_3\)
\end{proof}


\begin{definition}[]
If \(X\) is a set and \(G\) is a group, then \(G\) \textbf{acts} on \(X\) if there is a
function \(G\times X\to X\), denoted by \((g,x)\to gx\) s.t.
\begin{enumerate}
\item (gh)x=g(hx) for all \(g,h\in G\) and \(x\in X\)
\item \(1x=x\) for all \(x\in X\)
\end{enumerate}


\(X\) is a \(G\)-set if \(G\) acts on \(X\)
\end{definition}

\begin{definition}[]
If \(G\) acts on \(X\) and \(x\in X\), then the \textbf{orbit} of \(x\), denoted by
\(\calo(x)\), is the subset of \(X\)
\begin{equation*}
\calo(x)=\{gx:g\in G\}\subseteq X
\end{equation*}
the \textbf{stabilizer} of \(x\), denoted by \(G_x\), is the subgroup
\begin{equation*}
G_x=\{g\in G:gx=x\}\le G
\end{equation*}
\end{definition}

\(G\) acts \textbf{transitively} on \(X\) if there is only one orbit.
*centralizer\} \(C_G(x)=\{g\in G:gxg^{-1}=x\*\)

\textbf{Normalizer}
\begin{equation*}
N_G(H)=\{g\in G:gHg^{-1}=H\}
\end{equation*}

When a group \(G\) acts on itself by conjugation, then
\begin{equation*}
\calo(x)=\{y\in G:y=axa^{-1} \text{ for some } a\in G\}
\end{equation*}
In this case, \(\calo(x)\) is called the \textbf{conjugacy class} of \(x\), denoted
by \(x^G\)

\begin{proposition}[]
If \(G\) acts on a set \(X\), then \(X\) is the disjoint union of the orbits. If
\(X\) is finite, then
\begin{equation*}
\abs{X}=\displaystyle\sum_i\abs{\calo(x_i)}
\end{equation*}
where \(x_i\) is chosen from each orbit
\end{proposition}

\begin{proof}
\(x\equiv y\Leftrightarrow\) there exists \(g\in G\) with \(y=gx\) is an
equivalence relation
\end{proof}


\begin{theorem}[]
\label{89}
If \(G\) acts on a set \(X\) and \(x\in X\) then
\begin{equation*}
\abs{\calo(x)}=[G:G_x]
\end{equation*}
\end{theorem}

\begin{proof}
Let \(G/G_x\) denote the family of cosets. Construct a bijection
\(\varphi:G/G_x\to \calo(x)\)
\end{proof}

\begin{corollary}[]
If a finite group \(G\) acts on a set X, then the number of elements in any
orbit is a divisor of \(\abs{G}\)
\end{corollary}

\begin{corollary}[]
If \(x\) lies in a finite group \(G\), then the number of conjugates of \(x\) is
the index of its centralizer
\begin{equation*}
\abs{x^G}=[G:C_G(x)]
\end{equation*}
and hence it's a divisor of \(G\)
\end{corollary}

\begin{proposition}[]
If \(H\) is a subgroup of a finite group \(G\), then the number of conjugates of
\(H\) in \(G\) is \([G:N_G(H)]\)
\end{proposition}

\begin{proof}
Similar to theorem \ref{89}
\end{proof}

\begin{theorem}[Cauchy]
If \(G\) is a finite group whose order is divisible by a prime \(p\), then \(G\)
contains an element of order \(p\)
\end{theorem}

\begin{proof}
Prove by induction on \(m\ge 1\), where \(\abs{G}=mp\). If \(m=1\), it's obvious.

If \(x\in HG\) , then \(\abs{x^G}=[G:C_G(x)]\). If \(x\not\in Z(G)\), then \(x^G\)
has more than one element, so \(abs{C_G(x)}<\abs{G}\). If \(p\mid \abs{C_G(x)}\), by
inductive hypothesis, we are done. Else if \(p\nmid \abs{C_G(x)}\) for all
noncentral \(x\) and \(\abs{G}=[G:C_G(x)]\abs{C_G(x)}\), we have
\begin{equation*}
p\mid[G:C_G(x)]
\end{equation*}
\(Z(G)\) consists of all those elements with \(\abs{X^G}=1\), we have
\begin{equation*}
\abs{G}=\abs{Z(G)}+\displaystyle\sum_i[G:C_G(x_i)]
\end{equation*}
Hence \(p\mid\abs{Z(G)}\) and by proposition \ref{111}
\end{proof}

\begin{definition}[]
The \textbf{class equation} of a finite group \(G\) is
\begin{equation*}
\abs{G}=\abs{Z(G)}+\displaystyle\sum_i[G:C_G(x_i)]
\end{equation*}
where each \(x_i\) is selected from each conjugacy class having more than one element
\end{definition}

\begin{definition}[]
If \(p\) is a prime, then a finite group \(G\) is called a \textbf{p-group} if
\(\abs{G}=p^n\) for some \(n\ge 0\)
\end{definition}

\begin{theorem}[]
If \(p\) is a prime and \(G\) is a p-group, then \(Z(G)\neq\{1\}\)
\end{theorem}

\begin{proof}
Consider 
\begin{equation*}
\abs{G}=\abs{Z(G)}+\displaystyle\sum_i[G:C_G(x_i)]
\end{equation*}
\end{proof}
\end{document}