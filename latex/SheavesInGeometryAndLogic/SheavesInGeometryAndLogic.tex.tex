% Created 2020-04-13 一 16:19
% Intended LaTeX compiler: pdflatex
\documentclass[11pt]{article}
\usepackage[utf8]{inputenc}
\usepackage[T1]{fontenc}
\usepackage{graphicx}
\usepackage{grffile}
\usepackage{longtable}
\usepackage{wrapfig}
\usepackage{rotating}
\usepackage[normalem]{ulem}
\usepackage{amsmath}
\usepackage{textcomp}
\usepackage{amssymb}
\usepackage{capt-of}
\usepackage{imakeidx}
\usepackage{hyperref}
\usepackage{minted}
\author{wugouzi}
\date{\today}
\title{}
\hypersetup{
 pdfauthor={wugouzi},
 pdftitle={},
 pdfkeywords={},
 pdfsubject={},
 pdfcreator={Emacs 26.3 (Org mode 9.3.6)}, 
 pdflang={English}}
\begin{document}

\tableofcontents \clearpage\% Created 2020-04-13 一 16:18
\% Intended \LaTeX{} compiler: pdflatex
\documentclass[11pt]{article}
\usepackage[utf8]{inputenc}
\usepackage[T1]{fontenc}
\usepackage{graphicx}
\usepackage{grffile}
\usepackage{longtable}
\usepackage{wrapfig}
\usepackage{rotating}
\usepackage[normalem]{ulem}
\usepackage{amsmath}
\usepackage{textcomp}
\usepackage{amssymb}
\usepackage{capt-of}
\usepackage{imakeidx}
\usepackage{hyperref}
\usepackage{minted}
% TIPS
% \substack{a\\b} for multiple lines text





% pdfplots will load xolor automatically without option
\usepackage[dvipsnames]{xcolor}

\usepackage{forest}
% two-line text in node by [two \\ lines]
% \begin{forest} qtree, [..] \end{forest}
\forestset{
  qtree/.style={
    baseline,
    for tree={
      parent anchor=south,
      child anchor=north,
      align=center,
      inner sep=1pt,
    }}}
%\usepackage{flexisym}
% load order of mathtools and mathabx, otherwise conflict overbrace

\usepackage{mathtools}
%\usepackage{fourier}
\usepackage{pgfplots}
\usepackage{amsthm}
\usepackage{amsmath}
%\usepackage{unicode-math}
%
\usepackage{commath}
%\usepackage{,  , }
\usepackage{amsfonts}
\usepackage{amssymb}
% importing symbols https://tex.stackexchange.com/questions/14386/importing-a-single-symbol-from-a-different-font
%mathabx change every symbol
% use instead stmaryrd
%\usepackage{mathabx}
\usepackage{stmaryrd}
\usepackage{empheq}
\usepackage{tikz}
\usepackage{tikz-cd}
%\usepackage[notextcomp]{stix}
\usetikzlibrary{arrows.meta}
\usepackage[most]{tcolorbox}
%\utilde
%\usepackage{../../latexpackage/undertilde/undertilde}
% left and right superscript and subscript
\usepackage{actuarialsymbol}
\usepackage{threeparttable}
\usepackage{scalerel,stackengine}
\usepackage{stackrel}
% \stackrel[a]{b}{c}
\usepackage{dsfont}
% text font
\usepackage{newpxtext}
%\usepackage{newpxmath}

%\newcounter{dummy} \numberwithin{dummy}{section}
\newtheorem{dummy}{dummy}[section]
\theoremstyle{definition}
\newtheorem{definition}[dummy]{Definition}
\newtheorem{corollary}[dummy]{Corollary}
\newtheorem{lemma}[dummy]{Lemma}
\newtheorem{proposition}[dummy]{Proposition}
\newtheorem{theorem}[dummy]{Theorem}
\theoremstyle{definition}
\newtheorem{example}[dummy]{Example}
\theoremstyle{remark}
\newtheorem*{remark}{Remark}


\newcommand\what[1]{\ThisStyle{%
    \setbox0=\hbox{$\SavedStyle#1$}%
    \stackengine{-1.0\ht0+.5pt}{$\SavedStyle#1$}{%
      \stretchto{\scaleto{\SavedStyle\mkern.15mu\char'136}{2.6\wd0}}{1.4\ht0}%
    }{O}{c}{F}{T}{S}%
  }
}

\newcommand\wtilde[1]{\ThisStyle{%
    \setbox0=\hbox{$\SavedStyle#1$}%
    \stackengine{-.1\LMpt}{$\SavedStyle#1$}{%
      \stretchto{\scaleto{\SavedStyle\mkern.2mu\AC}{.5150\wd0}}{.6\ht0}%
    }{O}{c}{F}{T}{S}%
  }
}

\newcommand\wbar[1]{\ThisStyle{%
    \setbox0=\hbox{$\SavedStyle#1$}%
    \stackengine{.5pt+\LMpt}{$\SavedStyle#1$}{%
      \rule{\wd0}{\dimexpr.3\LMpt+.3pt}%
    }{O}{c}{F}{T}{S}%
  }
}

\newcommand{\bl}[1] {\boldsymbol{#1}}
\newcommand{\Wt}[1] {\stackrel{\sim}{\smash{#1}\rule{0pt}{1.1ex}}}
\newcommand{\wt}[1] {\widetilde{#1}}
\newcommand{\tf}[1] {\textbf{#1}}


%For boxed texts in align, use Aboxed{}
%otherwise use boxed{}

\DeclareMathSymbol{\widehatsym}{\mathord}{largesymbols}{"62}
\newcommand\lowerwidehatsym{%
  \text{\smash{\raisebox{-1.3ex}{%
    $\widehatsym$}}}}
\newcommand\fixwidehat[1]{%
  \mathchoice
    {\accentset{\displaystyle\lowerwidehatsym}{#1}}
    {\accentset{\textstyle\lowerwidehatsym}{#1}}
    {\accentset{\scriptstyle\lowerwidehatsym}{#1}}
    {\accentset{\scriptscriptstyle\lowerwidehatsym}{#1}}
}

\usepackage{graphicx}
    
% text on arrow for xRightarrow
\makeatletter
%\newcommand{\xRightarrow}[2][]{\ext@arrow 0359\Rightarrowfill@{#1}{#2}}
\makeatother


\newcommand{\dom}[1]{%
\mathrm{dom}{(#1)}
}

% Roman numerals
\makeatletter
\newcommand*{\rom}[1]{\expandafter\@slowromancap\romannumeral #1@}
\makeatother

\def \fR {\mathfrak{R}}
\def \bx {\boldsymbol{x}}
\def \bz {\boldsymbol{z}}
\def \ba {\boldsymbol{a}}
\def \bh {\boldsymbol{h}}
\def \bo {\boldsymbol{o}}
\def \bU {\boldsymbol{U}}
\def \bc {\boldsymbol{c}}
\def \bV {\boldsymbol{V}}
\def \bI {\boldsymbol{I}}
\def \bK {\boldsymbol{K}}
\def \bt {\boldsymbol{t}}
\def \bb {\boldsymbol{b}}
\def \bA {\boldsymbol{A}}
\def \bX {\boldsymbol{X}}
\def \bu {\boldsymbol{u}}
\def \bS {\boldsymbol{S}}
\def \bZ {\boldsymbol{Z}}
\def \bz {\boldsymbol{z}}
\def \by {\boldsymbol{y}}
\def \bw {\boldsymbol{w}}
\def \bT {\boldsymbol{T}}
\def \bF {\boldsymbol{F}}
\def \bS {\boldsymbol{S}}
\def \bm {\boldsymbol{m}}
\def \bW {\boldsymbol{W}}
\def \bR {\boldsymbol{R}}
\def \bQ {\boldsymbol{Q}}
\def \bS {\boldsymbol{S}}
\def \bP {\boldsymbol{P}}
\def \bT {\boldsymbol{T}}
\def \bY {\boldsymbol{Y}}
\def \bH {\boldsymbol{H}}
\def \bB {\boldsymbol{B}}
\def \blambda {\boldsymbol{\lambda}}
\def \bPhi {\boldsymbol{\Phi}}
\def \btheta {\boldsymbol{\theta}}
\def \bTheta {\boldsymbol{\Theta}}
\def \bmu {\boldsymbol{\mu}}
\def \bphi {\boldsymbol{\phi}}
\def \bSigma {\boldsymbol{\Sigma}}
\def \lb {\left\{}
\def \rb {\right\}}
\def \la {\langle}
\def \ra {\rangle}
\def \caln {\mathcal{N}}
\def \dissum {\displaystyle\Sigma}
\def \dispro {\displaystyle\prod}
\def \E {\mathbb{E}}
\def \Q {\mathbb{Q}}
\def \N {\mathbb{N}}
\def \V {\mathbb{V}}
\def \R {\mathbb{R}}
\def \P {\mathbb{P}}
\def \A {\mathbb{A}}
\def \Z {\mathbb{Z}}
\def \I {\mathbb{I}}
\def \C {\mathbb{C}}
\def \cala {\mathcal{A}}
\def \calb {\mathcal{B}}
\def \calq {\mathcal{Q}}
\def \calp {\mathcal{P}}
\def \cals {\mathcal{S}}
\def \calg {\mathcal{G}}
\def \caln {\mathcal{N}}
\def \calr {\mathcal{R}}
\def \calm {\mathcal{M}}
\def \calc {\mathcal{C}}
\def \calf {\mathcal{F}}
\def \calk {\mathcal{K}}
\def \call {\mathcal{L}}
\def \calu {\mathcal{U}}
\def \bcup {\bigcup}


\def \uin {\underline{\in}}
\def \oin {\overline{\in}}
\def \uR {\underline{R}}
\def \oR {\overline{R}}
\def \uP {\underline{P}}
\def \oP {\overline{P}}

\def \Ra {\Rightarrow}

\def \e {\enspace}

\def \sgn {\operatorname{sgn}}
\def \gen {\operatorname{gen}}
\def \ker {\operatorname{ker}}
\def \im {\operatorname{im}}

\def \tril {\triangleleft}

% \varprod
\DeclareSymbolFont{largesymbolsA}{U}{txexa}{m}{n}
\DeclareMathSymbol{\varprod}{\mathop}{largesymbolsA}{16}

% \bigtimes
\DeclareFontFamily{U}{mathx}{\hyphenchar\font45}
\DeclareFontShape{U}{mathx}{m}{n}{
      <5> <6> <7> <8> <9> <10>
      <10.95> <12> <14.4> <17.28> <20.74> <24.88>
      mathx10
      }{}
\DeclareSymbolFont{mathx}{U}{mathx}{m}{n}
\DeclareMathSymbol{\bigtimes}{1}{mathx}{"91}
% \odiv
\DeclareFontFamily{U}{matha}{\hyphenchar\font45}
\DeclareFontShape{U}{matha}{m}{n}{
      <5> <6> <7> <8> <9> <10> gen * matha
      <10.95> matha10 <12> <14.4> <17.28> <20.74> <24.88> matha12
      }{}
\DeclareSymbolFont{matha}{U}{matha}{m}{n}
\DeclareMathSymbol{\odiv}         {2}{matha}{"63}


\newcommand\subsetsim{\mathrel{%
  \ooalign{\raise0.2ex\hbox{\scalebox{0.9}{$\subset$}}\cr\hidewidth\raise-0.85ex\hbox{\scalebox{0.9}{$\sim$}}\hidewidth\cr}}}
\newcommand\simsubset{\mathrel{%
  \ooalign{\raise-0.2ex\hbox{\scalebox{0.9}{$\subset$}}\cr\hidewidth\raise0.75ex\hbox{\scalebox{0.9}{$\sim$}}\hidewidth\cr}}}

\newcommand\simsubsetsim{\mathrel{%
  \ooalign{\raise0ex\hbox{\scalebox{0.8}{$\subset$}}\cr\hidewidth\raise1ex\hbox{\scalebox{0.75}{$\sim$}}\hidewidth\cr\raise-0.95ex\hbox{\scalebox{0.8}{$\sim$}}\cr\hidewidth}}}
\newcommand{\stcomp}[1]{{#1}^{\mathsf{c}}}


\author{Sanders Mac Lane \& Leke Moerdijk}
\date{\today}
\title{\aunclfamily\Huge Sheaves in Geometry and Logic}
\hypersetup\{
 pdfauthor=\{Sanders Mac Lane $\backslash$& Leke Moerdijk\},
 pdftitle=\{\aunclfamily\Huge Sheaves in Geometry and Logic\},
 pdfkeywords=\{\},
 pdfsubject=\{\},
 pdfcreator=\{Emacs 26.3 (Org mode 9.3.6)\}, 
 pdflang=\{English\}\}
\begin{document}

\maketitle \clearpage
\tableofcontents \clearpage
\section{Categorical Preliminaries}
\label{sec:org1080b60}
A \textbf{category} \(\bC\) consists of a collection of \textbf{objects}, a collection of
\textbf{morphisms} and four operations; two of these operations associate with each
morphism \(f\) of \(\bC\) its \textbf{domain} \(\dom(f)\) or \(\textd_0(f)\) and its
\textbf{codomain} \(\cod(f)\) or \(\textd_1(f)\), respectively, both of which are objects of
\(\bC\). The other two operations are operation which associates with each
object \(C\) of \(\bC\) a morphism \(1_C\) (or \(\id_C\)) of \(\bC\) called the
\textbf{identity morphism} of \(C\) and an operation of \(\bC\) s.t.
\(\textd_0(f)=\textd_1(g)\) another morphism \(f\circ g\). These operations
are required to satisfy the following axioms
\begin{enumerate}
\item \(\textd_0(1_C)=C=\textd_1(1_C)\)
\item \(\td_0(f\circ g)=\td_0(g),\td_1(f\circ g)=\td_1(f)\)
\item \(1_D\circ f=f,f\circ 1_C=f\)
\item \((f\circ g)\circ h=f\circ(g\circ h)\)
\end{enumerate}


In an arbitrary category \(\bC\), a morphism \(f:C\to D\) in \(\bC\) is called
an \textbf{isomorphism} if there exists a morphism \(g:D\to C\) s.t. \(f\circ g=1_D\)
and \(g\circ f=1_C\). If such a morphism \(f\) exists, one says that \(C\) is
isomorphic to \(D\) and one writes \(f:C\xrightarrow{\sim}D\) and \(C\cong D\)


\index{subobject} \index{equivalent}
A morphism \(f:C\to D\) is called an \textbf{epi(morphism)} if for any object \(E\) and
any two parallel morphisms \(g,h:D\rightrightarrows E\) in \(\bC\), \(gf=hf\)
implies \(g=h\); one writes \(f:C\twoheadrightarrow D\) to indicate that \(f\)
is an epimorphism. Dually, \(f:C\to D\) is called a \textbf{mono(morphism)} if for any
object \(B\) and any two parallel morphisms \(g,h:B\rrarrow C\) in \(\bC\),
(fg=fh) implies \(g=h\); in this case, one writes \(f:C\monoarrow D\). Two
monomorphisms \(f:A\monoarrow D\) and \(g:B\monoarrow D\) with a common
codomain \(D\) are called \textbf{equivalent} if there exists an isomorphism
\(h:A\xrightarrow{\sim}B\) with \(gh=f\). A \textbf{subobject} of \(D\) is an equivalence
class of monomorphisms into \(D\). The collection \(\Sub_{\bC}(D)\) of
subobjects of \(D\) carries a natural partial order defined by \([f]\le[g]\) iff
there is an \(h:A\to B\) s.t. \(f=gh\), where \([f]\) and \([g]\) are the
classes of \(f:A\monoarrow D\) and \(g:B\monoarrow D\)

\begin{center}
\begin{tikzcd}
A \arrow[r,rightarrowtail,"f"] \arrow[d,"h"] & D\\
B \arrow[ur,rightarrowtail,"g"']
\end{tikzcd}
\end{center}

If \(\bC\) is a category, we sometimes write \(\bC_0\) for its collection of
objects and \(\bC_1\) for its collection of mophisms. For two objects \(C\) and
\(D\), the collection of morphisms with domain \(C\) and codomain \(D\) is denoted
by one of the following three symbols

\begin{equation*}
\Hom_{\bC}(C,D),\quad\Hom(C,D),\quad\bC(C,D)
\end{equation*}

\index{locally small}
We shall tacitly assume we are working in some fixed universe \(U\) of sets.
Members of \(U\) are then called \textbf{small} sets, whereas a collection of members of
\(U\) which doesnot itself belong to \(U\) will sometimes be referred to as a \textbf{large} set.
Given such an ambient universe \(U\), a category \(\bC\) is \textbf{locally small} if for
any two objects \(C\) and \(D\) of \(\bC\) the hom-set \(\Hom_{\bC}(C,D)\) is a
small set, while \(\bC\) is called \textbf{small} if both \(\bC_0\) and \(\bC_1\) are
small sets.

Given a category \(\bC\), one can form a new category \(\bC^{\op}\), called
the \textbf{opposite} or \textbf{dual} category of \(\bC\), by taking the same objects but
reversing the direction of all the morphisms and the order of all
compositions.

Given a category \(\bC\) and an object \(C\) of \(\bC\), one can construct the
\textbf{comma category} or the \textbf{slice category} \(\bC/C\) (read: \(\bC\) over \(C\)):
object of \(\bC/C\) are morphisms of \(\bC\) with codomain \(C\), and morphisms
in \(\bC/C\) from one such object \(f:D\to C\) to another \(g:E\to C\) are
commutative triangles in \(\bC\)

\begin{center}
\begin{tikzcd}
D \arrow[dr,"f"'] \arrow[rr,"h"]& & E \arrow[dl,"g"]\\
&C
\end{tikzcd}
\end{center}

Given two categories \(\bC\) and \(\bD\), a \textbf{functor} from \(\bC\) to \(\bD\) is
an operation \(F\) which assigns to each objects \(C\) of \(\bC\) an object \(F(C)\)
of \(\bD\) and to each morphism \(f\) of \(\bC\) a morphism \(F(f)\) of \(\bD\) in
such a way that \(F\) respects the domain and codomain as well as the identities
and compositions.

For a category \(\bC\) there is an \textbf{identity functor} \(\id_{\bC}:\bC\to\bC\), and
for two functors \(F:\bC\to\bD\) and \(G:\bD\to\bE\) one can form a new functor
\(G\circ F:\bC\to\bE\) by \textbf{composition}

\index{natural transformation}
Let \(F\) and \(G\) be two functors from a category \(\bC\) to a category \(\bD\). A
\textbf{natural transformation} \(\alpha\) from \(F\) to \(G\), written \(\alpha:F\to G\), is
an operation associating with each object \(C\) of \(\bC\) a morphism
\(\alpha_C:FC\to GC\) of \(\bD\) in such a way that for any morphism 
\(f:C'\to C\) in \(\bC\), the diagram

\begin{center}
\begin{tikzcd}
FC' \arrow[r,"\alpha_{C'}"] \arrow[d,"F(f)"] & GC' \arrow[d,"G(f)"]\\
FC \arrow[r,"\alpha_C"] & GC
\end{tikzcd}
\end{center}

commutes. The morphism \(\alpha_C\) is called the \textbf{component} of \(\alpha\) at \(C\). If
every component of \(\alpha\) is an isomorphism, \(\alpha\) is said to be a \textbf{natural isomorphism}.
If \(\alpha:F\to G\) and \(\beta:G\to H\) are two natural transformation between
functors \(\bC\to\bD\), one can define composite natural transformation
\(\beta\circ\alpha\) by setting 
\begin{equation*}
(\beta\circ\alpha)_C=\beta_{G(C)}\circ\alpha_C
\end{equation*}

By fixed categories \(\bC\) and \(\bD\) this yields a new category \(\bD^{\bC}\):
the objects of \(\bD^{\bC}\) are functors from \(\bC\) to \(\bD\) while the
morphisms of \(\bD^{\bC}\) are natural transformations between such functors.
Categories so constructed are called \textbf{functor categories}

For categories \(\bC\) and \(\bD\), a functor \(F:\bC^{\op}\to\bD\) is also called a
\textbf{contravariant functor} from \(\bC\) to \(\bD\). In contrast, ordinary functors from
\(\bC\) to \(\bD\) are sometimes called *covariant. Thus
\(C'\mapsto\Hom_{\bC}(C',C)\) for fixed \(C\) yields a contravariant functor from
\(\bC\) to \(\Sets\), while \(C\mapsto\Hom_{\bC}(C',C)\) for fixed \(C'\) is the
covariant Hom-functor.
\begin{center}
\begin{tikzcd}
C'\arrow[r]\arrow[d]&\Hom_{\bC}(C',C)
\arrow[d]\\
C''\arrow[r]&\Hom_{\bC}(C'',C)
\end{tikzcd}
\end{center}

\index{full functor} \index{faithful functor}
A functor \(F:\bC\to\bD\) is called \textbf{full} (respectively \textbf{faithful}) if for any two
objects \(C\) and \(C'\) of \(\bC\), the operation
\begin{equation*}
\Hom_{\bC}(C',C)\to\Hom_{\bD}(FC',FC);\quad f\mapsto F(f)
\end{equation*}
induced by \(F\) is surjective (respectively injective). A functor
\(F:\bC\to\bD\) is called an \textbf{equivalence of categories} if \(F\) is full and
faithful and if any object of \(\bD\) is isomorphic to an object in the image of
\(F\). For example, if \(F:\bC\to\bD\) is a functor s.t. there exists a functor
\(G:\bD\to\bC\) and natural isomorphism \(\alpha:F\circ
  G\xrightarrow{\sim}\id_{\bD}\) and \(\beta:G\circ F\xrightarrow{\sim}\id_{\bC}\),
then \(F\) is an equivalence (and \(G\) is sometimes called a \textbf{quasi-inverse} for
\(F\)).

We say that an object \(X\) equipped with morphsims \(\pi_1:X\to A\) and
\(\pi_2:X\to B\) is a \textbf{product} of \(A\) and \(B\) if for any other object \(Y\) and any
two maps \(f:Y\to A\) and \(g:Y\to B\) there is a \textbf{unique} map \(h:Y\to X\) s.t.
\(\pi_1\circ h=f\) and \(\pi_2\circ h=g\) [this unique is denoted by 
\((f,g):Y\to X\) or sometimes \(\la f,g\ra\)]

\begin{center}
\begin{tikzcd}
& Y \arrow[d,dashrightarrow,"!"] \arrow[dl,"f"'] \arrow[dr,"g"]\\
A&X\arrow[l,"\pi_1"'] \arrow[r,"\pi_2"]&B
\end{tikzcd}
\end{center}


\section{Index}
\label{sec:org4490641}
\renewcommand{\indexname}{}
\printindex
\end{document}
\end{document}